\hypersetup{%
  pdfauthor={Thomas~Tuerk}
  pdftitle={Deep Embeeding of Separation Logic}
}

\title[Deep Embeeding of Separation Logic]{A Deep Embedding of a \\Decidable Fragment of Separation Logic in HOL}
\author[Tuerk, Gordon]{Thomas Tuerk \and Mike Gordon}
\date{ARG Lunch, 26th June 2006}

\newcommand{\smallfoot}{{\sf Smallfoot}}
\newcommand{\HOL}{{\sf HOL}}
\newcommand{\nil}{{\textsf{nil}}}
\newcommand{\pftrue}{{\textsf{true}}}
\newcommand{\pfequal}[2]{\ensuremath{#1 \doteq #2}}
\newcommand{\pfunequal}[2]{\ensuremath{#1 \not\doteq #2}}
\newcommand{\values}{{\emph{Values}}}
\newcommand{\valuesnil}{\ensuremath{\values_\nil}}
\newcommand{\vars}{{\emph{Vars}}}
\newcommand{\expr}{{\emph{Exp}}}
\newcommand{\fields}{{\emph{Fields}}}
\newcommand{\pf}{{\emph{pf}}}
\newcommand{\sfset}{{\emph{sf}}}
\newcommand{\modelspf}{{\models_{\textit{pf}}\ }}
\newcommand{\modelssf}{{\models_{\textit{sf}}\ }}
\newcommand{\modelsds}{{\models_{\textit{ds}}\ }}


\newcommand{\sfemp}{{\textsf{emp}}}
\newcommand{\sftree}{{\textsf{tree}}}
\newcommand{\sfpointsto}[2]{#1 \hookrightarrow [#2]}
\newcommand{\sfbintree}{{\textsf{bin-tree}}}
\newcommand{\sflist}{{\textsf{ls}}}

\newcommand{\varpf}[1]{\textit{pf}_{#1}}
\newcommand{\varsf}[1]{\textit{sf}_{#1}}
\newcommand{\varel}{\eta}
\newcommand{\varepl}{\pi}

\newcommand{\heapdistinct}{\textit{heap\_distinct}}
\newcommand{\entailment}[2]{#1\ \vdash\ #2}
\newcommand{\dom}{{\text{dom}}}
\newcommand{\tofin}{\xrightarrow{fin}}

\newcommand{\eqinferstyle}{
\mprset{fraction={={\raisebox{0pt}[5pt][5pt]{=}}=}}}


\begin{document}

\frame{\titlepage}

\frame{\frametitle{Overview}\tableofcontents[hideallsubsections]}

\section{Motivation}
\subsection*{}

\begin{frame}
\frametitle{Smallfoot}
\begin{itemize}
\item "\smallfoot\ is an automatic verification tool which checks separation
logic specifications of concurrent programs which manipulate
dynamically-allocated recursive data structures." (\smallfoot\ documentation)
\item developed by
\begin{itemize}
\item Cristiano Calcagno
\item Josh Berdine
\item Peter O'Hearn
\end{itemize}
\end{itemize}
\end{frame}

\begin{frame}[fragile]
\frametitle{Smallfoot II}

\begin{exampleblock}{Example from \texttt{list.sf}}
\begin{semiverbatim}
list_copy(p) [list(p)] \{
  local t;
  t = p;
  q = NULL;
  while(t != NULL) [list(q) * lseg(p,t) * list(t)] \{
    sq = q;
    q = new();
    q->tl = sq;
    t = t->tl;
  \}
\} [list(p) * list(q)]
\end{semiverbatim}
\end{exampleblock}
\end{frame}


\begin{frame}
\frametitle{Motivation}
\begin{itemize}
\item we deeply embedded the fragment of separation logic used by \smallfoot\
  in \HOL
\item a decision procedure for entailments has been implemented in \HOL
\item this formalisation may increase the trust in \smallfoot
\item it may be used as a basis for non decidable fragments of separation
  logic and an interactive proof environment
\item it may be extended to a \HOL\ implementation of \smallfoot
\end{itemize}
\end{frame}


\section{Basic Definitions}
\subsection*{}

\begin{frame}
\frametitle{Pure Formulae}
\begin{itemize}
\item an \alert{expression} is either a constant or a variable
\item \alert{nil} is a special constant
\item a \alert{stack} is a variable assignment
\item \alert{pure formulae} are inductively defined by
\begin{itemize}
\item $\pftrue$
\item $\pfequal {e_1} {e_2}$,  $\pfunequal {e_1} {e_2}$
\item $pf_1 \wedge pf_2$
\end{itemize}
\item the semantics of pure formulae with respect to a stack are defined in
  the natural way
\end{itemize}
\end{frame}


\begin{frame}
\frametitle{Spatial Formulae}
\begin{itemize}
\item a \alert{heap} is finite map $h : (Values \setminus \{nil\}) \tofin Fields \tofin Values$
\item \alert{spatial formulae} are inductively defined by
\begin{itemize}
\item $\sfemp$
\item $\sfpointsto {e} {t_1:e_1, \ldots, t_n : e_n}$
\item $sf_1 * sf_2$
\item $\sftree((t_1, \ldots, t_n), es, e)$
\end{itemize}
\item list segments and binary trees are defined as syntactic sugar
\begin{itemize}
\item $\sfbintree(l, r, e) := \sftree((l,r), nil, e)$
\item $\sflist(tl, e_1, e_2) := \sftree(tl, e_2, e_1)$
\end{itemize}
\end{itemize}
\end{frame}


\begin{frame}
\frametitle{Spatial Formulae II}
\begin{columns}[c]
\begin{column}{0.2\textwidth}
\begin{pgfpicture}{0cm}{0cm}{1.5 cm}{2cm}
   \color{black}
   \color<7->{alert}
   \pgfcircle[fill]{\pgfxy(1,0)}{0.2 cm}
   \pgfcircle[fill]{\pgfxy(0,0)}{0.2 cm}



   \color<6->{alert}
   \pgfsetendarrow{\pgfarrowtriangle{2pt}}

   \pgfline{\pgfxy(0.6,0.8)}{\pgfxy(0.9,0.2)}
   \pgfline{\pgfxy(0.4,0.8)}{\pgfxy(0.1,0.2)}


   \pgfputat{\pgfxy(0.1,0.6)}{\pgfbox[center,center]{l}}
   \pgfputat{\pgfxy(0.9,0.6)}{\pgfbox[center,center]{r}}

   \pgfcircle[stroke]{\pgfxy(0.5,1)}{0.2 cm}
   \pgfputat{\pgfxy(0.5,1)}{\pgfbox[center,center]{2}}


   \color<5->{alert}
   \pgfcircle[fill]{\pgfxy(1.5,1)}{0.2 cm}



   \color<4->{alert}
   \pgfputat{\pgfxy(0.6,1.6)}{\pgfbox[center,center]{l}}
   \pgfputat{\pgfxy(1.4,1.6)}{\pgfbox[center,center]{r}}

   \pgfline{\pgfxy(0.9,1.8)}{\pgfxy(0.6,1.2)}
   \pgfline{\pgfxy(1.1,1.8)}{\pgfxy(1.4,1.2)}


   \pgfcircle[stroke]{\pgfxy(1,2)}{0.2 cm}
   \pgfputat{\pgfxy(1,2)}{\pgfbox[center,center]{1}}


\end{pgfpicture}
\end{column}
\begin{column}{0.8\textwidth}
$s (x_i) := i$\\
$\alert<7->{h := [\alert<4->{1 \to [l \to 2, r \to nil]}, \alert<6->{2 \to [l \to nil, r \to nil]}]}$
\end{column}
\end{columns}
\bigskip
\bigskip
\begin{center}
\only<1>{{$\sfbintree(l,r,x_1)$}}
\only<2>{{$\sftree((l,r),\nil,x_1)$}}
\only<3>{{$\exists e_l, e_r. \ \sfpointsto{x_1} {l:e_l, r:e_r} *
    \sftree((l,r),\nil,e_l) * \sftree((l,r),\nil,e_r)$}}
\only<4>{{$\alert{\sfpointsto{x_1} {l:2, r:\nil}} * \sftree((l,r),\nil,2) *
    \sftree((l,r),\nil,\nil)$}}
\only<5>{{$\alert{\sfpointsto{x_1} {l:2, r:\nil}} * \sftree((l,r),\nil,2)
    * \alert{\sftree((l,r),\nil,\nil)}$}}
\only<6>{{$\alert{\sfpointsto{x_1} {l:2, r:\nil}} * \alert{\sfpointsto{2}
      {l:\nil, r:\nil}}\ *$ $\sftree((l,r),\nil,\nil) *
    \sftree((l,r),\nil,\nil) * \alert{\sftree((l,r),\nil,\nil)}$}}
\only<7>{\alert{$\sfpointsto{x_1} {l:2, r:\nil} * \sfpointsto{2}
      {l:\nil, r:\nil}\ *$ $\sftree((l,r),\nil,\nil) *
    \sftree((l,r),\nil,\nil) * \sftree((l,r),\nil,\nil)$}}
\end{center}
\end{frame}

\section[Entailments]{A Decision Procedure for Entailments}
\subsection*{}

\begin{frame}[fragile]
\frametitle{Entailments}

\begin{itemize}
\item entailments are important for \smallfoot
\item example from \texttt{list.sf}
\[\small\begin{array}{lr}
\pfunequal {x_0} \nil,\pfunequal {x_1} \nil, \pfunequal {x_2} \nil,
\pfunequal {x_0} {x_3}, \pfunequal {x_0} {x_2}, \pfunequal {x_4} {x_5},\\
\pfunequal {x_1} {x_3},
\pfunequal {x_1} {x_2}, \pfunequal {x_3} {x_2},\\
\sfpointsto {x_2} {\textit{hd}:x_5, \textit{tl}:x_3},
\sflist(\textit{tl},x_3,\nil), \sflist(\textit{tl},x_1,x_0),
\sfpointsto {x_0} {\textit{tl},x_2} & \vdash \\
\sflist(\textit{tl},x_1,x_2), \sfpointsto {x_2} {\textit{tl}:x_3}, \sflist(\textit{tl},x_3,\nil)
\end{array}
\]
\item entailments in this fragment of separation logic are decidable
\item inferences can be easily combined to form a decision procedure
\item inferences and decision procedure are presented in
\begin{semiverbatim}\small
Josh Berdine, Cristiano Calcagno, and Peter W. O'Hearn.
Symbolic Execution with Separation Logic.
In K. Yi (Ed.): APLAS 2005, LNCS 3780, pp. 52-68, 2005.
\end{semiverbatim}
\end{itemize}
\end{frame}

\begin{frame}
\frametitle{\textsf{RemoveTrivial}}
\small\[\begin{array}{l}
\inferrule[RemoveTrivial-EQ-L]{\entailment{\Pi,\Sigma}{\Pi',\Sigma'}}
{\entailment{\alert{\pfequal e e},\Pi,\Sigma}{\Pi', \Sigma'}}
\bigskip\\
\inferrule[RemoveTrivial-EQ-R]{\entailment{\Pi,\Sigma}{\Pi',\Sigma'}}
{\entailment{\Pi,\Sigma}{\alert{\pfequal e e}, \Pi', \Sigma'}}
\bigskip\\
\inferrule[RemoveTrivial-EmpTree-L]{\entailment{\Pi,\Sigma}{\Pi',\Sigma'}}
{\entailment{\Pi,\alert{\sftree((t_1, \ldots, t_k),e,e)},\Sigma}{\Pi', \Sigma'}}
\bigskip\\
\inferrule[RemoveTrivial-EmpTree-R]{\entailment{\Pi,\Sigma}{\Pi',\Sigma'}}
{\entailment{\Pi,\Sigma}{\Pi',\alert{\sftree((t_1, \ldots, t_k),e,e)}, \Sigma'}}
\end{array}
\]
\end{frame}


\begin{frame}
\frametitle{\textsf{Hypothesis}}
\small\[
\inferrule[Hypothesis]{\entailment{\alert{\varpf{}},\Pi,\Sigma}{\Pi',\Sigma'}}
{\entailment{\alert{\varpf{}},\Pi,\Sigma}{\alert{\varpf {}},\Pi', \Sigma'}}
\]
\end{frame}

\begin{frame}
\frametitle{\textsf{Axiom / Inconsistent}}
\small\[\begin{array}{ll}
\inferrule[Axiom]{\ }
{\entailment{\Pi}{\ }}
\bigskip\\
\inferrule[Inconsistent-unequal]
{\ }
{\entailment{\alert{\pfunequal e e},\Pi,\Sigma}{\Pi',\Sigma'}}
&
\inferrule[Inconsistent-pointsto-nil]
{\ }
{\entailment{\Pi,\alert{\sfpointsto \nil {\ldots}}, \Sigma}{\Pi',\Sigma'}}
\end{array}
\]
\end{frame}


\begin{frame}
\frametitle{\textsf{Frame}}
\small
\[
\inferrule[Frame-base]{\entailment{\Pi,\Sigma}{\Pi',\Sigma'}}
{\entailment{\Pi,\alert{\varsf{}},\Sigma}{\Pi', \alert{\varsf{}},\Sigma'}}
\]

\begin{itemize}
\item problem: this is a real implication, information is lost
\item thus, order of inference application matters
\item example: $\entailment{\alert{\sfpointsto e {f:e_1}}, \sfpointsto e
    {g:e_2}}{\alert{\sfpointsto e {f:e_1}}}$
\item solution: add additional informations to entailments
\end{itemize}
\end{frame}

\begin{frame}
\frametitle{\textsf{Frame II}}
\vspace{-1cm}
\[\small\begin{array}{l}
\eqinferstyle
\inferrule[Frame-points\_to]
{\entailment{\alert{e},\eta,\pi,\Pi,\Sigma}{\Pi',\Sigma'}}
{\entailment{\eta,\pi,\Pi,\alert{\sfpointsto e {t_1:e_1, \ldots,
      t_n:e_n}},\Sigma}{\Pi', \alert{\sfpointsto e {t_1:e_1, \ldots,
      t_m:e_m}},\Sigma'}}
m \leq n
\bigskip\\
\eqinferstyle
\inferrule[Frame-tree]
{\entailment{\eta,\alert{(e,es)},\pi,\Pi,\Sigma}{\Pi',\Sigma'}}
{\entailment{\eta,\pi,\Pi,\alert{\sftree((t_1, \ldots, t_k),es,e)},\Sigma}{\Pi', \alert{\sftree((t_1, \ldots, t_k),es,e)},\Sigma'}}
\end{array}
\]
\begin{itemize}
\item all used inferences are equivalences
\item order of application does not matter
\item continued applications of inferences will terminate
\end{itemize}
\end{frame}


\begin{frame}
\frametitle{\textsf{Substitution}}
\[\small
\eqinferstyle
\inferrule[Substitution]{\entailment{\eta[e/x],\pi[e/x],\Pi[e/x],\Sigma[e/x]}{\Pi'[e/x],\Sigma'[e/x]}}
{\entailment{\eta,\alert{\pfequal x e},\pi,\Pi,\Sigma}{\Pi', \Sigma'}}
\]
\end{frame}

\begin{frame}
\frametitle{\textsf{NIL-NOT-LVAL}}
\[\small\begin{array}{l}
\eqinferstyle
\inferrule[NIL-NOT-LVAL-pointsto]
{\entailment{\eta,\pi,\alert{\pfunequal e \nil}, \Pi,\alert{\sfpointsto e {\ldots}},\Sigma}{\Pi',\Sigma'}}
{\entailment{\eta,\pi,\Pi,\alert{\sfpointsto e {\ldots}},\Sigma}{\Pi',\Sigma'}}
\bigskip\\
\eqinferstyle
\inferrule[NIL-NOT-LVAL-tree]
{\entailment{\eta,\pi,\alert{\pfunequal e \nil}, \alert{\pfunequal e \textit{es}},\Pi,\alert{\sftree\left(\ldots,\textit{es},e\right)},\Sigma}{\Pi',\Sigma'}}
{\entailment{\eta,\pi,\alert{\pfunequal e \textit{es}},\Pi,\alert{\sftree\left(\ldots,\textit{es},e\right)},\Sigma}{\Pi',\Sigma'}}
\end{array}
\]
\begin{alertblock}{side condition}
In order to prevent looping only new facts are added.
\end{alertblock}
\end{frame}


\begin{frame}
\frametitle{\textsf{Partial}}
\[\small\begin{array}{l}
\eqinferstyle
\inferrule[Partial-pointsto-pointsto]
{\entailment{\eta,\pi,\alert{\pfunequal {e_1} {e_2}}, \Pi,\alert{\sfpointsto
      {e_1} {\ldots}},\alert{\sfpointsto {e_2} {\ldots}},\Sigma}{\Pi',\Sigma'}}
{\entailment{\eta,\pi,\Pi,\alert{\sfpointsto {e_1} {\ldots}},\alert{\sfpointsto {e_2}
    {\ldots}},\Sigma}{\Pi',\Sigma'}}
\bigskip\\
\eqinferstyle
\inferrule[Partial-pointsto-tree]
{\entailment{\eta,\pi,\alert{\pfunequal {e_1} {e_2}}, \alert{\pfunequal {e_2} {e_3}},
    \Pi,\alert{\sfpointsto {e_1} {\ldots}},\alert{\sftree (\ldots, {e_3}, e_2)},\Sigma}{\Pi',\Sigma'}}
{\entailment{\eta,\pi,\alert{\pfunequal {e_2} {e_3}},\Pi,\alert{\sfpointsto {e_1} {\ldots}},\alert{\sftree (\ldots, {e_3}, e_2)},\Sigma}{\Pi',\Sigma'}}
\bigskip\\
\vdots
\end{array}
\]
\begin{alertblock}{side condition}
In order to prevent looping only new facts are added.
\end{alertblock}
\end{frame}


\begin{frame}
\frametitle{\textsf{Simple Unroll}}
\vspace{-0.5cm}
\[\small\begin{array}{l}
\eqinferstyle
\inferrule[Unroll-right-list]
{\entailment{\alert{e_1},\eta,\pi,\alert{\pfunequal {e_1} {e_3}}, \Pi,\Sigma}{\Pi',\sflist(\textit{tl}, \alert{e_2}, e_3),\Sigma'}}
{\entailment{\eta,\pi,\alert{\pfunequal {e_1} {e_3}}, \Pi,\alert{\sfpointsto{e_1}
    {\textit{tl}:e_2, \ldots}}, \Sigma}{\Pi',\alert{\sflist(\textit{tl}, e_1, e_3)},\Sigma'}}
\bigskip\\
\eqinferstyle
\inferrule[Unroll-right-bintree]
{\entailment{\alert{e},\eta,\pi, \Pi,\Sigma}{\Pi',\alert{\sfbintree(l,r, e_l)},\alert{\sfbintree(l,r, e_r)},\Sigma'}}
{\entailment{\eta,\pi,\Pi,\alert{\sfpointsto{e}
    {l:e_l, r:e_r, \ldots}}, \Sigma}{\Pi',\alert{\sfbintree(l,r,e)},\Sigma'}}
\bigskip\\
\eqinferstyle
\inferrule[Unroll-NilList]
{\entailment{\eta,\pi,\alert{\pfequal {e} {\nil}}, \Pi,\Sigma}{\Pi',\Sigma'}}
{\entailment{\eta,\pi,\Pi,\alert{\sflist(\textit{tl}, \nil, e)},\Sigma}{\Pi',\Sigma'}}
\bigskip\\
\eqinferstyle
\inferrule[Unroll-precond-list]
{\entailment{\alert{e_1},\eta,\pi,\alert{\pfequal {e_1} {e_2}}, \Pi,\Sigma}{\Pi',\Sigma'}}
{\entailment{\alert{e_1},\eta,\pi,\Pi,\alert{\sflist(\textit{tl}, e_1, e_2)},\Sigma}{\Pi',\Sigma'}}
\end{array}
\]
\end{frame}

\begin{frame}
\frametitle{\textsf{Unroll}}
\[\small
\eqinferstyle
\inferrule[Unroll-list]
{\entailment{\eta,\pi,\alert{\pfequal {e_1} {e_2}}, \Pi,\Sigma}{\Pi',\Sigma'} \\\\
 \alert{\forall x.}\ \entailment{\eta,\pi,\alert{\pfunequal {e_1} {e_2}},\alert{\pfunequal {e_2} {x}},
   \Pi,\alert{\sfpointsto {e_1} {\textit{fl}:x}}, \alert{\sfpointsto {x} {\textit{fl}:e_2}},
   \Sigma}{\Pi',\Sigma'}}
{\entailment{\eta,\pi,\Pi,\alert{\sflist(\textit{tl}, e_1, e_2)},\Sigma}{\Pi',\Sigma'}}
\]
\begin{itemize}
\item logic can not consider the content of a list
\item therefore, two cases are sufficient
\item no induction needed!
\item similar inference exists for arbitrary trees
\item however, in general useful for decision procedure
\end{itemize}
\end{frame}

\begin{frame}
\frametitle{\textsf{Append-list}}
\[\small
\eqinferstyle
\inferrule[Append-list]
{\entailment{\eta,(e_1,e_2),\pi,\alert{\pfunequal{e_1}{e_3}},\Pi,\Sigma}{\Pi',\alert{\sflist(\textit{tl},e_2, e_3)},\Sigma'}}
{\entailment{\eta,\pi,\alert{\pfunequal{e_1}{e_3}},\Pi,\alert{\sflist(\textit{tl},e_1,
    e_2)},\Sigma}{\Pi',\alert{\sflist(\textit{tl},e_1, e_3)},\Sigma'}}
\]
\begin{itemize}
\item this inference holds under some complicated side condition
\item it is preferable to unrolling lists
\item its correctness proof uses unrolling of lists and simple unrolls and the
  frame inference
\end{itemize}
\end{frame}


\begin{frame}
\frametitle{\textsf{Decision Procedure}}
\begin{itemize}
\item these inferences can be easily combined to form a decision procedure for entailments
\item apply inferences in arbitrary order as long as possible
\item be careful with \textsf{NIL-NOT-LVAL}, \textsf{Partial},
  \textsf{Hypothesis} to avoid looping
\item iff the entailment could not be reduced to true, it is false
\item remaining entailments are as simple, that a concrete counterexample can
  be easily constructed
\end{itemize}

\end{frame}

\section{HOL embedding}
\subsection*{}

\begin{frame}
\frametitle{HOL embedding}
\begin{itemize}
\item deep embedding in HOL is straight forward except for trees
\item trees are introduced considering their maximal depth
\item equivalence with other recursive definition is formally proofed
\item inferences are implemented as conversions
\item the decision procedure is implemented as a conversion
\end{itemize}
\end{frame}

\begin{frame}
\frametitle{HOL embedding II}

\begin{tabular}{|l|r|}
\hline
  & \textbf{LOC} \\\hline\hline
deep embedding and inferences & approx. 10\,000 \\
special versions of inferences for conversions & approx. 2\,500 \\
conversions and decision procedure & approx. 2\,000 \\\hline
\end{tabular}

\end{frame}

\section{Example}
\subsection*{}

\begin{frame}
\frametitle{example from \texttt{list.sf}}
\[\small\begin{array}{lr}
\only<1>{\alert{\pfunequal {x_0} \nil,\pfunequal {x_1} \nil, \pfunequal {x_2} \nil,
\pfunequal {x_0} {x_3}, \pfunequal {x_0} {x_2}, \pfunequal {x_4} {x_5},}}\\
\only<1>{\alert{\pfunequal {x_1} {x_3},
\pfunequal {x_1} {x_2}, \pfunequal {x_3} {x_2}}}
\only<2>{\Pi},\\
\sfpointsto {x_2} {\textit{hd}:x_5, \textit{tl}:x_3},
\sflist(\textit{tl},x_3,\nil), \sflist(\textit{tl},x_1,x_0),
\sfpointsto {x_0} {\textit{tl},x_2} & \vdash \\
\sflist(\textit{tl},x_1,x_2), \sfpointsto {x_2} {\textit{tl}:x_3}, \sflist(\textit{tl},x_3,\nil)
\end{array}
\]
\end{frame}

\begin{frame}
\frametitle{example inference application}
\[\small\begin{array}{llc}
\Pi,\alert<2>{\sfpointsto {x_2} {\textit{hd}:x_5, \textit{tl}:x_3}},\\
\alert<2>{\sflist(\textit{tl},x_3,\nil)}, \sflist(\textit{tl},x_1,x_0),
\sfpointsto {x_0} {\textit{tl},x_2} & \vdash & \onslide<2->{\textsc{\tiny Frame}} \\
\sflist(\textit{tl},x_1,x_2), \alert<2>{\sfpointsto {x_2} {\textit{tl}:x_3}},
\alert<2>{\sflist(\textit{tl},x_3,\nil)} & & \pause \alert<2>{\Longleftrightarrow} \\
\\

\alert<2>{x_2},\alert<2>{(x_3,\nil)},\Pi, \alert<3>{\sflist(\textit{tl},x_1,x_0)},
\sfpointsto {x_0} {\textit{tl},x_2} & \vdash & \onslide<3->{\text{\textsc{\tiny Append-List}}} \\
\alert<3>{\sflist(\textit{tl},x_1,x_2)} & & \pause \alert<3>{\Longleftrightarrow} \\
\\


x_2,(x_3,\nil),\alert<3>{(x_1,x_0)},\Pi,
\alert<4>{\sfpointsto {x_0} {\textit{tl},x_2}} & \vdash & \onslide<4->{\textsc{\tiny Simple-Unroll}} \\
\alert<4>{\sflist(\textit{tl},\alert<3>{x_0},x_2)} & & \pause \alert<4>{\Longleftrightarrow} \\
\\


x_2,\alert<4>{x_0},(x_3,\nil),(x_1,x_0),\Pi & \vdash & \onslide<5->{\textsc{\tiny Remove-Trivial}} \\
\alert<5>{\sflist(\textit{tl},\alert<4>{x_2},x_2)} & & \pause\alert<5>{\Longleftrightarrow} \\
\\


x_2,x_0,(x_3,\nil),(x_1,x_0),\Pi & \vdash & \onslide<6->{\textsc{\tiny Axiom}} \\
 & & \pause \alert<6>{\Longleftrightarrow} \\

\top
\end{array}
\]
\end{frame}

\begin{frame}[fragile]
\frametitle{example HOL}
\begin{semiverbatim}\scriptsize
val t = ``LIST_DS_ENTAILS ([],[])
  ([pf_unequal (dse_var 0) dse_nil;
    pf_unequal (dse_var 1) dse_nil;
    pf_unequal (dse_var 2) dse_nil;
    pf_unequal (dse_var 0) (dse_var 3);
    pf_unequal (dse_var 0) (dse_var 2);
    pf_unequal (dse_var 4) (dse_var 5);
    pf_unequal (dse_var 1) (dse_var 3);
    pf_unequal (dse_var 1) (dse_var 2);
    pf_unequal (dse_var 3) (dse_var 2)],
   [sf_points_to (dse_var 2) [("hd",dse_var 5); ("tl", dse_var 3)];
    sf_ls "tl" (dse_var 3) dse_nil;
    sf_ls "tl" (dse_var 1) (dse_var 0);
    sf_points_to (dse_var 0) [("tl", (dse_var 2))]])

   ([],
    [sf_ls "tl" (dse_var 1) (dse_var 2);
     sf_points_to (dse_var 2) [("tl", dse_var 3)];
     sf_ls "tl" (dse_var 3) dse_nil])``;
\end{semiverbatim}
\end{frame}

\begin{frame}[fragile]
\frametitle{example HOL II}
\begin{semiverbatim}\scriptsize
val thm1 =
(ds_inference_FRAME___CONV THENC
 ds_inference_APPEND_LIST___CONV THENC
 ds_inference_SIMPLE_UNROLL___CONV THENC
 ds_inference_REMOVE_TRIVIAL___CONV THENC
 ds_inference_AXIOM___CONV) t;

val thm2 = ds_DECIDE_CONV t;
\end{semiverbatim}
\end{frame}

\section{Conclusions and Future Work}
\subsection*{}

\begin{frame}
\begin{block}{Conclusions}
\begin{itemize}
\item there is a deep embedding of the decidable fragment of separation logic
  used by \smallfoot
\item all inferences used by \smallfoot\ have been verified using HOL
\item a decision procedure for entailments has been implemented
\end{itemize}
\end{block}

\begin{block}{Future Work}
\begin{itemize}
\item add symbolic execution to build a \smallfoot\ implementation in HOL
\item extend the logic
\item try interactive proofs for more complicated fragments of separation logic
\end{itemize}
\end{block}

\end{frame}

\end{document}

%%% Local Variables:
%%% mode: latex
%%% TeX-master: "presentation"
%%% End:
