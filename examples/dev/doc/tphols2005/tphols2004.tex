\documentclass{llncs}
\usepackage{epic,eepic,subfigure,alltt}
%-------------------------------------------------------------------------------
% File: notation.tex
%-------------------------------------------------------------------------------

%-------------------------------------------------------------------------------
% BNF
%-------------------------------------------------------------------------------
\newcommand{\bnfSC}[1]{\ensuremath{\mathit{\langle#1\rangle}}} % Syntax construct
\newcommand{\bnfDef}{\ensuremath{\mathit{::=}}} % ::=
\newcommand{\bnfList}[1]{\ensuremath{\{#1\}}}
\newcommand{\bnfOR}[1]{\ensuremath{|\ #1}}
\newcommand{\bnfOpt}[1]{\ensuremath{[#1]}}

%-------------------------------------------------------------------------------
% Mathematics
%-------------------------------------------------------------------------------
\newcommand{\mE}[1]{\ensuremath{\mathit{#1}}} % Math environment
\newcommand{\argm}[1]{\ensuremath{\vec{#1}}} % list of arguments or parameters
\newcommand{\VAR}[1]{\ensuremath{\mathit{#1}}} % variable
\newcommand{\OR}{\ensuremath{\vee}} % logical OR
\newcommand{\AND}{\ensuremath{\wedge}} % logical AND
\newcommand{\NOT}{\ensuremath{\neg}} % logical NOT
%\newcommand{\IF}[3]{\ensuremath{#1\, \rightarrow\, #2\, \mid\ #3}} % logical IF
\newcommand{\IF}[3]{\ensuremath{\texttt{if}~#1~\texttt{then}~#2~\texttt{else}~#3}} % logical IF
\newcommand{\TURNST}{\ensuremath{\vdash}} % turnstile
\newcommand{\IMP}{\ensuremath{\Rightarrow}} % implies
\newcommand{\NIMP}{\ensuremath{\not\supset}} % doesn't imply
\newcommand{\DEF}[1]{\ensuremath{\mathsf{#1}}} % definition
\newcommand{\sVAR}[2]{\ensuremath{\mE{#1}_{#2}}} % variable with subscript
\newcommand{\LAMBDA}[2]{\ensuremath{\lambda\, #1.\ #2}} % lambda abstraction
\newcommand{\COMP}[2]{\ensuremath{#1 \circ #2}} % function compositon
\newcommand{\MULT}{\ensuremath{\cdot}} % multiplication

%\newtheorem{example}{Example}

% added by MJCG
\newcommand{\TY}[1]{\ensuremath{\sl{#1}}} % type
\newcommand{\TT}[1]{\ensuremath{\texttt{#1}}} % keyword


%-------------------------------------------------------------------------------
\title{Higher-Level Hardware Synthesis in HOL\\[-4mm]
%{\small(submitted to TPHOLs2004 under {\it  Emerging Trends})}
}
\author{Juliano Iyoda \and  Michael J. C. Gordon}
\institute{University of Cambridge Computer Laboratory \\
           William Gates Building, 15 JJ Thomson Avenue,
           Cambridge CB3 0FD, U.K. \\
           \begin{sl}\{Juliano.Iyoda,Mike.Gordon\}@cl.cam.ac.uk\end{sl}}
\begin{document}
\maketitle

\vspace*{-5mm}
%-------------------------------------------------------------------------------
\thispagestyle{empty}

\begin{abstract}
  A simple functional language embedded in higher order logic is used
  as a hardware description language. Our approach uses proof scripts
  to synthesise circuits directly from logical specifications.  As
  well as synthesising implementations, we also generate theorems
  exhibiting their correctness. Our goal is to experiment with
  synthesis by proof along a spectrum of automation ranging from
  push-button compilation to user guided refinement.  This paper
  describes formal compilation to synchronous implementations with a
  handshaking interface.
\vspace*{-2mm}
\end{abstract}

%-------------------------------------------------------------------------------
\section{Introduction}
\label{secIntroduction}
%-------------------------------------------------------------------------------


We describe an approach to hardware synthesis by mechanised proof.
A compiler, implemented as a proof rule, transforms specifications
expressed in a simple functional language embedded in higher order
logic into hardware devices that interact via a handshaking protocol.
This approach allows the designer to focus solely on the high level
behaviour of the system without having to reason about the correctness
of the circuit at the gate level.

Our compilation method is partly inspired by SAFL \cite{MS01b},
especially ideas in Richard Sharp's PhD \cite{Sha02}. Our long term
goal is to develop correct-by-construction SAFL-like formal synthesis
by proof. The current paper is only a very first proof-of-concept
step.


Higher order logic (HOL) \cite{GM93} has already been successfully
applied to specify and verify hardware \cite{Fox01,Gor86,Mel93}, and
functional programming languages have been used as hardware
description languages \cite{CP02,O'D87,Sha02}.  Formal synthesis by
proof has previously been investigated by, among others, Johnson and
Bose \cite{Johnson90}, Hanna \cite{Hanna92}, Fourman \cite{fourman89b}
and a researchers at Karlsruhe on high-level synthesis using the
{\it Gropius\/} language \cite{Gropius1,Gropius2}. 

The novelty of our work is (i) the details of the device interface,
(ii) the implementation of synthesis by deduction (rather than by the
application of pre-verified transformations) and (iii) the way
synthesis results are encoded as composable theorems certifying the
correctness of the synthesised implementations.

Section~\ref{secASimpleLanguage} introduces the simple functional
language used as source code.  Section~\ref{secHandshakingDevices}
defines the specification of generic handshaking devices to be used
during the compilation.

The implementation and the verification of handshaking devices are
presented in Section~\ref{secImplementingHandshakingDevices}. 
The synthesis-by-proof algorithm is described in Section~\ref{secCompilingByProof}
and is illustrated by a case study in Section~\ref{secTheFactorialCaseStudy}.
Finally, conclusions and future work are
outlined in Section~\ref{secConclusionAndFutureWork}.

%-------------------------------------------------------------------------------
\section{A Simple Language}
\label{secASimpleLanguage}
%-------------------------------------------------------------------------------
In collaboration with Konrad Slind of the University of Utah, we
eventually plan to compile from an ML-like subset of higher order
logic, but in this paper we start from an intermediate language
consisting of expressions built using a set of simple operators.
These are quite expressive, and the construction of a front end to
parse into the intermediate language is orthogonal to the work
described here.

We implement functions of type
$
\sigma_1\times\cdots\times\sigma_m \rightarrow \tau_1\times\cdots\times\tau_n
$
where $\sigma_1,\ldots,\sigma_m,\tau_1,\ldots,\tau_n$ are types of
values that can be carried on busses.  In real applications, these
types will often be words of different widths, but in this paper we
will use booleans (\DEF{T} and \DEF{F} are the only values of type
\TY{bool}) and natural numbers ($0,1,\ldots$ etc.~of type \TY{num}).
Let $f, f_1, f_2,\ldots$ range over such functions.   The constructs
of our language are expressions $e$ given in BNF by:
\[
\begin{array}{l}
   e\ \bnfDef\           \DEF{Atm}~f\ 
                  \bnfOR \DEF{Lib}~f\ 
                  \bnfOR \DEF{Seq}~e_1~e_2\ 
                  \bnfOR \DEF{Par}~e_1~e_2\ 
                  \bnfOR \DEF{Ite}~e_1~e_2~e_3\ 
                  \bnfOR \DEF{Rec}~e_1~e_2~e_3\ 
\end{array}
\]
Both $\DEF{Atm}~f$ and $\DEF{Lib}~f$ implement function $f$.  The
difference is that $\DEF{Atm}~f$ is constructed from a combinational
circuit (see definition of \DEF{ATM} in Section~\ref{secImplementingHandshakingDevices}
below) and $\DEF{Lib}~f$ assumes $f$ is in a library (initially
assumed empty) of previously designed components (see
Section~\ref{preverified} for an example). We make a
shallow embedding of expressions in higher order logic by defining functions with the
same names as the expression constructors by:
\[
\begin{array}{ll}
\DEF{Lib}~f     &   = f\\
\DEF{Atm}~f     &   = f\\
\DEF{Seq}~f_1~f_2     &   = \lambda x.~f_2(f_1~x)\\
\DEF{Par}~f_1~f_2     &   = \lambda x.~(f_1~x,~f_2~x)\\
\DEF{Ite}~f_1~f_2~f_3 &   = \lambda x.~ \TT{if}~f_1~x~\TT{then}~f_2~x~\TT{else}~f_3~x\\
\DEF{Rec}~f_1~f_2~f_3 &   = \epsilon f.~f~=~\lambda x.~\TT{if}~f_1~x~\TT{then}~f_2~x~\TT{else}~f(f_3~x)
\end{array}
\]
$\DEF{Rec}~f_1~f_2~f_3$ uses
Hilbert's $\epsilon$-operator, and so means ``choose a function f such
that $f$ satisfies the equation $f = \lambda
x.~\TT{if}~f_1~x~\TT{then}~f_2~x~\TT{else}~f(f_3~x)$''.
In practise, $f_1$, $f_2$ and $f_3$ will be such that $f$ is uniquely determined.
For example, taking:
\[
\begin{array}{ll}
 f_1 & = \lambda(n,acc).~n=0\\
 f_2 & = \lambda(n,acc).~(n,acc)\\
 f_3 & = \lambda(n,acc).~(n{-}1,~n{\times}acc)
\end{array}
\]
uniquely defines the function
\[
f=\lambda(n,acc).~\TT{if}~(n=0)~\TT{then}~(n,acc)~\TT{else}~f(n{-}1,~n{\times}acc)
\]

%In practise, $f_1$, $f_2$ and $f_3$ will be such that $f$ is uniquely determined.
%For example, taking:
%\[
%\begin{array}{ll}
% f_1 & = \lambda(n,acc).~n=0\\
% f_2 & = \lambda(n,acc).~(0,acc)\\
% f_3 & = \lambda(n,acc).~(n{-}1,~n{\times}acc)
%\end{array}
%\]
%uniquely defines $f$ to be $\lambda(n,acc).~\DEF{Funpow}~f_3~n~(n,acc)$, where, for an
%arbitrary function $g$, $\DEF{Funpow}~g~n$ is the composition of $g$
%with itself $n$ times, so $\DEF{Funpow}~g~n~x = g^n~x$. This example
%is used in Section~\ref{secTheFactorialCaseStudy} below.

A program $p$ is a list of declarations
$\langle c_1=e_1~\ldots~c_n=e_n\rangle$, where for $1\leq i\leq n$,
$c_i$ is a new name and $e_i$ is an expression built out of
library functions and $c_1,\ldots,c_{i-1}$.


%-------------------------------------------------------------------------------
\section{Handshaking Devices}
\label{secHandshakingDevices}
%-------------------------------------------------------------------------------


Our compiler takes a pair 
$(\langle c_1=e_1~\ldots~c_n=e_n\rangle,~e)$, consisting of a program
$\langle c_1=e_1~\ldots~c_n=e_n\rangle$ and expression
$e$.
% built
%out of library functions and $c_1,\ldots,c_{n}$. 
It generates a
clocked device that computes $e$ via a simple handshaking protocol.
This section describes the protocol and its definition in HOL.

Figure~\ref{figDev} shows a sequence of events that illustrates a
transaction in which a handshaking device performs a single
computation starting at a time $t$ and ending at a later time $t'$
(where time counts cycles). 
The variables \VAR{inp} and \VAR{out} represent the usual 
input and output data, respectively. The wires \VAR{load} 
and \VAR{done} control the access to the device.
If \VAR{done} is asserted, it means that
the device is idle and ready to compute a request. 
Once a positive edge on \VAR{load} is detected,
the device samples the input and starts to 
compute the result (see when 
(\mE{time = t}) and ($\mE{time} = t+1$) at Figure~\ref{figDev}).
During the computation, \VAR{done} remains low and
every call is ignored. Eventually, the device outputs the result
and indicates its completion by asserting \VAR{done}.


\begin{figure}[htb]
\hspace*{-13mm}   {\input{dev.eepic}}
   \caption{\label{figDev}A handshake protocol.}
\end{figure}


Suppose the device computes a function $f$.  At the start of a
transaction (time $t$) the device must be outputting \DEF{T} on $done$
(to indicate it is ready) and the environment must be asserting
\DEF{F} on $load$ (i.e.~in a state such that a positive edge on $load$
can be generated). A transaction is initiated by asserting (at time
$t{+}1$) the value \DEF{T} on $load$ (i.e.~$load$ has a positive edge
at time $t{+}1$), and this causes the device to read the value, $v$
say, being input on $inp$ (at time $t{+}1$) and to de-assert $done$.
The device then becomes insensitive to inputs until \DEF{T} is next
asserted on $done$, at which time (say time $t' > t{+}1$) the value
$f(v)$ computed will be output on $out$.

The behaviour of hardware is modelled in HOL as a boolean-valued
term whose free variables represent the external (observable) wires
of the circuit. This term evaluates to true if the values 
observed at the external wires could occur in the circuit.
The variables are functions from natural numbers (representing
time) to values. For a signal, the low value zero and
the high value one are represented by false (\DEF{F}) 
and true (\DEF{T}), respectively.

Before specifying the behaviour of a handshaking device,
the auxiliary predicates \DEF{Posedge} and \DEF{HoldF} are defined.

A positive edge of a signal is defined as the transition of its
value from low to high or, in our case, from \DEF{F} to \DEF{T}. 
\DEF{Posedge} is specified by:
\[
\TURNST\ \DEF{Posedge}\ s\ t~ ~=~~ \IF{~t{=}0~}{~\DEF{F}~}{~(\NOT\hspace*{0.8mm} s(t{-}1)\ \AND\ s\ t~})
\]
Note that if the time is zero, it is assumed that no positive edge has
occurred.

The term \mE{\DEF{HoldF}\ (t_1,t_2)\ s} says that a
signal \VAR{s} holds a low value \DEF{F} during a half-open interval
starting at \sVAR{t}{1} to just before \sVAR{t}{2}.
\[
\TURNST\ \DEF{HoldF}\ (t_1,t_2)\ s = \forall t.\ t_1 \leq t < t_2\ \IMP\ \NOT(s\ t)
\]

The behaviour of the handshaking device computing a function $f$ is described by the term 
$\DEF{Dev}\ f\ \VAR{(load,inp,done,out)}$ where:
\[
\begin{array}{l}
\TURNST\ \DEF{Dev}\ f\ \VAR{(load,inp,done,out)} = \\
~~\quad     (\forall t.\ \VAR{done}\ t\ \AND\ \DEF{Posedge}\ \VAR{load}\ (t{+}1)\ \\
\phantom{~~\quad     (\forall t.~} \IMP \\
\phantom{~~\quad     (\forall t.\ ~ } \exists t'.\ t' > t{+}1\ \AND\ \DEF{HoldF}\ (t{+}1,t')\ \VAR{done}\ \AND \\
\phantom{~~\quad     (\forall t.\ ~ \exists t'.\ }  \VAR{done}\ t'\ \AND\ (\VAR{out}\ t' = f (\VAR{inp}\ (t{+}1))))\ \\
~~\quad  \AND \\
~~\quad (\forall t.\ \VAR{done}\ t\ \AND\ \NOT(\DEF{Posedge}\ \VAR{load}\ (t{+}1))\ \IMP  \VAR{done}\ (t{+}1))
\end{array}
\]
The first conjunct in the right-hand side describes the context presented
in Figure~\ref{figDev}. If the device is available and a positive
edge occurs on \VAR{load}, there exists a time \VAR{t'} in future
when \VAR{done} signals its termination and the output is produced.
The value of the output at time \VAR{t'} is the result
of applying \VAR{f} to the value of the input at time $\VAR{t}{+}1$.
The signal \VAR{done} holds the value \DEF{F} during the computation.
The second conjunct specifies the situation where no call
is made on \VAR{load} and the device simply remains idle.


%-------------------------------------------------------------------------------
\section{Implementing Handshaking Devices}
\label{secImplementingHandshakingDevices}
%-------------------------------------------------------------------------------
This section describes how we implement our language. Our convention
is to use fully capitalised named for primitive circuits and circuit
constructors.  First, we describe a circuit constructor \DEF{ATM} that
builds a handshaking device from a combinational circuit. Next we
describe circuit constructors \DEF{SEQ}, \DEF{PAR}, \DEF{ITE} and
\DEF{REC} that compose handshaking devices corresponding to
$\DEF{Seq}~e_1~e_2$, $\DEF{Par}~e_1~e_2$, $\DEF{Ite}~e_1~e_2~e_3$ and
$\DEF{Rec}~e_1~e_2~e_3$, respectively.  The key property of these
constructors that ensures they are correct are the following theorems
(the notation \COMP{g}{f} denotes the function composition $\lambda
x.~g(f~x)$):
$$
\begin{array}{l}
\TURNST\ \DEF{ATM}\ f\ (\VAR{load,inp,done,out})\\
\phantom{\TURNST\ } \IMP \  \DEF{Dev}\ f\ (\VAR{load,inp,done,out})\\

\phantom{
\TURNST\ \DEF{Total}(f_1,f_2,f_3) \AND \DEF{REC}\ (\DEF{Dev}\ f_1)\ (\DEF{Dev}\ f_2)\ (\DEF{Dev}\ f_3)\ 
                   (\VAR{load,inp,done,out})}\\[-5mm]
\end{array}$$
$$\begin{array}{l}

\TURNST\ \DEF{SEQ}\ (\DEF{Dev}\ f_1)\ (\DEF{Dev}\ f_2)\ (\VAR{load,inp,done,out})\\
\phantom{\TURNST\ } \IMP \ \DEF{Dev}\ (\COMP{f_2}{f_1})\ (\VAR{load,inp,done,out})\\

\phantom{
\TURNST\ \DEF{Total}(f_1,f_2,f_3) \AND \DEF{REC}\ (\DEF{Dev}\ f_1)\ (\DEF{Dev}\ f_2)\ (\DEF{Dev}\ f_3)\ 
                   (\VAR{load,inp,done,out})}\\[-5mm]
\end{array}$$
$$\begin{array}{l}

\TURNST\ \DEF{PAR}\ (\DEF{Dev}\ f_1)\ (\DEF{Dev}\ f_2)\ (\VAR{load,inp,done,out})\\
\phantom{\TURNST\ } \IMP \ \DEF{Dev}\ (\LAMBDA{x}{(f_1\ x, f_2\ x)})\ (\VAR{load,inp,done,out}) \\

\phantom{
\TURNST\ \DEF{Total}(f_1,f_2,f_3) \AND \DEF{REC}\ (\DEF{Dev}\ f_1)\ (\DEF{Dev}\ f_2)\ (\DEF{Dev}\ f_3)\ 
                   (\VAR{load,inp,done,out})}\\[-5mm]
\end{array}$$
 $$\begin{array}{l}

\TURNST\ \DEF{ITE}\ (\DEF{Dev}\ f_1)\ (\DEF{Dev}\ f_2)\ (\DEF{Dev}\ f_3)\ 
                   (\VAR{load,inp,done,out})\\
\phantom{\TURNST\ }  \IMP \ \DEF{Dev}\ (\LAMBDA{x}{\IF{f_1\ x}{f_2\ x}{f_3\ x}})\ (\VAR{load,inp,done,out})\\

\phantom{
\TURNST\ \DEF{Total}(f_1,f_2,f_3) \AND \DEF{REC}\ (\DEF{Dev}\ f_1)\ (\DEF{Dev}\ f_2)\ (\DEF{Dev}\ f_3)\ 
                   (\VAR{load,inp,done,out})}\\[-5mm]
\end{array}$$
%We hope to prove a result for \DEF{REC} of the form:
$$
\begin{array}{l}
\TURNST\ \DEF{Total}(f_1,f_2,f_3) \AND \DEF{REC}\ (\DEF{Dev}\ f_1)\ (\DEF{Dev}\ f_2)\ (\DEF{Dev}\ f_3)\ 
                   (\VAR{load,inp,done,out})\\
\phantom{\TURNST\ \ }  \IMP \ \DEF{Dev}\ (\DEF{Rec}~f_1~f_2~f_3)\ (\VAR{load,inp,done,out})
\end{array}
$$
where $\DEF{Total}(f_1,f_2,f_3)$ is a predicate ensuring that 
there is a unique function satisfying $f = \lambda
x.~\TT{if}~f_1~x~\TT{then}~f_2~x~\TT{else}~f(f_3~x)$ and is defined by:
\[
\DEF{Total}(f_1,f_2,f_3)~=~\exists variant.~\forall x.~\neg(f_1~x)
                                        \IMP variant(f_3~x) < variant~x
\]

The constructors \DEF{ATM},
\DEF{SEQ}, \DEF{PAR}, \DEF{ITE} and \DEF{REC} use
some primitive combinational hardware components
\DEF{AND}, \DEF{OR}, \DEF{NOT} and \DEF{MUX}, and
two primitive sequential components \DEF{DEL} and \DEF{DFF}.
The behaviour of a combinational AND-gate
is specified as a relation that constrains the value of the output
to the conjunction of the inputs.
\[
\TURNST\ \DEF{AND}\ (\sVAR{in}{1},\sVAR{in}{2},\VAR{out}) = 
   \forall t.\ \VAR{out}\ t = (\sVAR{in}{1}\ t\ \AND\ \sVAR{in}{2}\ t)
\]
A combinational OR-gate with inputs \sVAR{in}{1} and \sVAR{in}{2}
and output \VAR{out} is defined in a similar way.
\[
\TURNST\ \DEF{OR}\ (\sVAR{in}{1},\sVAR{in}{2},\VAR{out}) = 
   \forall t.\ \VAR{out}\ t = (\sVAR{in}{1}\ t\ \OR\ \sVAR{in}{2}\ t)
\]
An inverter simply outputs the negation of the input.
\[
\TURNST\ \DEF{NOT}\ (\VAR{inp},\VAR{out}) = 
   \forall t.\ \VAR{out}\ t = \NOT(\VAR{inp}\ t)
\]
A multiplexer connects the input \sVAR{in}{1}
to the output \VAR{out} if the selector \VAR{sel} has the value~\DEF{T}.
Otherwise, it outputs the value of \sVAR{in}{2}.
\[
\TURNST\ \DEF{MUX}\ (\VAR{sel},\sVAR{in}{1},\sVAR{in}{2},\VAR{out}) = 
   \forall t.\ \VAR{out}\ t = \IF{\VAR{sel}\ t}{\sVAR{in}{1}\ t}{\sVAR{in}{2}\ t}
\]
In general, a combinational component computing a function \VAR{f} is specified by:
\[
\TURNST\ \DEF{COMB}\ f\ \VAR{(inp,out)} = \forall t.\ \VAR{out}\ t = f (\VAR{inp}\ t)
\]
At any given time, this generic combinational device
outputs \VAR{f} applied to the current value
of the input.

A delay outputs the value of the input at the previous time.
\[
\TURNST\ \DEF{DEL}\ (\VAR{inp},\VAR{out}) = 
    (\VAR{out}\ 0 = \VAR{inp}\ 0)\ \AND\ 
    (\forall t.\ \VAR{out} (t{+}1) = \VAR{inp}\ t)
\]
At time zero, the delay behaves as a wire.
A D-type flip-flop \DEF{DFF} outputs the value of the input \VAR{d} on the
positive edge of the signal~\VAR{clk}.
If no positive edge occurs, the output \VAR{q} remains
unchanged.
\[
\TURNST\ \DEF{DFF} (\VAR{d,clk,q})~ =~
 \forall t.\ q (t{+}1) =  \IF{\DEF{Posedge}\ \VAR{clk}\ (t{+}1)}{d (t{+}1)}{q\ t}
\]

The connection between two components is modelled
by the conjunction of their specifications.
The physical connection is represented by
the identically-labelled wires of the subcomponents.
Moreover,  the existential quantifier hides the 
internal wires of the composite device.

For example, $\DEF{POSEDGE}(\VAR{inp},\VAR{out})$ specifies a composite
device that asserts \DEF{T} on its output $\VAR{out}$ if and only if a positive edge
has occurred on the input $\VAR{inp}$. Our implementation is:
\[
\TURNST\ \DEF{POSEDGE} (\VAR{inp},\VAR{out}) = 
    \exists c_0~ c_1.\ \DEF{DEL} (\VAR{inp},c_0)\ \AND\
                  \DEF{NOT} (c_0, c_1)\ \AND\ \DEF{AND} (c_1,\VAR{inp},\VAR{out})
\]
This component connects a \DEF{DEL}, \DEF{NOT} and \DEF{AND}
by the internal wires \sVAR{c}{0} and \sVAR{c}{1}.
The wire \sVAR{c}{0} has the value of the
input at the previous time. The circuit outputs \DEF{T} if
\sVAR{c}{0} has the value \DEF{F} and the current input is \DEF{T} ---
which is exactly what characterises a positive edge.
It is easy to show that \DEF{POSEDGE} has the 
following property.
\[
\TURNST\ \DEF{POSEDGE} (\VAR{inp},\VAR{out})\ \IMP\
         (\forall t.\ \VAR{out}\ t = \DEF{Posedge}\ \VAR{inp}\ t)
\]

The circuit \DEF{ATM} implements an atomic device.
\[
\begin{array}{l}
\TURNST\ \DEF{ATM}\ f\ (\VAR{load,inp,done,out}) =\\
\phantom{\TURNST\ ~~}    \exists c_0~ c_1.\ \DEF{POSEDGE} (\VAR{load}, c_0)\ \AND \ 
\DEF{NOT} (c_0, \VAR{done})\ \AND\\
\phantom{\TURNST\ ~~\exists c_0, c_1.\ }  \DEF{COMB}\ f\ (\VAR{inp},c_1)\ \AND\ \DEF{DEL} (c_1,\VAR{out})
\end{array}
\]
This device takes one time unit to compute (see Figure~\ref{figAtom}).
Although a combinational circuit is 
clearly more efficient than an atomic device,
this device is suitable for composing with other
handshaking devices.

The constructor \DEF{SEQ} specifies a circuit which combines
two devices to compute in sequence.
\[
\begin{array}{l}
\TURNST\ \DEF{SEQ}\ f\ g\ (\VAR{load,inp,done,out}) = \\
\phantom{\TURNST\ ~~} \exists c_0~ c_1~ c_2~ c_3~ \VAR{data}.\\
\phantom{\TURNST\ ~~\exists~}                      \DEF{NOT} (c_2,c_3) \ \AND \ 
                      \DEF{OR} (c_3,\VAR{load},c_0) \ \AND \  f (c_0,\VAR{inp},c_1,\VAR{data}) \ \AND \\
\phantom{\TURNST\ ~~ \exists~} 
        g (c_1,\VAR{data},c_2,\VAR{out})\ \AND\ 
        \DEF{AND} (c_1,c_2,\VAR{done}) 
\end{array}
\]
The subcomponents \VAR{f} and \VAR{g} have the same interface of
a handshaking device. The output of the component \VAR{f} 
is the input of the component \VAR{g} (see the
variables \sVAR{c}{1} and \VAR{data} in Figure~\ref{figSeq}).
This composite device signals its completion when both \VAR{f} and 
\VAR{g} terminate.

\begin{figure}[htb]
   \centerline{
      \subfigure[$\DEF{ATM}\ f$]{
         \label{figAtom}\input{atm.eepic}\hspace*{0cm}}
      \subfigure[$\DEF{SEQ}\ f\ g$]{
         \label{figSeq}\input{seq.eepic}\hspace*{0cm}}
      \subfigure[$\DEF{PAR}\ f\ g$]{
         \label{figPar}\input{par.eepic}}}
      \caption{\label{figSeqPar}Implementation of composite devices.}
\end{figure}

The constructor \DEF{PAR} combines two devices in parallel.
\[
\begin{array}{l}
\TURNST\ \DEF{PAR}\ f\ g\ (\VAR{load,inp,done,out}) = \\
\phantom{\TURNST\ ~}     \exists c_0~c_1~\VAR{start}~\sVAR{done}{1}~\sVAR{done}{2}~
                                 \sVAR{data}{1}~\sVAR{data}{2}~\sVAR{out}{1}~\sVAR{out}{2}.\\
\phantom{\TURNST\ ~~ \exists~}
       \DEF{POSEDGE} (\VAR{load},c_0)\ \AND\  
       \DEF{DEL} (\VAR{done},c_1)\ \AND \ 
       \DEF{AND} (c_0,c_1,\VAR{start})\ \AND\\
\phantom{\TURNST\ ~~ \exists~}
       f (\VAR{start},\VAR{inp},\sVAR{done}{1},\sVAR{data}{1})\ \AND \ 
       g (\VAR{start},\VAR{inp},\sVAR{done}{2},\sVAR{data}{2})\ \AND\\
\phantom{\TURNST\ ~~ \exists~}
       \DEF{DFF} (\sVAR{data}{1},\sVAR{done}{1},\sVAR{out}{1})\ \AND \  
       \DEF{DFF} (\sVAR{data}{2},\sVAR{done}{2},\sVAR{out}{2})\ \AND\\
\phantom{\TURNST\ ~~ \exists~}
       \DEF{AND} (\sVAR{done}{1},\sVAR{done}{2},done)\ \AND \ 
       (\VAR{out} = \LAMBDA{t}{(\sVAR{out}{1}\ t,\sVAR{out}{2}\ t)})
\end{array}
\]
The devices \VAR{f} and \VAR{g} are triggered simultaneously
by \VAR{start}
and return \sVAR{data}{1} and \sVAR{data}{2}, respectively
(see Figure~\ref{figPar}). 
As \VAR{f} and \VAR{g} may terminate at different times,
their outputs are stored by DFFs and made available
by \sVAR{out}{1} and \sVAR{out}{2}.
The components \DEF{POSEDGE} and \DEF{DEL} prevent
calls to either \VAR{f} or \VAR{g} during their computation.

The conditional constructor \DEF{ITE} implements an if-then-else
circuit from three subcomponents.
\[
\begin{array}{l}
\TURNST\ \DEF{ITE}\ e\ f\ g\ (\VAR{load,inp,done,out}) =\\
\phantom{\TURNST\ ~}
   \exists c_0~c_1~c_2~\VAR{start}~\VAR{start'}~\VAR{done\_e}~\VAR{data\_e}~q~ 
                          \VAR{not\_e}~\VAR{data\_f}~\VAR{data\_g}~\VAR{sel}\\
\phantom{\TURNST\ ~\exists }
                  \VAR{done\_f}~\VAR{done\_g}~
                          \VAR{start\_f}~\VAR{start\_g}.\ \\
\phantom{\TURNST\ ~\exists ~}
           \DEF{POSEDGE} (\VAR{load},c_0)\ \AND\
           \DEF{DEL} (\VAR{done},c_1)\ \AND\ \DEF{AND} (c_0,c_1,\VAR{start})\ \AND \\
\phantom{\TURNST\ ~\exists ~}
           e (\VAR{start},\VAR{inp},\VAR{done\_e},\VAR{data\_e})\ \AND\
           \DEF{POSEDGE} (\VAR{done\_e},\VAR{start'})\ \AND \\
\phantom{\TURNST\ ~\exists ~}
           \DEF{DFF} (\VAR{data\_e},\VAR{done\_e},\VAR{sel})\ \AND\
           \DEF{DFF} (\VAR{inp},\VAR{start},q)\ \AND \\
\phantom{\TURNST\ ~\exists ~}
           \DEF{AND} (\VAR{start'},\VAR{data\_e},\VAR{start\_f})\ \AND\
           \DEF{NOT} (\VAR{data\_e},\VAR{not\_e})\ \AND \\
\phantom{\TURNST\ ~\exists ~}
           \DEF{AND} (\VAR{start'},\VAR{not\_e},\VAR{start\_g})\ \AND\
           f (\VAR{start\_f},q,\VAR{done\_f},\VAR{data\_f})\ \AND\\
\phantom{\TURNST\ ~\exists ~}
           g (\VAR{start\_g},q,\VAR{done\_g},\VAR{data\_g})\ \AND\
           \DEF{MUX} (\VAR{sel},\VAR{data\_f},\VAR{data\_g},out)\ \AND \\
\phantom{\TURNST\ ~\exists ~}
           \DEF{AND} (\VAR{done\_e},\VAR{done\_f},c_2)\ \AND\
           \DEF{AND} (c_2,\VAR{done\_g},\VAR{done}) 
\end{array}
\]
The device \VAR{e} implements a boolean test, while
\VAR{f} and \VAR{g} implement the conditional branches.
The output of \VAR{e} triggers either \VAR{f} or \VAR{g}
(see the variable \VAR{data\_e} in Figure~\ref{figIf}).
A multiplexer selects the right output based on the
(stored) value of \VAR{data\_e}.
The variable \VAR{done} is asserted if all subcomponents
have terminated.

\begin{figure}[htb]
   \centerline{\input{if.eepic}}
   \caption{\label{figIf}The conditional constructor: $\DEF{ITE}\ e\ f\ g$.}
\end{figure}

A function is tail-recursive if its recursive
calls are the very last executed statements in the
function. Tail-recursion is interesting
for hardware compilation because it does not 
require the compiler to allocate storage for 
every function call.

The language introduced in Section~\ref{secASimpleLanguage} 
has an operator \DEF{Rec} for specifying tail-recursive functions ${\cal{F}}$
of the form
\[
\begin{array}{l}
\VAR{{\cal{F}}}\ \VAR{x} ~=~ \IF{\VAR{e}\ \VAR{x}}
                         {\VAR{f}\ \VAR{x}}
                         {\VAR{{\cal{F}}}(\VAR{g}\  \VAR{x})}
\end{array}
\]
Such a function ${\cal{F}}$ is specified by $\DEF{Rec}~\VAR{e}~\VAR{f}~\VAR{g}$ as defined
above. A handshaking circuit that implements ${\cal{F}}$ (if it is well-defined) is
constructed using the \DEF{REC} constructor, where:
\[
\begin{array}{l}
\TURNST\ \DEF{REC}\ e\ f\ g\ (\VAR{load,inp,done,out}) = \\
\phantom{\TURNST\ ~}
     \exists \VAR{done\_g}~ \VAR{data\_g}~ \VAR{start\_e}~ q~ \VAR{done\_e}~ 
             \VAR{data\_e}~ \VAR{start\_f}~ \VAR{start\_g}~ \VAR{inp\_e}~ 
             \VAR{done\_f}\\[-1mm]
\phantom{\TURNST\ ~\exists }
             \sVAR{c}{0}~ \sVAR{c}{1}~ 
             \sVAR{c}{2}~ \sVAR{c}{3}~\sVAR{c}{4}~
             \VAR{start}~ \VAR{sel}~ \VAR{start'}~ \VAR{not\_e}. \\[0.5mm]
\phantom{\TURNST\ ~\exists~ }
        \DEF{POSEDGE}(\VAR{load},\sVAR{c}{0})\ \AND\
        \DEF{DEL}(\VAR{done},\sVAR{c}{1})\ \AND\
        \DEF{AND}(\sVAR{c}{0},\sVAR{c}{1},\VAR{start})\ \AND \\
\phantom{\TURNST\ ~\exists~ }
        \DEF{OR}(\VAR{start,sel,start\_e})\ \AND\
        \DEF{POSEDGE}(\VAR{done\_g,sel})\ \AND \\
\phantom{\TURNST\ ~\exists~ }
        \DEF{MUX}(\VAR{sel,data\_g,inp,inp\_e})\ \AND\
           \DEF{DFF}(\VAR{inp\_e},\VAR{start\_e},q)\ \AND \\
\phantom{\TURNST\ ~\exists~ }
           e (\VAR{start\_e},\VAR{inp\_e},\VAR{done\_e},\VAR{data\_e}) \AND\
        \DEF{POSEDGE}(\VAR{done\_e,start'})\ \AND \\
\phantom{\TURNST\ ~\exists~ }
        \DEF{AND}(\VAR{start',data\_e,start\_f})\ \AND\
        \DEF{NOT}(\VAR{data\_e,not\_e})\ \AND \\
\phantom{\TURNST\ ~\exists~ }
        \DEF{AND}(\VAR{not\_e,start',start\_g})\ \AND\
           f (\VAR{start\_f},q,\VAR{done\_f},\VAR{out})\ \AND\\
\phantom{\TURNST\ ~\exists~ }
           g (\VAR{start\_g},q,\VAR{done\_g},\VAR{data\_g})\ \AND\
        \DEF{DEL}(\VAR{done\_g},\sVAR{c}{3})\ \AND\\
\phantom{\TURNST\ ~\exists~ }
        \DEF{AND}(\VAR{done\_g},\sVAR{c}{3},\sVAR{c}{4})\ \AND\
        \DEF{AND}(\VAR{done\_f},\VAR{done\_e},\sVAR{c}{2})\ \AND\
        \DEF{AND}(\sVAR{c}{2},\sVAR{c}{4},\VAR{done})
\end{array}
\]
The recursive constructor is similar to the conditional one
(see Figure~\ref{figRec}).
The main difference is the connection between the
``else'' branch and the circuit itself --- characterising a
recursive call. A multiplexer selects the input from either
the external environment or from the recursive call.
The circuit terminates if every subcomponent terminates
(see the variables \VAR{done\_e}, \VAR{done\_f} and \VAR{done\_g}
in Figure~\ref{figRec}). Furthermore, the component \VAR{g}
must have terminated at least one time unit before. This is
necessary to distinguish a recursive call from the
complete termination of the computation.

\begin{figure}[htb]
   \centerline{\input{rec.eepic}}
   \caption{\label{figRec}The recursive constructor: $\DEF{REC}\ e\ f\ g$.}
\end{figure}


%-------------------------------------------------------------------------------
\section{Compiling by Proof}
\label{secCompilingByProof}
%-------------------------------------------------------------------------------


The prototype compiler we have implemented takes a program and an expression $(\langle
c_1=e_1~\ldots~c_n=e_n\rangle,~e)$, where the expression $e$ is built
out of library functions and $c_1,\ldots,c_{n}$. It generates
a circuit ${\cal C}(\VAR{load,inp,done,out})$, represented as a term in higher order logic, 
and returns a theorem:

\vspace*{-2mm}

\[
\begin{array}{l}
\TURNST\ \forall \hspace*{0.5mm} load~inp~done~out.~{\cal C}(\VAR{load,inp,done,out}) \IMP \DEF{Dev}~e~(\VAR{load,inp,done,out})
\end{array}
\]

\vspace*{-2mm}

\noindent The compilation procedure is a straightforward recursive application of the
following theorems (which are proved from the key properties given in 
Section~\ref{secImplementingHandshakingDevices}
and the semantics of the expression and circuit constructors):
$$\begin{array}{l}
\texttt{ATM\_INTRO}\\
~\TURNST\ \forall c~s.~\DEF{ATM}\ c\ s\  \IMP \  \DEF{Dev}\ c\ s\\
\phantom{~~~~~~~~~~~~~~~~~~~~~~~~~~~~~~~~~~~~~~~~~~~~~~~~~~~~~~~~~~~~~~~~~~~~~~~~~~~~~~~~~~~~~~~~~~~~~~~~~~~~~}\\[-3mm]
\end{array}$$
\vspace*{-3mm}
$$\begin{array}{l}
\texttt{SEQ\_INTRO}\\
~\TURNST\ \forall P_1~P_2~f_1~f_2.\\
\phantom{~~\TURNST\ \forall~}
(\forall s.~P_1~s \ \IMP\ \DEF{Dev}~f_1~s)~\AND~(\forall s.~P_2~s \ \IMP\ \DEF{Dev}~f_2~s)\\
\phantom{~~\TURNST\ \forall~}
 \IMP\\
\phantom{~~\TURNST\ \forall~}
\forall s.~\DEF{SEQ}\ P_1\ P_2\ s \ \IMP \ \DEF{Dev}\ (\DEF{Seq}~f_1~f_2)\ s\\
\phantom{~~~~~~~~~~~~~~~~~~~~~~~~~~~~~~~~~~~~~~~~~~~~~~~~~~~~~~~~~~~~~~~~~~~~~~~~~~~~~~~~~~~~~~~~~~~~~~~~~~~~~~}
\end{array}$$
\vspace*{-3mm}
$$\begin{array}{l}
\texttt{PAR\_INTRO}\\
~\TURNST\ \forall P_1~P_2~f_1~f_2.\\
\phantom{~~\TURNST\ \forall~}
(\forall s.~P_1~s \ \IMP\ \DEF{Dev}~f_1~s)~\AND~(\forall s.~P_2~s \ \IMP\ \DEF{Dev}~f_2~s)\\
\phantom{~~\TURNST\ \forall~}
\IMP\\
\phantom{~~\TURNST\ \forall~}
\forall s.~ \DEF{PAR}\ P_1\ P_2\ s\ \IMP \ \DEF{Dev}\ (\DEF{Par}~f_1~f_2)\ s \\
\phantom{~~~~~~~~~~~~~~~~~~~~~~~~~~~~~~~~~~~~~~~~~~~~~~~~~~~~~~~~~~~~~~~~~~~~~~~~~~~~~~~~~~~~~~~~~~~~~~~~~~~~~~}
\end{array}$$
\vspace*{-3mm}
$$\begin{array}{l}
\texttt{ITE\_INTRO}\\
~\TURNST\ \forall P_1~P_2~P_3~f_1~f_2~f_3.\\
\phantom{~~\TURNST\ \forall~}
(\forall s.~P_1~s \ \IMP\ \DEF{Dev}~f_1~s)~\AND\\
\phantom{~~\TURNST\ \forall~}
(\forall s.~P_2~s \ \IMP\ \DEF{Dev}~f_2~s)~\AND\\
\phantom{~~\TURNST\ \forall~}
(\forall s.~P_3~s \ \IMP\ \DEF{Dev}~f_3~s)\\
\phantom{~~\TURNST\ \forall~}
\IMP\\
\phantom{~~\TURNST\ \forall~}
\forall s.~ \DEF{ITE}\ P_1\ P_2\ P_3\ s\ 
 \IMP \ \DEF{Dev}\ (\DEF{Ite}~f_1~f_2~f_3)\ s\\
\phantom{~~~~~~~~~~~~~~~~~~~~~~~~~~~~~~~~~~~~~~~~~~~~~~~~~~~~~~~~~~~~~~~~~~~~~~~~~~~~~~~~~~~~~~~~~~~~~~~~~~~~~~}
\end{array}$$
\vspace*{-3mm}
$$\begin{array}{l}
\texttt{REC\_INTRO}\\
%{}[~\forall s.~\DEF{REC}\ (\DEF{Dev}\ f_1)\ (\DEF{Dev}\ f_2)\ (\DEF{Dev}\ f_3)\ s\ \IMP\ \DEF{Dev}\ 
% (\DEF{Rec}~f_1~f_2~f_3)~]{}\\
\TURNST\ \forall f_1~f_2~f_3~P_1~P_2~P_3.\\
\phantom{\TURNST\ \forall~}
\DEF{Total}(f_1,f_2,f_3)\\
\phantom{\TURNST\ \forall~}
\IMP\\
\phantom{\TURNST\ \forall~}
(\forall s.~P_1~s \ \IMP\ \DEF{Dev}~f_1~s)~\AND\\
\phantom{\TURNST\ \forall~}
(\forall s.~P_2~s \ \IMP\ \DEF{Dev}~f_2~s)~\AND\\
\phantom{\TURNST\ \forall~}
(\forall s.~P_3~s \ \IMP\ \DEF{Dev}~f_3~s)\\
\phantom{\TURNST\ \forall~}
\IMP\\
\phantom{\TURNST\ \forall~}
\forall s.~ \DEF{REC}\ P_1\ P_2\ P_3\ s\ 
 \IMP \ \DEF{Dev}\ (\DEF{Rec}~f_1~f_2~f_3)\ s\\
\phantom{~~~~~~~~~~~~~~~~~~~~~~~~~~~~~~~~~~~~~~~~~~~~~~~~~~~~~~~~~~~~~~~~~~~~~~~~~~~~~~~~~~~~~~~~~~~~~~~~~~~~~}
\end{array}$$

\noindent The theorem \texttt{REC\_INTRO} is an implication whose antecedent  is
$\DEF{Total}(f_1,f_2,f_3)$.

We will outline how our compiler works using an ML-style pseudo-code to
describe the inferences that deductively transform a specification to
an implementation. Theorems in the HOL system logic have the form
$\Gamma\TURNST t$ where $\Gamma$ is a set of assumptions and $t$ is a
conclusions that follows from the assumptions.


The ML pseudo-code \texttt{SPEC~$[t_1,\ldots,t_n]$~$(\Gamma \TURNST \forall x_1\ldots x_n.~P(x_1,\ldots,x_n))$}
evaluates to $\Gamma\TURNST P(t_1,\ldots,t_n)$. 
\texttt{UNDISCH$(\Gamma\TURNST t_1\IMP t_2)$}
evaluates to \texttt{$\Gamma\cup\{t_1\}\TURNST t_2$}.
$\texttt{MATCH\_MP}~(\Gamma_1 \TURNST t)~(\Gamma_2\TURNST t_1\IMP t_2)$ matches $t_1$
with $t$ and then instantiates the theorem $(\Gamma_2 \TURNST t_1\IMP t_2)$ to
$(\Gamma_2 \TURNST t\IMP t')$ (where $t'$ is the instance of $t_2$
corresponding to the match) and returns $\Gamma_1\cup\Gamma_2 \TURNST t'$, the result of
applying Modus Ponens.  
$(\Gamma_1 \TURNST
t_1)~\texttt{AND}~(\Gamma_2 \TURNST t_2)$ evaluates to $(\Gamma_1\cup\Gamma_2 \TURNST t_1 \AND t_2)$.
Evaluating \texttt{LibraryLookup~lib~$f$} searches the library \texttt{lib}
for a theorem  of the form:

%\vspace*{-4mm}

\[
\begin{array}{l}
\TURNST\ \forall \hspace*{0.5mm} load~inp~done~out.~{\cal C}(\VAR{load,inp,done,out}) \IMP \DEF{Dev}~f~(\VAR{load,inp,done,out})
\end{array}
\]

%\vspace*{-4mm}

\noindent and uses the first one it finds (or raises an exception if no matching theorem found in \texttt{lib}).


To compile $(p,~\texttt{e})$, first rewrite \texttt{e} with the
definitions in $p$ to obtain an expanded expression \texttt{e'} that
only contains atomic or library functions, and then recursively apply
the rules below to evaluate \texttt{Compile~e'}.

%\vspace*{-1mm}

{\baselineskip10pt\begin{alltt}
  Compile lib (Atm f) = 
   SPEC f ATM_INTRO

  Compile lib (Lib f) = 
   LibraryLookup lib f

  Compile lib (Seq e1 e2) = 
   MATCH_MP SEQ_INTRO (Compile lib e1 AND Compile lib e2)

  Compile lib (Par e1 e2) = 
   MATCH_MP PAR_INTRO (Compile lib e1 AND Compile lib e2)

  Compile lib (Ite e1 e2 e3) = 
   MATCH_MP 
    PAR_INTRO 
    (Compile lib e1 AND Compile lib e2 AND Compile lib e3)

  Compile lib (Rec e1 e2 e3) = 
   MATCH_MP 
    (UNDISCH(SPEC [e1,e2,e3] REC_INTRO))
    (Compile lib e1 AND Compile lib e2 AND Compile lib e3)
\end{alltt}}

%\vspace*{-1mm}

\noindent Note that evaluating \texttt{Compile lib (Rec e1 e2 e3)} 
will generate a theorem with an assumption
$\DEF{Total}(\texttt{e1},\texttt{e2},\texttt{e3})$.

%\vspace*{-2mm}

%-------------------------------------------------------------------------------
\section{The Factorial Case Study}
\label{secTheFactorialCaseStudy}
%-------------------------------------------------------------------------------

%\vspace*{-2mm}

The tail-recursive function \DEF{FactIter} defined below can be used
to compute the factorial function.

%\vspace*{-2mm}

\[
\TURNST\ \DEF{FactIter}(n,acc) = \IF{(n=0)}{(n,acc)}{\DEF{FactIter}(n{-}1,n{\times}acc)}
\]

%\vspace*{-2mm}

\noindent The variable \VAR{acc} accumulates the result of the
computation. 
Evaluating \DEF{FactIter}$(\VAR{n},1)$
returns $(0,~n!)$, where $n!$ is the factorial of \VAR{n}.

\subsection{Implementation with an atomic (combinational) multiplier}\label{atomic}

The following program in our language computes $n!$.

%\vspace*{-3mm}

$$
\begin{array}{ll}
\mE{FactProg}~ =~ \langle \DEF{Test0}    &= \DEF{Atm}~\lambda n.~ n=0,\\
\phantom{FactProg~ = ~ \langle}
  \DEF{Ident}    &= \DEF{Atm}~\lambda (n,acc).~ (n,acc),\\
\phantom{FactProg~ = ~ \langle}
  \DEF{Dec}      &= \DEF{Atm}~\lambda n.~ n{-}1,\\
\phantom{FactProg~ = ~ \langle}
  \DEF{Mult}     &= \DEF{Atm}~\lambda (n,acc).~ n{\times}acc,\\
\phantom{FactProg~ = ~ \langle}
  \DEF{Fst}      &= \DEF{Atm}~\lambda (n,acc).~ n,\\
\phantom{FactProg~ = ~ \langle}
  \DEF{Snd}      &= \DEF{Atm}~\lambda (n,acc).~ acc,\\
\phantom{FactProg~ = ~ \langle}
  \DEF{PairOne}  &= \DEF{Atm}~\lambda n.~ (n,1),\\
\phantom{FactProg~ = ~ \langle}
  \DEF{FactIter} &= \DEF{Rec~(Seq~Fst~Test0)~Ident~(Par~(Seq~Fst~Dec)~Mult)},\\
\phantom{FactProg~ = ~ \langle}
  \DEF{Fact}     &= \DEF{Seq~PairOne~(Seq~FactIter~Snd)} \rangle
\end{array}
$$

\noindent The expressions \DEF{Test0}, \DEF{Ident}, \DEF{Dec}, \DEF{Mult}, \DEF{Fst}, \DEF{Snd} and
\DEF{PairOne} are assumed atomic (i.e.~implementable by combinational circuits). This is unrealistic 
for \DEF{Mult}; see Section~\ref{preverified} for a (slightly) more realistic version.

If we invoke the compiler on the program $(\mE{FactProg},\DEF{Fact})$ the result
is:

{\small\begin{alltt}
[ TOTAL (Seq (\(\lambda\)(n,acc). n) (\(\lambda\)n. n = 0), (\(\lambda\)(n,acc). (n,acc)),
         Par (Seq (\(\lambda\)(n,acc). n) (\(\lambda\)n. n-1)) (\(\lambda\)(n,acc). n\(\times\)acc)) ]
    |- \(\forall\)load inp done out.
         SEQ (ATM (\(\lambda\)n. (n,1)))
           (SEQ
              (REC (SEQ (ATM (\(\lambda\)(n,acc). n)) (ATM (\(\lambda\)n. n = 0)))
                 (ATM (\(\lambda\)(n,acc). (n,acc)))
                 (PAR
                    (SEQ (ATM (\(\lambda\)(n,acc). n))
                       (ATM (\(\lambda\)n. n-1)))
                    (ATM (\(\lambda\)(n,acc). n\(\times\)acc))))
              (ATM (\(\lambda\)(n,acc). acc))) (load,inp,done,out) 
         \(\Rightarrow\)
         Dev Fact (load,inp,done,out)
\end{alltt}}

The outcome is a theorem of the form $\Gamma\TURNST t$ where $\Gamma$
is a singleton set consisting of an assumption expressing the totality
of \DEF{FactIter}. Simplifying the assumption with the definitions of
$\DEF{Seq}$ and $\DEF{Par}$ yields:
\[
\DEF{Total}((\lambda (n,acc).~n=0),(\lambda (n,acc).~(n,acc)), (\lambda (n,acc).~(n{-}1,n{\times}acc)))
\]
which is easily proved (with the function $(\lambda
(x,y).~x)$ as the variant).  Once the totality assumption has been proved it can be
eliminated. Furthermore, it is easy to prove by elementary arithmetic from the definitions of the
components of $FactProg$ and the meanings of \DEF{Atm}, \DEF{Seq}, \DEF{Par}, \DEF{Ite} and \DEF{Rec}, that
$\TURNST Fact = \lambda n.~n!$. The output of the compiler thus simplifies to:

\newpage

{\small\begin{alltt}
 |- \(\forall\)load inp done out.
         SEQ (ATM (\(\lambda\)n. (n,1)))
           (SEQ
              (REC (SEQ (ATM (\(\lambda\)(n,acc). n)) (ATM (\(\lambda\)n. n = 0)))
                 (ATM (\(\lambda\)(n,acc). (n,acc)))
                 (PAR
                    (SEQ (ATM (\(\lambda\)(n,acc). n))
                       (ATM (\(\lambda\)n. n-1)))
                    (ATM (\(\lambda\)(n,acc). n\(\times\)acc))))
              (ATM (\(\lambda\)(n,acc). acc))) (load,inp,done,out)
         \(\Rightarrow\)
         Dev \(({\lambda}n. n!)\) (load,inp,done,out)
\end{alltt}}

%\noindent which has the form: $\TURNST \mbox{\it implementation}\ \Rightarrow\ \mbox{\it specification}$.

\vspace*{-6mm}

\subsection{Implementation with a pre-verified multiplier}\label{preverified}

The example above used 
$\DEF{Mult}~=~\DEF{Atm}~\lambda (n,acc).~ n{\times}acc$. Such a combinational multiplier
is unrealistic (except for small words). However,
we can easily implement a (naive) sequential multiplier that works by
repeated addition and so, more realistically, only assumes
combinational addition (and decrementing):

\vspace*{-4mm}

$$
\begin{array}{ll}
\mE{MultProg}~ =\\
~~ \langle \DEF{Test0}    &= \DEF{Atm}~\lambda m.~ m=0,\\
\phantom{~~ \langle}
  \DEF{Ident}    &= \DEF{Atm}~\lambda (m,n,acc).~ (m,n,acc),\\
\phantom{~~ \langle}
  \DEF{Dec}      &= \DEF{Atm}~\lambda m.~ m{-}1,\\
\phantom{~~ \langle}
  \DEF{AddAcc}     &= \DEF{Atm}~\lambda (m,n,acc).~ n{+}acc,\\
\phantom{~~ \langle}
  \DEF{Fst}      &= \DEF{Atm}~\lambda (m,n,acc).~m,\\
\phantom{~~ \langle}
  \DEF{Snd}      &= \DEF{Atm}~\lambda (m,n,acc).~n,\\
\phantom{~~ \langle}
  \DEF{Thd}      &= \DEF{Atm}~\lambda (m,n,acc).~acc,\\
\phantom{~~ \langle}
  \DEF{PairZero}  &= \DEF{Atm}~\lambda(m,n).~ (m,n,0),\\
\phantom{~~ \langle}
  \DEF{MultIter} &= \DEF{Rec~(Seq~Fst~Test0)~Ident~(Par~(Seq~Fst~Dec)~(Par~Snd~AddAcc))},\\
\phantom{~~ \langle}
  \DEF{Mult}     &= \DEF{Seq~PairZero~(Seq~MultIter~Thd)} \rangle
\end{array}
$$

\vspace*{-2mm}

Note that we have used the same names in \DEF{FactProg} and \DEF{MultProg} for
different (though semantically related) expressions (e.g.~\DEF{Fst}).  This is not a problem
as names are local to the program they occur in.

Compiling $(\mE{MultProg},\DEF{Mult})$, simplifying and discharging
the totality proof obligation (in a way very similar to the factorial example) results in:

\vspace*{-2mm}

{\small\begin{alltt}
 |- \(\forall\)load inp done out.
      SEQ (ATM_IMP (\(\lambda\)(m,n). (m,n,0)))
        (SEQ
           (REC (SEQ (ATM_IMP (\(\lambda\)(m,n,acc). m)) (ATM_IMP (\(\lambda\)m. m = 0)))
              (ATM_IMP (\(\lambda\)(m,n,acc). (m,n,acc)))
              (PAR (SEQ (ATM_IMP (\(\lambda\)(m,n,acc). m)) (ATM_IMP (\(\lambda\)m. m - 1)))
                 (PAR (ATM_IMP (\(\lambda\)(m,n,acc). n))
                    (ATM_IMP (\(\lambda\)(m,n,acc). n {+} acc)))))
           (ATM_IMP (\(\lambda\)(m,n,acc). acc))) (load,inp,done,out) \(\Rightarrow\)
      Dev (\(\lambda\)(m,n). m \(\times\) n) (load,inp,done,out)
\end{alltt}}

\vspace*{-2mm}

\noindent After adding this theorem to the library, we can replace
the combinational multiplier in the factorial example by
$\DEF{Mult}~=~\DEF{Lib}~\lambda (n,acc).~ n{\times}acc$.

If we recompile the factorial program after this change, the
implementation of the multiplier is `inlined' and we get:



{\small\begin{alltt}
 |- \(\forall\)load inp done out.
      SEQ (ATM_IMP (\(\lambda\)n. (n,1)))
        (SEQ
           (REC (SEQ (ATM_IMP (\(\lambda\)(n,acc). n)) (ATM_IMP (\(\lambda\)n. n = 0)))
              (ATM_IMP (\(\lambda\)(n,acc). (n,acc)))
              (PAR (SEQ (ATM_IMP (\(\lambda\)(n,acc). n)) (ATM_IMP (\(\lambda\)n. n - 1)))
                 (SEQ (ATM_IMP (\(\lambda\)(m,n). (m,n,0)))
                    (SEQ
                       (REC
                          (SEQ (ATM_IMP (\(\lambda\)(m,n,acc). m))
                             (ATM_IMP (\(\lambda\)m. m = 0)))
                          (ATM_IMP (\(\lambda\)(m,n,acc). (m,n,acc)))
                          (PAR
                             (SEQ (ATM_IMP (\(\lambda\)(m,n,acc). m))
                                (ATM_IMP (\(\lambda\)m. m - 1)))
                             (PAR (ATM_IMP (\(\lambda\)(m,n,acc). n))
                                (ATM_IMP (\(\lambda\)(m,n,acc). n {+} acc)))))
                       (ATM_IMP (\(\lambda\)(m,n,acc). acc))))))
           (ATM_IMP (\(\lambda\)(n,acc). acc))) (load,inp,done,out) \(\Rightarrow\)
      Dev \(({\lambda}n. n!)\) (load,inp,done,out)
\end{alltt}}

This is an implementation of the factorial with an `inner-loop' for each multiplication.
Not an efficient circuit, but it illustrates hierarchical development.

\vspace*{-1mm}

%-------------------------------------------------------------------------------
\section{Future Work}
\label{secConclusionAndFutureWork}
%-------------------------------------------------------------------------------

\vspace*{-1mm}

%A simple functional language embedded in higher order logic is used as a
%hardware description language. We describe a proof-based methodology
%for formal synthesis to synchronous circuits that compute the
%top-level function of the source code with a composable handshaking interface.
%As the implementations are correct-by-construction,
%no verification at the gate-level of abstraction is necessary.

The handshaking protocol for devices is
preliminary and we plan to refine and extend it.  For example,
$\DEF{Dev}\ f\ \VAR{(load,inp,done,out)}$ holds if
\DEF{F} is continuously asserted on \VAR{done}. We need to prove some
liveness results saying that if there is no posedge on \VAR{load} then
eventually \VAR{done} will go to \DEF{T}. This property looks clearly
true of \DEF{ATM} and should be compositional with respect to
\DEF{Seq}, \DEF{Par}, \DEF{Ite} and \DEF{Rec} (assuming totality). 
The compiler should also be able to generate handshaking devices
that are shared by several callers. An arbiter would control
the concurrent calls and preserve the handshaking behaviour. This may
require us to extend the handshaking protocol to support more than one
request (\VAR{load}) and acknowledge (\VAR{done}) line per device.


In the future we plan to 
explore formally validated optimisations to the compiler,
perhaps using ideas from SAFL compilation \cite{Sha02}.

Finally, the compiler could provide the choice to
generate either machine code or pure hardware. 
This feature would allow the user to
partition the system into software and hardware
parts and explore different designs.

\vspace*{-1mm}

%-------------------------------------------------------------------------------
\section{Acknowledgements}
\label{secAcknowledgements}
%-------------------------------------------------------------------------------

\vspace*{-1mm}

Konrad Slind provided encouragement and motivation and also helped us
with the use of TFL and the formulation and proof of \texttt{REC\_INTRO}.

\vspace*{-1mm}

%--------------------------------------------------------------------------------
\bibliographystyle{plain}
\bibliography{tphols2004}
%-------------------------------------------------------------------------------
\end{document}
% LocalWords:  langle


