We are in the midst of two case-studies: a Booth multiplier, and an implementation
of the AES encryption algorithm.

\paragraph{Booth Multiplier}
\paragraph{AES}

AES is a recent encryption standard \cite{AES}, chosen to replace the
aging Digital Encryption Standard (DES). Building on a functional
correctness proof of AES \cite{slind:aes}, we are working on applying
the hardware synthesis infrastructure described in the previous
sections to produce a hardware implementation of AES. Although this
work is still underway, interesting observations can still be
made. Currently, all the major components of a `round' of encryption
are synthesized. In particular, in a round, the following sub-routines
are invoked: (a) application of an \emph{sbox}; (b) xor-ing the state
with a key from the key schedule; (c) shifting of rows, and (d) column
multiplication.  The application of the sbox is, so far, being treated
as combinational.  Xor-ing of the state synthesizes to a combinational
circuit, as does row shifting, which is quite nicely synthesized to a
circuit that merely does some rewiring. The most interesting component
so far has been column multiplication, which relies on a special
Galois Field multiplication.  In the following, the function
\verb+xtime+, which operates over eight-bit bytes, is used to
implement the (recursive) multiplication function which is written
infix \verb+b1 ** b2+. Left shift (\verb+<<+), right shift
(\verb+>>>+), and least-and-most significant bit operations
(\verb+MSB+,\verb+LSB+) are also used.

\begin{verbatim}
 xtime w = w << 1 # (if MSB w then 0x1Bw else 0w

 b_1 ** b_2 =
    if b_1 = 0w then 0w else 
    if LSB b_1 then b2 # ((b_1 >>> 1) ** xtime b_2)
              else       ((b_1 >>> 1) ** xtime b_2)
\end{verbatim}

There are at least five different ways this function could be presented:
\begin{enumerate}
\item Direct translation of the recursive algorithm
\item Tail recursive algorithm
\item Partial evaluation with respect to the first argument
\item The result of the previous partially evaluated with respect to all
second arguments. The results of this can be represented in tabular form,
or as a balanced \verb+if+-tree.
\end{enumerate}
