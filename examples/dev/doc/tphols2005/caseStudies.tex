We are in the midst of two case-studies: a Booth multiplier, and an implementation
of the AES encryption algorithm.

\paragraph{Booth Multiplier}
\paragraph{AES}

The Advanced Encryption Standard (AES) \cite{AES} for private-key encryption
specifies a multi-round algorithm with primitive computations based on finite
field operations.  Starting from an existing formalization of AES
\cite{slind:aes}, we have generated netlists and circuits for the major
components of an encryption (and decryption) round.  Although out work on AES
is incomplete, our current progress provides several illustrations of our
synthesis methodology.

The AES formalization includes a proof of functional correctness  for the
algorithm: specifically, encryption and decryption are inverse functions.
Deriving the hardware from the proven specification using logical inference
assures us that the hardware encrypter is the inverse of the hardware
decrypter.

An encryption round performs the following transformations on a 4-by-4 matrix
of input bytes:
\begin{enumerate}
\item
application of an \emph{sbox}, an invertible function from bytes to bytes,
to each byte
\item
a cyclical shift of each row
\item
multiplication of each column by a fixed degree 3 polynomial, with coefficients
in the 256 element finite field, GF($2^8$).
\item
adding a key to the matrix with exclusive or
\end{enumerate}

\begin{figure}
\begin{verbatim}
ShiftRows (b00,b01,b02,b03,b10,b11,b12,b13,b20,b21,b22,b23,b30,b31,b32,b33) =
          (b00,b01,b02,b03,b11,b12,b13,b10,b22,b23,b20,b21,b33,b30,b31,b32)
\end{verbatim}
\begin{verbatim}
|- (?v0 v1 v2.
      DEL (load,v2) /\ NOT (v2,v1) /\ AND (v1,load,v0) /\
      NOT (v0,done) /\ DEL (inp1,out1) /\ DEL (inp2,out2) /\
      DEL (inp3,out3) /\ DEL (inp4,out4) /\ DEL (inp6,out5) /\
      DEL (inp7,out6) /\ DEL (inp8,out7) /\ DEL (inp5,out8) /\
      DEL (inp11,out9) /\ DEL (inp12,out10) /\ DEL (inp9,out11) /\
      DEL (inp10,out12) /\ DEL (inp16,out13) /\ DEL (inp13,out14) /\
      DEL (inp14,out15) /\ DEL (inp15,out16)) ==>
    DEV ShiftRows (load, inp1 <> ... inp16, done, out1 <> ... <> out16) 
\end{verbatim}
\caption{Row shifting}
\label{AES:shift}
\end{figure}

The discussion about multiplication below applies to the sbox implementation,
although the sbox is a more complicated function.  The row shift operation
translates into a simple circuit whose wiring reflects the permutation
(Fig.~\ref{AES:shift}).  The multiplication algorithm admits several
implementation strategies, each with important hardware
differences.  The key addition is similar to the row shift.

\begin{figure}
\begin{verbatim}
xtime w = w << 1 # (if MSB w then 0x1B else 0x0)

b_1 ** b_2 =
   if b_1 = 0x0 then 0x0
   else if LSB b_1 then b2 # ((b_1 >>> 1) ** xtime b_2)
   else                      ((b_1 >>> 1) ** xtime b_2)

IterMult (b1,b2,acc) =
   if b1 = 0w then (b1,b2,acc)
   else IterMult (b1 >>> 1, xtime b2, if LSB b1 then (b2 # acc) else acc)
\end{verbatim}
\caption{AES Multipliers}
\label{AES:mult}
\end{figure}

The column multiplication step multiplies (in GF($2^8$)) each input byte by a
pre-specified constant, meaning that multiplications by only a small handful
of numbers are needed (0x2 and 0x3 for encryption, and 0x9, 0xB, 0xD, and 0xE
for decryption).  Fig.~\ref{AES:mult} gives the specification for this
multiplication, written infix \verb+b1 ** b2+.  The \verb+<<+ and \verb+>>>+
operators denote left shift and logical right shift; the \verb+#+ operator
denotes exclusive or; and the \verb+MSB+ and \verb+LSB+ functions return the
most and least significant bit, respectively.

The specification cannot be used in hardware synthesis, because it is not
tail-recursive.  The tail recursive function \verb+IterMult+
(Fig.~\ref{AES:mult}) is easily proven equivalent to \verb+**+.  A
straightforward application of the hardware synthesis methodology yields a
netlist or Verilog for \verb+IterMult+.  The occurrence of \verb+xtime+ in the
function yields a choice in hardware generation.  We can treat the \verb+xtime+
function as a handshaking device, and refine it into \verb+IterMult+ after both
are converted into circuit constructors.  Alternately, we can replace the
definition of \verb+xtime+ in \verb+IterMult+ with \verb+xtime+'s body.  Since
\verb+xtime+ is a small combinational circuit, the latter approach has the
advantage of generating less handshaking hardware.

Instead of generating an iterative circuit for the general multiplication
function, we have the option to generate combinational circuits for only the
six needed values of the first argument.  Partially evaluating the definition
of \verb+**+ on its potential first arguments automatically proves six theorems
similar to the following:
\begin{verbatim}0xB ** x = x # xtime (x # xtime (xtime x))\end{verbatim}
We then take these theorems as specifications for one argument multipliers and
transforms them into hardware.  The \verb+xtime+ component can be supplied
either before or after synthesis, as in the previous example.

We can apply partial evaluation even further, and supply all of the 256
possible second arguments as well, to derive multiplication tables, each 256
bytes in size.
\begin{verbatim}
(0xB ** 0x0 = 0x0) /\ ... /\ (0xB ** 0xFF = 0xA3)
\end{verbatim}
These tables could be incorporated into a RAM or ROM device.  For synthesizing
the tables directly into hardware, we have automated the definition of a
function on bytes as a balanced \verb+if+ expression, branching on each
successive bit of its input.
\begin{verbatim}
0xB ** x = if WORD_BIT 7 x then
             if WORD_BIT 6 x then 
               ...
                       if WORD_BIT 0 then 0xA3 else 0xA8
               ...
           else
             if WORD_BIT 6 x then
               ...
\end{verbatim}

