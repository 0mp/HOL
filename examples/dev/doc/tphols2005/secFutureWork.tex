

A first version of the compiler described here has only just been
completed and is still being tested (e.g. by simulating circuits and
viewing the resulting waveforms: we use Icarus Verilog and GTKWave,
with are both public domain).  The source code for our compiler and
the current state of various examples (including the Booth multiplier
and AES) can be inspected at
SourceForge\footnote{\url{http://cvs.sourceforge.net/viewcvs.py/hol/hol98/examples/dev/}}

Although we have a flow from higher
order logic to Verilog, there are still some unresolved issues. The
most urgent of these is that it appears that on some examples the
circuits we generate do not initialise properly under simulation!  We
think there may be an inadequacy in our high level models, as they
assume all bits are $1$ or $0$, so do not represent the possibility of
uninitialised states (\texttt{x}~in~Verilog). Thus although we create
correct-by-construction circuits, the model underlying the correctness
verification may be too crude. We hope to resolve this soon (e.g.~by
the time a final version of this paper is produced). If necessary, we
can add extra initialisation logic to the circuit constructors and
then reverify the synthesis laws using a 3-valued model of bits
(as in Verilog).



In the immediate future we plan to ruggedise and debug the existing
code and continue and complete the case studies described in
Section~\ref{secCaseStudy}. An important part of these case studies is
examining the output of our compiler and developing improvements to
ensure that it can produce reasonable quality designs.

At present all data-refinement (e.g.~from numbers or enumerated types
to words) must be done manually, by proof in higher order logic. The
HOL4 system has some `boolification' facilities that automatically
translate higher level data-types into bit-strings, and we hope to
integrate these into the compiler.

We want to investigate using the compiler to generate test-bench
monitors that can run in parallel simulation with designs that are not
correct by construction.  Thus our hardware can act as a ``golden''
reference against which to test other implementations.

The work described here is part of a bigger project to create
hardware/software combinations by proof.  We hope to investigate the
option of creating software for ARM processors and linking it to
hardware created by our compiler (probably packaged as an ARM
co-processor). Our emphasis is likely to be on cryptographic hardware
and software, because there is a clear need for high assurance of
correct implementation in this domain.
