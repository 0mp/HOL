\documentclass{scrartcl}

\usepackage{holindex}

\setHOLlinewidth{80}   %default is 80
%\setHOLoverrides{over} %option

\begin{document}

\section{Holindex usage}

Similar usage as bibtex, makeindex etc.

\begin{enumerate}
\item create jobname.tex with header
\begin{verbatim}
\usepackage{holindex}
\setHOLlinewidth{80}   %optional default is 80
\setHOLoverrides{file} %optional 
\end{verbatim}
\item make sure that \texttt{holindex} is loaded after packages like
  \texttt{natbib}, since it loads \texttt{underscore} and this will
  break commands like \texttt{cite} that get arguments with
  underscores if they are not properly protected
\item run \texttt{mkmunge.exe theories} to generate \texttt{munge.exe}
  for your theories
\item run latex on jobname.tex to generate jobname.hix
\item run \texttt{munge.exe -index jobname} to create jobname.tde and jobname.tid
\item rerun latex to use jobname.tid and jobname.tde
\item rerun the munger whenever some significant HOL stuff changed
\item if using emacs with AucTex you might want to add to your \texttt{.emacs}
\begin{alltt}
(eval-after-load "tex"
  '(add-to-list 'TeX-command-list
    '("Holindex" "munge.exe -index %s" 
       TeX-run-background t t :help "Run Holindex") t))
\end{alltt}
\end{enumerate}


\section{Defining}
\begin{verbatim}
   \defHOLtm{unique_id}{label}{def}
   \defHOLty{unique_id}{label}{def}
   \defHOLthm{unique_id}{label}{def}

   (Theorems don't need defining. Theorem IDs of the form
    theory.name are automatically recognised.
    The label is their name and the def
    is stored in the theory. Giving them an extra label
    will add this label after their name.

    You can define theorems anyhow. This is useful, when
    - the theorem name is too lengthy / contains special characters
    - you want to use one theorem with different formating options
    - you want to add a theorem several times to the index)

   Use the same formating options as the munger
   \formatHOLtm{unique_id}{options}
   \formatHOLty{unique_id}{options}
   \formatHOLthm{unique_id}{options}

   Combine definition and formating
   \formatDefHOLtm{unique_id}{label}{options}{def}
   \formatDefHOLty{unique_id}{label}{options}{def}

   Add it explicitly to the index (citations are added automatically)
   \indexHOLtm{unique_id}
   \indexHOLty{unique_id}
   \indexHOLthm{unique_id}

   Print long version in the index
   \longIndexHOLthm{unique_id}
   \longIndexHOLty{unique_id}
   \longIndexHOLtm{unique_id}
   \shortIndexHOLthm{unique_id}
   \shortIndexHOLty{unique_id}
   \shortIndexHOLtm{unique_id}

   Use boolean flags to determine the default
   \setboolean{holIndexLongTermFlag}{true}
   \setboolean{holIndexLongTypeFlag}{true}
   \setboolean{holIndexLongThmFlag}{false}

   Add comments in the index
   \commentHOLtm{unique_id}{comment}
   \commentHOLty{unique_id}{comment}
   \commentHOLthm{unique_id}{comment}
\end{verbatim}
%
Definitions like these inside a latex file are tedious,
especially for long terms. I strongly recommend using
an external file for all these definitions. There
is a construct
\begin{verbatim}
   \useHOLfile{filename}
\end{verbatim}
that parses files in a syntax similar to BibTex.
Printout of \texttt{holindex-demo.hdf}:

\begin{verbatim}
@Term{term_id_4,
   label = "a short description of term form external file",
   force_index,
   options = "width=20",
   content = ``SOME_FUN = SUC a < 0 /\ 0 > SUC b``
}

@Type{type_id_4,
   label = "a short description of the type from external file",
   force_index,
   short-index,
   content = ``:bool``   
}


@Thm{arithmetic.LESS_SUC_EQ_COR,
   force-index,
   long-index,
   latex = "My latex code for \texttt{LESS_SUC_EQ_COR}"
}

@Thm{arithmetic.LESS_EQ_0,
   force-index, long-index
}

@Thm{thm_1,
   label = "(second instance)",
   content = "arithmetic.LESS_SUC_EQ_COR",
   options = "width=20",
   comment = "some lengthy\\comment

              with \textbf{formats} and newlines",
   force-index,
   long-index
}

@Theorems{
   ids = [arithmetic.LESS_ADD_SUC,
          arithmetic.LESS_EQ*],
   force-index
}
\end{verbatim}


%use external file
\useHOLfile{holindex-demo.hdf}

%or define inline (recommended only for short, simple ones)
\defHOLtm{term_id_1}{The first term}{SUC a > 0 /\ X > 2}
\defHOLtm{term_id_2}{The second term}{SUC a < 0 /\ X > 3}
\defHOLty{type_id_1}{The first type}{:bool}
\defHOLty{type_id_2}{The second type}{:num}
\formatDefHOLtm{term_width_5}{Test term width=5}{width=5}{SUC a > 0 /\ X > 2}
\formatDefHOLtm{term_width_10}{Test term width=10}{width=10}{SUC a > 0 /\ X > 2}
\formatDefHOLtm{term_width_50}{Test term width=50}{width=50}{SUC a > 0 /\ X > 2}



\section{Printing}

\begin{verbatim}
   Print inline
   \inlineHOLtm{id}
   \inlineHOLty{id}
   \inlineHOLthm{id}

   Print as block
   \blockHOLtm{id}
   \blockHOLty{id}
   \blockHOLthm{id}

   Use def without encapsulation (seldom useful)
   \useHOLtm{id}
   \useHOLty{id}
   \useHOLthm{id}
\end{verbatim}

\subsection{Block Examples}
\begin{itemize}
\item Example 1 \blockHOLtm{term_width_5}
\item Example 2 \blockHOLtm{term_width_10}
\item Example 3 \blockHOLtm{term_width_50}
\item Example 4 \blockHOLthm{arithmetic.DIVMOD_THM}
\end{itemize}


\subsection{Inline Examples}
\begin{itemize}
\item Example 1 \inlineHOLtm{term_width_5}
\item Example 2 \inlineHOLtm{term_width_10}
\item Example 3 \inlineHOLtm{term_width_50}
\item Example 4 \inlineHOLthm{arithmetic.DIVMOD_THM}
\end{itemize}



\section{Citing}

\begin{verbatim}
   Pure numbers
   \citePureHOLtm{id}
   \citePureHOLty{id}
   \citePureHOLthm{id}

   Single citations
   \citeHOLtm{id}
   \citeHOLty{id}
   \citeHOLthm{id}

   Multiple citations 
   \mciteHOLtm{id,id,...}
   \mciteHOLty{id,id,...}
   \mciteHOLthm{id,id,...}

   Hidden citations (adds page to index, 
      but does not print anything)
   \citeHiddenHOLtm{id}
   \citeHiddenHOLty{id}
   \citeHiddenHOLthm{id}


   Printing Index 
   \printHOLIndex
   \printShortHOLIndex 
   \printLongHOLIndex 

   Sort the index by occurence in the text, not by names
   \sortHOLIndexOccurence
\end{verbatim}


\section{Printing}
There are some macros just concerned with the resulting output.
One can adapt these easily to personal preferences. Here the standard
definitions are printed. Just copy them to the header of your file
and modify them to your taste:

\begin{verbatim}
The environment for printing something as a 
block and the inline command (found in holtexbasic.sty):
\renewcommand{\HOLinline}[1]{\mbox{\textup{\texttt{#1}}}}
\renewenvironment{HOLblock}{\begin{alltt}}{\end{alltt}}

Format of citations
\renewcommand[\protect]{\citeHOLthm}[1]{(HOL-Thm~\citePureHOLthm{#1})}
\renewcommand[\protect]{\citeHOLtm}[1]{(HOL-Term~\citePureHOLtm{#1})}
\renewcommand[\protect]{\citeHOLty}[1]{(HOL-Type~\citePureHOLty{#1})}

\renewcommand[\protect]{\mciteHOLthm}[1]
   {(HOL-Thms~\foreach{\citePureHOLthm}{, }{#1})}
\renewcommand[\protect]{\mciteHOLtm}[1]
   {(HOL-Terms~\foreach{\citePureHOLtm}{, }{#1})}
\renewcommand[\protect]{\mciteHOLty}[1]
   {(HOL-Types~\foreach{\citePureHOLty}{, }{#1})}


How theorem names are printed in the index
\renewcommand{\HOLThmName}[1]{{\tt{}#1}}

Headers in the index
\rewenvironment{HOLTermIndex}{\section*{HOL-Term Index}}{}
\renewenvironment{HOLTypeIndex}{\section*{HOL-Type Index}}{}
\renewenvironment{HOLThmIndex}{\section*{HOL-Theorem Index}}{}
\renewcommand{\HOLThmIndexTheory}[1]{\subsection*{#1Theory}}
\end{verbatim}
\pagebreak
\begin{verbatim}
Much more complicated: Index entries themselves
% Arg 1 : number in the index
% Arg 2 : unique ID
% Arg 3 : Label
% Arg 4 : Pages
% Arg 5 : Long index?
% Arg 6 : is the pages argument non-empty?
% Arg 7 : comment
% Arg 8 : latex printed version of theorem/type/term
\renewcommand{\HOLIndexEntryFormat}[8]{% 
   \makebox[0.1\textwidth][r]{(#1)\ }%number
   \begin{minipage}[t]{0.9\textwidth}%everything else in minibox
      \begin{flushright}#3 \ldots$\!$\dotfill #4\end{flushright}{%first line
      \ifthenelse{\(\not\boolean{holShortIndexFlag}\) \and %use short or
                  \(\boolean{#5} \or \boolean{holLongIndexFlag}\)}% long version
          {#7\begin{blockHOL}#8\end{blockHOL}\medskip}%long version
          {}% else short version
     }%end if-then-else
   \end{minipage}\\}


And index entries for the theorems, types and terms using this definition

% Arg 1 : unique ID 
% Arg 2 : Label
% Arg 3 : Pages
% Arg 4 : Long index?
% Arg 5 : is the pages argument non-empty?
% Arg 6 : comment
% Arg 7 : latex printed version of theorem/type/term
\renewcommand{\HOLThmIndexEntryFormat}[7]{%
   \HOLIndexEntryFormat{\arabic{holThmCounter}}{thm@#1}{#2}{#3}{#4}{#5}{#6}{#7}%
}
\renewcommand{\HOLTypeIndexEntryFormat}[7]{%
   \HOLIndexEntryFormat{\arabic{holTypeCounter}}{type@#1}{#2}{#3}{#4}{#5}{#6}{#7}%
}
\renewcommand{\HOLTermIndexEntryFormat}[7]{%
   \HOLIndexEntryFormat{\arabic{holTermCounter}}{term@#1}{#2}{#3}{#4}{#5}{#6}{#7}%
}
\end{verbatim}



\section{Some citing to fill the index}

\citeHOLthm{arithmetic.LESS_SUC_EQ_COR}
\citeHOLtm{term_id_1}
\citeHOLty{type_id_2}

\mciteHOLthm{arithmetic.LESS_SUC_EQ_COR,prim_rec.INV_SUC_EQ,arithmetic.LESS_SUC_EQ_COR}

\pagebreak

\citeHOLthm{prim_rec.INV_SUC_EQ}
\citeHOLthm{arithmetic.PRE_SUC_EQ}

\pagebreak

\citeHOLthm{prim_rec.INV_SUC_EQ}
\citeHOLthm{arithmetic.PRE_SUC_EQ}

\pagebreak

\citeHOLthm{prim_rec.INV_SUC_EQ}

\pagebreak

\citeHOLthm{prim_rec.INV_SUC_EQ}
\citeHOLthm{arithmetic.PRE_SUC_EQ}
\citeHOLthm{list.list_repfns}

\longIndexHOLthm{arithmetic.PRE_SUC_EQ}
\shortIndexHOLty{type_id_2}
\commentHOLty{type_id_4}{Some short comment}

%Sort index by occurence
%\sortHOLIndexOccurence

\setboolean{holIndexLongTermFlag}{false}


\section{Long Index}
\printLongHOLIndex

\pagebreak
\section{Index}
\printHOLIndex

\pagebreak
\section{Short Index}
\printShortHOLIndex


\end{document}