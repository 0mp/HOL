\documentclass{article}

\usepackage{holindex}
\initHOLindex
\setHOLlinewidth{80}  %default is 80

\begin{document}


\section{Holindex usage}

Similar usage as bibtex, makeindex etc.

\begin{enumerate}
\item create jobname.tex with header
\begin{verbatim}
\usepackage{holindex}
\initHOLindex
\setHOLlinewidth{80}  %optional default is 80
\end{verbatim}
\item run latex on jobname.tex to generate jobname.hix
\item run \texttt{munge -index jobname} to create jobname.tix
\item rerun latex to use jobname.tix
\item rerun the munger whenever some significant HOL stuff changed
\end{enumerate}

\section{Defining}
\begin{verbatim}
   \defHOLtm{unique_id}{label}{def}
   \defHOLty{unique_id}{label}{def}

   (Theorems don't need defining, since there ID is 
    theory.name, there label is there name and the def
    is stored in the theory)

   Use the same formating options as the munger
   \formatHOLtm{unique_id}{options}
   \formatHOLty{unique_id}{options}
   \formatHOLthm{unique_id}{options}

   Combine definition and formating
   \formatDefHOLtm{unique_id}{label}{options}{def}
   \formatDefHOLty{unique_id}{label}{options}{def}

   Add it explicitly to the index (citations are added automatically)
   \indexHOLtm{unique_id}
   \indexHOLty{unique_id}
   \indexHOLthm{unique_id}


   This is tedious, especially for long terms, so do it
   in an external file and use that one (Syntax similar to bibtex)
   \useHOLfile{filename}
\end{verbatim}

%use external file
\useHOLfile{test.hdf}

%or define inline (recommended only for short, simple ones)
\defHOLtm{term_id_1}{The first term}{SUC a > 0 /\ X > 2}
\defHOLtm{term_id_2}{The second term}{SUC a < 0 /\ X > 3}
\defHOLty{type_id_1}{The first type}{:bool}
\defHOLty{type_id_2}{The second type}{:num}
\formatDefHOLtm{term_width_5}{Test term width=5}{width=5}{SUC a > 0 /\ X > 2}
\formatDefHOLtm{term_width_10}{Test term width=10}{width=10}{SUC a > 0 /\ X > 2}
\formatDefHOLtm{term_width_50}{Test term width=50}{width=50}{SUC a > 0 /\ X > 2}



\section{Printing}

\begin{verbatim}
   Print inline
   \inlineHOLtm{id}
   \inlineHOLty{id}
   \inlineHOLthm{id}

   Print as block
   \blockHOLtm{id}
   \blockHOLty{id}
   \blockHOLthm{id}
\end{verbatim}

\subsection{Block Examples}
\begin{itemize}
\item Example 1 \blockHOLtm{term_width_5}
\item Example 2 \blockHOLtm{term_width_10}
\item Example 3 \blockHOLtm{term_width_50}
\item Example 4 \blockHOLthm{arithmetic.DIVMOD_THM}
\end{itemize}


\subsection{Inline Examples}
\begin{itemize}
\item Example 1 \inlineHOLtm{term_width_5}
\item Example 2 \inlineHOLtm{term_width_10}
\item Example 3 \inlineHOLtm{term_width_50}
\item Example 4 \inlineHOLthm{arithmetic.DIVMOD_THM}
\end{itemize}



\section{Citing}

\begin{verbatim}
   Pure numbers
   \citePureHOLtm{id}
   \citePureHOLty{id}
   \citePureHOLthm{id}

   Single citations
   \citeHOLtm{id}
   \citeHOLty{id}
   \citeHOLthm{id}

   Multiple citations 
   \mciteHOLtm{id,id,...}
   \mciteHOLty{id,id,...}
   \mciteHOLthm{id,id,...}

   Printing Index 
   \printHOLIndex
   \printShortHOLIndex 
\end{verbatim}

\citeHOLthm{arithmetic.LESS_SUC_EQ_COR}
\citeHOLtm{term_id_1}
\citeHOLty{type_id_2}

\mciteHOLthm{arithmetic.LESS_SUC_EQ_COR,prim_rec.INV_SUC_EQ,arithmetic.LESS_SUC_EQ_COR}

\pagebreak

\citeHOLthm{prim_rec.INV_SUC_EQ}
\citeHOLthm{arithmetic.PRE_SUC_EQ}

\pagebreak

\citeHOLthm{prim_rec.INV_SUC_EQ}
\citeHOLthm{arithmetic.PRE_SUC_EQ}

\pagebreak

\citeHOLthm{prim_rec.INV_SUC_EQ}

\pagebreak

\citeHOLthm{prim_rec.INV_SUC_EQ}
\citeHOLthm{arithmetic.PRE_SUC_EQ}

\printHOLIndex

\pagebreak

\printShortHOLIndex


\end{document}