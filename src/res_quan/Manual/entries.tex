\chapter{ML Functions in the {\tt res\_quan} Library}
\label{entries}
This chapter provides documentation on all the \ML\ functions that are made
available in \HOL\ when the \ml{taut} library is loaded.  This documentation
is also available online via the \ml{help} facility.



\DOC{COND\_REWRITE1\_CONV}

\TYPE {\small\verb%COND_REWRITE1_CONV : (thm list -> thm -> conv)%}\egroup

\SYNOPSIS
A simple conditional rewriting conversion.

\DESCRIBE
{\small\verb%COND_REWRITE1_CONV%} is a front end of the conditional rewriting
conversion {\small\verb%COND_REWR_CONV%}. The input theorem should be in the following form
{\par\samepage\setseps\small
\begin{verbatim}
   A |- !x11 ... . P1 ==> ... !xm1 ... . Pm ==> (!x ... . Q = R)
\end{verbatim}
}
\noindent where each antecedent {\small\verb%Pi%} itself may be a conjunction or disjunction.
This theorem is transformed to a standard form expected by
{\small\verb%COND_REWR_CONV%} which carries out the actual rewriting.
The transformation is performed by {\small\verb%COND_REWR_CANON%}. The search function
passed to {\small\verb%COND_REWR_CONV%} is {\small\verb%search_top_down%}. The effect of applying
the conversion {\small\verb%COND_REWRITE1_CONV ths th%} to a term {\small\verb%tm%} is to derive a theorem
{\par\samepage\setseps\small
\begin{verbatim}
  A' |- tm = tm[R'/Q']
\end{verbatim}
}
\noindent where the right hand side of the equation is obtained by rewriting
the input term {\small\verb%tm%} with an instance of the conclusion of the input theorem. 
The theorems in the list {\small\verb%ths%} are used to discharge the assumptions
generated from the antecedents of the input theorem. 

\FAILURE
{\small\verb%COND_REWRITE1_CONV ths th%}  fails if {\small\verb%th%} cannot be transformed into the
required form by {\small\verb%COND_REWR_CANON%}. Otherwise, it fails if no match
is found or the theorem cannot be instantiated.

\EXAMPLE
The following example illustrates a straightforward use of {\small\verb%COND_REWRITE1_CONV%}.
We use the built-in theorem {\small\verb%LESS_MOD%} as the input theorem.
{\par\samepage\setseps\small
\begin{verbatim}
   #LESS_MOD;;
   Theorem LESS_MOD autoloading from theory `arithmetic` ...
   LESS_MOD = |- !n k. k < n ==> (k MOD n = k)

   |- !n k. k < n ==> (k MOD n = k)

   #COND_REWRITE1_CONV [] LESS_MOD "2 MOD 3";;
   2 < 3 |- 2 MOD 3 = 2

   #let less_2_3 = REWRITE_RULE[LESS_MONO_EQ;LESS_0]
   #(REDEPTH_CONV num_CONV "2 < 3");;
   less_2_3 = |- 2 < 3

   #COND_REWRITE1_CONV [less_2_3] LESS_MOD "2 MOD 3";;
   |- 2 MOD 3 = 2

\end{verbatim}
}
\noindent In the first example, an empty theorem list is supplied to
{\small\verb%COND_REWRITE1_CONV%} so the resulting theorem has an assumption
{\small\verb%2 < 3%}. In the second example, a list containing a theorem {\small\verb%|- 2 < 3%}
is supplied, the resulting theorem has no assumptions.

\SEEALSO
COND_REWR_TAC, COND_REWRITE1_TAC, COND_REWR_CONV, 
COND_REWR_CANON, search_top_down.

\ENDDOC

\DOC{COND\_REWRITE1\_TAC}

\TYPE {\small\verb%COND_REWRITE1_TAC : thm_tactic%}\egroup

\SYNOPSIS
A simple conditional rewriting tactic.

\DESCRIBE
{\small\verb%COND_REWRITE1_TAC%} is a front end of the conditional rewriting
tactic {\small\verb%COND_REWR_TAC%}. The input theorem should be in the following form
{\par\samepage\setseps\small
\begin{verbatim}
   A |- !x11 ... . P1 ==> ... !xm1 ... . Pm ==> (!x ... . Q = R)
\end{verbatim}
}
\noindent where each antecedent {\small\verb%Pi%} itself may be a conjunction or disjunction.
This theorem is transformed to a standard form expected by
{\small\verb%COND_REWR_TAC%} which carries out the actual rewriting.
The transformation is performed by {\small\verb%COND_REWR_CANON%}. The search function
passed to {\small\verb%COND_REWR_TAC%} is {\small\verb%search_top_down%}. The effect of applying
this tactic is to substitute into the goal instances of the right hand
side of the conclusion of the input theorem {\small\verb%Ri'%} for the
corresponding instances of the left hand side. The search is top-down
left-to-right. All matches found by the search function are
substituted. New subgoals corresponding to the instances of the
antecedents which do not appear in the assumption of the original goal
are created. See manual page of {\small\verb%COND_REWR_TAC%} for details of how the
instantiation and substitution are done.

\FAILURE
{\small\verb%COND_REWRITE1_TAC th%}  fails if {\small\verb%th%} cannot be transformed into the
required form by the function {\small\verb%COND_REWR_CANON%}. Otherwise, it fails if no match
is found or the theorem cannot be instantiated.

\EXAMPLE
The following example illustrates a straightforward use of {\small\verb%COND_REWRITE1_TAC%}.
We use the built-in theorem {\small\verb%LESS_MOD%} as the input theorem.
{\par\samepage\setseps\small
\begin{verbatim}
   #LESS_MOD;;
   Theorem LESS_MOD autoloading from theory `arithmetic` ...
   LESS_MOD = |- !n k. k < n ==> (k MOD n = k)

   |- !n k. k < n ==> (k MOD n = k)
\end{verbatim}
}
\noindent We set up a goal
{\par\samepage\setseps\small
\begin{verbatim}
   #g"2 MOD 3 = 2";;
   "2 MOD 3 = 2"

   () : void
\end{verbatim}
}
\noindent and then apply the tactic
{\par\samepage\setseps\small
\begin{verbatim}
   #e(COND_REWRITE1_TAC LESS_MOD);;
   OK..
   2 subgoals
   "2 = 2"
       [ "2 < 3" ]

   "2 < 3"

   () : void
\end{verbatim}
}

\SEEALSO
COND_REWR_TAC, COND_REWRITE1_CONV, COND_REWR_CONV, 
COND_REWR_CANON, search_top_down.

\ENDDOC

\DOC{COND\_REWR\_CANON}

\TYPE {\small\verb%COND_REWR_CANON : thm -> thm%}\egroup

\SYNOPSIS
Transform a theorem into a form accepted by {\small\verb%COND_REWR_TAC%}.

\DESCRIBE
{\small\verb%COND_REWR_CANON%} transforms a theorem into a form accepted by {\small\verb%COND_REWR_TAC%}.
The input theorem should be an implication of the following form
{\par\samepage\setseps\small
\begin{verbatim}
   !x1 ... xn. P1[xi] ==> ... ==> !y1 ... ym. Pr[xi,yi] ==>
     (!z1 ... zk. u[xi,yi,zi] = v[xi,yi,zi])
\end{verbatim}
}
\noindent where each antecedent {\small\verb%Pi%} itself may be a conjunction or
disjunction. The output theorem will have all universal quantifications
moved to the outer most level with possible renaming to prevent
variable capture, and have all antecedents which are a conjunction
transformed to implications. The output theorem will be in the
following form
{\par\samepage\setseps\small
\begin{verbatim}
   !x1 ... xn y1 ... ym z1 ... zk. 
    P11[xi] ==> ... ==> P1p[xi] ==> ... ==>
     Pr1[xi,yi] ==> ... ==> Prq[x1,yi] ==> (u[xi,yi,zi] = v[xi,yi,zi])
\end{verbatim}
}

\FAILURE
This function fails if the input theorem is not in the correct form.

\EXAMPLE
{\small\verb%COND_REWR_CANON%} transforms the built-in theorem {\small\verb%CANCL_SUB%} into the
form for conditional rewriting:
{\par\samepage\setseps\small
\begin{verbatim}
   #COND_REWR_CANON CANCEL_SUB;;
   Theorem CANCEL_SUB autoloading from theory `arithmetic` ...
   CANCEL_SUB = |- !p n m. p <= n /\ p <= m ==> ((n - p = m - p) = (n = m))

   |- !p n m. p <= n ==> p <= m ==> ((n - p = m - p) = (n = m))
\end{verbatim}
}

\SEEALSO
COND_REWRITE1_TAC, COND_REWR_TAC, COND_REWRITE1_CONV, COND_REWR_CONV, 
search_top_down.

\ENDDOC

\DOC{COND\_REWR\_CONV}

{\small
\begin{verbatim}
COND_REWR_CONV : ((term -> term ->
 ((term # term) list # (type # type) list) list) -> thm -> conv)
\end{verbatim}
}\egroup

\SYNOPSIS
A lower level conversion implementing simple conditional rewriting.

\DESCRIBE
{\small\verb%COND_REWR_CONV%} is one of the basic building blocks for the
implementation of the simple conditional rewriting conversions in the
HOL system. In particular, the conditional term replacement or
rewriting done by all the conditional 
rewriting conversions in this library is ultimately done by applications of
{\small\verb%COND_REWR_CONV%}.  The description given here for {\small\verb%COND_REWR_CONV%} may
therefore be taken as a specification of the atomic action of
replacing equals by equals in a term under certain conditions that
are used in all these higher level conditional rewriting conversions.

The first argument to {\small\verb%COND_REWR_CONV%} is expected to be a function
which returns a list of matches. Each of these matches is in the form
of the value returned by the built-in function {\small\verb%match%}. It is used to
search the input term for instances which may be rewritten.

The second argument to {\small\verb%COND_REWR_CONV%} is expected to be an
implicative theorem in the following form:
{\par\samepage\setseps\small
\begin{verbatim}
   A |- !x1 ... xn. P1 ==> ... Pm ==> (Q[x1,...,xn] = R[x1,...,xn])
\end{verbatim}
}
\noindent where {\small\verb%x1%}, ..., {\small\verb%xn%} are all the variables that occur free in the
left hand side of the conclusion of the theorem but do not occur free in the
assumptions. 

The last argument to {\small\verb%COND_REWR_CONV%} is the term to be rewritten.

If {\small\verb%fn%} is a function and {\small\verb%th%} is an implicative theorem of the kind
shown above, then {\small\verb%COND_REWR_CONV fn th%} will be a conversion.
When applying to a term {\small\verb%tm%}, it will return a theorem
{\par\samepage\setseps\small
\begin{verbatim}
   P1', ..., Pm' |- tm = tm[R'/Q']
\end{verbatim}
}
\noindent if evaluating {\small\verb%fn Q[x1,...,xn] tm%} returns a
non-empty list of matches.
The assumptions of the resulting theorem are instances of the
antecedents of the input theorem {\small\verb%th%}. The right hand side of the
equation is obtained by rewriting the input term {\small\verb%tm%} with instances of
the conclusion of the input theorem.

\FAILURE
{\small\verb%COND_REWR_CONV fn th%} fails if {\small\verb%th%} is not an implication of the form
described above.  If {\small\verb%th%} is such an equation, but the function {\small\verb%fn%}
returns a null list of matches, or the function {\small\verb%fn%} returns a
non-empty list of matches, but the term or type instantiation fails.

\EXAMPLE
The following example illustrates a straightforward use of {\small\verb%COND_REWR_CONV%}.
We use the built-in theorem {\small\verb%LESS_MOD%} as the input theorem, and the function
{\small\verb%search_top_down%} as the search function.
{\par\samepage\setseps\small
\begin{verbatim}
   #LESS_MOD;;
   Theorem LESS_MOD autoloading from theory `arithmetic` ...
   LESS_MOD = |- !n k. k < n ==> (k MOD n = k)

   |- !n k. k < n ==> (k MOD n = k)

   #search_top_down;;
   - : (term -> term -> ((term # term) list # (type # type) list) list)

   #COND_REWR_CONV search_top_down LESS_MOD "2 MOD 3";;
   2 < 3 |- 2 MOD 3 = 2
\end{verbatim}
}

\SEEALSO
COND_REWR_TAC, COND_REWRITE1_TAC, COND_REWRITE1_CONV, COND_REWR_CANON,
search_top_down.

\ENDDOC
\DOC{COND\_REWR\_TAC}

{\small
\begin{verbatim}
COND_REWR_TAC : ((term -> term ->
 ((term # term) list # (type # type) list) list) -> thm_tactic)
\end{verbatim}
}\egroup

\SYNOPSIS
A lower level tactic used to implement simple conditional rewriting tactic.

\DESCRIBE
{\small\verb%COND_REWR_TAC%} is one of the basic building blocks for the
implementation of conditional rewriting in the HOL system. In
particular, the conditional term replacement or rewriting done by all
the built-in  conditional
rewriting tactics is ultimately done by applications of
{\small\verb%COND_REWR_TAC%}.  The description given here for {\small\verb%COND_REWR_TAC%} may
therefore be taken as a specification of the atomic action of
replacing equals by equals in the goal under certain conditions that
aare used in all these higher level conditional rewriting tactics.

The first argument to {\small\verb%COND_REWR_TAC%} is expected to be a function
which returns a list of matches. Each of these matches is in the form
of the value returned by the built-in function {\small\verb%match%}. It is used to
search the goal for instances which may be rewritten.

The second argument to {\small\verb%COND_REWR_TAC%} is expected to be an implicative theorem
in the following form:
{\par\samepage\setseps\small
\begin{verbatim}
   A |- !x1 ... xn. P1 ==> ... Pm ==> (Q[x1,...,xn] = R[x1,...,xn])
\end{verbatim}
}
\noindent where {\small\verb%x1%}, ..., {\small\verb%xn%} are all the variables that occur free in the
left-hand side of the conclusion of the theorem but do not occur free in the
assumptions. 

If {\small\verb%fn%} is a function and {\small\verb%th%} is an implicative theorem of the kind
shown above, then {\small\verb%COND_REWR_TAC fn th%} will be a tactic which returns
a list of subgoals if evaluating 
{\par\samepage\setseps\small
\begin{verbatim}
   fn Q[x1,...,xn] gl
\end{verbatim}
}
\noindent returns a non-empty list of matches when applied to a goal {\small\verb%(asm,gl)%}.

Let {\small\verb%ml%} be the match list returned by evaluating {\small\verb%fn Q[x1,...,xn]
gl%}. Each element in this list is in the form of
{\par\samepage\setseps\small
\begin{verbatim}
   ([(e1,x1);...;(ep,xp)], [(ty1,vty1);...;(tyq,vtyq)])
\end{verbatim}
}
\noindent which specifies the term and type instantiations of the input theorem
{\small\verb%th%}. Either the term pair list or the type pair list may be empty. In
the case that both lists are empty, an exact match is found, i.e., no
instantiation is required. If {\small\verb%ml%} is an empty list, no match has been
found and the tactic will fail.

For each match in {\small\verb%ml%}, {\small\verb%COND_REWR_TAC%} will perform the following:
1) instantiate the input theorem {\small\verb%th%} to get
{\par\samepage\setseps\small
\begin{verbatim}
   th' = A |- P1' ==> ... ==> Pm' ==> (Q' = R')
\end{verbatim}
}
\noindent where the primed subterms are instances of the corresponding
unprimed subterms obtained by applying {\small\verb%INST_TYPE%} with
{\small\verb%[(ty1,vty1);...;(tyq,vtyq)]%} and then {\small\verb%INST%} with {\small\verb%[(e1,x1);...;(ep,xp)]%};
2) search the assumption list {\small\verb%asm%} for occurrences of any antecedents
{\small\verb%P1'%}, ..., {\small\verb%Pm'%};
3) if all antecedents appear in {\small\verb%asm%}, the goal {\small\verb%gl%} is reduced to
{\small\verb%gl'%} by substituting {\small\verb%R'%} for each free occurrence of {\small\verb%Q'%}, otherwise,
in addition to the substitution, all antecedents which do not appear
in {\small\verb%asm%} are added to it and new 
subgoals corresponding to these antecedents are created. For example,
if {\small\verb%Pk'%}, ..., {\small\verb%Pm'%} do not appear in {\small\verb%asm%}, the following subgoals
are returned:
{\par\samepage\setseps\small
\begin{verbatim}
   asm ?- Pk'  ...  asm ?- Pm'   {asm,Pk',...,Pm'} ?- gl'
\end{verbatim}
}

If {\small\verb%COND_REWR_TAC%} is given a theorem {\small\verb%th%}:
{\par\samepage\setseps\small
\begin{verbatim}
   A |- !x1 ... xn y1 ... yk. P1 ==> ... ==> Pm ==> (Q = R)
\end{verbatim}
}
\noindent where the variables {\small\verb%y1%}, ..., {\small\verb%ym%} do not occur free in the
left-hand side of the conclusion {\small\verb%Q%} but they do occur free in the
antecedents, then, when carrying out Step 2 described 
above, {\small\verb%COND_REWR_TAC%} will attempt to find instantiations for these
variables from the assumption {\small\verb%asm%}. For example, if {\small\verb%x1%} and {\small\verb%y1%}
occur free in {\small\verb%P1%}, and a match is found in which {\small\verb%e1%} is an
instantiation of {\small\verb%x1%}, then {\small\verb%P1'%} will become {\small\verb%P1[e1/x1, y1]%}. If a term
{\small\verb%P1'' = P1[e1,e1'/x1,y1]%} appears in  {\small\verb%asm%}, {\small\verb%th'%} is instantiated  with
{\small\verb%(e1', y1)%} to get
{\par\samepage\setseps\small
\begin{verbatim}
   th'' = A |- P1'' ==> ... ==> Pm'' ==> (Q' = R'')
\end{verbatim}
}
\noindent then {\small\verb%R''%} is substituted into {\small\verb%gl%} for all free occurrences
of {\small\verb%Q'%}. If no consistent instantiation is found, then {\small\verb%P1'%} which
contains the uninstantiated variable {\small\verb%y1%} will become one of the new
subgoals. In such a case, the user has no control over the
choice of the variable {\small\verb%yi%}.

\FAILURE
{\small\verb%COND_REWR_TAC fn th%} fails if {\small\verb%th%} is not an implication of the form
described above.  If {\small\verb%th%} is such an equation, but the function {\small\verb%fn%}
returns a null list of matches, or the function {\small\verb%fn%} returns a
non-empty list of matches, but the term or type instantiation fails.

\EXAMPLE
The following example illustrates a straightforward use of {\small\verb%COND_REWR_TAC%}.
We use the built-in theorem {\small\verb%LESS_MOD%} as the input theorem, and the function
{\small\verb%search_top_down%} as the search function.
{\par\samepage\setseps\small
\begin{verbatim}
   #LESS_MOD;;
   Theorem LESS_MOD autoloading from theory `arithmetic` ...
   LESS_MOD = |- !n k. k < n ==> (k MOD n = k)

   |- !n k. k < n ==> (k MOD n = k)

   #search_top_down;;
   - : (term -> term -> ((term # term) list # (type # type) list) list)
\end{verbatim}
}
\noindent We set up a goal
{\par\samepage\setseps\small
\begin{verbatim}
   #g"2 MOD 3 = 2";;
   "2 MOD 3 = 2"

   () : void
\end{verbatim}
}
\noindent and then apply the tactic
{\par\samepage\setseps\small
\begin{verbatim}
   #e(COND_REWR_TAC search_top_down LESS_MOD);;
   OK..
   2 subgoals
   "2 = 2"
       [ "2 < 3" ]

   "2 < 3"

    () : void
\end{verbatim}
}

\SEEALSO
COND_REWRITE1_TAC, COND_REWRITE1_CONV, COND_REWR_CONV, COND_REWR_CANON,
search_top_down.

\ENDDOC
\DOC{dest\_resq\_abstract}

\TYPE {\small\verb%dest_resq_abstract : (term -> (term # term # term))%}\egroup

\SYNOPSIS
Breaks apart a restricted abstract term into
the quantified variable, predicate and body.

\DESCRIBE
{\small\verb%dest_resq_abstract%} is a term destructor for restricted abstraction:
{\par\samepage\setseps\small
\begin{verbatim}
   dest_resq_abstract "\var::P. t"
\end{verbatim}
}
\noindent returns {\small\verb%("var","P","t")%}.

\FAILURE
Fails with {\small\verb%dest_resq_abstract%} if the term is not a restricted
abstraction.

\SEEALSO
mk_resq_abstract, is_resq_abstract, strip_resq_abstract.

\ENDDOC
\DOC{dest\_resq\_exists}

\TYPE {\small\verb%dest_resq_exists : (term -> (term # term # term))%}\egroup

\SYNOPSIS
Breaks apart a restricted existentially quantified term into
the quantified variable, predicate and body.

\DESCRIBE
{\small\verb%dest_resq_exists%} is a term destructor for restricted existential
quantification: 
{\par\samepage\setseps\small
\begin{verbatim}
   dest_resq_exists "?var::P. t"
\end{verbatim}
}
\noindent returns {\small\verb%("var","P","t")%}.

\FAILURE
Fails with {\small\verb%dest_resq_exists%} if the term is not a restricted
existential quantification.

\SEEALSO
mk_resq_exists, is_resq_exists, strip_resq_exists.

\ENDDOC
\DOC{dest\_resq\_forall}

\TYPE {\small\verb%dest_resq_forall : (term -> (term # term # term))%}\egroup

\SYNOPSIS
Breaks apart a restricted universally quantified term into
the quantified variable, predicate and body.

\DESCRIBE
{\small\verb%dest_resq_forall%} is a term destructor for restricted universal
quantification: 
{\par\samepage\setseps\small
\begin{verbatim}
   dest_resq_forall "!var::P. t"
\end{verbatim}
}
\noindent returns {\small\verb%("var","P","t")%}.

\FAILURE
Fails with {\small\verb%dest_resq_forall%} if the term is not a restricted
universal quantification.

\SEEALSO
mk_resq_forall, is_resq_forall, strip_resq_forall.

\ENDDOC
\DOC{dest\_resq\_select}

\TYPE {\small\verb%dest_resq_select : (term -> (term # term # term))%}\egroup

\SYNOPSIS
Breaks apart a restricted choice quantified term into
the quantified variable, predicate and body.

\DESCRIBE
{\small\verb%dest_resq_select%} is a term destructor for restricted choice
quantification: 
{\par\samepage\setseps\small
\begin{verbatim}
   dest_resq_select "@var::P. t"
\end{verbatim}
}
\noindent returns {\small\verb%("var","P","t")%}.

\FAILURE
Fails with {\small\verb%dest_resq_select%} if the term is not a restricted
choice quantification.

\SEEALSO
mk_resq_select, is_resq_select, strip_resq_select.

\ENDDOC
\DOC{GQSPECL}

\TYPE {\small\verb%GQSPECL : (term list -> thm -> thm)%}\egroup

\SYNOPSIS
Specializes zero or more variables in the conclusion of a 
universally quantified theorem.

\DESCRIBE
When applied to a term list {\small\verb%[u1;...;un]%} and a theorem whose conclusion has
zero or more ordinary or restricted universal quantifications,
the inference rule {\small\verb%GQSPECL%} returns a theorem which is the result of
specializing the quantified variables. The substitutions are made
sequentially left-to-right in the same way as for {\small\verb%GQSPEC%}, with the same
sort of alpha-conversions applied to the body of the conclusion.
The two kinds of universal quantification can be mixed.
{\par\samepage\setseps\small
\begin{verbatim}
       A |- !x1::P1. ... !xk. ... !xn::Pn. t
   --------------------------------------------------  GQSPECL "[u1;...;un]"
    A,P1 u1,...,Pn un |- t[u1/x1]...[uk/xk]...[un/xn]
\end{verbatim}
}
\noindent It is permissible for the term-list to be empty, in which case
the application of {\small\verb%GQSPECL%} has no effect.

\FAILURE
Fails if one of the specialization of the 
quantified variable in the original theorem fails.

\SEEALSO
GQSPEC, GQSPEC_ALL, SPECL, GENL, RESQ_GEN, RESQ_GENL, RESQ_GEN_ALL,
RESQ_GEN_TAC, RESQ_SPEC, RESQ_SPECL, RESQ_SPEC_ALL.

\ENDDOC
\DOC{GQSPEC\_ALL}

\TYPE {\small\verb%GQSPEC_ALL : (thm -> thm)%}\egroup

\SYNOPSIS
Specializes the conclusion of a theorem with its own quantified variables.

\DESCRIBE
When applied to a theorem whose conclusion has zero or more 
ordinary or restricted universal quantifications,
the inference rule {\small\verb%GQSPEC_ALL%} returns a theorem which is the result
of specializing the quantified variables with its own variables. If this
will cause name clashes, a variant of the variable is used instead.
Normally {\small\verb%xi'%} is just {\small\verb%xi%}, in which case {\small\verb%GQSPEC_ALL%} simply removes
all universal quantifiers.
{\par\samepage\setseps\small
\begin{verbatim}
       A |- !x1::P1. ...!xk. ... !xn::Pn. t
   ------------------------------------------------------  GQSPEC_ALL
    A,P1 x1,...,Pn xn |- t[x1'/x1]...[xk'/xk]...[xn'/xn]
\end{verbatim}
}
\FAILURE
Never fails.

\SEEALSO
GQSPEC, GQSPECL, SPEC, SPECL, SPEC_ALL, RESQ_GEN, RESQ_GENL,
RESQ_GEN_ALL, RESQ_GEN_TAC, RESQ_SPEC, RESQ_SPECL, RESQ_SPEC_ALL.

\ENDDOC
\DOC{IMP\_RESQ\_FORALL\_CONV}

\TYPE {\small\verb%IMP_RESQ_FORALL_CONV : conv%}\egroup

\SYNOPSIS
Converts an implication to a restricted universal quantification.

\DESCRIBE
When applied to a term of the form {\small\verb%!x.P x ==> Q%}, the conversion
{\small\verb%IMP_RESQ_FORALL_CONV%} returns the theorem:
{\par\samepage\setseps\small
\begin{verbatim}
   |- (!x. P x ==> Q) = !x::P. Q 
\end{verbatim}
}

\FAILURE
Fails if applied to a term not of the form {\small\verb%!x.P x ==> Q%}.

\SEEALSO
RESQ_FORALL_CONV, LIST_RESQ_FORALL_CONV.

\ENDDOC
\DOC{is\_resq\_abstract}

\TYPE {\small\verb%is_resq_abstract : (term -> bool)%}\egroup

\SYNOPSIS
Tests a term to see if it is a restricted abstraction.

\DESCRIBE
{\small\verb%is_resq_abstract "\var::P. t"%} returns {\small\verb%true%}. If the term is not a
restricted abstraction the result is {\small\verb%false%}.

\FAILURE
Never fails.

\SEEALSO
mk_resq_abstract, dest_resq_abstract.

\ENDDOC
\DOC{is\_resq\_exists}

\TYPE {\small\verb%is_resq_exists : (term -> bool)%}\egroup

\SYNOPSIS
Tests a term to see if it is a restricted existential quantification.

\DESCRIBE
{\small\verb%is_resq_exists "?var::P. t"%} returns {\small\verb%true%}. If the term is not a 
restricted existential quantification the result is {\small\verb%false%}.

\FAILURE
Never fails.

\SEEALSO
mk_resq_exists, dest_resq_exists.

\ENDDOC
\DOC{is\_resq\_forall}

\TYPE {\small\verb%is_resq_forall : (term -> bool)%}\egroup

\SYNOPSIS
Tests a term to see if it is a restricted universal quantification.

\DESCRIBE
{\small\verb%is_resq_forall "!var::P. t"%} returns {\small\verb%true%}. If the term is not a 
restricted universal quantification the result is {\small\verb%false%}.

\FAILURE
Never fails.

\SEEALSO
mk_resq_forall, dest_resq_forall.

\ENDDOC
\DOC{is\_resq\_select}

\TYPE {\small\verb%is_resq_select : (term -> bool)%}\egroup

\SYNOPSIS
Tests a term to see if it is a restricted choice quantification.

\DESCRIBE
{\small\verb%is_resq_select "@var::P. t"%} returns {\small\verb%true%}. If the term is not a
restricted choice quantification the result is {\small\verb%false%}.

\FAILURE
Never fails.

\SEEALSO
mk_resq_select, dest_resq_select.

\ENDDOC
\DOC{list\_mk\_resq\_exists}

\TYPE {\small\verb%list_mk_resq_exists : ((term # term) list # term) -> term)%}\egroup

\SYNOPSIS
Iteratively constructs a restricted existential quantification.

\DESCRIBE
{\par\samepage\setseps\small
\begin{verbatim}
   list_mk_resq_exists([("x1","P1");...;("xn","Pn")],"t")
\end{verbatim}
}
\noindent returns {\small\verb%"?x1::P1. ... ?xn::Pn. t"%}.

\FAILURE
Fails with {\small\verb%list_mk_resq_exists%} if the first terms {\small\verb%xi%} in the pairs are
not a variable or if the second terms {\small\verb%Pi%} in the pairs and {\small\verb%t%} 
are not of type {\small\verb%":bool"%} if the list is non-empty. If the list is
empty the type of {\small\verb%t%} can be anything.

\SEEALSO
strip_resq_exists, mk_resq_exists.

\ENDDOC
\DOC{list\_mk\_resq\_forall}

\TYPE {\small\verb%list_mk_resq_forall : ((term # term) list # term) -> term)%}\egroup

\SYNOPSIS
Iteratively constructs a restricted universal quantification.

\DESCRIBE
{\par\samepage\setseps\small
\begin{verbatim}
   list_mk_resq_forall([("x1","P1");...;("xn","Pn")],"t")
\end{verbatim}
}
\noindent returns {\small\verb%"!x1::P1. ... !xn::Pn. t"%}.

\FAILURE
Fails with {\small\verb%list_mk_resq_forall%} if the first terms {\small\verb%xi%} in the pairs are
not a variable or if the second terms {\small\verb%Pi%} in the pairs and {\small\verb%t%} 
are not of type {\small\verb%":bool"%} if the list is non-empty. If the list is
empty the type of {\small\verb%t%} can be anything.

\SEEALSO
strip_resq_forall, mk_resq_forall.

\ENDDOC
\DOC{LIST\_RESQ\_FORALL\_CONV}

\TYPE {\small\verb%LIST_RESQ_FORALL_CONV : conv%}\egroup

\SYNOPSIS
Converts restricted universal quantifications iteratively to implications.

\DESCRIBE
When applied to a term whose outer level is a series of restricted universal
quantifications, the conversion
{\small\verb%LIST_RESQ_FORALL_CONV%} returns the theorem:
{\par\samepage\setseps\small
\begin{verbatim}
   |- !x1::P1. ... !xn::Pn. Q = (!x1...xn. P1 x1 ==> ... ==> Pn xn ==> Q)
\end{verbatim}
}

\FAILURE
Never fails.

\SEEALSO
IMP_RESQ_FORALL_CONV, RESQ_FORALL_CONV.

\ENDDOC
\DOC{mk\_resq\_abstract}

\TYPE {\small\verb%mk_resq_abstract : ((term # term # term) -> term)%}\egroup

\SYNOPSIS
Term constructor for restricted abstraction.

\DESCRIBE
{\small\verb%mk_resq_abstract("var","P","t")%} returns {\small\verb%"\var :: P . t"%}.

\FAILURE
Fails with {\small\verb%mk_resq_abstract%} if the first term is not a variable or if {\small\verb%P%} and {\small\verb%t%}
are not of type {\small\verb%":bool"%}.

\SEEALSO
dest_resq_abstract, is_resq_abstract, list_mk_resq_abstract.

\ENDDOC
\DOC{mk\_resq\_exists}

\TYPE {\small\verb%mk_resq_exists : ((term # term # term) -> term)%}\egroup

\SYNOPSIS
Term constructor for restricted existential quantification.

\DESCRIBE
{\small\verb%mk_resq_exists("var","P","t")%} returns {\small\verb%"?var :: P . t"%}.

\FAILURE
Fails with {\small\verb%mk_resq_exists%} if the first term is not a variable or if {\small\verb%P%} and {\small\verb%t%}
are not of type {\small\verb%":bool"%}.

\SEEALSO
dest_resq_exists, is_resq_exists, list_mk_resq_exists.

\ENDDOC
\DOC{mk\_resq\_forall}

\TYPE {\small\verb%mk_resq_forall : ((term # term # term) -> term)%}\egroup

\SYNOPSIS
Term constructor for restricted universal quantification.

\DESCRIBE
{\small\verb%mk_resq_forall("var","P","t")%} returns {\small\verb%"!var :: P . t"%}.

\FAILURE
Fails with {\small\verb%mk_resq_forall%} if the first term is not a variable or if {\small\verb%P%} and {\small\verb%t%}
are not of type {\small\verb%":bool"%}.

\SEEALSO
dest_resq_forall, is_resq_forall, list_mk_resq_forall.

\ENDDOC
\DOC{mk\_resq\_select}

\TYPE {\small\verb%mk_resq_select : ((term # term # term) -> term)%}\egroup

\SYNOPSIS
Term constructor for restricted choice quantification.

\DESCRIBE
{\small\verb%mk_resq_select("var","P","t")%} returns {\small\verb%"@var :: P . t"%}.

\FAILURE
Fails with {\small\verb%mk_resq_select%} if the first term is not a variable or if {\small\verb%P%} and {\small\verb%t%}
are not of type {\small\verb%":bool"%}.

\SEEALSO
dest_resq_select, is_resq_select, list_mk_resq_select.

\ENDDOC
\DOC{new\_binder\_resq\_definition}

\TYPE {\small\verb%new_binder_resq_definition : ((string # term) -> thm)%}\egroup

\SYNOPSIS
Declare a new binder and install a definitional axiom in the current theory.

\DESCRIBE
The function {\small\verb%new_binder_resq_definition%} provides a facility for definitional
extensions to the current theory. The new constant defined using this
function may take arguments which are restricted quantified.  The
function {\small\verb%new_binder_resq_definition%} takes a pair argument consisting 
of the name under which the resulting definition will be saved
in the current theory segment, and a term giving the desired definition.  The
value returned by {\small\verb%new_binder_resq_definition%} is a theorem which states the
definition requested by the user.

Let {\small\verb%x_1,...,x_n%} be distinct variables.  Evaluating
{\par\samepage\setseps\small
\begin{verbatim}
   new_binder_resq_definition (`name`,
    "!x_i::P_i. ... !x_j::P_j. B x_1 ... x_n = t")
\end{verbatim}
}
where {\small\verb%B%} is not already a constant, {\small\verb%i%} is greater or equal to 1 and
{\small\verb%i <= j <= n%}, declares {\small\verb%B%} to be a new constant in the current theory
with this definition as its specification. 
This constant specification is returned as a theorem with the form
{\par\samepage\setseps\small
\begin{verbatim}
   |- !x_i::P_i. ... !x_j::P_j. !x_k .... B x_1 ... x_n = t
\end{verbatim}
}
\noindent where the variables {\small\verb%x_k%} are the free variables occurring on
the left hand side of the definition and are not restricted
quantified. This theorem is saved in the current theory under
(the name) {\small\verb%name%}. 

The constant {\small\verb%B%} defined by this function will have the binder status
only after the definition has been processed. It is therefore necessary to use
the constant in normal prefix position when making the definition.
 
If the restricting predicates {\small\verb%P_l%} contains free occurrence of
variable(s) of the left hand side, the constant {\small\verb%B%} will stand for a
family of functions.

\FAILURE
{\small\verb%new_binder_resq_definition%} fails if called when HOL is not in draft mode.  It also
fails if there is already an axiom, definition or specification of the given
name in the current theory segment; if {\small\verb%`B`%} is already a constant in the
current theory or is not an allowed name for a constant; if {\small\verb%t%} contains free
variables that do not occur in the left hand side, or if
any variable occurs more than once in {\small\verb%x_1, ..., x_n%}.  Finally, failure occurs
if there is a type variable in {\small\verb%x_1%}, ..., {\small\verb%x_n%} or {\small\verb%t%} that does not occur in
the type of {\small\verb%B%}.

\SEEALSO
new_infix_resq_definition, new_resq_definition,
new_definition, new_specification.


\ENDDOC
\DOC{new\_infix\_resq\_definition}

\TYPE {\small\verb%new_infix_resq_definition : ((string # term) -> thm)%}\egroup

\SYNOPSIS
Declare a new infix constant and install a definitional axiom in the current theory.

\DESCRIBE
The function {\small\verb%new_infix_resq_definition%} provides a facility for definitional
extensions to the current theory. The new constant defined using this
function may take arguments which are restricted quantified.  The
function {\small\verb%new_infix_resq_definition%} takes a pair argument consisting 
of the name under which the resulting definition will be saved
in the current theory segment, and a term giving the desired definition.  The
value returned by {\small\verb%new_infix_resq_definition%} is a theorem which states the
definition requested by the user.

Let {\small\verb%x_1,...,x_n%} be distinct variables.  Evaluating
{\par\samepage\setseps\small
\begin{verbatim}
   new_infix_resq_definition (`name`,
    "!x_i::P_i. ... !x_j::P_j. IX x_1 ... x_n = t")
\end{verbatim}
}
where {\small\verb%IX%} is not already a constant, {\small\verb%i%} is greater or equal to 1 and
{\small\verb%i <= j <= n%}, declares {\small\verb%IX%} to be a new constant in the current theory
with this definition as its specification. 
This constant specification is returned as a theorem with the form
{\par\samepage\setseps\small
\begin{verbatim}
   |- !x_i::P_i. ... !x_j::P_j. !x_k .... IX x_1 ... x_n = t
\end{verbatim}
}
\noindent where the variables {\small\verb%x_k%} are the free variables occurring on
the left hand side of the definition and are not restricted
quantified. This theorem is saved in the current theory under
(the name) {\small\verb%name%}. 

The constant {\small\verb%IX%} defined by this function will have the infix status
only after the definition has been processed. It is therefore necessary to use
the constant in normal prefix position when making the definition.
 
If the restricting predicates {\small\verb%P_l%} contains free occurrence of
variable(s) of the left hand side, the constant {\small\verb%IX%} will stand for a
family of functions.


\FAILURE
{\small\verb%new_infix_resq_definition%} fails if called when HOL is not in draft mode.  It also
fails if there is already an axiom, definition or specification of the given
name in the current theory segment; if {\small\verb%`IX`%} is already a constant in the
current theory or is not an allowed name for a constant; if {\small\verb%t%} contains free
variables that do not occur in the left hand side, or if
any variable occurs more than once in {\small\verb%x_1, ..., x_n%}.  Finally, failure occurs
if there is a type variable in {\small\verb%x_1%}, ..., {\small\verb%x_n%} or {\small\verb%t%} that does not occur in
the type of {\small\verb%IX%}.

\EXAMPLE
A function for indexing list element starting from 1 can be defined as follows:
{\par\samepage\setseps\small
\begin{verbatim}
   #let IXEL1_DEF = new_infix_resq_definition (`IXEL1_DEF`,
   # "!n:: (\k. 0 < k). IXEL1 n (l:* list) = EL (n -1) l");;
   IXEL1_DEF = |- !n :: \k. 0 < k. !l. IXEL1 n l = EL(n - 1)l
\end{verbatim}
}
One can then use {\small\verb%IXEL1%} as an infix and do the following proof:
{\par\samepage\setseps\small
\begin{verbatim}
   #g"2 IXEL1 [1;2;3] = 2";;
   "2 IXEL1 [1;2;3] = 2"

   #e(RESQ_REWRITE1_TAC IXEL1_DEF THENL[
   #   CONV_TAC(ONCE_DEPTH_CONV num_CONV) THEN MATCH_ACCEPT_TAC LESS_0;
   #   CONV_TAC((LHS_CONV o LHS_CONV)(REDEPTH_CONV num_CONV))
   #   THEN REWRITE_TAC[SUB_MONO_EQ;SUB_0;EL;HD;TL]]);;
   OK..
   goal proved
   |- 2 IXEL1 [1;2;3] = 2

   Previous subproof:
   goal proved
   () : void
\end{verbatim}
}
\SEEALSO
new_binder_resq_definition, new_resq_definition,
new_definition, new_specification.


\ENDDOC
\DOC{new\_resq\_definition}

\TYPE {\small\verb%new_resq_definition : ((string # term) -> thm)%}\egroup

\SYNOPSIS
Declare a new constant and install a definitional axiom in the current theory.

\DESCRIBE
The function {\small\verb%new_resq_definition%} provides a facility for definitional
extensions to the current theory. The new constant defined using this
function may take arguments which are restricted quantified.  The
function {\small\verb%new_resq_definition%} takes a pair argument consisting 
of the name under which the resulting definition will be saved
in the current theory segment, and a term giving the desired definition.  The
value returned by {\small\verb%new_resq_definition%} is a theorem which states the
definition requested by the user.

Let {\small\verb%x_1,...,x_n%} be distinct variables.  Evaluating
{\par\samepage\setseps\small
\begin{verbatim}
   new_resq_definition (`name`,
    "!x_i::P_i. ... !x_j::P_j. C x_1 ... x_n = t")
\end{verbatim}
}
\noindent where {\small\verb%C%} is not already a constant, {\small\verb%i%} is greater or equal to 1 and
{\small\verb%i <= j <= n%}, declares {\small\verb%C%} to be a new constant in the current theory
with this definition as its specification. 
This constant specification is returned as a theorem with the form
{\par\samepage\setseps\small
\begin{verbatim}
   |- !x_i::P_i. ... !x_j::P_j. !x_k .... C x_1 ... x_n = t
\end{verbatim}
}
\noindent where the variables {\small\verb%x_k%} are the free variables occurring on
the left hand side of the definition and are not restricted
quantified. This theorem is saved in the current theory under
(the name) {\small\verb%name%}. 

If the restricting predicates {\small\verb%P_l%} contains free occurrence of
variable(s) of the left hand side, the constant {\small\verb%C%} will stand for a
family of functions.


\FAILURE
{\small\verb%new_resq_definition%} fails if called when HOL is not in draft mode.  It also
fails if there is already an axiom, definition or specification of the given
name in the current theory segment; if {\small\verb%`C`%} is already a constant in the
current theory or is not an allowed name for a constant; if {\small\verb%t%} contains free
variables that do not occur in the left hand side, or if
any variable occurs more than once in {\small\verb%x_1, ..., x_n%}.  Finally, failure occurs
if there is a type variable in {\small\verb%x_1%}, ..., {\small\verb%x_n%} or {\small\verb%t%} that does not occur in
the type of {\small\verb%C%}.

\EXAMPLE
A function for indexing list elements starting from 1 can be defined as follows:
{\par\samepage\setseps\small
\begin{verbatim}
   #new_resq_definition (`EL1_DEF`,
   # "!n:: (\k. 0 < k). EL1 n (l:* list) = EL (n - 1) l");;
   |- !n :: \k. 0 < k. !l. EL1 n l = EL(n - 1)l
\end{verbatim}
}
The following example shows how a family of constants may be defined
if the restricting predicate involves free variable on the left hand
side of the definition.
{\par\samepage\setseps\small
\begin{verbatim}
   #new_resq_definition (`ELL_DEF`,
   # "!n:: (\k. k < (LENGTH l)). ELL n (l:* list) = EL n  l");;
   |- !l. !n :: \k. k < (LENGTH l). !l'. ELL l n l' = EL n l'
\end{verbatim}
}

\SEEALSO
new_resq_binder_definition, new_resq_infix_definition,
new_definition, new_specification.


\ENDDOC
\DOC{RESQ\_EXISTS\_CONV}

\TYPE {\small\verb%RESQ_EXISTS_CONV : conv%}\egroup

\SYNOPSIS
Converts a restricted existential quantification to a conjunction.

\DESCRIBE
When applied to a term of the form {\small\verb%?x::P. Q[x]%}, the conversion
{\small\verb%RESQ_EXISTS_CONV%} returns the theorem:
{\par\samepage\setseps\small
\begin{verbatim}
   |- ?x::P. Q[x] = (?x. P x /\ Q[x])
\end{verbatim}
}
\noindent which is the underlying semantic representation of the restricted
existential quantification.

\FAILURE
Fails if applied to a term not of the form {\small\verb%?x::P. Q%}.

\SEEALSO
RESQ_FORALL_CONV, RESQ_EXISTS_TAC.

\ENDDOC
\DOC{RESQ\_EXISTS\_TAC}

\TYPE {\small\verb%RESQ_EXISTS_TAC : term -> tactic%}\egroup

\SYNOPSIS
Strips the outermost restricted existential quantifier from
the conclusion of a goal.

\DESCRIBE
When applied to a goal {\small\verb%A ?- ?x::P. t%}, the tactic {\small\verb%RESQ_EXISTS_TAC%}
reduces it to a new subgoal {\small\verb%A ?- P x' /\ t[x'/x]%} where {\small\verb%x'%} is a variant
of {\small\verb%x%} chosen to avoid clashing with any variables free in the goal's
assumption list. Normally {\small\verb%x'%} is just {\small\verb%x%}.
{\par\samepage\setseps\small
\begin{verbatim}
     A ?- ?x::P. t
   ======================  RESQ_EXISTS_TAC
    A ?- P x' /\ t[x'/x]
\end{verbatim}
}
\FAILURE
Fails unless the goal's conclusion is a restricted extistential quantification.

\SEEALSO
RESQ_HALF_EXISTS.

\ENDDOC
\DOC{RESQ\_FORALL\_AND\_CONV}

\TYPE {\small\verb%RESQ_FORALL_AND_CONV : conv%}\egroup

\SYNOPSIS
Splits a restricted universal quantification across a conjunction.

\DESCRIBE
When applied to a term of the form {\small\verb%!x::P. Q /\ R%}, the conversion
{\small\verb%RESQ_FORALL_AND_CONV%} returns the theorem:
{\par\samepage\setseps\small
\begin{verbatim}
   |- (!x::P. Q /\ R)  = ((!x::P. Q) /\ (!x::P. R))
\end{verbatim}
}

\FAILURE
Fails if applied to a term not of the form {\small\verb%!x::P. Q /\ R%}.

\SEEALSO
AND_RESQ_FORALL_CONV.

\ENDDOC
\DOC{RESQ\_FORALL\_CONV}

\TYPE {\small\verb%RESQ_FORALL_CONV : conv%}\egroup

\SYNOPSIS
Converts a restricted universal quantification to an implication.

\DESCRIBE
When applied to a term of the form {\small\verb%!x::P. Q%}, the conversion
{\small\verb%RESQ_FORALL_CONV%} returns the theorem:
{\par\samepage\setseps\small
\begin{verbatim}
   |- !x::P. Q = (!x. P x ==> Q)
\end{verbatim}
}
\noindent which is the underlying semantic representation of the restricted
universal quantification.

\FAILURE
Fails if applied to a term not of the form {\small\verb%!x::P. Q%}.

\SEEALSO
IMP_RESQ_FORALL_CONV, LIST_RESQ_FORALL_CONV.

\ENDDOC
\DOC{RESQ\_FORALL\_SWAP\_CONV}

\TYPE {\small\verb%RESQ_FORALL_SWAP_CONV : conv%}\egroup

\SYNOPSIS
Changes the order of two restricted universal quantifications.

\DESCRIBE
When applied to a term of the form {\small\verb%!x::P. !y::Q. R%}, the conversion
{\small\verb%RESQ_FORALL_SWAP_CONV%} returns the theorem:
{\par\samepage\setseps\small
\begin{verbatim}
   |- (!x::P. !y::Q. R) =  !y::Q. !x::P. R
\end{verbatim}
}
\noindent providing that {\small\verb%x%} does not occur free in {\small\verb%Q%} and {\small\verb%y%} does not
occur free in {\small\verb%P%}.

\FAILURE
Fails if applied to a term not of the correct form.

\SEEALSO
RESQ_FORALL_CONV.

\ENDDOC
\DOC{RESQ\_GEN}

\TYPE {\small\verb%RESQ_GEN : ((term # term) -> thm -> thm)%}\egroup

\SYNOPSIS
Generalizes the conclusion of a theorem to a restricted universal quantification.

\DESCRIBE
When applied to a pair of terms {\small\verb%x%}, {\small\verb%P%} and a theorem {\small\verb%A |- t%},
the inference rule {\small\verb%RESQ_GEN%} returns the theorem {\small\verb%A |- !x::P. t%},
provided that {\small\verb%P%} is a predicate taking an argument of the same type
as {\small\verb%x%} and that {\small\verb%x%} is a variable not free in any of the assumptions
except {\small\verb%P x%} if it occurs. There is no compulsion that {\small\verb%x%} should
be free in {\small\verb%t%} or {\small\verb%P x%} should be in the assumptions.
{\par\samepage\setseps\small
\begin{verbatim}
      A |- t
   --------------- RESQ_GEN ("x","P") [where x is not free in A except P x]
    A |- !x::P. t
\end{verbatim}
}
\FAILURE
Fails if {\small\verb%x%} is not a variable, or if it is free in any of the assumptions
other than {\small\verb%P x%}.

\SEEALSO
RESQ_GENL, RESQ_GEN_ALL, RESQ_GEN_TAC, RESQ_SPEC, RESQ_SPECL, RESQ_SPEC_ALL.

\ENDDOC
\DOC{RESQ\_GENL}

\TYPE {\small\verb%RESQ_GENL : ((term # term) list -> thm -> thm)%}\egroup

\SYNOPSIS
Generalizes zero or more variables to restricted universal quantification
in the conclusion of a theorem.

\DESCRIBE
When applied to a term-pair list {\small\verb%[(x1,P1);...;(xn,Pn)]%} and a theorem 
{\small\verb%A |- t%}, the inference rule {\small\verb%RESQ_GENL%} returns the theorem 
{\small\verb%A |- !x1::P1. ... !xn::Pn. t%}, provided none of the
variables {\small\verb%xi%} are free in any of the assumptions except in the corresponding
{\small\verb%Pi%}. It is not necessary that any or all of the {\small\verb%xi%} should be free in {\small\verb%t%}.
{\par\samepage\setseps\small
\begin{verbatim}
         A |- t
   ------------------------------  RESQ_GENL "[(x1,P1);...;(xn,Pn)]" 
    A |- !x1::P1. ... !xn::Pn. t   [where no xi is free in A except in Pi]
\end{verbatim}
}
\FAILURE
Fails unless all the terms {\small\verb%xi%} in the list are variables, none of which are
free in the assumption list except in {\small\verb%Pi%}.

\SEEALSO
RESQ_GEN, RESQ_GEN_ALL, RESQ_GEN_TAC, RESQ_SPEC, RESQ_SPECL, RESQ_SPEC_ALL.

\ENDDOC
\DOC{RESQ\_GEN\_ALL}

\TYPE {\small\verb%RESQ_GEN_ALL : (thm -> thm)%}\egroup

\SYNOPSIS
Generalizes the conclusion of a theorem over its own assumptions.

\DESCRIBE
When applied to a theorem {\small\verb%A |- t%}, the inference rule {\small\verb%RESQ_GEN_ALL%} returns
the theorem {\small\verb%A' |- !x1::P1. ...!xn::Pn. t%}, where the {\small\verb%Pi xi%} are in the 
assumptions.
{\par\samepage\setseps\small
\begin{verbatim}
         A |- t
   ------------------------------------------------  RESQ_GEN_ALL
   A - (P1 x1,...,Pn xn) |- !x1::P1. ... !xn::Pn. t
\end{verbatim}
}
\FAILURE
Never fails.

\SEEALSO
RESQ_GEN, RESQ_GENL, GEN_ALL, RESQ_SPEC, RESQ_SPECL, RESQ_SPEC_ALL.

\ENDDOC
\DOC{RESQ\_GEN\_TAC}

\TYPE {\small\verb%RESQ_GEN_TAC : tactic%}\egroup

\SYNOPSIS
Strips the outermost restricted universal quantifier from
the conclusion of a goal.

\DESCRIBE
When applied to a goal {\small\verb%A ?- !x::P. t%}, the tactic {\small\verb%RESQ_GEN_TAC%}
reduces it to a new goal {\small\verb%A,P x' ?- t[x'/x]%} where {\small\verb%x'%} is a variant of {\small\verb%x%}
chosen to avoid clashing with any variables free in the goal's
assumption list. Normally {\small\verb%x'%} is just {\small\verb%x%}.
{\par\samepage\setseps\small
\begin{verbatim}
     A ?- !x::P. t
   ===================  RESQ_GEN_TAC
    A,P x' ?- t[x'/x]
\end{verbatim}
}
\FAILURE
Fails unless the goal's conclusion is a restricted universal quantification.

\USES
The tactic {\small\verb%REPEAT RESQ_GEN_TAC%} strips away a series of restricted
universal quantifiers, and is commonly used before tactics relying on
the  underlying term structure. 

\SEEALSO
RESQ_HALF_GEN_TAC, RESQ_GEN, RESQ_GENL, RESQ_GEN_ALL, RESQ_SPEC, RESQ_SPECL,
RESQ_SPEC_ALL, GGEN_TAC, STRIP_TAC, GEN_TAC, X_GEN_TAC.

\ENDDOC
\DOC{RESQ\_HALF\_EXISTS}

\TYPE {\small\verb%RESQ_HALF_EXISTS : (thm -> thm)%}\egroup

\SYNOPSIS
Strip a restricted existential quantification from the conclusion of a theorem.

\DESCRIBE
When applied to a theorem {\small\verb%A |- ?x::P. t%}, {\small\verb%RESQ_HALF_EXISTS%} returns
the theorem
{\par\samepage\setseps\small
\begin{verbatim}
   A |- ?x. P x /\ t
\end{verbatim}
}
\noindent i.e., it transforms the restricted existential
quantification to its underlying semantic representation.
{\par\samepage\setseps\small
\begin{verbatim}
      A |- ?x::P. t
   --------------------  RESQ_HALF_EXISTS
    A |- ?x. P x ==> t
\end{verbatim}
}
\FAILURE
Fails if the theorem's conclusion is not a restricted existential quantification.

\SEEALSO
RESQ_EXISTS_TAC, EXISTS.

\ENDDOC
\DOC{RESQ\_HALF\_GEN\_TAC}

\TYPE {\small\verb%RESQ_HALF_GEN_TAC : tactic%}\egroup

\SYNOPSIS
Strips the outermost restricted universal quantifier from
the conclusion of a goal.

\DESCRIBE
When applied to a goal {\small\verb%A ?- !x::P. t%}, {\small\verb%RESQ_GEN_TAC%}
reduces it to {\small\verb%A ?- !x. P x ==> t%} which is the underlying semantic
representation of the restricted universal quantification.
{\par\samepage\setseps\small
\begin{verbatim}
     A ?- !x::P. t
   ====================  RESQ_HALF_GEN_TAC
    A ?- !x. P x ==> t
\end{verbatim}
}
\FAILURE
Fails unless the goal's conclusion is a restricted universal quantification.

\USES
The tactic {\small\verb%REPEAT RESQ_GEN_TAC%} strips away a series of restricted
universal quantifiers, and is commonly used before tactics relying on
the  underlying term structure. 

\SEEALSO
RESQ_GEN_TAC, RESQ_GEN, RESQ_GENL, RESQ_GEN_ALL, RESQ_SPEC, RESQ_SPECL,
RESQ_SPEC_ALL, GGEN_TAC, STRIP_TAC, GEN_TAC, X_GEN_TAC.

\ENDDOC
\DOC{RESQ\_HALF\_SPEC}

\TYPE {\small\verb%RESQ_HALF_SPEC : (thm -> thm)%}\egroup

\SYNOPSIS
Strip a restricted universal quantification in the conclusion of a theorem.

\DESCRIBE
When applied to a theorem {\small\verb%A |- !x::P. t%}, the derived inference rule
{\small\verb%RESQ_HALF_SPEC%} returns
the theorem {\small\verb%A |- !x. P x ==> t%}, i.e., it transforms the restricted universal
quantification to its underlying semantic representation.
{\par\samepage\setseps\small
\begin{verbatim}
      A |- !x::P. t
   --------------------  RESQ_HALF_SPEC
    A |- !x. P x ==> t
\end{verbatim}
}
\FAILURE
Fails if the theorem's conclusion is not a restricted universal quantification.

\SEEALSO
RESQ_SPEC, RESQ_SPECL, RESQ_SPEC_ALL, RESQ_GEN, RESQ_GENL, RESQ_GEN_ALL.

\ENDDOC
\DOC{RESQ\_IMP\_RES\_TAC}

\TYPE {\small\verb%RESQ_IMP_RES_TAC : thm_tactic%}\egroup

\SYNOPSIS
Repeatedly resolves a restricted universally quantified theorem with
the assumptions of a goal.

\DESCRIBE
The function {\small\verb%RESQ_IMP_RES_TAC%} performs repeatedly
resolution using a restricted quantified theorem.
It takes a restricted quantified theorem and transforms it into an
implication. This resulting theorem is used in the resolution.

Given a theorem {\small\verb%th%}, the theorem-tactic {\small\verb%RESQ_IMP_RES_TAC%}
applies {\small\verb%RESQ_IMP_RES_THEN%} repeatedly to resolve the theorem with the
assumptions. 

\FAILURE
Never fails

\SEEALSO
RESQ_IMP_RES_THEN, RESQ_RES_THEN, RESQ_RES_TAC,
IMP_RES_THEN, IMP_RES_TAC, MATCH_MP, RES_CANON, RES_TAC, RES_THEN.

\ENDDOC
\DOC{RESQ\_IMP\_RES\_THEN}

\TYPE {\small\verb%RESQ_IMP_RES_THEN : thm_tactical%}\egroup

\SYNOPSIS
Resolves a restricted universally quantified theorem with
the assumptions of a goal.

\DESCRIBE
The function {\small\verb%RESQ_IMP_RES_THEN%} is the basic building block for
resolution using a restricted quantified theorem.
It takes a restricted quantified theorem and transforms it into an
implication. This resulting theorem is used in the resolution.

Given a theorem-tactic {\small\verb%ttac%} and a theorem {\small\verb%th%}, the theorem-tactical
{\small\verb%RESQ_IMP_RES_THEN%} transforms the theorem into an implication {\small\verb%th'%}. It
then passes {\small\verb%th'%} together with {\small\verb%ttac%} to {\small\verb%IMP_RES_THEN%} to carry out
the resolution.

\FAILURE
Evaluating {\small\verb%RESQ_IMP_RES_THEN ttac th%} fails if the supplied
theorem {\small\verb%th%} is not restricted universally quantified, or if the call
to {\small\verb%IMP_RES_THEN%} fails.


\SEEALSO
RESQ_IMP_RES_TAC, RESQ_RES_THEN, RESQ_RES_TAC,
IMP_RES_THEN, IMP_RES_TAC, MATCH_MP, RES_CANON, RES_TAC, RES_THEN.

\ENDDOC
\DOC{RESQ\_MATCH\_MP}

\TYPE {\small\verb%RESQ_MATCH_MP : (thm -> thm -> thm)%}\egroup

\SYNOPSIS
Eliminating a restricted universal quantification with automatic matching.

\DESCRIBE
When applied to theorems {\small\verb%A1 |- !x::P. Q[x]%} and {\small\verb%A2 |- P x'%}, the
derived inference rule {\small\verb%RESQ_MATCH_MP%} matches {\small\verb%x'%} to {\small\verb%x%} by instantiating 
free or universally quantified variables in the first theorem (only),
and returns a theorem {\small\verb%A1 u A2 |- Q[x'/x]%}. Polymorphic types are also
instantiated if necessary.

{\par\samepage\setseps\small
\begin{verbatim}
    A1 |- !x::P.Q[x]   A2 |- P x'
   --------------------------------------  RESQ_MATCH_MP
          A1 u A2 |- Q[x'/x]
\end{verbatim}
}
\FAILURE
Fails unless the first theorem is a (possibly repeatedly) restricted
universal quantification whose quantified variable can be instantiated
to match the conclusion of the second theorem, without instantiating
any variables which are free in {\small\verb%A1%}, the first theorem's assumption list.

\SEEALSO
MATCH_MP, RESQ_HALF_SPEC.

\ENDDOC
\DOC{RESQ\_RES\_TAC}

\TYPE {\small\verb%RESQ_RES_TAC : tactic%}\egroup

\SYNOPSIS
Enriches assumptions by repeatedly resolving restricted universal
quantifications in them against the others.

\DESCRIBE
{\small\verb%RESQ_RES_TAC%} uses those assumptions which are restricted universal
quantifications in resolution in a way similar to {\small\verb%RES_TAC%}. It calls
{\small\verb%RESQ_RES_THEN%} repeatedly until there is no more resolution can be done.
The conclusions of all the new results are returned as additional
assumptions of the subgoal(s).  The effect of {\small\verb%RESQ_RES_TAC%} 
on a goal is to enrich the assumption set with some of its collective
consequences.


\FAILURE
{\small\verb%RESQ_RES_TAC%} cannot fail and so should not be unconditionally {\small\verb%REPEAT%}ed.

\SEEALSO
RESQ_IMP_RES_TAC, RESQ_IMP_RES_THEN, RESQ_RES_THEN,
IMP_RES_TAC, IMP_RES_THEN, RES_CANON, RES_THEN, RES_TAC.

\ENDDOC
\DOC{RESQ\_RES\_THEN}

\TYPE {\small\verb%RESQ_RES_THEN : thm_tactic -> tactic%}\egroup

\SYNOPSIS
Resolves all restricted universally quantified assumptions against
other assumptions of a goal.

\DESCRIBE
Like the function {\small\verb%RESQ_IMP_RES_THEN%}, the function {\small\verb%RESQ_RES_THEN%}
performs a single step resolution. The difference is that the
restricted universal quantification used in the resolution is taken
from the assumptions.

Given a theorem-tactic {\small\verb%ttac%}, applying the tactic {\small\verb%RESQ_RES_THEN
ttac%} to a goal {\small\verb%(asml,gl)%} has the effect of:
{\par\samepage\setseps\small
\begin{verbatim}
   MAP_EVERY (mapfilter ttac [... ; (ai,aj |- vi) ; ...]) (amsl ?- g)
\end{verbatim}
}
where the theorems {\small\verb%ai,aj |- vi%} are all the consequences that can be
drawn by a (single) matching modus-ponens inference from the
assumptions {\small\verb%amsl%} and the implications derived from the restricted
universal quantifications in the assumptions.

\FAILURE
Evaluating {\small\verb%RESQ_RES_TAC ttac th%} fails if there are no restricted
universal quantifications in the assumptions, or if the theorem-tactic
{\small\verb%ttac%} applied to all the consequences fails.


\SEEALSO
RESQ_IMP_RES_TAC, RESQ_IMP_RES_THEN, RESQ_RES_TAC,
IMP_RES_THEN, IMP_RES_TAC, MATCH_MP, RES_CANON, RES_TAC, RES_THEN.

\ENDDOC
\DOC{RESQ\_REWRITE1\_CONV}

\TYPE {\small\verb%RESQ_REWRITE1_CONV : thm list -> thm -> conv%}\egroup

\SYNOPSIS
Rewriting conversion using a restricted universally quantified theorem.

\DESCRIBE
{\small\verb%RESQ_REWRITE1_CONV%} is a rewriting conversion similar to
{\small\verb%COND_REWRITE1_CONV%}. The only difference is the rewriting theorem it
takes. This should be an equation with restricted universal
quantification at the outer level. It is converted to a theorem in the
form accepted by the conditional rewriting conversion.

Suppose that {\small\verb%th%} is the following theorem
{\par\samepage\setseps\small
\begin{verbatim}
   A |- !x::P. Q[x] = R[x])
\end{verbatim}
}
\noindent evaluating {\small\verb%RESQ_REWRITE1_CONV thms th "t[x']"%}
will return a theorem
{\par\samepage\setseps\small
\begin{verbatim}
   A, P x' |- t[x'] = t'[x']
\end{verbatim}
}
\noindent where {\small\verb%t'%} is the result of substituting instances of
{\small\verb%R[x'/x]%} for corresponding instances of {\small\verb%Q[x'/x]%} in the original
term {\small\verb%t[x]%}. All instances of {\small\verb%P x'%} which 
do not appear in the original assumption {\small\verb%asml%} are added to the assumption.
The theorems in the list {\small\verb%thms%} are used to eliminate the instances {\small\verb%P
x'%} if it is possible.

\FAILURE
{\small\verb%RESQ_REWRITE1_CONV%}  fails if {\small\verb%th%} cannot be transformed into the
required form by the function {\small\verb%RESQ_REWR_CANON%}. Otherwise, it fails if no 
match is found or the theorem cannot be instantiated.


\SEEALSO
RESQ_REWRITE1_TAC, RESQ_REWR_CANON, COND_REWR_TAC,
COND_REWRITE1_CONV, COND_REWR_CONV, COND_REWR_CANON, search_top_down.

\ENDDOC

\DOC{RESQ\_REWRITE1\_TAC}

\TYPE {\small\verb%RESQ_REWRITE1_TAC : thm_tactic%}\egroup

\SYNOPSIS
Rewriting with a restricted universally quantified theorem.

\DESCRIBE
{\small\verb%RESQ_REWRITE1_TAC%} takes an equational theorem which is restricted universally
quantified at the outer level. It calls {\small\verb%RESQ_REWR_CANON%} to convert
the theorem to the form accepted by {\small\verb%COND_REWR_TAC%} and passes the
resulting theorem to this tactic which carries out conditional
rewriting.

Suppose that {\small\verb%th%} is the following theorem
{\par\samepage\setseps\small
\begin{verbatim}
   A |- !x::P. Q[x] = R[x])
\end{verbatim}
}
\noindent Applying the tactic {\small\verb%RESQ_REWRITE1_TAC th%} to a goal {\small\verb%(asml,gl)%}
will return a main subgoal {\small\verb%(asml',gl')%} where {\small\verb%gl'%} is obtained by
substituting instances of {\small\verb%R[x'/x]%} for corresponding instances of
{\small\verb%Q[x'/x]%} in the original goal {\small\verb%gl%}. All instances of {\small\verb%P x'%} which
do not appear in the original assumption {\small\verb%asml%} are added to it to
form {\small\verb%asml'%}, and they also become new subgoals {\small\verb%(asml,P x')%}.

\FAILURE
{\small\verb%RESQ_REWRITE1_TAC th%}  fails if {\small\verb%th%} cannot be transformed into the
required form by the function {\small\verb%RESQ_REWR_CANON%}. Otherwise, it fails if no 
match is found or the theorem cannot be instantiated.


\SEEALSO
RESQ_REWRITE1_CONV, RESQ_REWR_CANON, COND_REWR_TAC,
COND_REWRITE1_CONV, COND_REWR_CONV, COND_REWR_CANON, search_top_down.

\ENDDOC

\DOC{RESQ\_REWR\_CANON}

\TYPE {\small\verb%RESQ_REWR_CANON : thm -> thm%}\egroup

\SYNOPSIS
Transform a theorem into a form accepted for rewriting.

\DESCRIBE
{\small\verb%RESQ_REWR_CANON%} transforms a theorem into a form accepted by {\small\verb%COND_REWR_TAC%}.
The input theorem should be headed by a series of restricted universal
quantifications in the following form
{\par\samepage\setseps\small
\begin{verbatim}
   !x1::P1. ... !xn::Pn. u[xi] = v[xi])
\end{verbatim}
}
\noindent Other variables occurring in {\small\verb%u%} and {\small\verb%v%} may be universally quantified.
The output theorem will have all ordinary universal quantifications
moved to the outer most level with possible renaming to prevent
variable capture, and have all restricted universal quantifications
converted to implications. The output theorem will be in the
form accepted by {\small\verb%COND_REWR_TAC%}.

\FAILURE
This function fails is the input theorem is not in the correct form.


\SEEALSO
RESQ_REWRITE1_TAC, RESQ_REWRITE1_CONV,
COND_REWR_CANON, COND_REWR_TAC, COND_REWR_CONV,.


\ENDDOC

\DOC{RESQ\_SPEC}

\TYPE {\small\verb%RESQ_SPEC : (term -> thm -> thm)%}\egroup

\SYNOPSIS
Specializes the conclusion of a restricted universally quantified theorem.

\DESCRIBE
When applied to a term {\small\verb%u%} and a theorem {\small\verb%A |- !x::P. t%}, {\small\verb%RESQ_SPEC%} returns
the theorem {\small\verb%A, P u |- t[u/x]%}. If necessary, variables will be renamed prior
to the specialization to ensure that {\small\verb%u%} is free for {\small\verb%x%} in {\small\verb%t%}, that is,
no variables free in {\small\verb%u%} become bound after substitution.
{\par\samepage\setseps\small
\begin{verbatim}
      A |- !x::P. t
   ------------------  RESQ_SPEC "u"
    A, P u |- t[u/x]
\end{verbatim}
}
\FAILURE
Fails if the theorem's conclusion is not restricted universally quantified,
or if type instantiation fails.

\EXAMPLE
The following example shows how {\small\verb%RESQ_SPEC%} renames bound variables if necessary,
prior to substitution: a straightforward substitution would result in the
clearly invalid theorem {\small\verb%(\y. 0 < y)y |- y = y%}.
{\par\samepage\setseps\small
\begin{verbatim}
   #let th = RESQ_GEN "x:num" "\y.0<y" (REFL "x:num");;
   th = |- !x :: \y. 0 < y. x = x

   #RESQ_SPEC "y:num" th;;
   (\y'. 0 < y')y |- y = y
\end{verbatim}
}
\SEEALSO
RESQ_SPECL, RESQ_SPEC_ALL, RESQ_GEN, RESQ_GENL, RESQ_GEN_ALL.

\ENDDOC
\DOC{RESQ\_SPECL}

\TYPE {\small\verb%RESQ_SPECL : (term list -> thm -> thm)%}\egroup

\SYNOPSIS
Specializes zero or more variables in the conclusion of a restricted
universally quantified theorem.

\DESCRIBE
When applied to a term list {\small\verb%[u1;...;un]%} and a theorem
{\small\verb%A |- !x1::P1. ... !xn::Pn. t%}, the inference rule {\small\verb%RESQ_SPECL%} returns
the theorem
{\par\samepage\setseps\small
\begin{verbatim}
   A,P1 u1,...,Pn un |- t[u1/x1]...[un/xn]
\end{verbatim}
}
\noindent where the substitutions are made
sequentially left-to-right in the same way as for {\small\verb%RESQ_SPEC%}, with the same
sort of alpha-conversions applied to {\small\verb%t%} if necessary to ensure that no
variables which are free in {\small\verb%ui%} become bound after substitution.
{\par\samepage\setseps\small
\begin{verbatim}
           A |- !x1::P1. ... !xn::Pn. t
   --------------------------------------------  RESQ_SPECL "[u1;...;un]"
     A,P1 u1, ..., Pn un |- t[u1/x1]...[un/xn]
\end{verbatim}
}
\noindent It is permissible for the term-list to be empty, in which case
the application of {\small\verb%RESQ_SPECL%} has no effect.

\FAILURE
Fails if one of the specialization of the 
restricted universally quantified variable in the original theorem fails.

\SEEALSO
RESQ_GEN, RESQ_GENL, RESQ_GEN_ALL, RESQ_GEN_TAC, RESQ_SPEC, RESQ_SPEC_ALL.

\ENDDOC
\DOC{RESQ\_SPEC\_ALL}

\TYPE {\small\verb%RESQ_SPEC_ALL : (thm -> thm)%}\egroup

\SYNOPSIS
Specializes the conclusion of a theorem with its own restricted
quantified variables.

\DESCRIBE
When applied to a theorem {\small\verb%A |- !x1::P1. ...!xn::Pn. t%},
the inference rule {\small\verb%RESQ_SPEC_ALL%}
returns the theorem {\small\verb%A,P1 x1',...,Pn xn' |- t[x1'/x1]...[xn'/xn]%}
 where the {\small\verb%xi'%} are distinct
variants of the corresponding {\small\verb%xi%}, chosen to avoid clashes with any variables
free in the assumption list and with the names of constants. Normally {\small\verb%xi'%} is
just {\small\verb%xi%}, in which case {\small\verb%RESQ_SPEC_ALL%} simply removes all restricted
 universal quantifiers.
{\par\samepage\setseps\small
\begin{verbatim}
       A |- !x1::P1. ... !xn::Pn. t
   -------------------------------------------  RESQ_SPEC_ALL
    A,P1 x1,...,Pn xn |- t[x1'/x1]...[xn'/xn]
\end{verbatim}
}

\FAILURE
Never fails.

\SEEALSO
RESQ_GEN, RESQ_GENL, RESQ_GEN_ALL, RESQ_GEN_TAC, RESQ_SPEC, RESQ_SPECL.

\ENDDOC
\DOC{search\_top\_down}

{\small
\begin{verbatim}
search_top_down
 : (term -> term -> ((term # term) list # (type # type) list) list)
\end{verbatim}
}\egroup

\SYNOPSIS
Search a term in a top-down fashion to find matches to another term.

\DESCRIBE
{\small\verb%search_top_down tm1 tm2%} returns a list of instantiations which make
the whole  or part of {\small\verb%tm2%} match {\small\verb%tm1%}. The first term should not have
a quantifier at the outer most level. {\small\verb%search_top_down%} first
attempts to match the whole second term to {\small\verb%tm1%}. If this fails, it
recursively descend into the subterms of {\small\verb%tm2%} to find all matches.

The length of the returned list indicates the number of matches found.
An empty list means no match can be found between {\small\verb%tm1%} and {\small\verb%tm2%} or
any subterms of {\small\verb%tm2%}.
The instantiations returned in the list are in the same format as for  the
function {\small\verb%match%}. Each instantiation is a pair of lists: the first is
a list of term pairs and the second is a list of type pairs. Either of
these lists may be empty. The situation in which both lists are empty
indicates that there is an exact match between the two terms, i.e., no
instantiation is required to make the entire {\small\verb%tm2%} or a part of {\small\verb%tm2%}
the same as {\small\verb%tm1%}.

\FAILURE
Never fails.

\EXAMPLE
{\par\samepage\setseps\small
\begin{verbatim}
   #search_top_down "x = y:*" "3 = 5";;
   [([("5", "y"); ("3", "x")], [(":num", ":*")])]
   : ((term # term) list # (type # type) list) list

   #search_top_down "x = y:*" "x =y:*";;
   [([], [])] : ((term # term) list # (type # type) list) list

   #search_top_down "x = y:*" "0 < p ==> (x <= p = y <= p)";;
   [([("y <= p", "y"); ("x <= p", "x")], [(":bool", ":*")])]
   : ((term # term) list # (type # type) list) list
\end{verbatim}
}
\noindent The first example above shows the entire {\small\verb%tm2%} matching {\small\verb%tm1%}.
The second example shows the two terms match exactly. No
instantiation is required. The last example shows that a subterm of
{\small\verb%tm2%} can be instantiated to match {\small\verb%tm1%}.

\SEEALSO
match, COND_REWR_TAC, CONV_REWRITE_TAC, COND_REWR_CONV, CONV_REWRITE_CONV.

\ENDDOC


\DOC{strip\_resq\_exists}

\TYPE {\small\verb%strip_resq_exists : (term -> ((term # term) list # term))%}\egroup

\SYNOPSIS
Iteratively breaks apart a restricted existentially quantified term.

\DESCRIBE
{\small\verb%strip_resq_exists%} is an iterative term destructor for restricted existential
quantifications. It iteratively breaks apart a restricted existentially
quantified term into a list of pairs which are the restricted quantified
variables and predicates and the body.
{\par\samepage\setseps\small
\begin{verbatim}
   strip_resq_exists "?x1::P1. ... ?xn::Pn. t"
\end{verbatim}
}
\noindent returns {\small\verb%([("x1","P1");...;("xn","Pn")],"t")%}.

\FAILURE
Never fails.

\SEEALSO
list_mk_resq_exists, is_resq_exists, dest_resq_exists.

\ENDDOC
\DOC{strip\_resq\_forall}

\TYPE {\small\verb%strip_resq_forall : (term -> ((term # term) list # term))%}\egroup

\SYNOPSIS
Iteratively breaks apart a restricted universally quantified term.

\DESCRIBE
{\small\verb%strip_resq_forall%} is an iterative term destructor for restricted universal
quantifications. It iteratively breaks apart a restricted universally
quantified term into a list of pairs which are the restricted quantified
variables and predicates and the body.
{\par\samepage\setseps\small
\begin{verbatim}
   strip_resq_forall "!x1::P1. ... !xn::Pn. t"
\end{verbatim}
}
\noindent returns {\small\verb%([("x1","P1");...;("xn","Pn")],"t")%}.

\FAILURE
Never fails.

\SEEALSO
list_mk_resq_forall, is_resq_forall, dest_resq_forall.

\ENDDOC
