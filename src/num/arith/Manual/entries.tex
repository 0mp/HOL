\chapter{ML Functions in the arith Library}
This chapter provides documentation on all the \ML\ functions that are made
available in \HOL\ when the \ml{taut} library is loaded.  This documentation
is also available online via the \ml{help} facility.



\DOC{ARITH\_CONV}{ARITH\_CONV}

\noindent{\small\verb%ARITH_CONV : conv%}\egroup

\SYNOPSIS
Partial decision procedure for a subset of linear natural number arithmetic.

\DESCRIBE
{\small\verb%ARITH_CONV%} is a partial decision procedure for Presburger natural
arithmetic. Presburger natural arithmetic is the subset of arithmetic formulae
made up from natural number constants, numeric variables, addition,
multiplication by a constant, the relations {\small\verb%<%}, {\small\verb%<=%}, {\small\verb%=%}, {\small\verb%>=%}, {\small\verb%>%} and the
logical connectives {\small\verb%~%}, {\small\verb%/\%}, {\small\verb%\/%}, {\small\verb%==>%}, {\small\verb%=%} (if-and-only-if), {\small\verb%!%} (`forall')
and {\small\verb%?%} (`there exists'). Products of two expressions which both contain
variables are not included in the subset, but the functions {\small\verb%SUC%} and {\small\verb%PRE%}
which are not normally included in a specification of Presburger arithmetic
are allowed in this HOL implementation.

{\small\verb%ARITH_CONV%} further restricts the subset as follows: when the formula has been
put in prenex normal form it must contain only one kind of quantifier, that is
the quantifiers must either all be universal (`forall') or all existential.
Variables may appear free (unquantified) provided any quantifiers that do
appear in the prenex normal form are universal; free variables are taken as
being implicitly universally quantified so mixing them with existential
quantifiers would violate the above restriction.

Given a formula in the permitted subset, {\small\verb%ARITH_CONV%} attempts to prove that
it is equal to {\small\verb%T%} (true). For universally quantified formulae the procedure
only works if the formula would also be true of the non-negative rationals;
it cannot prove formulae whose truth depends on the integral properties of the
natural numbers. The procedure is also incomplete for existentially quantified
formulae, but in this case there is no rule-of-thumb for determining whether
the procedure will work.

The function features a number of preprocessors which extend the coverage
beyond the subset specified above. In particular, natural number subtraction
and conditional statements are allowed. Another permits substitution instances
of universally quantified formulae to be accepted. Note that Boolean-valued
variables are not allowed.

\FAILURE
The function can fail in two ways. It fails if the argument term is not a
formula in the specified subset, and it also fails if it is unable to prove
the formula. The failure strings are different in each case. However, the
function may announce that it is unable to prove a formula that one would
expect it to reject as being outside the subset. This is due to it looking for
substitution instances; it has generalised the formula so that the new formula
is in the subset but is not valid.

\EXAMPLE
A simple example containing a free variable:
{\par\samepage\setseps\small
\begin{verbatim}
   - ARITH_CONV ``m < SUC m``;
   > val it = |- m < (SUC m) = T : thm
\end{verbatim}}\noindent 
A more complex example with subtraction and universal quantifiers, and
which is not initially in prenex normal form:
{\par\samepage\setseps\small
\begin{verbatim}
   #ARITH_CONV
   # "!m p. p < m ==> !q r. (m < (p + q) + r) ==> ((m - p) < q + r)";;
   |- (!m p. p < m ==> (!q r. m < ((p + q) + r) ==> (m - p) < (q + r))) = T
\end{verbatim}}\noindent 
Two examples with existential quantifiers:
{\par\samepage\setseps\small
\begin{verbatim}
   #ARITH_CONV "?m n. m < n";;
   |- (?m n. m < n) = T

   #ARITH_CONV "?m n. (2 * m) + (3 * n) = 10";;
   |- (?m n. (2 * m) + (3 * n) = 10) = T
\end{verbatim}}\noindent 
An instance of a universally quantified formula involving a conditional
statement and subtraction:
{\par\samepage\setseps\small
\begin{verbatim}
   #ARITH_CONV
   # "((p + 3) <= n) ==> (!m. ((m EXP 2 = 0) => (n - 1) | (n - 2)) > p)";;
   |- (p + 3) <= n ==> (!m. ((m EXP 2 = 0) => n - 1 | n - 2) > p) = T
\end{verbatim}}\noindent 
Failure due to mixing quantifiers:
{\par\samepage\setseps\small
\begin{verbatim}
   #ARITH_CONV "!m. ?n. m < n";;
   evaluation failed     ARITH_CONV -- formula not in the allowed subset
\end{verbatim}}\noindent 
Failure because the truth of the formula relies on the fact that the
variables cannot have fractional values:
{\par\samepage\setseps\small
\begin{verbatim}
   #ARITH_CONV "!m n. ~(SUC (2 * m) = 2 * n)";;
   evaluation failed     ARITH_CONV -- cannot prove formula
\end{verbatim}}

\SEEALSO
NEGATE\_CONV, EXISTS\_ARITH\_CONV, FORALL\_ARITH\_CONV, INSTANCE\_T\_CONV, PRENEX\_CONV, SUB\_AND\_COND\_ELIM\_CONV.

\ENDDOC

\DOC{ARITH\_FORM\_NORM\_CONV}{ARITH\_FORM\_NORM\_CONV}

\noindent{\small\verb%ARITH_FORM_NORM_CONV : conv%}\egroup

\SYNOPSIS
Normalises an unquantified formula of linear natural number arithmetic.

\DESCRIBE
{\small\verb%ARITH_FORM_NORM_CONV%} converts a formula of natural number arithmetic into a
disjunction of conjunctions of less-than-or-equal-to inequalities. The
arithmetic expressions are only allowed to contain natural number constants,
numeric variables, addition, the {\small\verb%SUC%} function, and multiplication by a
constant. The formula must not contain quantifiers, but may have disjunction,
conjunction, negation, implication, equality on Booleans (if-and-only-if), and
the natural number relations: {\small\verb%<%}, {\small\verb%<=%}, {\small\verb%=%}, {\small\verb%>=%}, {\small\verb%>%}. The formula must not
contain products of two expressions which both contain variables.

The inequalities in the result are normalised so that each variable appears on
only one side of the inequality, and each side is a linear sum in which any
constant appears first followed by products of a constant and a variable. The
variables are ordered lexicographically, and if the coefficient of the
variable is {\small\verb%1%}, the product of {\small\verb%1%} and the variable appears in the term
rather than the variable on its own.

\FAILURE
The function fails if the argument term is not a formula in the specified
subset.

\EXAMPLE
{\par\samepage\setseps\small
\begin{verbatim}
#ARITH_FORM_NORM_CONV "m < n";;
|- m < n = (1 + (1 * m)) <= (1 * n)

#ARITH_FORM_NORM_CONV
# "(n < 4) ==> ((n = 0) \/ (n = 1) \/ (n = 2) \/ (n = 3))";;
|- n < 4 ==> (n = 0) \/ (n = 1) \/ (n = 2) \/ (n = 3) =
   4 <= (1 * n) \/
   (1 * n) <= 0 /\ 0 <= (1 * n) \/
   (1 * n) <= 1 /\ 1 <= (1 * n) \/
   (1 * n) <= 2 /\ 2 <= (1 * n) \/
   (1 * n) <= 3 /\ 3 <= (1 * n)
\end{verbatim}}

\USES
Useful in constructing decision procedures for linear arithmetic.

\ENDDOC

\DOC{COND\_ELIM\_CONV}{COND\_ELIM\_CONV}

\noindent{\small\verb%COND_ELIM_CONV : conv%}\egroup

\SYNOPSIS
Eliminates conditional statements from a formula.

\DESCRIBE
This function moves conditional statements up through a term and if at any
point the branches of the conditional become Boolean-valued the conditional is
eliminated. If the term is a formula, only an abstraction can prevent a
conditional being moved up far enough to be eliminated.

\FAILURE
Never fails.

\EXAMPLE
{\par\samepage\setseps\small
\begin{verbatim}
#COND_ELIM_CONV "!f n. f ((SUC n = 0) => 0 | (SUC n - 1)) < (f n) + 1";;
|- (!f n. (f((SUC n = 0) => 0 | (SUC n) - 1)) < ((f n) + 1)) =
   (!f n.
     (~(SUC n = 0) \/ (f 0) < ((f n) + 1)) /\
     ((SUC n = 0) \/ (f((SUC n) - 1)) < ((f n) + 1)))

#COND_ELIM_CONV "!f n. (\m. f ((m = 0) => 0 | (m - 1))) (SUC n) < (f n) + 1";;
|- (!f n. ((\m. f((m = 0) => 0 | m - 1))(SUC n)) < ((f n) + 1)) =
   (!f n. ((\m. ((m = 0) => f 0 | f(m - 1)))(SUC n)) < ((f n) + 1))
\end{verbatim}}

\USES
Useful as a preprocessor to decision procedures which do not allow conditional
statements in their argument formula.

\SEEALSO
SUB\_AND\_COND\_ELIM\_CONV.

\ENDDOC

\DOC{DISJ\_INEQS\_FALSE\_CONV}{DISJ\_INEQS\_FALSE\_CONV}

\noindent{\small\verb%DISJ_INEQS_FALSE_CONV : conv%}\egroup

\SYNOPSIS
Proves a disjunction of conjunctions of normalised inequalities is false,
provided each conjunction is unsatisfiable.

\DESCRIBE
{\small\verb%DISJ_INEQS_FALSE_CONV%} converts an unsatisfiable normalised arithmetic
formula to false. The formula must be a disjunction of conjunctions of
less-than-or-equal-to inequalities. The inequalities must have the following
form: Each variable must appear on only one side of the inequality and each
side must be a linear sum in which any constant appears first followed by
products of a constant and a variable. On each side the variables must be
ordered lexicographically, and if the coefficient of the variable is {\small\verb%1%}, the
{\small\verb%1%} must appear explicitly.

\FAILURE
Fails if the formula is not of the correct form or is satisfiable. The
function will also fail on certain unsatisfiable formulae due to
incompleteness of the procedure used.

\EXAMPLE
{\par\samepage\setseps\small
\begin{verbatim}
#DISJ_INEQS_FALSE_CONV
# "(1 * n) <= ((1 * m) + (1 * p)) /\
#  ((1 * m) + (1 * p)) <= (1 * n) /\
#  (5 + (4 * n)) <= ((3 * m) + (1 * p)) \/
#  2 <= 0";;
|- (1 * n) <= ((1 * m) + (1 * p)) /\
   ((1 * m) + (1 * p)) <= (1 * n) /\
   (5 + (4 * n)) <= ((3 * m) + (1 * p)) \/
   2 <= 0 =
   F
\end{verbatim}}

\SEEALSO
ARITH\_FORM\_NORM\_CONV.

\ENDDOC

\DOC{EXISTS\_ARITH\_CONV}{EXISTS\_ARITH\_CONV}

\noindent{\small\verb%EXISTS_ARITH_CONV : conv%}\egroup

\SYNOPSIS
Partial decision procedure for non-universal Presburger natural arithmetic.

\DESCRIBE
{\small\verb%EXISTS_ARITH_CONV%} is a partial decision procedure for formulae of Presburger
natural arithmetic which are in prenex normal form and have all variables
existentially quantified. Presburger natural arithmetic is the subset of
arithmetic formulae made up from natural number constants, numeric variables,
addition, multiplication by a constant, the relations {\small\verb%<%}, {\small\verb%<=%}, {\small\verb%=%}, {\small\verb%>=%}, {\small\verb%>%}
and the logical connectives {\small\verb%~%}, {\small\verb%/\%}, {\small\verb%\/%}, {\small\verb%==>%}, {\small\verb%=%} (if-and-only-if),
{\small\verb%!%} (`forall') and {\small\verb%?%} (`there exists'). Products of two expressions which
both contain variables are not included in the subset, but the function {\small\verb%SUC%}
which is not normally included in a specification of Presburger arithmetic is
allowed in this HOL implementation.

Given a formula in the specified subset, the function attempts to prove that
it is equal to {\small\verb%T%} (true). The procedure is incomplete; it is not able to
prove all formulae in the subset.

\FAILURE
The function can fail in two ways. It fails if the argument term is not a
formula in the specified subset, and it also fails if it is unable to prove
the formula. The failure strings are different in each case.

\EXAMPLE
{\par\samepage\setseps\small
\begin{verbatim}
#EXISTS_ARITH_CONV "?m n. m < n";;
|- (?m n. m < n) = T

#EXISTS_ARITH_CONV "?m n. (2 * m) + (3 * n) = 10";;
|- (?m n. (2 * m) + (3 * n) = 10) = T
\end{verbatim}}

\SEEALSO
NEGATE\_CONV, FORALL\_ARITH\_CONV, ARITH\_CONV.

\ENDDOC

\DOC{FORALL\_ARITH\_CONV}{FORALL\_ARITH\_CONV}

\noindent{\small\verb%FORALL_ARITH_CONV : conv%}\egroup

\SYNOPSIS
Partial decision procedure for non-existential Presburger natural arithmetic.

\DESCRIBE
{\small\verb%FORALL_ARITH_CONV%} is a partial decision procedure for formulae of Presburger
natural arithmetic which are in prenex normal form and have all variables
either free or universally quantified. Presburger natural arithmetic is the
subset of arithmetic formulae made up from natural number constants, numeric
variables, addition, multiplication by a constant, the relations {\small\verb%<%}, {\small\verb%<=%},
{\small\verb%=%}, {\small\verb%>=%}, {\small\verb%>%} and the logical connectives {\small\verb%~%}, {\small\verb%/\%}, {\small\verb%\/%}, {\small\verb%==>%},
{\small\verb%=%} (if-and-only-if), {\small\verb%!%} (`forall') and {\small\verb%?%} (`there exists'). Products of two
expressions which both contain variables are not included in the subset, but
the function {\small\verb%SUC%} which is not normally included in a specification of
Presburger arithmetic is allowed in this HOL implementation.

Given a formula in the specified subset, the function attempts to prove that
it is equal to {\small\verb%T%} (true). The procedure only works if the formula would also
be true of the non-negative rationals; it cannot prove formulae whose truth
depends on the integral properties of the natural numbers.

\FAILURE
The function can fail in two ways. It fails if the argument term is not a
formula in the specified subset, and it also fails if it is unable to prove
the formula. The failure strings are different in each case.

\EXAMPLE
{\par\samepage\setseps\small
\begin{verbatim}
#FORALL_ARITH_CONV "m < SUC m";;
|- m < (SUC m) = T

#FORALL_ARITH_CONV "!m n p q. m <= p /\ n <= q ==> (m + n) <= (p + q)";;
|- (!m n p q. m <= p /\ n <= q ==> (m + n) <= (p + q)) = T

#FORALL_ARITH_CONV "!m n. ~(SUC (2 * m) = 2 * n)";;
evaluation failed     FORALL_ARITH_CONV -- cannot prove formula
\end{verbatim}}

\SEEALSO
NEGATE\_CONV, EXISTS\_ARITH\_CONV, ARITH\_CONV, ARITH\_FORM\_NORM\_CONV, DISJ\_INEQS\_FALSE\_CONV.

\ENDDOC

\DOC{INSTANCE\_T\_CONV}{INSTANCE\_T\_CONV}

\noindent{\small\verb%INSTANCE_T_CONV : ((term -> term list) -> conv -> conv)%}\egroup

\SYNOPSIS
Function which allows a proof procedure to work on substitution instances of
terms that are in the domain of the procedure.

\DESCRIBE
This function generalises a conversion that is used to prove formulae true. It
does this by first replacing any syntactically unacceptable subterms with
variables. It then attempts to prove the resulting generalised formula and if
successful it re-instantiates the variables.

The first argument should be a function which computes a list of subterms of a
term which are syntactically unacceptable to the proof procedure. This
function should include in its result any variables that do not appear in
other subterms returned. The second argument is the proof procedure to be
generalised; this should be a conversion which when successful returns an
equation between the argument formula and {\small\verb%T%} (true).

\FAILURE
Fails if either of the applications of the argument functions fail, or if the
conversion does not return an equation of the correct form.

\EXAMPLE
{\par\samepage\setseps\small
\begin{verbatim}
#FORALL_ARITH_CONV "!f m (n:num). (m < (f n)) ==> (m <= (f n))";;
evaluation failed     FORALL_ARITH_CONV -- formula not in the allowed subset

#INSTANCE_T_CONV non_presburger_subterms FORALL_ARITH_CONV
# "!f m (n:num). (m < (f n)) ==> (m <= (f n))";;
|- (!f m n. m < (f n) ==> m <= (f n)) = T
\end{verbatim}}

\ENDDOC

\DOC{is\_prenex}{is\_prenex}

\noindent{\small\verb%is_prenex : (term -> bool)%}\egroup

\SYNOPSIS
Determines whether a formula is in prenex normal form.

\DESCRIBE
This function returns true if the term it is given as argument is in prenex
normal form. If the term is not a formula the result will be true provided
there are no nested Boolean expressions involving quantifiers.

\FAILURE
Never fails.

\EXAMPLE
{\par\samepage\setseps\small
\begin{verbatim}
#is_prenex "!x. ?y. x \/ y";;
true : bool

#is_prenex "!x. x ==> (?y. x /\ y)";;
false : bool
\end{verbatim}}

\USES
Useful for determining whether it is necessary to apply a prenex normaliser to
a formula before passing it to a function which requires the formula to be in
prenex normal form.

\SEEALSO
PRENEX\_CONV.

\ENDDOC

\DOC{is\_presburger}{is\_presburger}

\noindent{\small\verb%is_presburger : (term -> bool)%}\egroup

\SYNOPSIS
Determines whether a formula is in the Presburger subset of arithmetic.

\DESCRIBE
This function returns true if the argument term is a formula in the Presburger
subset of natural number arithmetic. Presburger natural arithmetic is the
subset of arithmetic formulae made up from natural number constants, numeric
variables, addition, multiplication by a constant, the natural number
relations {\small\verb%<%}, {\small\verb%<=%}, {\small\verb%=%}, {\small\verb%>=%}, {\small\verb%>%} and the logical connectives {\small\verb%~%}, {\small\verb%/\%},
{\small\verb%\/%}, {\small\verb%==>%}, {\small\verb%=%} (if-and-only-if), {\small\verb%!%} (`forall') and {\small\verb%?%} (`there exists').

Products of two expressions which both contain variables are not included in
the subset, but the function {\small\verb%SUC%} which is not normally included in a
specification of Presburger arithmetic is allowed in this HOL implementation.
This function also considers subtraction and the predecessor function, {\small\verb%PRE%},
to be part of the subset.

\FAILURE
Never fails.

\EXAMPLE
{\par\samepage\setseps\small
\begin{verbatim}
#is_presburger "!m n p. m < (2 * n) /\ (n + n) <= p ==> m < SUC p";;
true : bool

#is_presburger "!m n p q. m < (n * p) /\ (n * p) < q ==> m < q";;
false : bool

#is_presburger "(m <= n) ==> !p. (m < SUC(n + p))";;
true : bool

#is_presburger "(m + n) - m = n";;
true : bool
\end{verbatim}}

\USES
Useful for determining whether a decision procedure for Presburger arithmetic
is applicable to a term.

\SEEALSO
non\_presburger\_subterms, FORALL\_ARITH\_CONV, EXISTS\_ARITH\_CONV, is\_prenex.

\ENDDOC

\DOC{NEGATE\_CONV}{NEGATE\_CONV}

\noindent{\small\verb%NEGATE_CONV : (conv -> conv)%}\egroup

\SYNOPSIS
Function for negating the operation of a conversion that proves a formula to
be either true or false.

\DESCRIBE
This function negates the operation of a conversion that proves a formula to
be either true or false. For example, if {\small\verb%conv%} proves {\small\verb%"t"%} to be equal to
{\small\verb%"T"%} then {\small\verb%NEGATE_CONV conv%} will prove {\small\verb%"~t"%} to be {\small\verb%"F"%}.

\FAILURE
Fails if the application of the conversion to the negation of the formula does
not yield either {\small\verb%"T"%} or {\small\verb%"F"%}.

\EXAMPLE
{\par\samepage\setseps\small
\begin{verbatim}
#ARITH_CONV "!n. 0 <= n";;
|- (!n. 0 <= n) = T

#NEGATE_CONV ARITH_CONV "~(!n. 0 <= n)";;
|- ~(!n. 0 <= n) = F

#NEGATE_CONV ARITH_CONV "?n. ~(0 <= n)";;
|- (?n. ~0 <= n) = F
\end{verbatim}}

\ENDDOC

\DOC{non\_presburger\_subterms}{non\_presburger\_subterms}

\noindent{\small\verb%non_presburger_subterms : (term -> term list)%}\egroup

\SYNOPSIS
Computes the subterms of a term that are not in the Presburger subset of
arithmetic.

\DESCRIBE
This function computes a list of subterms of a term that are not in the
Presburger subset of natural number arithmetic. All numeric variables in the
term are included in the result. Presburger natural arithmetic is the subset
of arithmetic formulae made up from natural number constants, numeric
variables, addition, multiplication by a constant, the natural number
relations {\small\verb%<%}, {\small\verb%<=%}, {\small\verb%=%}, {\small\verb%>=%}, {\small\verb%>%} and the logical connectives {\small\verb%~%}, {\small\verb%/\%},
{\small\verb%\/%}, {\small\verb%==>%}, {\small\verb%=%} (if-and-only-if), {\small\verb%!%} (`forall') and {\small\verb%?%} (`there exists').

Products of two expressions which both contain variables are not included in
the subset, so such products will appear in the result list. However, the
function {\small\verb%SUC%} which is not normally included in a specification of Presburger
arithmetic is allowed in this HOL implementation. This function also considers
subtraction and the predecessor function, {\small\verb%PRE%}, to be part of the subset.

\FAILURE
Never fails.

\EXAMPLE
{\par\samepage\setseps\small
\begin{verbatim}
#non_presburger_subterms "!m n p. m < (2 * n) /\ (n + n) <= p ==> m < SUC p";;
["m"; "n"; "p"] : term list

#non_presburger_subterms "!m n p q. m < (n * p) /\ (n * p) < q ==> m < q";;
["m"; "n * p"; "q"] : term list

#non_presburger_subterms "(m + n) - m = f n";;
["m"; "n"; "f n"] : term list
\end{verbatim}}

\SEEALSO
INSTANCE\_T\_CONV, is\_presburger.

\ENDDOC

\DOC{PRENEX\_CONV}{PRENEX\_CONV}

\noindent{\small\verb%PRENEX_CONV : conv%}\egroup

\SYNOPSIS
Puts a formula into prenex normal form.

\DESCRIBE
This function puts a formula into prenex normal form, and in the process splits
any Boolean equalities (if-and-only-if) into two implications. If there is a
Boolean-valued subterm present as the condition of a conditional, the subterm
will be put in prenex normal form, but quantifiers will not be moved out of the
condition. Some renaming of variables may take place.

\FAILURE
Never fails.

\EXAMPLE
{\par\samepage\setseps\small
\begin{verbatim}
#PRENEX_CONV "!m n. (m <= n) ==> !p. (m < SUC(n + p))";;
|- (!m n. m <= n ==> (!p. m < (SUC(n + p)))) =
   (!m n p. m <= n ==> m < (SUC(n + p)))

#PRENEX_CONV "!p. (!m. m >= p) = (p = 0)";;
|- (!p. (!m. m >= p) = (p = 0)) =
   (!p m. ?m'. (m' >= p ==> (p = 0)) /\ ((p = 0) ==> m >= p))

#PRENEX_CONV "!m. (((m = 0) ==> (!n. m <= n)) => 0 | m) + m = m";;
|- (!m. (((m = 0) ==> (!n. m <= n)) => 0 | m) + m = m) =
   (!m. ((!n. (m = 0) ==> m <= n) => 0 | m) + m = m)
\end{verbatim}}

\USES
Useful as a preprocessor to decision procedures which require their argument
formula to be in prenex normal form.

\SEEALSO
is\_prenex.

\ENDDOC

\DOC{SUB\_AND\_COND\_ELIM\_CONV}{SUB\_AND\_COND\_ELIM\_CONV}

\noindent{\small\verb%SUB_AND_COND_ELIM_CONV : conv%}\egroup

\SYNOPSIS
Eliminates natural number subtraction, PRE, and conditional statements from a
formula.

\DESCRIBE
This function eliminates natural number subtraction and the predecessor
function, {\small\verb%PRE%}, from a formula, but in doing so may generate conditional
statements, so these are eliminated too. The conditional statements are moved
up through the term and if at any point the branches of the conditional become
Boolean-valued the conditional is eliminated. Subtraction operators are moved
up until a relation (such as less-than) is reached. The subtraction can then
be transformed into an addition. Provided the argument term is a formula, only
an abstraction can prevent a conditional being moved up far enough to be
eliminated. If the term is not a formula it may not be possible to eliminate
the subtraction. The function is also incapable of eliminating subtractions
that appear in arguments to functions other than the standard operators of
arithmetic.

The function is not as delicate as it could be; it tries to eliminate all
conditionals in a formula when it need only eliminate those that have to be
removed in order to eliminate subtraction.

\FAILURE
Never fails.

\EXAMPLE
{\par\samepage\setseps\small
\begin{verbatim}
#SUB_AND_COND_ELIM_CONV
# "((p + 3) <= n) ==> (!m. ((m = 0) => (n - 1) | (n - 2)) > p)";;
|- (p + 3) <= n ==> (!m. ((m = 0) => n - 1 | n - 2) > p) =
   (p + 3) <= n ==>
   (!m. (~(m = 0) \/ n > (1 + p)) /\ ((m = 0) \/ n > (2 + p)))

#SUB_AND_COND_ELIM_CONV
# "!f n. f ((SUC n = 0) => 0 | (SUC n - 1)) < (f n) + 1";;
|- (!f n. (f((SUC n = 0) => 0 | (SUC n) - 1)) < ((f n) + 1)) =
   (!f n.
     (~(SUC n = 0) \/ (f 0) < ((f n) + 1)) /\
     ((SUC n = 0) \/ (f((SUC n) - 1)) < ((f n) + 1)))

#SUB_AND_COND_ELIM_CONV
# "!f n. (\m. f ((m = 0) => 0 | (m - 1))) (SUC n) < (f n) + 1";;
|- (!f n. ((\m. f((m = 0) => 0 | m - 1))(SUC n)) < ((f n) + 1)) =
   (!f n. ((\m. ((m = 0) => f 0 | f(m - 1)))(SUC n)) < ((f n) + 1))
\end{verbatim}}

\USES
Useful as a preprocessor to decision procedures which do not allow natural
number subtraction in their argument formula.

\SEEALSO
COND\_ELIM\_CONV.

\ENDDOC

