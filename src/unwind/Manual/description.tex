\chapter{The unwind Library}

This document describes the facilities provided by the \ml{unwind} library
for the HOL system~\cite{description}. The library provides conversions and
rules for unfolding, unwinding and pruning device implementations (logical
representations of hardware). For a detailed description of these techniques,
see~\cite{HVusingHOL}.

Most of the functions fall into one of five groups. The first group consists
of conversions and inference rules for moving universal quantifiers up and
down through conjunctions; they have names beginning with either
\ml{CONJ\_FORALL} or \ml{FORALL\_CONJ}. The second group of functions are for
unfolding, that is expanding sub-components using their definitions. The names
of these begin with \ml{UNFOLD}. The functions in the third group perform
unwinding and have names beginning with \ml{UNWIND}. The fourth group of
functions prune internal lines that have been unwound. Their names begin with
\ml{PRUNE}. The final group of functions combine unfolding, unwinding and
pruning. They have names beginning with \ml{EXPAND}.

I have tried to make the behaviour of the functions uniform. The conversions
apply to the smallest term possible, to provide maximum flexibility. The
inference rules, on the other hand, are designed to apply to the definition
of a hardware component. They expect to be given a theorem of the form:

\begin{small}\begin{verbatim}
   |- !x1 ... xn. DEVICE (x1,...,xn) = ?l1 ... lm. t1 /\ ... /\ tp
\end{verbatim}\end{small}


\section{Using the library}

The \ml{unwind} library can be loaded into a \HOL\ session using the function
\ml{load\_library}\index{load\_library@{\ptt load\_library}} (see the \HOL\
manual for a general description of library loading). The first action in the
load sequence initiated by \ml{load\_library} is to update the \HOL\
help\index{help!updating search path} search path. The help search path is
updated with a pathname to online help files for the \ML\ functions in the
library. After updating the help search path, the \ML\ functions in the
library are loaded into \HOL.

The following session shows how the \ml{unwind} library may be loaded using
\ml{load\_library}:

\setcounter{sessioncount}{1}
\begin{session}\begin{verbatim}
#load_library `unwind`;;
Loading library `unwind` ...
Updating help search path
..................................
Library `unwind` loaded.
() : void
\end{verbatim}\end{session}

We now illustrate the use of the library on the parity-checker example.
Firstly, we begin a new theory:

\begin{session}\begin{verbatim}
#new_theory `PARITY`;;
() : void
\end{verbatim}\end{session}

\vfill

\noindent
We define the sub-components used:

\vfill

\begin{session}\begin{verbatim}
#let ONE_DEF =
# new_definition
#  (`ONE_DEF`, "ONE(out:num->bool) = !t. out t = T");;
ONE_DEF = |- !out. ONE out = (!t. out t = T)
\end{verbatim}\end{session}

\vfill

\begin{session}\begin{verbatim}
#let NOT_DEF =
# new_definition
#  (`NOT_DEF`, "NOT(in,out:num->bool) = !t. out t = ~(in t)");;
NOT_DEF = |- !in out. NOT(in,out) = (!t. out t = ~in t)
\end{verbatim}\end{session}

\vfill

\begin{session}\begin{verbatim}
#let MUX_DEF =
# new_definition
#  (`MUX_DEF`,
#   "MUX(sw,in1,in2,out:num->bool) =
#     !t. out t = (sw t => in1 t | in2 t)");;
MUX_DEF = 
|- !sw in1 in2 out.
    MUX(sw,in1,in2,out) = (!t. out t = (sw t => in1 t | in2 t))
\end{verbatim}\end{session}

\vfill

\begin{session}\begin{verbatim}
#let REG_DEF =
# new_definition
# (`REG_DEF`, "REG(in,out:num->bool) =
#              !t. out t = ((t=0) => F | in(t-1))");;
REG_DEF = 
|- !in out. REG(in,out) = (!t. out t = ((t = 0) => F | in(t - 1)))
\end{verbatim}\end{session}

\vfill

\noindent
Now we define the parity-checker implementation:

\begin{session}\begin{verbatim}
#let PARITY_IMP_DEF =
# new_definition
#  (`PARITY_IMP_DEF`,
#   "PARITY_IMP(in,out) =
#    ?l1 l2 l3 l4 l5.
#     NOT(l2,l1) /\ MUX(in,l1,l2,l3) /\ REG(out,l2) /\
#     ONE l4     /\ REG(l4,l5)       /\ MUX(l5,l3,l4,out)");;
PARITY_IMP_DEF = 
|- !in out.
    PARITY_IMP(in,out) =
    (?l1 l2 l3 l4 l5.
      NOT(l2,l1) /\
      MUX(in,l1,l2,l3) /\
      REG(out,l2) /\
      ONE l4 /\
      REG(l4,l5) /\
      MUX(l5,l3,l4,out))
\end{verbatim}\end{session}

\noindent
The function \ml{EXPAND\_AUTO\_RIGHT\_RULE} can be used to unfold, unwind and
prune the body of this definition:

\begin{session}\begin{verbatim}
#EXPAND_AUTO_RIGHT_RULE [ONE_DEF;NOT_DEF;MUX_DEF;REG_DEF] PARITY_IMP_DEF;;
|- !in out.
    PARITY_IMP(in,out) =
    (!t.
      out t =
      (((t = 0) => F | T) => 
       (in t => 
        ~((t = 0) => F | out(t - 1)) | 
        ((t = 0) => F | out(t - 1))) | 
       T))
\end{verbatim}\end{session}


\section{Automatic unwinding}

\def\putbox(#1,#2){\put(#1,#2){\framebox(2,2){}}}

Hardware implementations often contain feedbacks. This presents a problem
when trying to unwind and prune the internal lines in the logical
representation. The mutual dependencies between lines can cause a brute-force
unwind to loop indefinitely. To avoid this one has to be selective about
which lines to unwind. The tools in the \ml{unwind} library allow the user
to be selective in this way. However, it is possible for the machine itself to
be selective. The function \ml{UNWIND\_AUTO\_CONV} attempts to analyze the
dependencies between lines and unwind as far as possible without looping.

Consider the following term which arises in the parity-checker example:

\begin{small}\begin{verbatim}
   "?l1 l2 l3 l4 l5.
     (!t. l1 (t:num) = ~l2 t) /\
     (!t. l3 t = (in t => l1 t | l2 t)) /\
     (!t. l2 t = ((t = 0) => F | out (t - 1))) /\
     (!t. l4 t = T) /\
     (!t. l5 t = ((t = 0) => F | l4 (t - 1))) /\
     (!t. out t = (l5 t => l3 t | l4 t))"
\end{verbatim}\end{small}

\noindent
We can represent the dependencies of the lines using a directed graph:

{\setlength{\unitlength}{4mm}
\begin{center}
\begin{picture}(14,10)(0,0)
\put(0,2){\makebox(2,2){\small{\tt l2}}}
\put(4,2){\makebox(2,2){\small{\tt l1}}}
\put(8,4){\makebox(2,2){\small{\tt l3}}}
\put(12,6){\makebox(2,2){\small{\tt out}}}
\put(2,8){\makebox(2,2){\small{\tt l4}}}
\put(6,8){\makebox(2,2){\small{\tt l5}}}

\put(2,3){\vector(1,0){2}}
\put(2,3){\vector(3,1){6}}
\put(6,3){\vector(1,1){2}}
\put(10,5){\vector(1,1){2}}
\put(4,9){\vector(1,0){2}}
\put(4,9){\vector(4,-1){8}}
\put(8,9){\vector(2,-1){4}}

\put(13,6){\line(0,-1){6}}
\put(13,0){\line(-1,0){12}}
\put(1,0){\vector(0,1){2}}
\end{picture}
\end{center}}

\noindent
which can in turn be represented by the following list:

\begin{small}\begin{verbatim}
   l1, [l2]
   l3, [l1;l2]
   l2, [out]
   l4, []
   l5, [l4]
   out,[l5;l3;l4]
\end{verbatim}\end{small}

Since we wish to eliminate the internal lines, we want to be left with a
recursive equation for {\small\verb%out%} in terms of itself. We can do this
be `breaking the loop' at {\small\verb%out%}, giving the following structure:

\begin{small}\begin{verbatim}
   l1, [l2]
   l3, [l1;l2]
   l2, []
   l4, []
   l5, [l4]
\end{verbatim}\end{small}

\noindent
Note that {\small\verb%out%} has been removed from the structure. From the
graph we can see that {\small\verb%l2%} and {\small\verb%l4%} do not Manual/Makefile100644   2316     24        3252  6040457173  11761 0ustar  kxsuser