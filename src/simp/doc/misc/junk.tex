\chapter*{Preface}

The purpose of the chapters which follow is to document some of the 
more advanced techniques that experienced users employ 
when using the \HOL system.
Chapter~\ref{simplification} describes the \HOL\ simplifier, a 
powerful new proof tool available with this release of \HOL.  
%Chapter~\ref{abstract-theories} describes how abstract (or
%parameterized) theories
%can be developed in the \HOL\ system, for example theories
%for algebraic notions like groups and rings.  
%Chapter~\ref{choice} describes utilities that are available for 
%dealing with the troublesome Hilbert-choice operator, and also some techniques
%for avoiding its use altogether.
%Chapter~\ref{decision-procedures} describes some of the advanced
%decision procedures included with the \HOL\ system.
%Chapter~\ref{subtyping} describes the latest ideas on how
%subtyping and dependent types can be handled in the \HOL\ framework.
%Chapter~\ref{wisdom} is a distillation of some of the wisdom
%that has appeared on the \verb%info-hol% mailing list over the
%last 10 years.


Many of these topics are ongoing areas of research, and users are 
invited to pursue better solutions than those 
presented here.  The \HOL\ system provides an
ideal environment for exploring solutions to these problems,
due to its high level of programmability and simple conceptual
framework.

All of these tools are developed on top of the
core \HOL\ system, without changing the essential logical
basis being used.  In different ways this demonstrates both
the power of the \HOL\ design philosophy --- a
simple core allows safe, easy extensibility via programming.
However, this can be also be seen as a weakness, as the advanced
facilities described here are needed in most applications, and 
significant effort is needed to program them on top of the \HOL\ core.



\chapter{Simplification}

\label{simplification}
\label{simplifier}

This chapter describes `simplification', which is a 
powerful new proof technique available in the latest version
of \HOL.  

First a word of warning! As always, {\em the more you know about what
an automated tool can do for you, the more effective it will
be for you}. Users should ensure they have
a good understanding of what simplification is
before they make large sale use of it in their poofs.
This will help avoid the plethora of problems
that develop from the misapplication of automated tools.

In particular, users new to theorem proving
systems should probably use simplification sparingly,
until they have a thorough understanding of the basics of how
a theorem proving system works.

\section{What is simplification?}



Some of the basic functions for creating and manipulating simpsets
are:
\begin{boxed} \begin{verbatim}
    val addrewrs : simpset * thm list -> simpset
    val addconvs : simpset * convdata list -> simpset
    val addcongs : simpset * thm list -> simpset
    val adddprocs : simpset * dproc list -> simpset
\end{verbatim} \end{boxed}
Some of inbuilt simpsets of the \HOL\ system are:
\begin{center}
\index{simpsets@!inbuilt}
\index{mk_vartype@\ml{mk\_vartype}}
\index{mk_type@\ml{mk\_type}}
\begin{tabular}{|l|l|} \hline
{\it Simpset} & {\it Purpose}  \\ \hline
 & \\ \hline
\ml{pure\_ss} & Used for building other simpsets \\ \hline
\ml{bool\_ss} & Simplifies ``basic'' constructs \\ \hline
\ml{pair\_ss} & Simplifies tupled expressions \\ \hline
\ml{list\_ss} & Simplifies list constructs \\ \hline
\ml{combin\_ss} & Simplifies combinator constructs \\ \hline
\ml{arith\_ss} & Simplification combined arithmetic decision procedures \\ \hline
\ml{hol\_ss} & Combines all inbuilt simplification strategies \\ \hline
\end{tabular}
\end{center}
Simpsets are usually constructed to perform a particular
task, e.g. to rewrite out all occurrences of a group of definitions,
or to apply a set of reductions which will always reduce a term
to a normal form.

The above functions 
are usually called using the \ml{|>} infix operator.  For example:
\begin{session} \begin{verbatim}
- val BOOL_ss = 
      pure_ss |> addrewrs [FORALL_SIMP, EXISTS_SIMP,
                           IMP_CLAUSES, AND_CLAUSES,
                           COND_CLAUSES, OR_CLAUSES]
              |> addcongs [imp_congrule, cond_congrule]
\end{verbatim} \end{session}

The basic routines used to invoke simplification are:
\begin{boxed} \begin{verbatim}
    val SIMP_CONV : simpset -> thm list -> conv
    val SIMP_PROVE : simpset -> thm list -> term -> thm
    val SIMP_RULE : simpset -> thm list -> thm -> thm
    val SIMP_TAC : simpset -> thm list -> tactic
    val ASM_SIMP_TAC : simpset -> thm list -> tactic
    val FULL_SIMP_TAC : simpset -> thm list -> tactic
\end{verbatim} \end{boxed}






\subsection{Adding Decsion Procedures}

\label{adding-dprocs}
During application, each decision procedure maintains
a private copy of the working context.
Each procedure is allowed to organise this data according to its needs.  
For exampe, in the implementation of the simplification, the rewriter counts
as just one decision procedure, which happens to store its working
context as a term net.

A decision procedure is added to a simpset using \ml{adddprocs}
and \ml{mk\_dproc}:
\begin{boxed} \begin{verbatim}
    val adddprocs : simpset * dproc list -> simpset
    val mk_dproc : {
         name : string,
         relation : term,
         initial: context,
         addcontext : context * Thm.thm list -> context,
         apply: {solver:term -> thm, context: context} -> Abbrev.conv
       } -> dproc
\end{verbatim} \end{boxed}
Do not be overwhelemed by the number of fields in \ml{mk\_dproc} --- only
\ml{initial}, \ml{addcontext}\ and \ml{apply}\ need much thought:
\begin{itemize}
   \item \ml{name} field should be a unique name for the decision procedure
   \item \ml{relation} should be the term \verb%(--`$=`--)% 
(i.e equality), since we are always reducing under equality during
simplification.
   \item \ml{initial} should return a value of type \ml{context}.  This
is the value of the context storage for the decision procedure before
simplification starts.  If the context is being stored as a list of
theorems, this should be \ml{THMLIST []}.
   \item \ml{addcontext} is called every time a new fact becomes known
during simplification.  Facts become known from three sources:
   \begin{itemize}
       \item Theorems passed as arguments to the simplification routines.
       \item Theorems from the assumption list when using \ml{ASM\_SIMP\_TAC}
or \ml{FULL\_SIMP\_TAC}.
       \item Facts derived because of contextual rewriting.
   \end{itemize}
   \ml{addcontext} is passed the previous working context and the new
   facts.  It is expected to return the next working context.
   \item \ml{apply} is how the decision procedure is actually
invoked.  The current working context is passed as an argument.
A {\em solver} is also provided which can be used to help solve
any side conditions that arise from applying the decision procedure.
\ml{apply} should return a theorem of the form \ml{|- $t_1$ = $t_2$},
and it should also fail quickly for terms to which it does not apply.
\end{itemize}.

Consider the example of \ml{ARITH}.  This decision procedure can
make use of any contextual facts relating to Presburger arithmetic.
We shall store the context as a theorem list - hence we use the
constructor \ml{THMLIST} to produce values of type \ml{context}.
Assuming the the function \ml{is\_arith\_thm} determines if a
theorem expresses a fact about Presburger arithmetic, and the
function \ml{ARITH\_CCONV} accepts a list of theorems and a term,
calls the arithmetic decision procedure
and returns an equality theorem, then a simpset utilising the
decision procedure is created as follows:
\begin{session} \begin{verbatim}
val ARITH_DPROC = 
   let fun addcontext (THMLIST thms, newthms) = 
       let val thms' = flatten (map CONJUNCTS newthms
       in THMLIST (filter is_arith_thm thms'@thms)
       end
       fun apply {context=THMLIST thms,solver} tm = ARITH_CCONV thms tm
   in mk_reducer {
         addcontext= addcontext,
         apply=apply,
         initial=THMLIST [],
         name="ARITH",
         relation = #const (const_decl "=")
      }
   end;

val ARITH_ss = pure_ss |> adddprocs [ARITH_DPROC];
\end{verbatim} \end{session}

\shadowbox{The working context must be stored as a value of type \ml{context}.
This would seem to limit the ways in which the context can be organised,
however see the notes in the implementation for a method using
exceptions to allow the storage of arbitrary data.}

\subsection{Programmable Hooks: Reduction Conversions and
Congruence Procedures}

<not yet written - see examples in the inbuilt simpsets>

\subsection{Generalized Term Traversal}

Simplification is implemented as specific instance of {\em 
contextual term traversal}.  Term traversal generalizes
simplification in that it is capable of reducing a term
under relations other than equality.  

Non-contextual term traversal
is essentially encompassed by the functions

In many applications it is common to define equivalence or
preorder relations other than logical equality.  For instance, for
a small embedded language we may define a relation 
\verb%(--`$==: prog -> prog -> bool`--)% which tests
whether two expressions are observationally equivalent,
e.g. whether they have the same input/output characteristics.  The
following theorems should then hold:
\begin{hol} \begin{verbatim}
|- (p == p)
|- (p1 == p2) /\ (p2 = p3) ==> (p1 == p3)
\end{verbatim} \end{hol}
which makes \ml{==} a preorder (it is, of course a congruence also).
These theorems, and appropriate congruence rules, are sufficient
to enable automated term traversal and reduction of a kind very similar
to that for simplification.  The similarities are close enough that
they deserve to be implemented in the same mechanism.

Some congruence rules for the above construct might be:
\begin{hol} \begin{verbatim}
|- (p == p') ==> (Succ p == Succ p')
|- (p == p') ==> (q == q') ==> (Appl p q == Appl p' q')
\end{verbatim} \end{hol}
Some theorems we might want to automatically apply are:
\begin{hol} \begin{verbatim}
|- (Skip ;; P) == P
|- (P ;; Skip) == P
\end{verbatim} \end{hol}
Other theorems may help normalize fully specified programs
to sequences of input/output actions, thus aiding automatic
proof of program equivalence.



