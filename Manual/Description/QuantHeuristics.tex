\index{quantHeuristicsLib|(}
\index{Quantifier Instantiation|see {quantHeuristicsLib}}

\setcounter{sessioncount}{0}

\subsection{Motivation}

Often interactive proofs can be simplified by instantiating
quantifiers. The \ml{Unwind} library\index{Unwind}, which is part of the simplifier, allows
instantiations of ``trivial'' quantifiers:

\[ \forall x_1\ \ldots x_i \ldots x_n.\ P_1 \wedge \ldots \wedge x_i = c \wedge \ldots \wedge P_n \Longrightarrow Q \]
and
\[ \exists x_1\ \ldots x_i \ldots x_n.\ P_1 \wedge \ldots \wedge x_i =
c \wedge \ldots \wedge P_n \] can be simplified by
instantiating $x_i$ with $c$. Because unwind-conversions are
part of \holtxt{bool\_ss}, they are used with nearly every call of the simplifier
and often simplify proofs considerably. However, the \ml{Unwind} library can only handle these common cases. If the term structure is
only slightly more complicated, it fails. For example, $\exists x.\ P(x) \Longrightarrow (x = 2) \wedge Q(x)$
cannot be tackled.

There is also the \ml{Satisfy} library\index{Satisfy}, which uses
unification to show existentially quantified formulas. It can handle
problems like $\exists x.\ P_1(x,c_1)\ \wedge \ldots P_n(x,c_n)$ if
given theorems of the form $\forall x\ c.\ P_i(x, c)$. This is often
handy, but still rather limited.

The quantifier heuristics library (\ml{quantHeuristicsLib}) provides more power
and flexibility. A few simple examples of what it can do
are shown in Table~\ref{table-qh-examples}. Besides the power demonstrated
by these examples, the library is highly flexible as well.  At its
core, there is a modular, syntax driven search for instantiation.
This search consists of a collection of interleaved heuristics.  Users
can easily configure existing heuristics and add own ones. Thereby, it
is easy to teach the library about new predicates, logical connectives
or datatypes.

\newcommand{\mytablehead}[1]{\\\multicolumn{2}{l}{\textit{#1}}\\}
\begin{table}[h]
\centering\scriptsize
\begin{tabular}{lll}
\textbf{Problem} & \textbf{Result} \\\hline

\mytablehead{basic examples}
$\exists x.\ x = 2 \wedge P (x)$ & $P(2)$ \\
$\forall x.\ x = 2 \Longrightarrow P (x)$ & $P(2)$ \\

\mytablehead{solutions and counterexamples}
$\exists x.\ x = 2$ & $\textit{true}$ \\
$\forall x.\ x = 2$ & $\textit{false}$ \\

\mytablehead{complicated nestings of standard operators}
$\exists x_1. \forall x_2.\ (x_1 = 2) \wedge P(x_1, x_2)$ &
$\forall x_2.\ P(2, x_2)$ \\

$\exists x_1, x_2.\ P_1(x_2) \Longrightarrow (x_1 = 2) \wedge P(x_1, x_2)$ &
$\exists x_2.\ P_1(x_2) \Longrightarrow P(2, x_2)$ \\
$\exists x.\ ((x = 2) \vee (2 = x)) \wedge P(x)$ & $P(2)$ \\

\mytablehead{exploiting unification}
$\exists x.\ (f (8 + 2) = f (x + 2)) \wedge P (f(10))$ & $P (f(10))$ \\
$\exists x.\ (f (8 + 2) = f (x + 2)) \wedge P (f(x + 2))$ & $P (f(8 + 2))$ \\
$\exists x.\ (f (8 + 2) = f (x + 2)) \wedge P (f(x))$ & - (\textrm{no instantiation found}) \\

\mytablehead{partial instantiation for datatypes}
$\forall p.\ c = \textsf{FST}(p) \Longrightarrow P(p)$ & $\forall p_2.\ P(c, p_2)$ \\

$\forall x.\ \textsf{IS\_NONE}(x) \vee P(x)$ & $\forall x'.\ P (\textsf{SOME}(x'))$ \\

$\forall l.\ l \neq [\,] \Longrightarrow P(l)$ & $\forall \textit{hd}, \textit{tl}.\
P(\textit{hd} :: \textit{tl})$ \\

\mytablehead{context}
$P_1(c) \Longrightarrow \exists x.\ P_1(x) \vee P_2(x)$ & \textit{true} \\
$P_1(c) \Longrightarrow \forall x.\ \neg P_1(x) \wedge P_2(x)$ & $\neg P_1(c)$ \\

$(\forall x.\ P_1(x) \Rightarrow (x = 2)) \Longrightarrow (\forall x.\ P_1(x) \Rightarrow P_2(x))$ &
$(\forall x.\ P_1(x) \Rightarrow (x = 2)) \Rightarrow (P_1(2) \Rightarrow P_2(2))$ \\

$\big((\forall x.\ P_1(x) \Rightarrow P_2(x)) \wedge P_1(2)\big) \Longrightarrow \exists x.\ P_2(x)$ &
\textit{true} \\
\hline
\end{tabular}
\caption{Examples}
\label{table-qh-examples}
\end{table}

\subsection{User Interface}\label{sec:qh-interface}

The quantifier heuristics library can be found in the sub-directory
\holtxt{src/quantHeuristics}.  The entry point to the framework is the
library \holtxt{quantHeuristicsLib}.

\subsubsection{Conversions}
Usually the library is used for
converting a term containing quantifiers to an equivalent one. For this,
the following high level entry points exists:
\bigskip

\noindent
\begin{tabular}{@{}ll}
\texttt{QUANT\_INSTANTIATE\_CONV} & \texttt{: quant\_param list -> conv} \\
\texttt{QUANT\_INST\_ss} & \texttt{: quant\_param list -> ssfrag} \\
\texttt{QUANT\_INSTANTIATE\_TAC} & \texttt{: quant\_param list -> tactic} \\
\texttt{ASM\_QUANT\_INSTANTIATE\_TAC} & \texttt{: quant\_param list -> tactic}
\end{tabular}
\bigskip

All these functions get a list of \emph{quantifier heuristic parameters} as arguments. These
parameters essentially configure, which heuristics are used during the guess-search. If
an empty list is provided, the tools know about the standard Boolean combinators, equations and context.
\texttt{std\_qp} adds support for common datatypes like pairs or lists.
Quantifier heuristic parameters are explained in more detail in
Section~\ref{quantHeu-subsec-qps}.

So, some simple usage of the quantifier heuristic library looks like:

\begin{session}
\begin{verbatim}
- QUANT_INSTANTIATE_CONV [] ``?x. (!z. Q z /\ (x=7)) /\ P x``;
> val it = |- (?x. (!z. Q z /\ (x = 7)) /\ P x) <=> (!z. Q z) /\ P 7: thm

- QUANT_INSTANTIATE_CONV [std_qp] ``!x. IS_SOME x ==> P x``
> val it = |- (!x. IS_SOME x ==> P x) <=> !x_x'. P (SOME x_x'): thm
\end{verbatim}
\end{session}

Usually, the quantifier heuristics library is used together with the
simplifier using \holtxt{QUANT\_INST\_ss}. Besides interleaving
simplification and quantifier instantiation, this has the benefit of
being able to use context information collected by the simplifier:

\begin{session}
\begin{verbatim}
- QUANT_INSTANTIATE_CONV [] ``P m ==> ?n. P n``
Exception- UNCHANGED raised

- SIMP_CONV (bool_ss ++ QUANT_INST_ss []) [] ``P m ==> ?n. P n``
> val it = |- P m ==> (?n. P n) <=> T: thm
\end{verbatim}
\end{session}

It's usually best to use \holtxt{QUANT\_INST\_ss}
together with e.\,g.\ \holtxt{SIMP\_TAC} when using the library with tactics.
However, if free variables of the goal should be instantiated, then
\holtxt{ASM\_QUANT\_INSTANTIATE\_TAC} should be used:

\begin{session}
\begin{verbatim}
P x
------------------------------------
  IS_SOME x
  : proof

- e (ASM_QUANT_INSTANTIATE_TAC [std_qp])
> P (SOME x_x') : proof
\end{verbatim}
\end{session}

There is also \holtxt{QUANT\_INSTANTIATE\_TAC}. This tactic does not
instantiate free variables. Neither does it take assumptions into consideration.
It is just a shortcut for using \holtxt{QUANT\_INSTANTIATE\_CONV} as a tactic.


\subsubsection{Unjustified Guesses}

Most heuristics justify the guesses they produce and therefore allow to
prove equivalences of e.\,g.\ the form $\exists x.\ P(x) \Leftrightarrow P(i)$.
However, the implementation also supports unjustified guesses, which may be bogus.
Let's consider e.\,g.\ the formula $\exists x.\ P(x) \Longrightarrow (x = 2)\ \wedge\ Q(x)$.
Because nothing is known about $P$ and $Q$, we can't find a safe instantiation for $x$ here.
However, $2$ looks tempting and is probably sensible in many situations. (Counterexample:
$P(2)$, $\neg Q(2)$ and $\neg P(3)$ hold)

\texttt{implication\_concl\_qp} is a quantifier parameter that looks for valid guesses in the conclusion of an implication.
Then, it assumes without justification that these guesses are probably sensible for the whole implication as well.
Because these guesses might be wrong, one can either use implications or
expansion theorems like $\exists x.\ P(x)\ \Longleftrightarrow (\forall x.\ x \neg c \Rightarrow \neg P(x)) \Rightarrow P(c)$.

\begin{session}
\begin{verbatim}
- QUANT_INSTANTIATE_CONV [implication_concl_qp]
     ``?x. P x ==> (x = 2) /\ Q x``
Exception- UNCHANGED raised

- QUANT_INSTANTIATE_CONSEQ_CONV [implication_concl_qp]
     CONSEQ_CONV_STRENGTHEN_direction
     ``?x. P x ==> (x = 2) /\ Q x``
> val it =
   |- (P 2 ==> Q 2) ==> ?x. P x ==> (x = 2) /\ Q x: thm

- EXPAND_QUANT_INSTANTIATE_CONV [implication_concl_qp]
    ``?x. P x ==> (x = 2) /\ Q x``
> val it = |- (?x. P x ==> (x = 2) /\ Q x) <=>
              (!x. x <> 2 ==> ~(P x ==> (x = 2) /\ Q 2)) ==> P 2 ==> Q 2

- SIMP_CONV (std_ss++EXPAND_QUANT_INST_ss [implication_concl_qp]) []
    ``?x. P x ==> (x = 2) /\ Q x``
> val it =
   |- (?x. P x ==> (x = 2) /\ Q x) <=>
      (!x. x <> 2 ==> P x) ==> P 2 ==> Q 2: thm
\end{verbatim}
\end{session}

The following entry points should be used to exploit unjustified guesses:
\bigskip

\noindent
\begin{tabular}{@{}ll}
\texttt{QUANT\_INSTANTIATE\_CONSEQ\_CONV} & \texttt{: quant\_param list -> directed\_conseq\_conv} \\
\texttt{EXPAND\_QUANT\_INSTANTIATE\_CONV} & \texttt{: quant\_param list -> conv} \\
\texttt{EXPAND\_QUANT\_INST\_ss} & \texttt{: quant\_param list -> ssfrag} \\
\texttt{QUANT\_INSTANTIATE\_CONSEQ\_TAC} & \texttt{: quant\_param list -> tactic}
\end{tabular}


\subsubsection{Explicit Instantiations}

A special (degenerated) use of the framework, is turning guess search off completely and
providing instantiations explicitly. The tactic \holtxt{QUANT\_TAC} allows this. This means that
it allows to partially instantiate quantifiers at subpositions
with explicitly given terms. As such, it can be seen as
a generalisation of \holtxt{EXISTS\_TAC}\index{EXISTS\_TAC}.
%
\begin{session}
\begin{verbatim}
- val it = !x. (!z. P x z) ==> ?a b.    Q a        b z : proof

> e( QUANT_INST_TAC [("z", `0`, []), ("a", `SUC a'`, [`a'`])] )
- val it = !x. (    P x 0) ==> ?  b a'. Q (SUC a') b z : proof
\end{verbatim}
\end{session}
%
This tactic is implemented using unjustified guesses. It normally
produces implications, which is fine when used as a tactic. There is
also a conversion called \holtxt{INST\_QUANT\_CONV} with the same
functionality. For a conversion, implications are
problematic. Therefore, the simplifier and Metis are used to prove
the validity of the explicitly given instantiations. This succeeds
only for simple examples.


\subsection{Quantifier Heuristic Parameters}\label{quantHeu-subsec-qps}

Quantifier heuristic parameters play a similar role for the quantifier
instantiation library as simpsets do for the simplifier. They contain
theorems, ML code and general configuration parameters that allow to configure
guess-search. There are predefined parameters that handle
common constructs and the user can define own parameters.

\subsubsection{Quantifier Heuristic Parameters for Common Datatypes}

There are \holtxt{option\_qp}, \holtxt{list\_qp}, \holtxt{num\_qp} and \holtxt{sum\_qp} for option types, lists,
natural numbers and sum types respectively.
Some examples are displayed in the following table:
%
\[\begin{array}{r@{\quad \Longleftrightarrow \quad}l}
\forall x.\ \holtxt{IS\_SOME}(x) \Rightarrow P(x) & \forall x'.\ P (\holtxt{SOME}(x')) \\
\forall x.\ \holtxt{IS\_NONE}(x)& \textit{false} \\
\forall l.\ l \neq [\,] \Rightarrow P(l)& \forall h, l'.\ P(h::l')  \\
\forall x.\ x = c + 3& \textit{false} \\
\forall x.\ x \neq 0 \Rightarrow P(x)& \forall x'.\ P(\holtxt{SUC}(x'))
\end{array}\]

\subsubsection{Quantifier Heuristic Parameters for Tuples}

For tuples the situation is peculiar, because each quantifier over a variable of a product type
can be instantiated. The challenge is to decide which quantifiers should be instantiated and
which new variable names to use for the components of the pair.
There is a quantifier heuristic parameter called \holtxt{pair\_default\_qp}. It first looks for
subterms of the form $(\lambda (x_1, \ldots, x_n).\ \ldots)\ x$. If such a term is found $x$ is instantiated with
$(x_1, \ldots, x_n)$. Otherwise, subterms of the form $\holtxt{FST}(x)$ and $\holtxt{SND}(x)$ are searched. If such a term
is found, $x$ is instantiated as well. This parameter therefore allows simplifications like:
%
\[\begin{array}{r@{\quad \Longleftrightarrow \quad}l}
\forall p.\ (x = \holtxt{SND}(p)) \Rightarrow P(p)& \forall p_1.\ P(p_1, x) \\
\exists p.\ (\lambda (p_a, p_b, p_c). P(p_a, p_b, p_c))\ p & \exists p_a, p_b, p_c.\ P(p_a, p_b, p_c)
\end{array}\]

\holtxt{pair\_default\_qp} is implemented in terms of the more general
quantifier heuristic parameter \holtxt{pair\_qp}, which allows the
user to provide a list of ML functions. These functions get the
variable and the term. If they return a tuple of variables, these
variables are used for the instantiation, otherwise the next function
in the list is called or - if there is no function left - the variable
is not instantiated. In the example of $\exists p.\ (\lambda (p_a,
p_b, p_c). P(p_a, p_b, p_c))\ p$ these functions are given the
variable $p$ and the term $(\lambda (p_a, p_b, p_c). P(p_a, p_b,
p_c))\ p$ and return $\holtxt{SOME} (p_a, p_b, p_c)$.  This simple
ML-interface gives the user full control over what quantifier over
product types to expand and how to name the new variables.

\subsubsection{Quantifier Heuristic Parameter for Records}

Records are similar to pairs, because they can always be instantiated. Here, it is interesting that the necessary
monochotomy lemma comes from HOL~4's \holtxt{Type\_Base} library. This means that \holtxt{record\_qp} is stateful.
If a new record type is defined, the automatically proven monochotomy lemma is then automatically used
by \holtxt{record\_qp}. In contrast to the pair parameter, the one for records gets only one function instead of a
list of functions to decide which variables to instantiate. However, this function is simpler, because it just needs
to return true or false. The names of the new variables are constructed from the field-names of the record.
The quantifier heuristic parameter \holtxt{default\_record\_qp} expands all records.

\subsubsection{Stateful Quantifier Heuristic Parameters}

The parameter for records is stateful, as it uses knowledge from
\holtxt{Type\_Base}. Such information is not only useful for records
but for general datatypes. The quantifier heuristic parameter
\holtxt{TypeBase\_qp} uses automatically proven theorems about new
datatypes to exploit mono- and dichotomies. Moreover, there is also a
stateful \holtxt{pure\_stateful\_qp} that allows the user to
explicitly add other parameters to it.  \holtxt{stateful\_qp} is a
combination of \holtxt{pure\_stateful\_qp} and \holtxt{TypeBase\_qp}.

\subsubsection{Standard Quantifier Heuristic Parameter}

The standard quantifier heuristic parameter \holtxt{std\_qp} combines
the parameters for lists, options, natural numbers, the default one
for pairs and the default one for records.


\subsection{User defined Quantifier Heuristic Parameters}\label{sec:qps-user}

The user is also able to define own parameters. There
is \holtxt{empty\_qp}, which does not contain any information. Several
parameters can be combined using
\holtxt{combine\_qps}. Together with the basic types of user defined
parameters that are explained below, these functions provide an
interface for user defined quantifier heuristic parameters.

\subsubsection{Rewrites / Conversions}

A very powerful, yet simple technique for teaching the guess search
about new constructs are rewrite rules. For example, the standard rules
for equations and basic logical operations
cannot generate guesses for the predicate \holtxt{IS\_SOME}. By
rewriting \holtxt{IS\_SOME(x)} to \holtxt{?x'. x =
SOME(x')}, however, these rules fire.

\holtxt{option\_qp} uses this rewrite to implement support for
\holtxt{IS\_SOME}. Similarly support for predicates like \holtxt{NULL} is
implemented using rewrites. Even adding
rewrites like $\textsf{append}(l_1, l_2) = [\,] \Longleftrightarrow (l_1 =
[\,]\ \wedge\ l_2 = [\,])$ for list-append turned out to be beneficial
in practice.
\bigskip

\holtxt{rewrite\_qp} allows to provide rewrites in the form of rewrite theorems.
For the example of \holtxt{IS\_SOME} this looks like:

\begin{session}
\begin{verbatim}
> val thm = QUANT_INSTANTIATE_CONV [] ``!x. IS_SOME x ==> P x``
Exception- UNCHANGED raised

> val IS_SOME_EXISTS = prove (``IS_SOME x = (?x'. x = SOME x')``,
   Cases_on `x` THEN SIMP_TAC std_ss []);
val IS_SOME_EXISTS = |- IS_SOME x <=> ?x'. x = SOME x': thm

> val thm = QUANT_INSTANTIATE_CONV [rewrite_qp[IS_SOME_EXISTS]]
    ``!x. IS_SOME x ==> P x``
val thm = |- (!x. IS_SOME x ==> P x) <=>
             !x'. IS_SOME (SOME x') ==> P (SOME x'): thm
\end{verbatim}
\end{session}

To clean up the result after instantiation, theorems used to rewrite the result after instantiation can be provided via
\holtxt{final\_rewrite\_qp}.
\begin{session}
\begin{verbatim}
> val thm = QUANT_INSTANTIATE_CONV [rewrite_qp[IS_SOME_EXISTS],
                                    final_rewrite_qp[option_CLAUSES]]
      ``!x. IS_SOME x ==> P x``
val thm = |- (!x. IS_SOME x ==> P x) <=> !x'. P (SOME x'): thm
\end{verbatim}
\end{session}

If rewrites are not enough, \holtxt{conv\_qp} can be used to add conversions:
\begin{session}
\begin{verbatim}
- val thm = QUANT_INSTANTIATE_CONV [] ``?x. (\y. y = 2) x``
Exception- UNCHANGED raised

- val thm = QUANT_INSTANTIATE_CONV [convs_qp[BETA_CONV]] ``?x. (\y. y = 2) x``
> val thm = |- (?x. (\y. y = 2) x) <=> T: thm
\end{verbatim}
\end{session}

\subsubsection{Strengthening / Weakening}

In rare cases, equivalences that can be used for rewrites are unavailable. There might be just implications that
can be used for strengthening or weakening. The function
\holtxt{imp\_qp} might be used to provide such implication.

\begin{session}
\begin{verbatim}
- val thm = QUANT_INSTANTIATE_CONV [list_qp] ``!l. 0 < LENGTH l ==> P l``
Exception- UNCHANGED raised

- val LENGTH_LESS_IMP = prove (``!l n. n < LENGTH l ==> l <> []``,
    Cases_on `l` THEN SIMP_TAC list_ss []);
> val LENGTH_LESS_IMP = |- !l n. n < LENGTH l ==> l <> []: thm

- val thm = QUANT_INSTANTIATE_CONV [imp_qp[LENGTH_LESS_IMP], list_qp]
    ``!l. 0 < LENGTH l ==> P l``
> val thm =
   |- (!l. 0 < LENGTH l ==> P l) <=>
      !l_t l_h. 0 < LENGTH (l_h::l_t) ==> P (l_h::l_t): thm

- val thm = SIMP_CONV (list_ss ++
              QUANT_INST_ss [imp_qp[LENGTH_LESS_IMP], list_qp]) []
              ``!l. SUC (SUC n) < LENGTH l ==> P l``
> val thm =
   |- (!l. SUC (SUC n) < LENGTH l ==> P l) <=>
      !l_h l_t_h l_t_t_t l_t_t_h. n < SUC (LENGTH l_t_t_t) ==>
                                  P (l_h::l_t_h::l_t_t_h::l_t_t_t): thm
\end{verbatim}
\end{session}


\subsubsection{Filtering}
Sometimes, one might want to avoid to instantiate certain quantifiers.
The function \holtxt{filter\_qp} allows to add ML-functions that filter the handled
quantifiers. These functions are given a variable $x$ and a term $P(x)$.
The tool only tries to instantiate $x$ in $P(x)$, if all filter functions
return \textit{true}.

\begin{session}
\begin{verbatim}
- val thm = QUANT_INSTANTIATE_CONV []
     ``?x y z. (x = 1) /\ (y = 2) /\ (z = 3) /\ P (x, y, z)``
> val thm = |- (?x y z. (x = 1) /\ (y = 2) /\ (z = 3) /\ P (x,y,z)) <=>
               P (1,2,3): thm

- val thm = QUANT_INSTANTIATE_CONV
     [filter_qp [fn v => fn t => (v = ``y:num``)]]
     ``?x y z. (x = 1) /\ (y = 2) /\ (z = 3) /\ P (x, y, z)``
> val thm = |- (?x y z. (x = 1) /\ (y = 2) /\ (z = 3) /\ P (x,y,z)) <=>
                ?x   z. (x = 1) /\            (z = 3) /\ P (x,2,z): thm
\end{verbatim}
\end{session}

\subsubsection{Satisfying and Contradicting Instantiations}

As the satisfy library demonstrates, it is often
useful to use unification and explicitly given theorems to
find instantiations. In addition to satisfying instantiations, the quantifier heuristics framework
is also able to use contradicting ones. The theorems used for finding instantiations usually come from
the context. However, \holtxt{instantiation\_qp} allows to add additional ones:

\begin{session}
\begin{verbatim}
> val thm = SIMP_CONV (std_ss++QUANT_INST_ss[]) []
    ``P n ==> ?m:num. n <= m /\ P m``
Exception- UNCHANGED raised

> val thm = SIMP_CONV (std_ss++
               QUANT_INST_ss[instantiation_qp[LESS_EQ_REFL]]) []
               ``P n ==> ?m:num. n <= m /\ P m``
> val thm = |- P n ==> ?m:num. n <= m /\ P m = T : thm
\end{verbatim}
\end{session}

\subsubsection{Di- and Monochotomies}

Dichotomies can be exploited for guess search.
\holtxt{distinct\_qp} provides an interface to add theorems
of the form $\forall x.\ c_1(x) \neq c_2(x)$.
\holtxt{cases\_qp} expects theorems of the form
$\forall x. \ (x = \exists \textit{fv}. c_1(\textit{fv}))\ \vee \ldots \vee (x = \exists \textit{fv}. c_n(\textit{fv}))$.
However, only theorems for $n = 2$ and $n = 1$ are used. All other cases are currently ignored.

\subsubsection{Oracle Guesses}

Sometimes, the user does not want to justify guesses. The tactic
\holtxt{QUANT\_TAC} is implemented using oracle guesses for example.
A simple interface to oracle guesses is provided by \holtxt{oracle\_qp}.
It expects a ML function that given a variable and a term returns
a pair of an instantiation and the free variables in this instantiation.

As an example, lets define a parameter that states that every list is non-empty:
\begin{verbatim}
   val dummy_list_qp = oracle_qp (fn v => fn t =>
     let
        val (v_name, v_list_ty) = dest_var v;
        val v_ty = listSyntax.dest_list_type v_list_ty;

        val x = mk_var (v_name ^ "_hd", v_ty);
        val xs = mk_var (v_name ^ "_tl", v_list_ty);
        val x_xs = listSyntax.mk_cons (x, xs)
     in
        SOME (x_xs, [x, xs])
     end)
\end{verbatim}

\noindent
Notice, that an option type is returned and that the function is
allowed to throw \holtxt{HOL\_ERR} exceptions.
With this definition, we get

\begin{session}
\begin{verbatim}
- NORE_QUANT_INSTANTIATE_CONSEQ_CONV [dummy_list_qp]
    CONSEQ_CONV_STRENGTHEN_direction ``?x:'a list y:'b. P (x, y)``
> val it = ?y x_hd x_tl. P (x_hd::x_tl,y)) ==> ?x y. P (x,y) : thm
\end{verbatim}
\end{session}

\subsubsection{Lifting Theorems}

The function \holtxt{inference\_qp} enables the
user to provide theorems that allow lifting guesses over
user defined connectives. As writing these lifting theorems requires
deep knowledge about guesses, it is not discussed here. Please have a
look at the detailed documentation of the quantifier heuristics library as
well as its sources. You might also want to contact
Thomas Tuerk (\url{tt291@cl.cam.ac.uk}).


\subsubsection{User defined Quantifier Heuristics}

At the lowest level, the tool searches guesses using ML-functions
called \emph{quantifier heuristics}. Slightly simplified, such a
quantifier heuristic gets a variable and a term and returns a set of
guesses for this variable and term. Heuristics allow full
flexibility. However, to write your own heuristics a lot of knowledge
about the ML-datastructures and auxiliary functions is
required. Therefore, no details are discussed here. Please have a look
at the source code and contact Thomas Tuerk
(\url{tt291@cl.cam.ac.uk}), if you have questions.
\holtxt{heuristics\_qp} and \holtxt{top\_heuristics\_qp} provide
interfaces to add user defined heuristics to a quantifier heuristics
parameter.

\index{quantHeuristicsLib|)}


%%% Local Variables:
%%% mode: latex
%%% TeX-master: "description"
%%% End:
