\chapter{Commonly-used Libraries}\label{HOLlibraries}

\section{A simple proof manager}\label{goalstack}

The {\it goal stack\/} provides a simple interface to tactic-based
proof. When one uses tactics to decompose a proof, many intermediate
states arise; the goalstack takes care of the necessary bookeeping. The
implementation of goalstacks reported here is a re-design of Larry
Paulson's original conception.

The goalstack library is automatically loaded when \HOL\ starts up.

The abstract types {\it goalstack\/} and {\it proofs\/} are the focus of
backwards proof operations. The type \verb+proofs+ can be regarded as a
list of independent goalstacks. Most operations act on the head of the
list of goalstacks; there are operations so that the focus can
be changed.

\subsection{Starting a goalstack proof}

\begin{verbatim}
    g        : term quotation -> proofs
    set_goal : goal -> proofs
\end{verbatim}

Recall that the type \verb+goal+ is an abbreviation for
\verb+term list * term+. To start on a new goal, one gives
\verb+set_goal+ a goal. This creates a new goalstack and makes it the
focus of further operations.

A shorthand for \verb+set_goal+ is the function \verb+g+: it
invokes the parser automatically, and it doesn't allow the the goal to
have any assumptions.


Calling \verb+set_goal+, or \verb+g+, adds a new proof attempt to
the existing ones, {\it i.e.}, rather than overwriting the current
proof attempt, the new attempt is stacked on top.

\subsection{Applying a tactic to a goal}

\begin{verbatim}
    expandf : tactic -> goalstack
    expand  : tactic -> goalstack
    e       : tactic -> goalstack
\end{verbatim}

How does one actually do a goalstack proof then? In most cases, the
application of tactics to the current goal is done with the function
\verb+expand+. In the rare case that one wants to apply an
{\it invalid\/} tactic, then \verb+expandf+ is used. (For an
explanation of invalid tactics, see Chapter 24 of Gordon \& Melham.) The
abbreviation \verb+e+ may also be used to expand a tactic.


\subsection{Undo}

\begin{verbatim}
    b          : unit -> goalstack
    drop       : unit -> proofs
    dropn      : int  -> proofs
    backup     : unit -> goalstack
    restart    : unit -> goalstack
    set_backup : int  -> unit
\end{verbatim}

Often (we are tempted to say {\it usually}!) one takes a wrong path
in doing a proof, or makes a mistake when setting a goal. To undo a step
in the goalstack, the function \verb+backup+ and its abbreviation
\verb+b+ are used. This will restore the goalstack to its previous
state.


To directly back up all the way to the original goal, the function
\verb+restart+ may be used. Obviously, it is also important to get
rid of proof attempts that are wrong; for that there is \verb+drop+,
which gets rid of the current proof attempt, and \verb+dropn+, which
eliminates the top $n$ proof attempts.


Each proof attempt has its own {\it undo-list\/} of previous
states. The undo-list for each attempt is of fixed size (initially
12). If you wish to set this value for the current proof attempt, the
function \verb+set_backup+ can be used. If the size of the backup
list is set to be smaller than it currently is, the undo list will be
immediately truncated. You can not undo a ``proofs-level'' operation, such
as \verb+set_goal+ or \verb+drop+.

\subsection{Viewing the state of the proof manager}

\begin{verbatim}
    p            : unit -> goalstack
    status       : unit -> proofs
    top_goal     : unit -> goal
    top_goals    : unit -> goal list
    initial_goal : unit -> goal
    top_thm      : unit -> thm
\end{verbatim}

To view the state of the proof manager at any time, the functions
\verb+p+ and \verb+status+ can be used. The former only shows
the top subgoals in the current goalstack, while the second gives a
summary of every proof attempt.

To get the top goal or goals of a proof attempt, use \verb+top_goal+
and \verb+top_goals+. To get the original goal of a proof attempt,
use \verb+initial_goal+.

Once a theorem has been proved, the goalstack that was used to derive it
still exists (including its undo-list): its main job now is to
hold the theorem. This theorem can be retrieved with
\verb+top_thm+.

\subsection{Switch focus to a different subgoal or proof attempt}

\begin{verbatim}
    r             : int -> goalstack
    R             : int -> proofs
    rotate        : int -> goalstack
    rotate_proofs : int -> proofs
\end{verbatim}

Often we want to switch our attention to a different goal in the current
proof, or a different proof. The functions that do this are
\verb+rotate+ and \verb+rotate_proofs+, respectively. The abbreviations
\verb+r+ and \verb+R+ are simpler to type in.
\section{The {\tt boss} library}
\newcommand\bossLib{{\tt bossLib}}

The library \bossLib\ marshalls some of the most widely used theorem
proving tools in \HOL\ and provides them with a convenient interface
for interaction. The library currently focuses on three things:
definition of datatypes and functions; high-level interactive proof
operations, and composition of automated reasoners. Loading \bossLib\
commits one to working in a context that already supplies the theories
of booleans, pairs, sums, the option type, arithmetic, and lists.

\subsection{Datatype definition}

There are several useful consequences of an object logic datatype
definition: structural induction, rewrite rules for constructors,
\etc\ However, these have not traditionally been automatically derived
at the invocation of the definition package: the user would have to
build the required theorems by explicitly invoking various proof
procedures.  To remedy this, \bossLib\ offers the \verb+Hol_datatype+
function.  This function allows for the definition of mutually
recursive types, nested recursive types and record types.  The syntax
of declarations that \verb+Hol_datatype+ accepts is found in Table
\ref{datatype}.

\newcommand{\ident}      {\mbox{\it ident}}
\newcommand{\clause}      {\mbox{\it clause}}
\newcommand{\type}       {\mbox{\it hol\_type}}
{
\newcommand{\binding} {\mbox{\it binding}}
\newcommand{\recdspec}  {\mbox{\it record-spec}}
\newcommand{\constr} {\mbox{\it constructor-spec}}

\begin{table}[h]
\begin{center}
\begin{tabular}{|rcl|}
\hline
\multicolumn{3}{|l|}
{\texttt{Hol\_datatype `}[\binding\ \texttt{;}]* \binding\texttt{`}}\\
& &\\
\binding & \verb+::=+ & \ident\ \verb+=+ \constr\\
         & \verb+|+ & \ident\ \verb+=+ \recdspec\\
& & \\
\constr & \verb+::=+ & [\clause\ \verb+|+]* \clause \\
& & \\
\clause & \verb+::=+ & \ident \\
        & \verb+|+ & \ident\ \verb+of+\ [\type\ \verb+=>+]* \type\\
& & \\
\recdspec & \verb+::=+ & \verb+<|+ [\ident\ \verb+:+ \type\ \verb+;+]*
                                   \ident\ \verb+:+ \type\ \verb+|>+\\

\hline
\end{tabular}
\caption{Datatype Declaration}\label{datatype}
\end{center}
\end{table}
}
There is an underlying database of datatype facts that supports the
activities of \verb+bossLib+. This database already contains the
relevant entries for the types \verb+bool+, \verb+prod+, \verb+num+,
\verb+option+, and \verb+list+.  When a datatype is defined by
\verb+Hol_datatype+, the following information is derived and stored in
the database.

\begin{itemize}
\item initiality theorem for the type
\item injectivity of the constructors
\item distinctness of the constructors
\item structural induction theorem
\item case analysis theorem
\item definition of the `case' constant for the type
\item congruence theorem for the case constant
\item definition of the `size' of the type
\end{itemize}

\subsection{Support for high-level proof steps}

The following functions use information in the database to ease the
application of \holn{}'s underlying functionality:

\begin{verbatim}
        type_rws     : string -> thm list
        Induct       : tactic
        Cases        : tactic
        Cases_on     : term quotation -> tactic
        Induct_on    : term quotation -> tactic
\end{verbatim}

\begin{itemize}

\item
The function \underline{\tt type\_rws} will search for the given type
by name in the underlying database and return useful
rewrite rules for that type. The rewrite rules of the
datatype are built from the injectivity and distinctness theorems, along
with the case constant definition. The pre-existing rewrite rules in the
database are already integrated into the simplification sets provided
by \verb+bossLib+; however rewrite rules arising from an invocation of
\verb+Hol_datatype+, or which come from a user-defined theory, will have
to be manually added into the simpsets used by the simplifier.

\item
The \underline{\tt Induct} tactic makes it convenient to invoke
induction. When it is applied to a goal, the leading universal
quantifier is examined; if its type is that of a known datatype, the
appropriate structural induction tactic is extracted and applied.

\item
The \underline{\tt Cases} tactic makes it convenient to invoke case
analysis. The leading universal quantifier in the goal is examined; if
its type is that of a known datatype, the appropriate structural
case analysis theorem is extracted and applied.

\item The \underline{\tt Cases\_on} tactic takes a quotation, which is
parsed into a term $M$, and then $M$ is searched for in the goal. If $M$
is a variable, then a variable with the same name is searched for. Once
the term to split over is known, its type and the associated facts are
obtained from the underlying database and used to perform the case
split. If some free variables of $M$ are bound in the goal, an attempt
is made to remove (universal) quantifiers so that the case split has
force. Finally, $M$ need not appear in the goal, although it should at
least contain some free variables already appearing in the goal. Note
that the \verb+Cases_on+ tactic is more general than \verb+Cases+, but
it does require an explicit term to be given.

\item The \underline{\tt Induct\_on} tactic takes a quotation, which is
parsed into a term $M$, and then $M$ is searched for in the goal. If $M$
is a variable, then a variable with the same name is searched for. Once
the term to induct on is known, its type and the associated facts are
obtained from the underlying database and used to perform the induction.
If $M$ is not a variable, a new variable $v$ not already occurring in
the goal is created, and used to build a term $v = M$ which the goal is
made conditional on before the induction is performed. First however,
all terms containing free variables from $M$ are moved from the
assumptions to the conclusion of the goal, and all free variables of $M$
are universally quantified. \verb+Induct_on+ is more general than
\verb+Induct+, but it does require an explicit term to be given.

\end{itemize}

Two supplementary entrypoints have been provided for more exotic
inductions:
\begin{description}
\item [completeInduct\_on] performs complete
induction on the term denoted by the given quotation. Complete induction
allows a seemingly \footnote{Complete
induction and ordinary mathematical induction are each derivable from
the other.} stronger induction hypothesis than ordinary mathematical
induction: to wit, when inducting on
$n$, one is allowed to assume the property holds for {\it all\/} $m$
smaller than $n$. Formally: $\forall P.\ (\forall x.\ (\forall y.\ y < x
\supset P\, y) \supset P\,x) \supset \forall x.\ P\,x$. This allows the
inductive hypothesis to be used more than once, and also allows
instantiating the inductive hypothesis to other than the predecessor.

\item [measureInduct\_on] takes a quotation, and breaks it apart
to find a term and a measure function with which to induct. For
example, if one wanted to induct on the length of a list \verb+L+, the
invocation {\small\verb+measureInduct_on `LENGTH L`+} would be be
appropriate.

\end{description}

\subsection{Function definition}

\begin{verbatim}
    Define   : term quotation -> thm
    xDefine  : string -> term quotation -> thm
    Hol_defn : string -> term quotation -> Defn.defn
\end{verbatim}

The \underline{\tt Define} function is a general-purpose function definition
mechanism. The \underline{\tt xDefine} function is identical to
{\small\verb+Define+} except that it takes an explicit name to use when
storing the definition in the current theory. {\small\tt Define}
accepts the following syntax:

\begin{enumerate}

\item Non-recursive definition, varstructs allowed on lhs.
\begin{verbatim}
         Define `f w (x, y, z) = x + y / w + z`;
\end{verbatim}

\item Primitive recursive (or non-recursive) over known datatype.

\begin{verbatim}
 Define
     `(fold b f [] = b) /\
      (fold b f (h::t) = f h (fold b f t))`;

\end{verbatim}

\item Non-recursive definition, over complex patterns:

\begin{verbatim}
Define
    `(g (0,x,y,z) = 1)
  /\ (g (w,0,y,z) = 2)
  /\ (g (w,x,0,z) = 3)
  /\ (g (w,x,y,0) = 4)`;
\end{verbatim}

\item Recursions (not mutual or nested) that aren't handled by 2.

\begin{verbatim}
  Define `(flatten  []           = [])
     /\   (flatten ([]::rst)     = flatten rst)
     /\   (flatten ((h::t)::rst) = h::flatten(t::rst))`;
\end{verbatim}

\item Nested recursions.
\begin{verbatim}
 Define `N x = if x>100 then x-10 else N(N(x+11))`;
\end{verbatim}

\item Mutual recursion.
\begin{verbatim}
  xDefine "even_odd"
     `(even 0 = T)
  /\  (even (SUC n) = odd n)
  /\  (odd 0 = F)
  /\  (odd (SUC n) = even n)`;
\end{verbatim}

\item Schematic definitions (mutual and nested recursive schemata are
         accepted).
\begin{verbatim}
      Define `While s = if B s then While (C s) else s`;
\end{verbatim}
\end{enumerate}

For complex recursions, {\small\verb+Define+} attempts to find a measure
under which recursive calls become smaller (and to prove that they do
indeed become smaller). Currently, it examines the domain type of the
function being defined and synthesizes a ``size'' measure.  Then it does
some basic simplifications and then attempts to automatically prove the
termination constraints.  If this termination proof fails, then the
definition attempt fails. If the termination proof succeeds, an
induction theorem for the function is also automatically derived and
stored in the current theory.

\noindent {\bf Example.} Invoking
\begin{verbatim}
      Define
        `(gcd 0 y = y)
    /\   (gcd (SUC x) 0 = SUC x)
    /\   (gcd (SUC x) (SUC y) =
             if y <= x then gcd (x-y)   (SUC y)
                       else gcd (SUC x) (y-x))`;
\end{verbatim}
proves all termination conditions and stores the theorem
\begin{verbatim}
      |- (gcd 0 y = y)           /\
         (gcd (SUC x) 0 = SUC x) /\
         (gcd (SUC x) (SUC y) =
              if y <= x then gcd (x - y) (SUC y)
                        else gcd (SUC x) (y - x))
\end{verbatim}
\noindent in the current theory under the name \verb+"gcd_def"+ and
also stores the theorem
\begin{verbatim}
          !P. (!y. P 0 y)       /\
              (!x. P (SUC x) 0) /\
              (!x y. (~(y <= x) ==> P (SUC x) (y - x)) /\
                       (y <= x  ==> P (x - y) (SUC y))
                        ==> P (SUC x) (SUC y))
                 ==>
                   !v v1. P v v1.
\end{verbatim}
\noindent in the current theory under the name \verb+"gcd_ind"+ before
returning the requested recursion equations.

Recall that, if the termination proof fails, an invocation of
\verb+Define+  (or \verb+xDefine+) fails. In such situations, the \ML\
function \verb+Hol_defn+ should be used.

\begin{verbatim}
   Hol_defn     : string -> term quotation -> Defn.defn
   WF_REL_TAC   : Defn.defn -> term quotation -> tactic
\end{verbatim}

\verb+Hol_defn+ makes the requested definition, but defers the proof of
termination to the user. For setting up termination proofs, there are
several useful entrypoints, namely

\begin{verbatim}
   Defn.tgoal  : Defn.defn -> GoalstackPure.proofs
   Defn.tprove : Defn.defn * tactic -> thm * thm
\end{verbatim}

{\small\verb+Defn.tgoal+} is analogous to {\small\verb+set_goal+} and
{\small\verb+Defn.tprove+} is analogous to {\small\verb+prove+}.

\noindent {\bf Example.} An invocation of {\small\verb+Define+} on
the following equations for Quicksort will currently fail, since the
termination proof is beyond the capabilities of our naive termination
prover. Instead, we make an application of {\small\verb+Hol_defn+}:

\begin{verbatim}
    val qsort_def =
      Hol_defn "qsort"
        `(qsort r [] = []) /\
         (qsort r (h::t) =
             APPEND (qsort r (FILTER (\x. r x h) t))
               (h :: qsort r (FILTER (\x. ~(r x h)) t)))`;
\end{verbatim}
which returns a {\small\verb+defn+}, but does not try to prove
termination. Although it is possible to directly work with elements of
type {\small\verb+defn+}, it is more convenient to invoke
`{\small\tt Defn.tgoal qsort\_def}', which sets up a termination
proof in a goalstack. The goal is just to get the unrestricted recursion
equations and induction theorem.

\begin{verbatim}
Defn.tgoal qsort_def;

> val it =
>    Proof manager status: 1 proof.
>    1. Incomplete:
>         Initial goal:
>         ((qsort r [] = []) /\
>          (qsort r (h::t) =
>           APPEND (qsort r (FILTER (\x. r x h) t))
>             (h::qsort r (FILTER (\x. ~r x h) t)))) /\
>         !P.
>           (!r. P r []) /\
>           (!r h t. P r (FILTER (\x. r x h) t) /\
>                    P r (FILTER (\x. ~r x h) t) ==> P r (h::t))
>           ==> !v v1. P v v1
\end{verbatim}

How to proceed? The function {\small\verb+WF_REL_TAC+} now shows its
utility. When given a {\small\verb+defn+} and a quotation denoting a
termination relation for the function, {\small\verb+WF_REL_TAC+}
initiates the termination proof. For our example, we obtain two subgoals
both of which are easy to prove.

\begin{verbatim}
    - e (WF_REL_TAC qsort_def `measure (LENGTH o SND)`
>   OK..
>
>   2 subgoals:
>   val it =
>      !t h r. LENGTH (FILTER (\x. r x h) t) < LENGTH (h::t)
>
>
>      !t h r. LENGTH (FILTER (\x. ~r x h) t) < LENGTH (h::t)
\end{verbatim}


Both goals are provable; once the proof is completed, we can encapsulate
it with {\small\verb+Defn.tprove+}, which takes a
{\small\verb+defn+}, builds a termination goal from it, applies the
given tactic, and, if the initial goal is proved, returns a pair
comprising the requested equations and the induction theorem.

\begin{verbatim}
   val (qsort_eqns,qsort_ind) =
      Defn.tprove
       (qsort_def,
        WF_REL_TAC qsort_def `measure (LENGTH o SND)`
        THEN ...);

   > val qsort_eqns =
   >  |- (qsort r [] = []) /\
   >     (qsort r (h::t) =
   >        APPEND (qsort r (FILTER (\x. r x h) t))
   >           (h::qsort r (FILTER (\x. ~r x h) t)))  : thm

   >  val qsort_ind =
   >   |- !P.
   >      (!r. P r []) /\
   >      (!r h t. P r (FILTER (\x. r x h) t) /\
   >               P r (FILTER (\x. ~r x h) t) ==> P r (h::t))
   >      ==> !v v1. P v v1
\end{verbatim}

\subsection{Automated reasoners}

\verb+bossLib+ brings together the most powerful reasoners in \holn{} and
tries to make it easy to compose them in a simple way. We take our basic
reasoners from \verb+mesonLib+, \verb+simpLib+, and \verb+decisionLib+,
but the point of \verb+bossLib+ is to provide a layer of abstraction so
the user has to know only a few entrypoints.\footnote{In the mid 1980's
Graham Birtwistle advocated such an approach, calling it `Ten Tactic
HOL'.}

\begin{verbatim}
    PROVE      : thm list -> term quotation -> thm
    PROVE_TAC  : thm list -> tactic

    DECIDE     : term quotation -> thm
    DECIDE_TAC : tactic
\end{verbatim}

The inference rule \underline{\tt PROVE} (and the corresponding tactic
\underline{\tt PROVE\_TAC}) takes a list of theorems and a quotation, and
attempts to prove the term using a first order reasoner. The inference
rule \underline{\tt DECIDE} (and the corresponding tactic \underline{\tt
DECIDE\_TAC}) applies a decision procedure that (at least) handles
statements of linear arithmetic.

\begin{verbatim}
    RW_TAC   : simpset -> thm list -> tactic
    &&       : simpset * thm list -> simpset  (* infix *)
    std_ss   : simpset
    arith_ss : simpset
    list_ss  : simpset
\end{verbatim}

The rewriting tactic \underline{\tt RW\_TAC} works by
first adding the given theorems into the given \underline{\tt simpset}; then it
simplifies the goal as much as possible; then it performs case splits on any
conditional expressions in the goal; then it repeatedly (1) eliminates
all hypotheses of the form $v = M$ or $M = v$ where $v$ is a variable
not occurring in $M$, (2) breaks down any equations between constructor
terms occurring anywhere in the goal. The infix combinator \verb+&&+ is used to
build a new simpset from a given simpset and a list of theorems.

Simplification sets for its native datatypes are provided
by \verb+bossLib+. In general, these are extended versions of those
found in \verb+simpLib+. The simpset for pure logic, sums, pairs, and the
\verb+option+ type is named \verb+std_ss+. The simpset for arithmetic
is named \verb+arith_ss+, and the simpset for lists is named
\verb+list_ss+. The simpsets provided by {\tt bossLib} strictly increase
in strength: {\tt std\_ss} is contained in {\tt arith\_ss}, and {\tt
arith\_ss} is contained in {\tt list\_ss}.

\begin{verbatim}
    STP_TAC  : simpset -> tactic -> tactic
    ZAP_TAC  : simpset -> thm list -> tactic
\end{verbatim}

The compound reasoners of \verb+bossLib+ take a basic approach: they
simplify the goal as much as possible with \verb+RW_TAC+ and then a
`finishing' tactic is applied. The primitive entrypoint for this is
\underline{\tt STP\_TAC}. Currently, the most powerful reasoner is
\underline{\tt ZAP\_TAC}, which features a finishing tactic that first
tries a tautology checking tactic; if that fails, \verb+DECIDE_TAC+ is
called; if that fails, \verb+PROVE_TAC+ is called with the second
argument. Although this general approach (simplify as much as possible,
then apply automated reasoners in sequence) is crude, we have found that
it allows one to make good progress in a high percentage of proof
situations.

\begin{verbatim}
    by : term quotation * tactic -> tactic (* infix 8 *)
    SPOSE_NOT_THEN : (thm -> tactic) -> tactic
\end{verbatim}

The function \underline{\tt by} is an infix operator that takes a
quotation and a tactic $tac$. The quotation is parsed into a term
$M$. When the invocation ``$M\ \mbox{\tt by}\ tac$'' is applied to a
goal $(A,g)$, a new subgoal $(A,M)$ is created and $tac$ is applied to
it. If the goal is proved, the resulting theorem is broken down and
added to the assumptions of the original goal; thus the proof proceeds
with the goal $((M::A), g)$. (Note however, that case-splitting will
happen if the breaking-down of $\ \vdash M$ exposes disjunctions.) Thus
\underline{\tt by} allows a useful style  of `assertional' or
`Mizar-like' reasoning to be mixed with ordinary tactic
proof`\footnote{Proofs in the Mizar system are readable documents,
unlike almost all tactic-based proofs.}


\underline{\tt SPOSE\_NOT\_THEN} initiates a proof by
contradiction by assuming the negation of the goal and driving the
negation inwards through quantifiers. It provides the resulting theorem
as an argument to the supplied function, which will use the theorem to
build and apply a tactic.

\noindent{\bf Note.} When the library \verb+bossLib+ is loaded, the
infix parsing status of \verb+&&+ and ``{\tt by}'' must be re-asserted
by the user.
\section{Record types}

Record types are convenient ways of bundling together a number of
component types, and giving those components names so as to facilitate
access to them.  Record types are semantically equivalent to big pair
(cross-product) types, but the ability to label the fields with names
of one's own choosing is a great convenience.  Record types as
implemented in \holn{} are similar to C's {\tt struct} types and to
Pascal's records.

Done correctly, record types provide useful maintainability features.
If one can always access the {\tt fieldn} field of a record type by
simply writing {\tt record.fieldn}, then changes to the type that
result in the addition or deletion of other fields will not invalidate
this reference.  One failing in SML's record types is that they do not
allow the same maintainability as far as (functional) updates of
records are concerned.  The HOL implementation allows one to write
{\tt rec with fieldn := new\_value}, which replaces the old value of
{\tt fieldn} in the record {\tt rec} with {\tt new\_value}.  This
expression will not need to be changed if another field is added,
modified or deleted from the record's original definition.

\subsection{Defining a record type}

Defining a record type is achieved with the function
\texttt{Hol\_datatype}, as previously discussed.  For example, to
create a record type called {\tt person} with boolean, string and
number fields called {\tt employed}, {\tt name} and {\tt age}, one
would enter:
\begin{verbatim}
val _ = Hol_datatype `person = <| employed : bool ; age : num ;
                                  name : string
                               |>`;
\end{verbatim}
The order in which the fields are entered is not significant. As well
as defining the type (called {\tt person}), the datatype definition
function also defines three other sets of constants.  These are the
field access functions, update functions, and functional update
functions.  The field access functions have names of the form
``$\langle$\textsl{record-type\/}$\rangle$\_$\langle$\textsl{field\/}$\rangle$''.
These functions can be used directly, or one can use standard field
selection notation to access the values of a record's field.  Thus,
one would write the expression: \mbox{\tt ``bob.employed``} in order
to return the value of {\tt bob}'s {\tt employed} field.  The
alternative, \texttt{``person\_employed bob``}, works, but would be
printed using the first syntax, with the full-stop.

The update functions are given the names
\mbox{``$\langle$\textsl{record-type\/}$\rangle$\_%
$\langle$\textsl{field\/}$\rangle$\_\texttt{update}''} for each
field in the type.  They take a value of the type of the field in
question and a record value to be modified.  They return a new record
value that is otherwise the same as the old value but with the
specified field having the new value.  They can be written with the
keyword \texttt{with} and the \texttt{:=} operator:
\begin{verbatim}
  ``bob with employed := T``
\end{verbatim}
\noindent If a chain of updates is desired, then multiple updates can
be specified inside \texttt{<|}-\texttt{|>} pairs, separated by
semi-colons, thus:

\begin{verbatim}
  ``bob with <| age := 10; name := "Child labourer" |>``
\end{verbatim}

\noindent Finally, the second sort of update functions, the so-called
``functional'' updates have names of the form
\mbox{``$\langle$\textsl{record-type\/}$\rangle$\_%
$\langle$\textsl{field\/}$\rangle$\_\texttt{fupd}''}. Rather than
specifying a new value for the record, these functions take a function
as their first parameter, which will be an endomorphism on the field
type, so that the resulting record is the same as the original, except
that the specified field has had the given function applied to it to
generate the new value for that field.  The functional update
functions allow more concision when writing updates on a record that
depend on the field's old value.

The special syntax for writing these updates is to again use the
\texttt{with} keyword, but to use the infix \texttt{updated\_by}
rather than \texttt{:=}.  Thus
\begin{alltt}
  ``bob with employed updated_by \$~``
\end{alltt} \noindent
is a record value with the opposite boolean value in the
\texttt{employed} field as held by \texttt{bob}.

\subsection{Specifying record literals}

The parser accepts lists of field specifications between
\texttt{<|}-\texttt{|>} pairs without the \texttt{with} keyword.
These translate to sequences of updates of an arbitrary value
(literally, the HOL value \texttt{ARB}), and are treated as literals.
Thus,
\begin{verbatim}
  ``<| age := 21; employed := F; name := "Layabout" |>``
\end{verbatim}

\subsection{Using the theorems produced by record definition}

As well as defining the type and the functions described above, record
type definition also proves a suite of useful theorems.  Most of these
are returned in a big record; all are stored using {\tt save\_thm} so
that they can be recovered.

The record returned has the following fields:
\newcommand{\rewruse}{This theorem should be included in rewrites used
  for this type.}
\newcommand{\field}[1]{\mbox{\it field}_{#1}}
\newcommand{\update}{\mbox{\tt\_update}}
\begin{description}
\item[{\tt type\_axiom}] The type axiom for the record type, as
  returned by the standard datatype definition package.
\item[{\tt accessor\_fns}] The definitions of the accessor functions.
  \rewruse
\item[{\tt update\_fns}] The definitions of the update functions.
  \rewruse
\item[{\tt cases\_thm}] The usual cases theorem for a type, stating
  that for all record values, there exist component values making it
  up.
\item[{\tt fn\_upd\_thm}] The definitions of the functional update
  functions.  \rewruse
\item[{\tt acc\_upd\_thm}] A theorem stating simpler forms for
  expressions of the form $\field{i}\, (\field{j}\update\;v\; r)$.  If
  $i = j$, then the RHS is $v$, if not, it is $(\field{i}\;r)$.
  \rewruse
\item[{\tt upd\_acc\_thm}] A theorem stating that $\field{i}\update
  \;(\field{i}\;r)\;r = r$ for all of
  the fields defined in the type. \rewruse
\item[{\tt upd\_upd\_thm}] A thereom stating that $\field{i}\update
  \;v_1 \,(\field{i}\update \;v_2\;r) = \field{i}\update\;v_1\;r$.
  \rewruse
\item[{\tt upd\_canon\_thm}] A theorem that states commutativity results
  for all possible pairs of field updates.  They are constructed in
  such a way that if used as rewrites, they will canonicalise
  sequences of updates. \rewruse
\item[{\tt cons\_11\_thm}] The standard result stating the type
  constructor is injective.  \rewruse
\item[{\tt create\_term}] This last component of the record returned
  is not a theorem, but rather an ML function.  It is identical to the
  {\tt create\_term\_fn} already defined in {\tt RecordType}, but is
  pre-applied to the relevant arguments, so that it is of the type
  string-value list to term.
\end{description}


\section{The {\tt meson} library}
\section{The {\tt simp} library}
\section{The {\tt num} library}

\newpage
\section{The type definition package}\label{types-package}\index{extension, of HOL logic@extension, of \HOL\ logic!by type definition|(}
\index{type definition package, in HOL system@type definition package, in \HOL\ system|(}

\begin{center}
\framebox{\parbox{0.8\textwidth}{\it All of this section of the
    documentation is out of date.  Users of \holn{} should use
    \texttt{Hol\_datatype} to define types, and \texttt{Define} or
    \texttt{xDefine} to define functions over them.  The functions
    described here do exist in \holn{} but generally with different
    types.  This section was written for HOL88 and has not been
    updated since.}}
\end{center}

In the \HOL\ system, new types and type operators can be
introduced\index{extension, of HOL logic@extension, of \HOL\ logic}
using the consistency-preserving definitional mechanism of type
definitions\index{type definition extension, in HOL logic@type
  definition extension, in \HOL\ logic|(} (see Sections~\ref{tydefs}
and~\ref{type-defs}).  The \ML\ rule for introducing a new type is:

\begin{hol}
\index{new_type_definition@\ml{new\_type\_definition}}
\begin{verbatim}
   new_type_definition : (string # term # thm) -> thm
\end{verbatim}\end{hol}

\noindent This rule allows
axioms of a restricted form to be added to the primitive basis of the logic.
These axioms are analogous to definitional axioms for new constants: they
define new types in terms of other type expressions already present in the
logic. Like the rule \ml{new\_definition} for making constant definitions,
the rule  \ml{new\_type\_definition}
 for type definitions
ensures that adding a new syntactic entity (in this case, a type or
type operator) is a conservative extension of the logic.

The basic idea behind \ml{new\_type\_definition} is  that a  type definition is
made by  adding an  axiom to  the logic  which asserts  that the  set of values
denoted by a  new type  is isomorphic\index{isomorphism of types, in HOL logic@isomorphism of types, in \HOL\ logic}  to an  appropriate subset  of the values
denoted by  a type  expression already  present in  the logic.   A definitional
axiom\index{definitional axioms} of this form merely states
that a  new type  is isomorphic  to a particular
subset of an existing type.  From such type definition  axioms, it  is usual to
prove theorems that characterize newly-defined types more abstractly.  The idea
is to prove a collection of theorems that state  the essential  properties of a
new type without reference to how it is defined.   These  theorems then
constitute a derived `abstract axiomatization' of the new type, and
once  they have  been proved they
become the basis for all further reasoning about it.

With this approach, introducing a new type (or type operator) in \HOL\
involves two distinct steps:

\setcounter{myenumi}{1}
\begin{list}{\arabic{myenumi}.}{\usecounter{myenumi}
\setlength{\leftmargin}{10mm}
\setlength{\rightmargin}{5mm}
\setlength{\labelwidth}{3mm}
\setlength{\labelsep}{2mm}
\setlength{\listparindent}{0mm}
\setlength{\itemsep}{8pt plus1pt minus2pt}
\setlength{\topsep}{3mm}
\setlength{\parsep}{0mm}}
\setlength{\abovedisplayshortskip}{8pt plus1pt minus1pt}
\setlength{\belowdisplayshortskip}{8pt plus1pt minus1pt}

\item Finding an appropriate representation for the new type, and making a type \mbox{definition} using \ml{new\_type\_definition} based
 on this representation.

\item Using the axiomatic definition of the new type and the properties of its
representation to prove a set of theorems that abstractly characterizes it.

\end{list}

Defining a new type using this approach can be hard work.  But a set of tools
is provided in the system which---for a certain class of commonly-used {\it
concrete recursive types\/}\index{types, in HOL logic@types, in \HOL\
logic!tools for construction of}\index{concrete recursive types, in HOL
logic@concrete recursive types, in \HOL\ logic}---automatically carries out all
the formal proofs necessary to define these types and derive abstract
characterizations from their definitions.  This section provides a user-level
overview of these tools.  Details of the formal proofs carried out by these
tools are discussed in~\cite{Melham-banff}.

\subsection{Defining types}

\index{type definition package, in HOL system@type definition package, in \HOL\ system!use of, in recursive type definition|(}
\index{recursive types, in HOL logic@recursive types, in \HOL\ logic!tools for construction of|(}
The main \ML\ function in the \HOL\ type definition package is

\begin{boxed}
\index{define_type@\ml{define\_type}|pin}
\begin{verbatim}
   define_type : string -> string -> thm
\end{verbatim}\end{boxed}

\noindent This function can be used to define any concrete recursive type in
the \HOL\ system.  These are types whose values are generated by a set of {\it
constructors\/} (i.e.\ functions) which yield concrete representations for
these values.  Examples include types which denote finite sets of atomic values
(enumerated types), types which denote sets of structured values (record types)
or finite disjoint unions of structured values (variant records), and types
which denote sets of recursive data structures (recursive types).

The two inputs to \ml{define\_type} are both strings.  The first string
is a name  under which the results of making
the type definition will be stored in the current theory segment. The second is
a user-supplied informal\footnote{ In this context, {\it informal\/} means not
in the language of higher order logic.} specification of the concrete
recursive type to be defined.  This type specification is written in
a notation (explained below) which resembles
a data type declaration in functional programming languages like Standard
\ML~\cite{sml}.  It simply states the names of the new type's
constructors and the logical types of their \mbox{arguments.}  The output is a
theorem which abstractly characterizes the properties of
the desired recursive type---i.e.\ a
derived `abstract axiomatization' of the type.

\subsubsection{Input syntax}

The type specification given as input\index{type definition package, in HOL system@type definition package, in \HOL\ system!input to|(} to \ml{define\_type}
must be an \ML\ string\index{strings, in ML@strings, in \ML!as input to HOL type definition package@as input to \HOL\ type definition package} (of \ML\ type \ml{string}) of the form:

{\def\op{{\normalsize\sl op}}
\begin{hol}\begin{alltt}
   `{\op} = \(C\sb{1}\;ty{}\sb{1}\sp{1}\;\ldots\;ty{}\sb{1}\sp{k\sb{1}} \) | \(\cdots\) | \(C\sb{m}\;ty{}\sb{m}\sp{1}\;\ldots\;ty{}\sb{m}\sp{k\sb{m}}\)`
\end{alltt}\end{hol}}

\noindent where each $ty_i^{j}$ is either a type expression already defined
as a type in the current theory (this type expression must not
contain \ty{op}) or is the name \ty{op} itself.  A string of this form
describes an $n$-ary type operator \ty{op},
where $n$ is the number of distinct type variables in the types
$ty_i^{j}$ on the right hand side of the equation.
If $n = 0$ then \ty{op} is a type constant;
otherwise \ty{op} is an $n$-ary type operator.  The concrete
type described has $m$
distinct constructors \m{C_1, \dots, C_m} where $m \geq 1$.
Each constructor \m{C_i} takes $k_i$ arguments, where $k_i \geq 0$;
and the types of these arguments are given by the type
expressions $ty_i^j$ for $1 \leq j \leq k_i$.  If one or more of the type
expressions $ty_i^{j}$ is the type \ty{op} itself, then the \mbox{equation}
specifies a {\it recursive\/} type.  In any specification of a recursive type,
at least one constructor must be non-recursive---i.e.\ all its arguments must
have types which already exist in the current theory.

The input parser for \ml{define\_type} treats type expressions exactly as the
\HOL\ quotation parser does, with precedences among the various built-in type
operators in force.\index{type definition package, in HOL system@type definition package, in \HOL\ system!input to|)}

\subsubsection{The type specified}

The logical type described by an input string of the form
shown above is intended
to denote the set of all values which can be finitely
generated using the constructors \m{C_1, \dots, C_m},
where each constructor is one-to-one and any two
different constructors yield different values.   Every value of this
type will be denoted by some term of the form:

\[ C_i\;x_i^1\;\ldots\;x_i^{k_i} \]

\noindent where $x_i^j$ is a term of type $ty_i^j$ for $1 \leq j \leq
k_i$. In addition, any two terms:

\[ C_i\;x_i^1\;\ldots\;x_i^{k_i} \qquad {\rm and} \qquad
   C_j\;x_j^1\;\ldots\;x_j^{k_j} \]

\noindent denote equal values exactly when their constructors are the same
(i.e.\ $i=j$) and these constructors are applied to equal arguments
(i.e.\ $x_i^n = x_j^n\;\,{\rm for}\;\,1\leq n\leq k_i$).

\subsubsection{The output}\label{define-type-output}
\index{type definition package, in HOL system@type definition package, in \HOL\ system!output of|(}

\noindent For any type specification
in the form of an equation of the kind discussed above, executing:

{\def\op{{\normalsize\sl op}}
\begin{hol}
\index{define_type@\ml{define\_type}}
\begin{alltt}
   define\_type `\m{name}` `{\op} = \(C\sb{1}\;ty{}\sb{1}\sp{1}\;\ldots\;ty{}\sb{1}\sp{k\sb{1}} \) | \(\cdots\) | \(C\sb{m}\;ty{}\sb{m}\sp{1}\;\ldots\;ty{}\sb{m}\sp{k\sb{m}}\)`
\end{alltt}\end{hol}}

\noindent will make a formal definition for a
type (or  type operator) \ty{op}
in the current theory segment, make appropriate definitions for
constants  \m{C_1, C_2,\dots, C_m},
and automatically prove a theorem which
provides an abstract
characterization\index{characterizing theorem!for defined types}\index{automated derivation!of characterizing theorems for recursive types}
of the newly-defined type \ty{op}.  This theorem, which is
stored in the current theory segment under the name $name$ and also returned by
\ml{define\_type}, has the form shown below:

{\def\op{{\normalsize\sl op}}
\begin{hol}\begin{alltt}
   |- !f\(\sb{1}\:\cdots\:\)f\(\sb{m}\). ?!fn:{\op}->*.
         !x\(\sb{1}\sp{1}\;\,\cdots\,\;\)x\(\sb{1}\sp{k\sb{1}}\). fn(\m{C}\(\sb{1}\,\) x\(\sb{1}\sp{1}\;\,\ldots\,\;\)x\(\sb{1}\sp{k\sb{1}}\)) = f\(\sb{1}\) (fn x\(\sb{1}\sp{1}\))\(\;\ldots\;\)(fn x\(\sb{1}\sp{k\sb{1}}\)) x\(\sb{1}\sp{1}\;\ldots\;\)x\(\sb{1}\sp{k\sb{1}}\)
                                      \(\vdots\)
         !x\(\sb{m}\sp{1}\;\cdots\;\)x\(\sb{1}\sp{k\sb{m}}\!\). fn(\m{C}\(\sb{m}\) x\(\sb{m}\sp{1}\;\ldots\;\)x\(\sb{m}\sp{k\sb{m}}\)) = f\(\sb{m}\) (fn x\(\sb{m}\sp{1}\))\(\;\ldots\;\)(fn x\(\sb{m}\sp{k\sb{m}}\)) x\(\sb{m}\sp{1}\;\ldots\;\)x\(\sb{m}\sp{k\sb{m}}\)
\end{alltt}\end{hol}}


\noindent where the right hand sides of the equations include recursive
applications `$\ml{fn}\;\ml{x}_i^j$' only for variables  $\ml{x}_i^j$ of type
\ty{op}.  (See the examples given below.) A theorem of this form asserts the
unique existence of primitive recursive functions defined by cases on the
constructors \m{C_1, C_2,\dots,C_m}.  This is a slight
extension of the {\it initiality\/}\index{initiality} property by which
structures of this kind are characterized in the `initial algebra' approach to
specifying abstract data types~\cite{goguen}.  This property provides an
abstract characterization of the type \ty{op} which is both succinct and
complete, in the sense that it completely determines the structure of the
values of \ty{op} up to isomorphism.

The call to \ml{define\_type} shown above fails if:

\begin{myenumerate}

\item not in draft mode\index{draft mode, in HOL system@draft mode, in \HOL\ system};

\item \ty{op} is already the name of a type constant or type operator in the
current theory;

\item any one of $C_1,\dots,C_{m}$ is already the name
of a constant in the current theory.

\item either \ty{op} or any one $C_{1},\dots,C_{m}$ is not a
legal identifier.  Identifiers must start with a letter (as defined by
\ml{is\_letter}) and contain only alphanumeric characters (as defined by
\ml{is\_alphanum})

\item $\ml{ABS\_}\ty{op}$\index{ABS_@\ml{ABS\_}$\ldots$}
 or $\ml{REP\_}\ty{op}$\index{REP_@\ml{REP\_}$\ldots$}
 are already constants in the
current theory;

\item there is already an axiom, definition, constant specification or type
definition stored under either the name
$\ty{op}\ml{\_TY\_DEF}$\index{TY_DEF@$\ldots$\ml{\_TY\_DEF}} or the name
$\ty{op}\ml{\_ISO\_DEF}$ in the current
theory segment.

\item there is already a theorem stored under the name \ml{`\m{name}`} in the
current theory segment.

\item the input type specification does not conform to the syntax described
above.

\end{myenumerate}
\index{recursive types, in HOL logic@recursive types, in \HOL\ logic!tools for construction of|)}\index{type definition extension, in HOL logic@type definition extension, in \HOL\ logic|)}
\index{type definition package, in HOL system@type definition package, in \HOL\ system!output of|)}
\index{type definition package, in HOL system@type definition package, in \HOL\ system!use of, in recursive type definition|)}

\subsubsection{Examples}\label{define-type-example}

\index{primitive recursion theorem!a degenerate case|(}
The session that follows illustrates the use of \ml{define\_type} in defining
a variety of simple concrete types.  It is assumed that the session begins
with the user in draft mode.

The first definition is simple, the definition of a type \ml{three} with
exactly three distinct values: \ml{ONE}, \ml{TWO}, and \ml{THREE}.

\setcounter{sessioncount}{1}\label{types-session}
\begin{session}\begin{verbatim}
#let three_Axiom = define_type `three_Axiom` `three = ONE | TWO | THREE`;;
three_Axiom =
|- !e0 e1 e2. ?! fn. (fn ONE = e0) /\ (fn TWO = e1) /\ (fn THREE = e2)
\end{verbatim}\end{session}

\noindent The theorem returned by \ml{define\_type} provides a complete
and abstract characterization of
a defined logical type \ml{three} which denotes a set of
exactly three elements.  This characterization takes the form of a
degenerate `primitive recursion'
 theorem for the concrete type \ml{three}.
Since \ml{three} is an enumerated type with no recursive constructors,
the theorem returned by \ml{define\_type} simply
states that any function defined by cases on the three constants  \ml{ONE},
\ml{TWO}, and \ml{THREE} exists and is uniquely defined.

It follows immediately from this theorem that the type constant \ml{three}
denotes a set containing exactly three values: the fact that the function
\ml{fn} always exists implies that the constants \ml{ONE}, \ml{TWO}, and
\ml{THREE} denote distinct values of type \ml{three}, and the fact that \ml{fn}
is uniquely determined by its values for \ml{ONE}, \ml{TWO}, and \ml{THREE}
implies that these constants denote the only values of type \ml{three}.
\index{primitive recursion theorem!a degenerate case|)}

The next call to \ml{define\_type} defines a `record type' \ml{rec}, values of
which are records with three boolean fields (essentially 3-tuples):

\begin{session}\begin{verbatim}
#let rec_Axiom = define_type `rec_Axiom` `rec = REC bool bool bool`;;
rec_Axiom = |- !f. ?! fn. !b0 b1 b2. fn(REC b0 b1 b2) = f b0 b1 b2
\end{verbatim}\end{session}\label{rec-def}

\noindent Here, the resulting theorem states that a function \ml{fn} on
record values of type \ml{rec} can be
uniquely defined in terms of a function \ml{f} of the three components of
the record.

A more interesting {\it recursive\/}
  example is the type of natural
numbers\index{number theory, in HOL logic@number theory, in \HOL\ logic!type definition package version of}
\index{type definition package, in HOL system@type definition package, in \HOL\ system!use of, to build number theory}, which can be defined using \ml{define\_type} as follows:

\begin{session}\begin{verbatim}
#let nat_Axiom = define_type `nat_Axiom` `nat = Z | Suc nat`;;
nat_Axiom = |- !e f. ?! fn. (fn Z = e) /\ (!n. fn(Suc n) = f(fn n)n)
\end{verbatim}\end{session}

\noindent Here, the input string describes a type \ml{nat} with two
constructors: \ml{Z}, which stands for zero; and \ml{Suc}, which is the
successor function on natural numbers.  (The names
\ml{Z}, and \ml{Suc} are used here because \ml{0} and \ml{SUC} are already
constants in the built-in \HOL\ theory \ml{num}.)
The output theorem is just the primitive recursion
theorem\footnote{See Section~\ref{prim_rec} for a discussion of
the primitive recursion theorem.}
for the natural numbers; it states that any primitive recursive definition on
the natural numbers (\ie\ on values of type \ml{nat})
uniquely defines a total function.

A recursive type of labelled binary trees\index{binary tree theory, in HOL logic@binary tree theory, in \HOL\ logic|(}, where labels of type \ml{*}
appear only on leaf nodes, can likewise be defined using \ml{define\_type}.
The input states that a binary tree is either a leaf node (\ml{LEAF})
labelled by a value of type \ml{*} or an internal node \ml{NODE} with
two binary trees as subtrees:

\begin{session}\begin{verbatim}
#let btree_Axiom =
#    define_type `btree_Axiom` `btree = LEAF * | NODE btree btree`;;
btree_Axiom =
|- !f0 f1.
    ?! fn.
     (!x. fn(LEAF x) = f0 x) /\
     (!b1 b2. fn(NODE b1 b2) = f1(fn b1)(fn b2)b1 b2)
\end{verbatim}\end{session}\label{btree-def}

\noindent The result returned by the call to {\small \verb!define_type!} is,
in this case, an abstract
characterization for a defined type {\small\verb!(*)btree!},
in the form of a `primitive recursion theorem' for the required
type of labelled binary trees.\index{binary tree theory, in HOL logic@binary tree theory, in \HOL\ logic|)}

Any simple  concrete  recursive  type  can  be  defined  automatically  from  a
user-supplied equation using \ml{define\_type} in exactly the same way.

\subsection{Defining recursive functions}\label{prim-rec-defs}

\index{type definition package, in HOL system@type definition package, in \HOL\ system!use of, in recursive function definition|(}
\index{recursive definitions, in HOL logic@recursive definitions, in \HOL\ logic!automated, for recursive types|(}
\index{primitive recursion theorem!for binary trees|(}
An important property of the characterizing theorems for concrete types
shown in the examples given above is that they
provide a formal
means for defining recursive functions on those types.
When a concrete
recursive type  \ty{op} is
 characterized by a theorem of the kind returned by
\ml{define\_type}\index{types, in HOL logic@types, in \HOL\ logic!tools for construction of} (see Section~\ref{define-type-output}) this theorem
can be used to
prove the existence of any
{\it primitive recursive\/} function on \ty{op} and to
define constants which denote such functions.

This is illustrated
for a particular
example by the method of defining primitive recursive functions on
the natural numbers  discussed in Section~\ref{num-prim-rec}.  In that section,
an \ML\ function \ml{new\_prim\_rec\_definition}\index{new_prim_rec_definition@\ml{new\_prim\_rec\_definition}}
 was described which automates
the logical inferences necessary to derive particular primitive recursive
definitions on the built-in defined type \ml{num} of natural numbers.  The
basis of this function is the primitive recursion theorem

\begin{hol}
\index{num_Axiom@\ml{num\_Axiom}}
\begin{verbatim}
   num_Axiom   |- !x f. ?!fn. (fn 0 = x) /\ (!n. fn(SUC n) = f (fn n) n)
\end{verbatim}\end{hol}

\noindent which is pre-proved and stored in the built-in theory \ml{prim\_rec}\index{prim_rec@\ml{prim\_rec}}
(see Section~\ref{prim_rec}).  The \ML\ function
\ml{new\_prim\_rec\_definition} uses \ml{num\_Axiom} to automate the
justification of any user-supplied primitive recursive definition on the
natural numbers.

The type definition package\index{primitive recursion theorem!automated use of, in HOL system@automated use of, in \HOL\ system|(} provides a similar function for defining
primitive recursive functions on
arbitrary concrete recursive
types.\footnote{In fact, {\tt new\_prim\_rec\_definition} is defined in ML
using the more general tools provided by the type definition package.}
The \ML\ function

\begin{boxed}\index{new_recursive_definition@\ml{new\_recursive\_definition}|pin}
\begin{verbatim}
   new_recursive_definition : bool -> thm -> string -> term -> thm
\end{verbatim}\end{boxed}

\noindent automates the
inferences necessary to justify any given primitive recursive definition on a
concrete recursive type of the kind definable by \ml{define\_type}.
 It takes four arguments.  The first is a boolean
flag which indicates if the function to be defined will be an infix\index{infixes, in HOL logic@infixes, in \HOL\ logic!in recursive type definitions} or not.
The second is the primitive recursion theorem for the concrete type in question
(\ie\ a theorem obtained from {\small\verb!define_type!}).\index{define_type@\ml{define\_type}}  The third
argument is a name under which the resulting definition will be saved in the
current theory segment.
The fourth argument is a term giving the desired primitive recursive
definition.  The value returned
by  {\small\verb!new_recursive_definition!} is a theorem
which states the primitive recursive definition requested by the
user.  This theorem is derived by formal proof from an instance of the general
primitive recursion theorem\index{automated derivation!of recursive definitions}
 given as the second argument.

If the \ML\ variable \ty{op}\ml{\_Axiom} is bound to a theorem of the form
returned by \ml{define\_type},
then evaluating:

{\def\op{{\normalsize\sl op}}
\begin{hol}\begin{alltt}
   new_recursive_definition
     `\m{flag}` \op\_Axiom `\m{name}` "{\normalsize\it primitive recursive definition on \op}"
\end{alltt}\end{hol}}

\noindent  automatically proves the existence of the primitive recursive
function supplied as the fourth argument,
and then declares a new constant in the current theory
with this definition as its
specification. This constant specification is returned as a theorem
and is saved in the current theory segment under the name
$name$.
If $flag$ is \ml{true}, the constant is given infix status.
Failure occurs if:

\begin{myenumerate}
\item \HOL\ cannot prove there is a function
satisfying the defining equations supplied by the user
(\ie\ the term supplied to \ml{new\_recursive\_definition}
 is not a well-formed primitive recursive definition on values
of type \ty{op});
\item any other condition for making a constant specification is violated
(see the failure conditions for \ml{new\_specification} in
Section~\ref{conspec}).
\end{myenumerate}

Curried\index{currying, in ML@currying, in \ML!in recursive definitions}
functions defined using \ml{new\_recursive\_definition} can be
recursive on any one of their arguments.  Furthermore, defining equations need
not be given for all the constructors of the concrete type in question.  See
the examples given in the next section, or the examples of functions
defined on \ml{num} given in Section~\ref{num-prim-rec} for more details.

The \ML\ function

\begin{boxed}
\index{prove_rec_fn_exists@\ml{prove\_rec\_fn\_exists}|pin}
\begin{verbatim}
   prove_rec_fn_exists : thm -> term -> thm
\end{verbatim}\end{boxed}

\noindent is a version of \ml{new\_recursive\_definition} which proves only
that the required function exists; it does not make a constant specification.
The first argument is a theorem of the form returned by \ml{define\_type},
and the second is a user-supplied primitive recursive function definition.
The theorem which is returned asserts the existence of the recursively-defined
function in question (if it is primitive recursive over the type characterized
by the theorem given as the first argument).
\index{recursive definitions, in HOL logic@recursive definitions, in \HOL\ logic!automated, for recursive types|)}
\index{type definition package, in HOL system@type definition package, in \HOL\ system!use of, in recursive function definition|)}
\index{primitive recursion theorem!automated use of, in HOL system@automated use of, in \HOL\ system|)}

\subsubsection{More examples}

Continuing the example session started above in Section~\ref{define-type-example},
the following interactions with the system show how the \ML\ function
\ml{new\_recursive\_definition} can be used to define functions on concrete types,
which have themselves been defined using \ml{define\_type}.

Given the characterizing theorem
\ml{btree\_Axiom} for the type of labelled binary trees
defined in Section~\ref{types-session}, a recursive function \ml{Leaves}, which
computes the number of leaf nodes in a binary tree,
can be defined recursively in \HOL\ as shown below:

\begin{session}\begin{verbatim}
#let Leaves =
#    new_recursive_definition false btree_Axiom `Leaves`
#      "(Leaves (LEAF (x:*)) = 1) /\
#       (Leaves (NODE t1 t2) = (Leaves t1) + (Leaves t2))";;
Leaves =
|- (!x. Leaves(LEAF x) = 1) /\
   (!t1 t2. Leaves(NODE t1 t2) = (Leaves t1) + (Leaves t2))
\end{verbatim}\end{session}

\noindent The result of the call to {\small\verb!new_recursive_definition!} is
a theorem which states that the constant {\small\verb!Leaves!} satisfies the
primitive-recursive defining equations supplied by the user.  This theorem is
derived automatically from an instance of the general primitive recursion
theorem for binary trees ({\small\verb!btree_Axiom!}) and an appropriate
constant specification for the constant {\small\verb!Leaves!}.

The function defined using \ml{new\_recursive\_definition}\index{type
definition package, in HOL system@type definition package, in \HOL\
system!use of, in case definition} need not, in fact, be recursive.
Here is the definition of a predicate \ml{IsLeaf}, which is true of
binary trees which are leaves, but is false of the internal nodes in a
binary tree:

\begin{session}\begin{verbatim}
#let IsLeaf =
#    new_recursive_definition false btree_Axiom `IsLeaf`
#      "(IsLeaf (NODE t1 t2) = F) /\ (IsLeaf (LEAF (x:*)) = T)";;
IsLeaf = |- (!t1 t2. IsLeaf(NODE t1 t2) = F) /\ (!x. IsLeaf(LEAF x) = T)
\end{verbatim}\end{session}

\noindent Note that two equations defining a (recursive or non-recursive)
function on binary trees by cases can be given in either order.  Here, the
\ml{NODE} case is given first, and the \ml{LEAF} case second.  The reverse
order was used in the above definition of \ml{Leaves}.

The \ML\ function  {\small\verb!new_recursive_definition!} also allows the user
to partially specify\index{type definition package, in HOL system@type definition package, in \HOL\ system!use of, in partial definition} the value of a function defined on a concrete type, by
allowing defining equations for some of the constructors to be omitted.  Here,
for example, is the definition of a function \ml{Label} which extracts the
label from a leaf node.  The value of \ml{Label} applied to an internal node
is left unspecified:

\begin{session}\begin{verbatim}
#let Label =
#    new_recursive_definition false btree_Axiom `Label`
#      "Label (LEAF (x:*)) = x";;
Label = |- !x. Label(LEAF x) = x
\end{verbatim}\end{session}

\index{type definition package, in HOL system@type definition package, in \HOL\ system!use of, in curried infix definition|(}
Curried functions can also be defined, and the recursion can be on any
argument.  The next definition defines an infix (curried)
function \ml{<<} which expresses the idea that one tree is a proper
subtree of another.

\begin{session}\begin{verbatim}
#let Subtree =
#    new_recursive_definition true btree_Axiom `Subtree`
#      "(<< (t:(*)btree) (LEAF (x:*)) = F) /\
#       (<< t (NODE t1 t2) = ((t=t1) \/ (t=t2) \/ (<< t t1) \/ (<< t t2)))";;
Subtree =
|- (!t x. t << (LEAF x) = F) /\
   (!t t1 t2.
     t << (NODE t1 t2) = (t = t1) \/ (t = t2) \/ t << t1 \/ t << t2)
\end{verbatim}\end{session}

\noindent Note that the first argument to the \ML\ function is \ml{true}
(to indicate that the function being defined is to have infix status) and that
the constant \ml{<<} is an infix after the definition has been made.
Furthermore, the function \ml{<<} is recursive on its second argument\index{type definition package, in HOL system@type definition package, in \HOL\ system!use of, in curried infix definition|)}.

Finally, the function {\small\verb!new_recursive_definition!} can also be used
to define functions by cases on enumerated types.  For example, a predicate
\ml{One}, which is true of only the value \ml{ONE} of the three-valued type
\ml{three} defined above in Section~\ref{types-session}, can be defined as
follows:

\begin{session}\begin{verbatim}
#let One = new_recursive_definition false three_Axiom `One`
#            "(One ONE = T) /\ (One TWO = F) /\ (One THREE = F)";;
One = |- (One ONE = T) /\ (One TWO = F) /\ (One THREE = F)
\end{verbatim}\end{session}

The existence only of
any function definable using \ml{new\_recursive\_definition} can be proved
using \ml{prove\_rec\_fn\_exists}.  For example:

\begin{session}\begin{verbatim}
#close_theory();;
() : void

#let exists = prove_rec_fn_exists three_Axiom
#             "(f ONE = T) /\ (f TWO = F) /\ (f THREE = F)";;
exists = |- ?f. (f ONE = T) /\ (f TWO = F) /\ (f THREE = F)
\end{verbatim}\end{session}

\noindent The resulting theorem simply states the existence of the
required function.  Here, a constant is not defined, and the user need
not be in draft mode.
\index{primitive recursion theorem!for binary trees|)}

\subsection{Structural induction}

\index{type definition package, in HOL system@type definition package, in \HOL\ system!use of, to build induction tools|(}
\index{induction rule!structural, derivation of|(}
For any concrete recursive type definable
 using the \HOL\ type definition package there is a structural induction
theorem which states the validity of proof by induction
on the structure of the type's values.  The \ML\ function

\begin{boxed}
\index{prove_induction_thm@\ml{prove\_induction\_thm}|pin}
\begin{verbatim}
   prove_induction_thm : thm -> thm
\end{verbatim}\end{boxed}


\noindent can be used to derive a structural induction\index{automated derivation!of structural induction theorems}\index{induction rule!for concrete recursive types}
theorem for any concrete
recursive type defined using \ml{define\_type}.  If the \ML\
variable \ty{op}\ml{\_Axiom} is bound to a theorem of the form
returned by \ml{define\_type},
then executing

\[ \ml{prove\_induction\_thm}\;\ty{op}\ml{\_Axiom} \]

\noindent will prove  and  return a  structural induction  theorem for
the concrete type
\ty{op}.  The `induction' theorem is degenerate in the case of non-recursive
types (see the examples given below). Failure occurs, or an unpredictable
output theorem is returned, if the input theorem does not have the form
of a theorem  returned by \ml{define\_type}.

\subsubsection{Examples}

A structural induction theorem on the type of binary trees defined in the session
beginning on Section~\ref{types-session} can be proved by:

\begin{session}\begin{verbatim}
#let btree_Induct = prove_induction_thm btree_Axiom;;
btree_Induct =
|- !P.
    (!x. P(LEAF x)) /\ (!b1 b2. P b1 /\ P b2 ==> P(NODE b1 b2)) ==>
    (!b. P b)
\end{verbatim}\end{session}


\noindent The output theorem states that a predicate \ml{P} is true of all
binary trees if it is true of all labelled leaf nodes, and whenever it is true
of two binary trees \ml{b1} and \ml{b2} it is also true of the binary tree
\ml{NODE b1 b2}, in which \ml{b1} and \ml{b2} occur as immediate left and right
subtrees.

For non-recursive types, the induction theorem returned by
\ml{prove\_induction\_thm} is degenerate: there are no `step' cases in the
induction.  For the two types \ml{three} and \ml{rec} defined in the preceding
interactions of this session, the induction theorems are:

\begin{session}\begin{verbatim}
#let three_Induct = prove_induction_thm three_Axiom;;
three_Induct = |- !P. P ONE /\ P TWO /\ P THREE ==> (!t. P t)

#let rec_Induct = prove_induction_thm rec_Axiom;;
rec_Induct = |- !P. (!b0 b1 b2. P(REC b0 b1 b2)) ==> (!r. P r)
\end{verbatim}\end{session}

\noindent Here, induction simply reduces to the consideration of
cases,\index{case analysis, in HOL logic@case analysis, in \HOL\ logic!as instance  of induction} one for each of the constructors for the concrete type involved.
\index{induction rule!structural, derivation of|)}

\subsection{Structural induction tactics}
\label{avrasi}
%I added one paragraph because this section is so out of sequence. Avra 9/11/89.

\index{theorem continuations!use of, in derivation of induction tactics|(}
This section has been included here for reference
because it relates chiefly to the
type definition package, but it involves concepts not defined until
later, in Chapter~\ref{tactics-and-tacticals}. Tactics, goals
and subgoals are
defined in Section~\ref{tactics}; and theorem continuations, in
Section~\ref{asm-manip}. \ml{MAP\_EVERY} is defined in Section~\ref{avra_manip1}.
\ml{ASSUME\_TAC} is defined in Section~\ref{avra_builtin}.
\ml{MP\_TAC}  and \ml{INDUCT\_TAC} can be found in \REFERENCE.

The \ML\ function

\begin{boxed}
\index{INDUCT_THEN@\ml{INDUCT\_THEN}|pin}
\index{induction tactics!derivation of|(}
\begin{verbatim}
   INDUCT_THEN : thm -> (thm -> tactic) -> tactic
\end{verbatim}\end{boxed}

\noindent can be used to generate
a structural induction tactic\index{automated derivation!of structural induction tactics}
for any concrete types definable using \ml{define\_type}.\index{define_type@\ml{define\_type}}  The first argument
is an induction theorem of the form returned by the function
\ml{prove\_induction\_thm}\index{prove_induction_thm@\ml{prove\_induction\_thm}} discussed in the previous section.  The second
argument is a theorem continuation\index{characterizing theorem!use of, in deriving induction} (see
Chapter~\ref{tactics-and-tacticals}) that determines what is to be done with
the induction hypotheses when the resulting tactic is applied to a goal.

If $th$ is an induction theorem for a concrete type \ty{op} with
$m$ constructors  \mbox{$C_1$, \dots, $C_m$}
(\ie\ a theorem of the kind returned by
\ml{prove\_induction\_thm}) and $F$ is a theorem continuation, then the
tactic $\ml{INDUCT\_THEN}\;th\;F$ will reduce a goal
{\small\verb%(%}$\Gamma${\small\verb%,"!%}$x{:}
\ty{op}${\small\verb%.%}$t[x]${\small\verb%")%} to the collection of $m$
induction subgoals generated by:

\[ \begin{array}[t]{@{}l@{}l}
 \ml{MAP\_EVERY } F \ml{ [}th_1^1\ml{;}\;\ldots\ml{;}\;th_1^{k_1}\ml{]} &
\ml{ (}\Gamma\ml{, "}t[ C_1\;x_1^1\;\ldots\;x_1^{k_1}]\ml{")},\\
\qquad\qquad \vdots & \mbox{} \\
 \ml{MAP\_EVERY } F \ml{ [}th_m^1\ml{;}\;\ldots\ml{;}\;th_m^{k_m}\ml{]} &
\ml{ (}\Gamma\ml{, "}t[ C_m\;x_m^1\;\ldots\;x_m^{k_m}]\ml{")}
 \end{array}
\]\index{MAP_EVERY@\ml{MAP\_EVERY}}

\noindent where $th_i^j$ is a theorem of the form $\ml{|- }t[x_i^j]$ asserting
the truth of $t[x_i^j]$ for the $j$th recursive argument (for non-recursive
arguments, there will be no $th_i^j$ in the list) of the $i$th constructor
$C_i$ (for $1 \leq i \leq m$).

The most common use of \ml{INDUCT\_THEN} is in conjunction with the theorem
continuation \ml{ASSUME\_TAC}.  For example, the built-in
induction tactic \ml{INDUCT\_TAC} for mathematical induction on
the natural numbers is defined in \ML\ by:

\begin{hol}
\index{INDUCT_TAC@\ml{INDUCT\_TAC}}
\begin{verbatim}
   let INDUCT_TAC = INDUCT_THEN INDUCTION ASSUME_TAC
\end{verbatim}\end{hol}

\noindent This built-in tactic reduces a goal
 {\small\verb%(%}$\Gamma${\small\verb%,"!%}$n${\small\verb%.%}$t[n]${\small\verb%")%} to a basis subgoal
{\small\verb%(%}$\Gamma${\small\verb%,"%}$t[${\small\verb%0%}$]${\small\verb%")%}
and a step subgoal
{\small\verb%(%}$\Gamma\cup\{${\small\verb%"%}$t[n]${\small\verb%"%}$\}${\small\verb%,"%}$t[${\small\verb%SUC %}$n]${\small\verb%")%}.
The extra assumption  {\small\verb%"%}$t[n]${\small\verb%"%} (\ie\ the
induction hypothesis)
is added to the assumptions $\Gamma$ by \ml{ASSUME\_TAC}.

By contrast, the
induction tactic \ml{INDUCT\_MP\_TAC} (which is not built-in) defined
by:

\begin{hol}\begin{verbatim}
   let INDUCT_MP_TAC = INDUCT_THEN INDUCTION MP_TAC
\end{verbatim}\end{hol}

\noindent reduces a goal
 {\small\verb%(%}$\Gamma${\small\verb%,"!%}$n${\small\verb%.%}$t[n]${\small\verb%")%} to a basis subgoal
{\small\verb%(%}$\Gamma${\small\verb%,"%}$t[${\small\verb%0%}$]${\small\verb%")%}
and an induction step subgoal
{\small\verb%(%}$\Gamma${\small\verb%, "%}$t[n]${\small\verb% ==> %}$t[${\small\verb%SUC %}$n]${\small\verb%")%}.
Here, the theorem continuation \ml{MP\_TAC} makes the induction hypothesis
an antecedent of the step subgoal, rather than an assumption.

As this example illustrates, the theorem continuation $F$ in
an induction tactic

\[ \ml{INDUCT\_THEN }th\;\;F \]

\noindent generated using an induction theorem $th$ can be thought of as a
function which determines what is to be done with the induction hypotheses
corresponding to the recursive arguments of constructors in the step
cases of a proof by structural induction.  When $F$ is \ml{ASSUME\_TAC},
the induction hypotheses become assumptions in the subgoals generated; and when
$F$ is \ml{MP\_TAC}, the induction hypotheses become the antecedents of
implicative subgoals.  Other theorem continuations (for which, see
Chapter~\ref{tactics-and-tacticals} and \REFERENCE) can also be used\index{theorem continuations!use of, in derivation of induction tactics|)}.
\index{type definition package, in HOL system@type definition package, in \HOL\ system!use of, to build induction tools|)}\index{induction tactics!derivation of|)}

\subsection{Other tools}

The function

\begin{boxed}
\index{prove_constructors_one_one@\ml{prove\_constructors\_one\_one}|pin}
\begin{verbatim}
   prove_constructors_one_one : thm -> thm
\end{verbatim}\end{boxed}

\noindent proves that the constructors\index{constructors, of concrete types in HOL logic@constructors, of concrete types in \HOL\ logic!proving one-to-one}
 of a concrete type which take arguments
are one-to-one\index{automated derivation!of one-to-one theorems}.  The argument to \ml{prove\_constructors\_one\_one} is a
theorem of the form returned by \ml{define\_type}.

The function

\begin{boxed}
\index{prove_constructors_distinct@\ml{prove\_constructors\_distinct}|pin}
\begin{verbatim}
   prove_constructors_distinct : thm -> thm
\end{verbatim}\end{boxed}

\noindent proves that the constructors\index{constructors, of concrete types in HOL logic@constructors, of concrete types in \HOL\ logic!proving distinct}
 of a concrete type yield distinct\index{automated derivation!of
distinctness theorems} values.  The argument to
\ml{prove\_constructors\_distinct} is again
a theorem of the form returned by \ml{define\_type}.

The function

\begin{boxed}
\index{prove_cases_thm@\ml{prove\_cases\_thm}|pin}
\begin{verbatim}
   prove_cases_thm : thm -> thm
\end{verbatim}\end{boxed}

\noindent proves a cases\index{automated derivation!of case analysis theorems}
 theorem\index{case analysis, in HOL logic@case analysis, in \HOL\ logic!theorems for} for any concrete type.  Such a theorem states that every
value can be constructed using one of the type's constructors.  This property
follows more easily (and therefore is faster to prove) from induction than from
primitive recursion, so the function \ml{prove\_cases\_thm} takes as an
argument an induction theorem of the kind returned by
\ml{prove\_induction\_thm}.\index{prove_induction_thm@\ml{prove\_induction\_thm}}

These auxiliary tools work for any concrete type definable using
\ml{define\_type}.

\subsubsection{Examples}

The following interactions with the system show the proof that the constructor
\ml{LEAF} for the type \ml{(*)btree}  is one-one,
and also that the constructor \ml{REC} for the type \ml{rec} is one-to-one.

\begin{session}\begin{verbatim}
#let LEAF_one_one = prove_constructors_one_one btree_Axiom;;
LEAF_one_one =
|- (!x x'. (LEAF x = LEAF x') = (x = x')) /\
   (!b1 b2 b1' b2'.
     (NODE b1 b2 = NODE b1' b2') = (b1 = b1') /\ (b2 = b2'))

#let REC_one_one = prove_constructors_one_one rec_Axiom;;
REC_one_one =
|- !b0 b1 b2 b0' b1' b2'.
    (REC b0 b1 b2 = REC b0' b1' b2') =
    (b0 = b0') /\ (b1 = b1') /\ (b2 = b2')
\end{verbatim}\end{session}

The function \ml{prove\_constructors\_one\_one} fails when the concrete
type involved has no constructors that take arguments.  For example:

\begin{session}\begin{verbatim}
#let th = prove_constructors_one_one three_Axiom;;
evaluation failed     prove_constructors_one_one: invalid input theorem
\end{verbatim}\end{session}

The function \ml{prove\_constructors\_distinct} returns the theorem stating
that the constructors of a concrete type yield pair-wise distinct values.  For
example:

\begin{session}\begin{verbatim}
#let NOT_LEAF_NODE = prove_constructors_distinct btree_Axiom;;
NOT_LEAF_NODE = |- !x b1 b2. ~(LEAF x = NODE b1 b2)

#let three_distinct = prove_constructors_distinct three_Axiom;;
three_distinct = |- ~(ONE = TWO) /\ ~(ONE = THREE) /\ ~(TWO = THREE)
\end{verbatim}\end{session}

Cases theorems\index{case analysis, in HOL logic@case analysis, in \HOL\ logic!as instance  of induction} are proved from structural induction theorems. For the binary
tree example considered in the present session, here is the cases theorem:

\begin{session}\begin{verbatim}
#let btree_cases = prove_cases_thm btree_Induct;;
btree_cases = |- !b. (?x. b = LEAF x) \/ (?b1 b2. b = NODE b1 b2)
\end{verbatim}\end{session}

\noindent Note that the structural induction theorem for binary trees,
\ml{btree\_Induct}, is used.
\index{extension, of HOL logic@extension, of \HOL\ logic!by type definition|)}
\index{type definition package, in HOL system@type definition package, in \HOL\ system|)}



%%% Local Variables:
%%% mode: latex
%%% TeX-master: "description"
%%% End:
