\index{HolSmtLib|(}
\index{SMT solvers|see {HolSmtLib}}

\setcounter{sessioncount}{0}

The purpose of \ml{HolSmtLib} is to provide a platform for
experimenting with combinations of theorem proving and Satisfiability
Modulo Theories~(SMT) solvers.  \ml{HolSmtLib} was developed as part
of a research project on {\it Expressive Multi-theory Reasoning for
  Interactive Verification} (EPSRC grant EP/F067909/1) from 2008
to~2011.  It is loosely inspired by \ml{HolSatLib}
(Section~\ref{sec:HolSatLib}), and has been described in parts in the
following publications:
\begin{itemize}
\item Tjark Weber: {\it SMT Solvers: New Oracles for the HOL Theorem
  Prover}. Submitted for publication, 2009.%TODO
\item Sascha B{\"o}hme, Tjark Weber: {\it Fast LCF-Style Proof
  Reconstruction for Z3}. Submitted for publication, 2009.%TODO
\end{itemize}
\ml{HolSmtLib} uses external SMT solvers to prove instances of SMT
tautologies, \ie, formulas that are provable using (a combination of)
propositional logic, equality reasoning, linear arithmetic on integers
and reals, and decision procedures for bit vectors and arrays.  The
supported fragment of higher-order logic varies with the SMT solver
used.  At least for Yices, it is a superset of the fragment supported
by \ml{bossLib.DECIDE} (and the performance of \ml{HolSmtLib},
especially on big problems, should be much better).

\subsection{Interface}

The library provides several tactics to invoke different SMT solvers,
namely \ml{YICES\_TAC}, \ml{CVC3\_TAC}, \ml{Z3\_ORACLE\_TAC}, and
\ml{Z3\_TAC}.  These tactics are defined in the \ml{HolSmtLib}
structure, which is the library's main entry point.  Given a
goal~$(\Gamma, \varphi)$ (where $\Gamma$ is a list of assumptions, and
$\varphi$ is the goal's conclusion), each tactic returns an empty list
of new goals and a validation function that returns a theorem~$\Gamma'
\vdash \varphi$ (with $\Gamma' \subseteq \Gamma$), or fails if the SMT
solver cannot prove the goal.\footnote{Internally, the assumptions and
  the \emph{negated} goal are passed to the SMT solver.  If the SMT
  solver determines that these formulas are unsatisfiable, then the
  (unnegated) goal must be provable from the assumptions.}  In other
words, these tactics solve the goal (or fail).
\ml{Tactical.TAC\_PROOF} can be used to derive functions of type
\ml{goal -> thm} from these tactics.

\paragraph{Oracles vs.\ proof reconstruction}

\ml{YICES\_TAC}, \ml{CVC3\_TAC}, and \ml{Z3\_ORACLE\_TAC} each use the
SMT solver (Yices, CVC3, and Z3, respectively) as an oracle: the
solver's result is trusted.  Bugs in the SMT solver or in
\ml{HolSmtLib} could potentially lead to inconsistent theorems.
Accordingly, the returned theorem is tagged with an oracle
tag. \ml{Z3\_TAC}, on the other hand, performs proof reconstruction.
One obtains a proper \HOL{} theorem, without (additional) oracle tags.

\paragraph{Supported subsets of higher-order logic}

\ml{YICES\_TAC} employs a translation into Yices's native input
format.  The interface supports types \holtxt{bool}, \holtxt{num},
\holtxt{int}, \holtxt{real}, \holtxt{->} (\ie, function types),
\holtxt{prod} (\ie, tuples), fixed-width word types, and the following
terms: equality, Boolean connectives (\holtxt{T}, \holtxt{F},
\holtxt{==>}, \holtxt{/\bs}, \holtxt{\bs /}, negation,
\holtxt{if-then-else}, \holtxt{bool-case}), quantifiers (\holtxt{!},
\holtxt{?}), numeric literals, arithmetic operators (\holtxt{SUC},
\holtxt{+}, \holtxt{-}, \holtxt{*}, \holtxt{/}, unary minus,
\holtxt{DIV}, \holtxt{MOD}, \holtxt{ABS}, \holtxt{MIN}, \holtxt{MAX}),
comparison operators (\holtxt{<}, \holtxt{<=}, \holtxt{>},
\holtxt{>=}, both on \holtxt{num}, \holtxt{int}, and \holtxt{real}),
function application, lambda abstraction, tuple selectors \holtxt{FST}
and \holtxt{SND}, and various word operations.  Other terms are
treated as uninterpreted, \ie, replaced by fresh variables.

CVC3 and Z3 are integrated via a much more restrictive translation
into SMT-LIB format.  The interface at the moment only supports
equality, Boolean operators, type \holtxt{int}, and a few arithmetic
and comparison operations on integers.  Other terms are again replaced
by fresh variables.  Therefore, Yices is typically the solver of
choice at the moment (unless you need proof reconstruction, which is
available for Z3 only).

\paragraph{Tracing}

Tracing output can be controlled via \ml{Feedback.set\_trace
  "HolSmtLib"}.  See the source code in \ml{SolverSpec.sml} for
possible values.

Communication between \HOL{} and external SMT solvers is via temporary
files.  The default behavior is to delete these files after successful
invocation of the SMT solver.  However, this can be changed via the
above tracing variable.  If there is an error, files are retained in
any case.

To protect your data, \ml{HolSmtLib} will not overwrite existing
files.  Thus in case of an error, you will need to remove temporary
files manually before you can invoke the SMT solver again.  If you
forget to do so, or if the file exists for some other reason, you'll
see an error message \texttt{file '<name>' exists, please remove it}.

\begin{session}
\begin{verbatim}
- load "HolSmtLib"; open Tactical; open HolSmtLib;
(* output omitted *)
> val it = () : unit

- show_tags := true;
> val it = () : unit

- TAC_PROOF YICES_TAC ``(a ==> b) /\ (b ==> a) = (a=b)``;
> val it = [oracles: DISK_THM, HolSmtLib] [axioms: ] []
           |- (a ==> b) /\ (b ==> a) = (a = b) : thm

- TAC_PROOF CVC3_TAC ``(a ==> b) /\ (b ==> a) = (a=b)``;
> val it = [oracles: DISK_THM, HolSmtLib] [axioms: ] []
           |- (a ==> b) /\ (b ==> a) = (a = b) : thm

- TAC_PROOF Z3_ORACLE_TAC ``(a ==> b) /\ (b ==> a) = (a=b)``;
> val it = [oracles: DISK_THM, HolSatLib] [axioms: ] []
           |- (a ==> b) /\ (b ==> a) = (a = b) : thm

- TAC_PROOF Z3_TAC ``(a ==> b) /\ (b ==> a) = (a=b)``;
> val it = [oracles: DISK_THM, HolSatLib] [axioms: ] []
           |- (a ==> b) /\ (b ==> a) = (a = b) : thm
\end{verbatim}
\end{session}

\subsection{Installing SMT solvers}

\ml{HolSmtLib} has been tested with CVC3~1.5, Yices~1.0.18 and Z3~2.0.
Later versions of these SMT solvers might work as well.  To use
\ml{HolSmtLib}, you need to install at least one of these SMT solvers
on your machine.  As mentioned before, Yices supports a much larger
fragment of higher-order logic than the other solvers, but proof
reconstruction has been implemented only for Z3.

Yices is available for various platforms from
\url{http://yices.csl.sri.com/}.  After installation, you must make
the Yices executable available as {\tt yices}, \eg, by placing it into
a folder that is in your {\tt \$PATH}.  This name is currently
hard-coded: there is no configuration option to tell \HOL{} about the
location and name of the Yices executable.

CVC3 can be obtained from \url{http://www.cs.nyu.edu/acsys/cvc3/}.
The CVC3 executable must be made available as {\tt cvc3-optimized}.

Z3 is probably hardest to set up.  A Windows version is available from
\url{http://research.microsoft.com/en-us/um/redmond/projects/z3/}.
For Linux and Mac OS~X, see the installation instructions at
\url{http://www4.in.tum.de/~boehmes/z3.html}.  The executable must be
made available as {\tt z3} after installation.  Additionally, to
enable proofs, you must create a file {\tt z3.ini} in your current
directory that contains (at least) the following lines:
\begin{verbatim}
PROOF_MODE=2
DISPLAY_PROOF=true
\end{verbatim}

\subsection{Wishlist}

The following features have not been implemented yet.  Please submit
additional feature requests (or code contributions) via
\url{http://hol.sf.net}.

\paragraph{Counterexamples}

For satisfiable input formulas, SMT solvers typically return a
satisfying assignment.  This assignment should be displayed to the
\HOL{} user as a counterexample.  It could also be turned into a
theorem, similar to the way \ml{HolSatLib} treats satisfying
assignments.

\paragraph{Proof reconstruction for bit vectors}

Proof reconstruction for Z3 is currently limited to first-order logic
(including uninterpreted functions and equality), linear arithmetic
and arrays.  In particular, the theory of bit vectors is not supported
yet.

\paragraph{Proof reconstruction for other SMT solvers}

Proof reconstruction has been implemented only for Z3.  Several other
SMT solvers can produce proofs, and it would be nice to offer \HOL{}
users more choice.  However, in the absence of a standard proof format
for SMT solvers, it is maybe not worth the implementation effort.

\paragraph{Support for SMT-LIB version~2.0}

The current implementation of \ml{HolSmtLib} uses version~1.2 of the
SMT-LIB language.  An update to version~2.0 could greatly simplify the
translation from higher-order logic into SMT-LIB format.

\paragraph{SMT solvers as a web service}

The need to install an SMT solver locally poses an entry barrier.  It
would be much more convenient to have a web server running one (or
several) SMT solvers, roughly similar to the ``System on TPTP''
interface that G.~Sutcliffe provides for first-order theorem provers.
For Isabelle/HOL, such a web service has been installed by
S.~B{\"o}hme in Munich, but unfortunately it is not publicly
available.  Perhaps the SMT-EXEC initiative
(\url{http://www.smtexec.org/}) could offer hardware or implementation
support.

\index{HolSmtLib|)}

%%% Local Variables:
%%% mode: latex
%%% TeX-master: "description"
%%% End:
