\chapter*{Preface}\markboth{Preface}{Preface}
\label{intro}

This volume contains the  description of  the \HOL\  system.
It is one of three volumes making up the documentation for \HOL:

\begin{myenumerate}
\item \TUTORIAL: a tutorial introduction to \HOL, with case studies.
\item \DESCRIPTION: a description of higher order logic,
the \ML\ programming language, and theorem proving methods in the \HOL\ system;
\item \REFERENCE: the reference manual for \HOL.
\end{myenumerate}

\noindent These three documents will be referred to by the short names (in
small slanted capitals) given above.

This document, \DESCRIPTION, is intended to serve both as a definition of \HOL\
and as an advanced guide for users with some prior experience of the system.
Beginners should start with the companion document \TUTORIAL.

The \HOL\ system is designed to support interactive theorem proving in higher
order logic (hence the acronym `\HOL').  To this end, the formal logic is
interfaced to a general purpose programming language (\ML, for meta-language)
in which terms and theorems of the logic can be denoted, proof strategies
expressed and applied, and logical theories developed.  This document presents
the definitions of the meta-language and the logic, and it explains the means
by which meta-language functions can be used to generate proofs in the logic.

The version of higher order logic used in \HOL\ is predicate calculus with
terms from the typed lambda calculus (\ie\ simple type theory). This was
originally developed as a foundation for mathematics \cite{Church}.  The
primary application area of \HOL\ was initially intended to be the
specification and verification of hardware designs.  (The use of higher order
logic for this purpose was first advocated by Keith Hanna \cite{Hanna-Daeche}.)
However, the logic does not restrict applications to hardware; \HOL\ has been
applied to many other areas.

The approach to mechanizing formal proof used in \HOL\ is due to Robin Milner
\cite{Edinburgh-LCF}, who also headed the team that designed and implemented
the language \ML.  That work centred on a system called \LCF\ (logic for
computable functions), which was intended for interactive automated reasoning
about higher order recursively defined functions.  The interface of the logic
to the meta-language was made explicit, using the type structure of \ML, with
the intention that other logics eventually be tried in place of the original
logic.  The \HOL\ system is a direct descendant of \LCF; this is reflected in
everything from its structure and outlook to its incorporation of \ML, and even
to parts of its implementation.  Thus \HOL\ satisfies the early plan to apply
the \LCF\ methodology to other logics.

The original \LCF\ was implemented at Edinburgh in the early 1970's, and is now
referred to as `Edinburgh \LCF'. Its code was ported from Stanford Lisp to
Franz Lisp by G\'erard Huet at {\small INRIA}, and was used in a French
research project called `Formel'.  Huet's Franz Lisp version of \LCF\ was
further developed at Cambridge by Larry Paulson, and became known as `Cambridge
\LCF'. The \HOL\ system is implemented on top of an early version of Cambridge
\LCF\ and consequently many features of both Edinburgh and Cambridge \LCF\ were
inherited by \HOL. For example, the axiomatization of higher order logic used
is not the classical one due to Church, but an equivalent formulation
influenced by \LCF.

The language \ML\ has now achieved status as a programming language in its own
right, although it was originally designed as the proof management language for
\LCF.  It is a functional language distinguished particularly for its type
inference mechanism, that gives type security without overburdening the user.
Types, and especially abstract types, are the basis for distinguishing the
theorems of a logic from arbitrary formulae, in a secure way.

A standard has now been established for \ML\ (\cite{sml}), but for
historical reasons, the \HOL\ system originally included an earlier
version of \ML.  The version documented here is \HOL 98, which is
written entirely in SML.

In this document, the acronym `\HOL' refers to both the interactive theorem
proving system and to the version of higher order logic that the system
supports; where there is serious ambiguity, the former is called `the \HOL\
system' and the latter `the \HOL\ logic'.

An enhanced and rationalized version of \HOL, called \HOL 88, was
released (in 1988), after the original \HOL\ system had been in use
for several years.  \HOL 90 was written by Konrad Slind and was a port
of \HOL 88 to SML.  \HOL 98 is the latest version of \HOL\ and has a
number of new and interesting features compared to its predecessors.
This is intended to serve as a stable platform for a number of
research projects and technology transfer activities that are in
progress at Cambridge, and elsewhere, at the time of writing.  It is
also the supported version of the system for the international \HOL\
community.  The main differences between the various versions and
releases of \HOL\ are described in Appendix~\ref{appendix}.

%%% Local Variables:
%%% mode: latex
%%% TeX-master: "description"
%%% End:
