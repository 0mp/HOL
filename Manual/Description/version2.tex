
\appendix

\chapter{Version 2.0}\label{appendix}

This appendix summarizes the differences between \HOL 88 Version 2.0
and the last major release, which was Version 1.11 (Version 1.12 is
essentially the Version 2.0 system with the Version 1.10
documentation).


Version 2.0 is primarily a rationalization of Version 1.11. Relics of
\LCF\ have been deleted and many `internal' \ML\ bindings have been
made local to the functions that use them. It is hoped that nothing
useful has been inadvertently removed.

The top-level directory of the \HOL\ distribution sources has been
reorganized.  The details are in Chapter 1 of \TUTORIAL\ (`Getting and
Installing \HOL').  An important addition is a directory
{\small\verb%contrib%} containing contributions from the \HOL\ user
community.  The aim of this directory is to make it easy for the
community to share theories, proofs, examples, tools, documents, and
other material which may be of general interest. Unlike the library,
there is no quality control on {\small\verb%contrib%}.

Two major revisions of the theorem-proving tools in Version 2.0 are the
complete reimplementations, by Tom Melham, of (i) the resolution rules and
tactics and (ii) the type definition package.


The resolution tools have been rewritten so that they are more
systematic and predictable. Unfortunately, this means that sometimes
different results are produced, and so code using
{\small\verb%RES_TAC%}, {\small\verb%IMP_RES_TAC%} \etc\ may need to
be modified to run in Version 2.0. This is particularly likely if
tactics have been used that depend on the positions of assumptions in
goals (this is generally considered `bad style'). The old
resolution tools are available in the directory
{\small\verb%contrib/resolve%}.

The type definition package (i.e. the \ML\ function
{\small\verb%define_type%}) has been rewritten to be faster and more
robust. A number of identifier names have changed and its internal
workings have been reorganized. A tool for mutually recursive type
definitions is available in {\small\verb%contrib/mut_rec_types%} (it
was produced by students at Aarhus University).

Four libraries from Version 1.11 of HOL have been temporarily withdrawn,
because the Cambridge group have been unable to rebuild them in Version 2.0.
These are:

\begin{hol}\begin{verbatim}
   card   well_order   zet   csp
\end{verbatim}\end{hol}

\noindent The first three are difficult to rebuild because of changes to
resolution.  They rely very heavily on the order in which the
assumptions of goals appear.  The fourth library uses the
{\small\verb%set%} library, which has been substantially modified.
These four libraries have been moved to
{\small\verb%contrib%} until they are updated by their authors.


Version 2.0 provides parser and pretty-printer support for restricted
quantification and set-theoretic notation.


The restricted quantification notation allows terms of
the form
\con{Q}$x${\small\verb%::%}$p${\small\verb%.%}$t[x]$, where
\con{Q} is a quantifier and
if $x:\alpha$ then $p$ can be any term of type $\alpha\fun\bool$; this
denotes the quantification of $x$ over those values
satisfying $p$.  The qualifier {\small\verb%::%} can be used with
{\small\verb%\%} and any
binder, including user defined ones; the appropriate meanings are
predefined for {\small\verb%\%} and the built-in binders
{\small\verb%!%}, {\small\verb%?%} and {\small\verb%@%}.  This provides a method of
simulating subtypes and dependent types; the qualifying predicate $p$ can be
an arbitrary term containing parameters. For example:
{\small\verb%!%}$w${\small\verb%::%}$\con{Word}(n)${\small\verb%. %}$t[w]$,
for a suitable constant \con{Word}, simulates a quantification over the
`type' of $n$-bit words. Experiments are needed to establish how satisfactory
this approach is. The syntactic mechanism,
although implemented initially to support restricted quantification, can
also be used to support better approximations to notations like
$\sum_{i=1}^{n}x_i$; this could, for example, be represented by
$\con{Sum}\ i ${\small\verb%::(1,%}$n${\small\verb%).%}$\ x_i$, via a
suitable definition of the constant $\con{Sum}$.


The set notation allows finite sets to be written in the form
{\small\verb%""%}. It also
supports set abstraction notation of the form
{\small\verb%"{%}$t${\small\verb%|%}$p${\small\verb%}"%}; for example
{\small\verb%"{2*x|x>0}"%} and {\small\verb%"{(x,y)|x=y}"%}.  Both
these notations can be mounted onto any underlying set theory (or even
onto theories of lists, bags etc.), although they were initially
designed for use with the library \ml{sets} (which was previously
called \ml{all\_sets}). This library has been greatly extended.


Version 2.0 has libraries {\small\verb%parser%} and
{\small\verb%prettyp%}, which provide parser and pretty-printer
support for embedded languages. These are currently being used to
provide theorem-proving tools for the hardware description languages
{\small ELLA}, {\small VHDL} and {\small SILAGE}, and for various
other special purpose languages.  A mechanism called ``syntax blocks''
allows user-specified parsers to be invoked from \ML. The development
of methodologies for embedding application-specific languages in \HOL\
is a major topic of research at Cambridge.

A facility is provided to support experiments with user-programmed
quotation typecheckers for the \HOL\ logic. This enables more type
inference to be implemented.

Many new conversions are provided. The most useful ones are described
in Section~\ref{CONVERSIONS} below.

In the sections that follow, those new things in Version 2.0 which
are likely to be of most general use are described. Section~\ref{adddel}
list the \ML\ bindings that have ben added and deleted in Version 2.0.


\section{Directory reorganization}

The help files and manual have been split apart into two separate directories.
The top-level of the \HOL\ distribution directory now contains:

\begin{itemize}
\item \ml{Manual}:  directory containing the manual sources;
\item \ml{help}:  directory containing text files for online help, as well
as for the \REFERENCE\ part of the manual.
\end{itemize}

\noindent The search path mechanism has been revised in various ways.


Library pathnames are now no longer on the default search path, nor
are they added to the search path when a library is loaded (it is
the responsibility of the library's load-file to do this, if desired).
As a consequence, the predefined \ML\ value \ml{Libraries} is no longer used.

The initial search path is now:

\begin{hol}\begin{alltt}
   [``; `~/`; `\(directory\)/theories/`]
\end{alltt}\end{hol}

\noindent where $directory$ is the site-specific absolute pathname in which
the \HOL\ distribution directory (`\ml{hol}') resides.

The \ML\ functions \ml{add\_to\_search\_path} and
\ml{append\_to\_search\_path} are
no longer built-in.  Users may define them, if required, by:

\begin{hol}\begin{verbatim}
   let add_to_search_path p = set_search_path(p.search_path())

   let append_to_search_path p = set_search_path(search_path()@[p])
\end{verbatim}\end{hol}

The \ml{install} function now sets the search path to:

\begin{hol}\begin{alltt}
        [``; `~/`; \(arg\)^`/theories/`]
\end{alltt}\end{hol}

\noindent where $arg$ is the string argument to install.

\section{Location of libraries}

The function

\begin{holboxed}\index{library_pathname@\ml{library\_pathname}}
\begin{verbatim}
   library_pathname : void -> string
\end{verbatim}\end{holboxed}

\noindent returns the internal pathname used by \HOL\ to
load libraries.  This pathname, which is site-specific and is given an initial
value when the system is built, should be the absolute pathname of the \HOL\
system library directory.  This pathname will typically have the form:

\begin{hol}\begin{alltt}
   `\(directory\)/hol/Library`
\end{alltt}\end{hol}

\noindent where $directory$ is the site-specific absolute pathname in which the \HOL\
distribution directory (`\ml{hol}') resides. The value returned by
\ml{library\_pathname} can be changed by users only via the \ml{install} function.

The string returned by \ml{library\_pathname} is primarily used in
library load-files to update the \HOL\ search path and help search
path.  For example, suppose that in a library \ml{lib} there is a
directory \ml{help} which contains online help files specific to this
library.  The load-file {\small\verb%lib.ml%} can then update the help
search path as follows:

\begin{hol}\begin{verbatim}
   let path = library_pathname() ^ `/lib/help/`
   in
   set_help_search_path (path . help_search_path())
\end{verbatim}\end{hol}

\noindent This will make the help files of the library \ml{lib}
available for online help
whenever the library is loaded.



\section{More flexible help system}\index{help@\ml{help}}\index{help system|(}


The help system now uses \ml{cat} rather than \ml{more} as the default for
displaying help files. This default can be changed with the \ML\ function:

\begin{holboxed}\index{set_help@\ml{set\_help}}
\begin{verbatim}
   set_help : string -> string
\end{verbatim}\end{holboxed}

\noindent This installs a new user-supplied help function, and returns the
previous one as result. The effect of

\begin{hol}\begin{alltt}
   help `\(file\)`
\end{alltt}\end{hol}

\noindent is to pipe the appropriate help file derived from $file${\small\verb%.doc%}
into the current help function,
with the top level of \ML\ being the standard output.
For example,

\begin{hol}\begin{verbatim}
   set_help `lpr`
\end{verbatim}\end{hol}

\noindent will cause the help file to be printed instead of being displayed and

\begin{hol}\begin{verbatim}
   set_help `lpr -Pbarf`
\end{verbatim}\end{hol}

\noindent will cause it to be printed to the printer \ml{barf}.


Two new functions:

\begin{holboxed}
\index{set_help_search_path@\ml{set\_help\_search\_path}}
\index{help_search_path@\ml{help\_search\_path}}
\begin{verbatim}
  set_help_search_path : string list -> void
  help_search_path     : void -> string list
\end{verbatim}\end{holboxed}

\noindent have been added to the system for accessing the
internal search path used by \HOL\ to find online help files.  The help
search path has precisely the same format as the regular search path,
and these two function are analogous to the \ML\ functions
\ml{search\_path} and \ml{set\_search\_path}.\index{help system|)}



\section{Syntax for restricted quantification}
\index{types, in HOL logic@types, in \HOL\ logic! dependent}
\index{dependent types in HOL logic@dependent types in \HOL\ logic}

Syntactic support for restricted quantification and abstraction is now
provided. This follows a suggestion discussed at the Second \HOL\ Users
Meeting and implements a method of simulating subtypes and dependent
types with predicates.

Currently no derived rules are provided to support this notation, so
any inferences will need to work on the underlying semantic
representation.

The new syntax automatically translates as follows:

\begin{hol}
{\small\verb%   \%}$v${\small\verb%::%}$P${\small\verb%.%}$B${\small\verb%    <---->   RES_ABSTRACT %}$P${\small\verb% (\%}$v${\small\verb%.%}$B${\small\verb%)%}\\
{\small\verb%   !%}$v${\small\verb%::%}$P${\small\verb%.%}$B${\small\verb%    <---->   RES_FORALL   %}$P${\small\verb% (\%}$v${\small\verb%.%}$B${\small\verb%)%}\\
{\small\verb%   ?%}$v${\small\verb%::%}$P${\small\verb%.%}$B${\small\verb%    <---->   RES_EXISTS   %}$P${\small\verb% (\%}$v${\small\verb%.%}$B${\small\verb%)%}\\
{\small\verb%   @%}$v${\small\verb%::%}$P${\small\verb%.%}$B${\small\verb%    <---->   RES_SELECT   %}$P${\small\verb% (\%}$v${\small\verb%.%}$B${\small\verb%)%}
\end{hol}

Anything can be written between the binder and `\ml{::}' that could be
written between the binder and `\ml{.}` in the old notation. See the
examples below.

The flag \ml{print\_restrict} has default \ml{true}, but if set to
\ml{false} will
disable the pretty printing. This is useful for seeing what the
semantics of particular restricted abstractions are.

The constants \ml{RES\_ABSTRACT}, \ml{RES\_FORALL}, \ml{RES\_EXISTS} and
\ml{RES\_SELECT} are
defined in the theory \ml{bool} by:

\begin{hol}\begin{verbatim}
   RES_ABSTRACT P B  =  \x:*. (P x => B x | ARB:**)

   RES_FORALL   P B  =  !x:*. P x ==> B x

   RES_EXISTS   P B  =  ?x:*. P x /\ B x

   RES_SELECT   P B  =  @x:*. P x /\ B x
\end{verbatim}\end{hol}

\noindent where the constant \ml{ARB} is defined in the theory \ml{bool} by:

\begin{hol}\begin{verbatim}
   ARB  =  @x:*. T
\end{verbatim}\end{hol}

User-defined binders can also have restricted forms, which are set up
with the function:

\begin{holboxed}\index{associate_restriction@\ml{associate\_restriction}}
\begin{verbatim}
   associate_restriction : (string # string) -> *
\end{verbatim}\end{holboxed}


\noindent If \m{B} is the name
of a binder and \ml{RES\_}$B$ is the name of a suitable constant (which
must be explicitly defined), then executing:

\begin{hol}
{\small\verb%   associate_restriction(`%}\m{B}{\small\verb%`, `RES_%}\m{B}{\small\verb%`)%}
\end{hol}

\noindent will cause the parser and pretty-printer to support:

\begin{hol}
{\small\verb%   %}$B$ $v${\small\verb%::%}$P${\small\verb%. %}$B${\small\verb%    <---->   RES_%}$B$ $P${\small\verb% (\%}$v${\small\verb%. %}$B${\small\verb%)%}
\end{hol}

\noindent Note that associations between user defined binders and their
restrictions are not stored in theory files, so they have to be set up
for each \HOL\ session (e.g. with a {\small\verb%hol-init.ml%} initialization file).

Here is an example session:

\setcounter{sessioncount}{1}
\begin{session}\begin{verbatim}
#"!x y::P. x<y";;
"!x y :: P. x < y" : term

#set_flag(`print_restrict`, false);;
true : bool

#"!x y::P. x<y";;
"RES_FORALL P(\x. RES_FORALL P(\y. x < y))" : term

#"?(x,y) p::(\(m,n).m<n). p=(x,y)";;
"RES_EXISTS
 (\(m,n). m < n)
 (\(x,y). RES_EXISTS(\(m,n). m < n)(\p. p = x,y))"
: term

#"\x y z::P.[0;x;y;z]";;
"RES_ABSTRACT P(\x. RES_ABSTRACT P(\y. RES_ABSTRACT P(\z. [0;x;y;z])))"
: term
\end{verbatim}\end{session}

A conversion that rewrites away the constants \ml{RES\_ABSTRACT},
\ml{RES\_FORALL}, \ml{RES\_EXISTS} and \ml{RES\_SELECT} is:

\begin{hol}\begin{verbatim}
   let RESTRICT_CONV =
    (REWRITE_CONV [definition `bool` `RES_ABSTRACT`;
                   definition `bool` `RES_FORALL`;
                   definition `bool` `RES_EXISTS`;
                   definition `bool` `RES_SELECT`])
    THENC (DEPTH_CONV BETA_CONV)
\end{verbatim}\end{hol}

\noindent This is a bit unsatisfactory, as is shown by the artificial
example below:

\setcounter{sessioncount}{1}
\begin{session}\begin{verbatim}
#let t = "!x y::P.?f:num->num::Q. f(@n::R.T) = (x+y)";;
t = "!x y :: P. ?f :: Q. f(@n :: R. T) = x + y" : term

#RESTRICT_CONV t;;
|- (!x y :: P. ?f :: Q. f(@n :: R. T) = x + y) =
   (!x. P x ==> (!x'. P x' ==> (?x. Q x /\ (x(@x. R x /\ T) = x + x'))))
\end{verbatim}\end{session}

The variable \ml{x} in the definitions of the constants
\ml{RES\_ABSTRACT}, \ml{RES\_FORALL}, \ml{RES\_EXISTS} and
\ml{RES\_SELECT} gets confused with the variable in the supplied term.
This can be avoided by changing \ml{RESTRICT\_CONV} to perform
explicit alpha-conversion. For example, by implementing a conversion:

\begin{hol}
{\small\verb%   RES_FORALL %}$P${\small\verb% (\%}$v${\small\verb%.%}$B${\small\verb%[%}$v${\small\verb%])  ---->  !%}$v${\small\verb%. %}$P$ $v${\small\verb% ==> %}$B${\small\verb%[%}$v${\small\verb%]%}
\end{hol}

\noindent Dealing with the case when
$v$ is a variable structure is also desirable. For example:

\begin{session}\begin{verbatim}
#let t1 = "!(m,n)::P. m<n";;
t1 = "!(m,n) :: P. m < n" : term

#RESTRICT_CONV t1;;
|- (!(m,n) :: P. m < n) = (!x. P x ==> (\(m,n). m < n)x)

\end{verbatim}\end{session}

\noindent If anyone writes the desired conversions please let us know!

Here is an example of a user-defined restriction:

\begin{session}\begin{verbatim}
#new_binder_definition(`DURING`, "DURING(p:num#num->bool) = $!p");;
|- !p. $DURING p = $! p

#"DURING x::(m,n). p x";;
no restriction constant associated with DURING
skipping: x " ;; parse failed

#new_definition
# (`RES_DURING`, "RES_DURING(m,n)p = !x. m<=x /\ x<=n ==> p x");;
|- !m n p. RES_DURING(m,n)p = (!x. m <= x /\ x <= n ==> p x)

#associate_restriction(`DURING`,`RES_DURING`);;
() : void

#"DURING x::(m,n). p x";;
"DURING x :: (m,n). p x" : term

#set_flag(`print_restrict`,false);;
true : bool

#"DURING x::(m,n). p x";;
"RES_DURING(m,n)(\x. p x)" : term
\end{verbatim}\end{session}


\section{Syntax for sets}\index{set theory notation}

The special purpose set-theoretic notations
{\small\verb%""%} and
{\small\verb%"{%}$t${\small\verb%|%}$p${\small\verb%}"%} are now available.
The normal interpretation of the former is the finite set containing
$t_1,t_2,\ldots, t_n$ and the normal interpretation of the latter
is the set of all $t$s such that $p$. These interpretations are predefined for
the library \ml{sets}, but the user can use the syntax for other purposes if
he or she wishes using the functions:

\begin{holboxed}
\index{define_finite_set_syntax@\ml{define\_finite\_set\_syntax}}
\index{define_set_abstraction_syntax@\ml{define\_set\_abstraction\_syntax}}
\begin{alltt}
   define_finite_set_syntax      : (string # string) -> void
   define_set_abstraction_syntax : string -> void
\end{alltt}\end{holboxed}

Executing:

\begin{hol}\begin{alltt}
   define_finite_set_syntax(`\(\con{c}\sb{1}\)`,`\(\con{c}\sb{2}\)`)
\end{alltt}\end{hol}

\noindent causes {\small\verb%""%}
to parse to:

\begin{hol}\begin{alltt}
   "\(c\sb{2}\) \(t\sb{1}\) (\(c\sb{2}\) \(t\sb{2}\) \(\cdots\) (\(c\sb{2}\) \(t\sb{n}\) \(c\sb{2}\)) \(\cdots\) ))"
\end{alltt}\end{hol}

\noindent with failure if either $c_1$ or $c_2$
is not the name of a constant.

In the library \ml{sets}, the empty set is \ml{EMPTY} and
the infixed function \ml{INSERT} adds an element to a set.
Executing:

\begin{hol}\begin{verbatim}
   define_finite_set_syntax(`EMPTY`,`INSERT`)
\end{verbatim}\end{hol}

\noindent will cause

\begin{hol}\begin{verbatim}
   "{1,2,3,4}"
\end{verbatim}\end{hol}

\noindent to parse to

\begin{hol}\begin{verbatim}
   "1 INSERT (2 INSERT (3 INSERT (4 INSERT EMPTY)))"
\end{verbatim}\end{hol}


Executing:

\begin{hol}\begin{alltt}
   define_set_abstraction_syntax `\(c\)`
\end{alltt}\end{hol}

\noindent causes  {\small\verb%"{%}$t${\small\verb%|%}$p${\small\verb%}"%}
to parse to:

\medskip

\noindent{\small
{\verb%   "%}}$c${\small{\verb%(\(%}$x_1${\verb%,%}$\ldots${\verb%,%}$x_n${\verb%).(%}$t${\verb%,%}$p${\verb%))"%}
}

\medskip

\noindent where $x_1$, $\ldots$ , $x_n$ are the free variables occurring in both $t$
and $p$.  If there are no such free variables then an error results.
The order in which the variables are listed in the variable structure
of the paired abstraction is an unspecified function of the structure
of $t$ (it is approximately left to right). Failure if $c$ is not the
name of a constant.

For example, if the library \ml{sets} (i.e. what used to be \ml{all\_sets})
is loaded, then

\begin{hol}\begin{verbatim}
   define_set_abstraction_syntax `GSPEC`
\end{verbatim}\end{hol}

\noindent will cause

\begin{hol}\begin{verbatim}
   "{x+y | (x < y) /\ (y < z)}"
\end{verbatim}\end{hol}

\noindent to parse to:

\begin{hol}\begin{verbatim}
   "GSPEC(\(x,y). ((x+y), (x < y) /\ (y < z)))"
\end{verbatim}\end{hol}

\noindent where \ml{GSPEC} is defined by:

\begin{hol}\begin{verbatim}
   |- !f. GSPEC f = SPEC(\x. ?y. x,T = f y)
\end{verbatim}\end{hol}

\noindent and \ml{SPEC} abstracts a predicate to a set (it is the abstraction
bijection used in the definition of the type operator \ml{set}).
Other examples are:

\begin{hol}\begin{verbatim}
   "{x+y+z | (x < y) /\ (y < z)}"
\end{verbatim}\end{hol}

\noindent will parse to:

\begin{hol}\begin{verbatim}
   "GSPEC(\(x,y,z). (x+(y+z), (x < y /\ y < z)))"
\end{verbatim}\end{hol}

\noindent and

\begin{hol}\begin{verbatim}
   "{x+y+w | (x < y) /\ (y < z)}"
\end{verbatim}\end{hol}

\noindent will parse to:

\begin{hol}\begin{verbatim}
   "GSPEC(\(x,y). (x+(y+w), (x < y /\ y < z)))"
\end{verbatim}\end{hol}


Note that the precedence of comma is increased in the contexts
``{\small\verb%%}'' and
``{\small\verb%{%}$\cdots${\small\verb%|%}''.
Terms will be printed in set notation if the flag \ml{print\_set} is
\ml{true}.
Note that

\medskip

\ml{"}$c${\small\verb%(\(%}$x_1$\ml{,}$\ldots$\ml{,}$x_n$\ml{).(}$t$\ml{,}$p$\ml{))"}

\medskip

\noindent will only print as
{\small\verb%"{%}$t${\small\verb%|%}$p${\small\verb%}"%}
if the variables $x_1$, $\ldots$ ,
$x_n$ occur free in both $t$ and $p$ (and \ml{print\_set} is \ml{true}) .

\section{Revised set theory libraries}\label{LIBRARY}
\index{set theory libraries}

In previous versions of the system, there were three libraries dealing with
elementary set theory:

\begin{myenumerate}
\item  \ml{sets}: a theory of finite sets (Windley and Leveilley);
\item \ml{all\_sets}: a theory of infinite and finite sets (Windley and Leveilley);
\item \ml{set}: a theory of predicates-as-sets (Kalker).
\end{myenumerate}

Each of these libraries is useful for certain applications, and all three will
therefore continue to be supported.  To better reflect the contents of these
libraries, they have been renamed as follows:

\begin{myenumerate}
\item \ml{sets} is now called \ml{finite\_sets};
\item \ml{all\_sets} is now called \ml{sets};
\item \ml{set} is now called \ml{pred\_sets}.
\end{myenumerate}

The \ml{sets} library (formerly, \ml{all\_sets}) has been extensively revised and
extended.  The library has been restructured, and several additional built-in
theorems are available.  Parser and pretty-printer support for the notation:

\begin{itemize}
\item \ml{\{$x_1,\ldots,x_n$\}}     denotes a finite set.

\item  \ml{\{$x$|$P[x]$\}}      denotes the set of all \ml{$x$} such that
\ml{$P[x]$}.

\item \ml{\{$t[x]$|$P[x]$\}}   denotes the set of all \ml{$t[x]$}
such that \ml{$P[x]$}.
\end{itemize}

\noindent is now provided by the parser/pretty-printer extension mentioned above.

Proof support is supplied for this notation in the form of a conversion

\begin{holboxed}\index{SET_SPEC_CONV@\ml{SET\_SPEC\_CONV}}
\begin{verbatim}
   SET_SPEC_CONV : conv
\end{verbatim}\end{holboxed}

\noindent which implements the axiom of specification for generalized set
specifications.

\ml{SET\_SPEC\_CONV} accepts two types of input.

\begin{hol}
{\small\verb%   SET_SPEC_CONV "%}$t${\small\verb% IN {%}$v${\small\verb% | %}$p${\small\verb%[%}$v${\small\verb%]}"  %}(where $v$ is a variable)
\end{hol}

\noindent this returns the theorem:

\begin{hol}\begin{alltt}
   |- \(t\) IN \{\(v\) | \(p[v]\)\}  =  \(p[t/v]\)
\end{alltt}\end{hol}

\noindent Furthermore

\begin{hol}\begin{alltt}
   SET_SPEC_CONV "\(t\sb{1}\) IN \{\(t\sb{2}[x\sb{1},\ldots,x\sb{n}]\) | \(p[x\sb{1}\ldots,x\sb{n}]\)\}"
\end{alltt}\end{hol}

\noindent returns the theorem

\medskip

{\small
\begin{tabular}{ll}
{\verb%|-%} & $t_1\ \ml{IN}\ ${\verb%{%}$t_2[x_1,\ldots,x_n]${\verb%|%}$p[x_1,\ldots,x_n]${\verb%}%}\\
 & \ml{=}\\
 & {\verb%?%}$x_1\cdots x_n${\verb%. (%}$t_1 \ml{=} t_2[x_1,\ldots,x_n]${\verb%) /\ %}$p[x_1,\ldots,x_n]$\\
\end{tabular}
}

\medskip

\begin{hol}\begin{verbatim}
   |- x IN {n | n < m}  =  x < m

   |- 5 IN {x+y | x < 2 /\ y < 3}  =  ?x y. (5 = x + y) /\ x < 2 /\ y < 3
\end{verbatim}\end{hol}

The library \ml{sets} also makes available an induction principle for
proving properties of finite sets.

\begin{holboxed}\index{SET_INDUCT_TAC@\ml{SET\_INDUCT\_TAC}}
\begin{verbatim}
   SET_INDUCT_TAC : tactic
\end{verbatim}\end{holboxed}

\begin{hol}\begin{alltt}
    "!\(s\). FINITE \(s\) ==>  \(P\)[\(s\)]"
   =============================     SET_INDUCT_TAC
    \(P\)[EMPTY]   \(P\)[\(x\) INSERT \(t\)]
                  [ "FINITE \(t\)" ]
                  [ "\(P\)[\(s\)]"]
                  [ "~\(x\) IN \(t\)"]
\end{alltt}\end{hol}

\noindent The file {\small\verb%Libaries/sets/COMPAT.READ.ME%} provides an
index to theorem names in the new library and other compatibility information.
This is intended to help users update proof scripts based on the old version of
the \ml{all\_sets} library.



\section{Syntax blocks}

\index{syntax blocks|(}
A syntax block starts with a keyword and ends with a terminator and is
associated with a function on strings. When such a block is
encountered in the input stream, all the characters between the start
keyword and the terminator are made into a string to which the
associated function is applied. Syntax blocks can be used with the
parser generator library {\small\verb%parser%} to parse user-specified
languages into \HOL\ terms. See the documentation of the library for
details and examples.

The ML function:

\begin{holboxed}\index{new_syntax_block@\ml{new\_syntax\_block}}
\begin{verbatim}
   new_syntax_block : string # string # string -> void
\end{verbatim}\end{holboxed}


\noindent declares a new syntax block. The first argument is the start keyword
of the block, the second argument is the terminator and the third
argument is the name of a function having a type which is an instance
of {\small\verb%string->*%}. The effect of

\begin{hol}
\ml{   new\_syntax\_block(`XXX`, `YYY`, `}$f$\ml{`);;}
\end{hol}

\noindent is that if subsequently

\begin{hol}\begin{verbatim}
   XXX   ...   YYY
\end{verbatim}\end{hol}

\noindent occurs in the input, then it is as though

\begin{hol}
\ml{   }$f$\ml{ `   ...   `}
\end{hol}

\noindent were input instead. For example:

\setcounter{sessioncount}{1}
\begin{session}\begin{verbatim}
#let foo s = print_string `Hello: `; print_string s; print_newline();;
foo = - : (string -> void)

#new_syntax_block(`<<`,`>>`, `foo`);;
() : void

#<< Mike >>;;
Hello: Mike
() : void

\end{verbatim}\end{session}

The following function is useful with syntax blocks, because it enables
backslash ({\small\verb%\%}) to be included in strings.

\begin{holboxed}\index{set_string_escape@\ml{set\_string\_escape}}
\begin{verbatim}
   set_string_escape : int -> int
\end{verbatim}\end{holboxed}

\noindent This changes the escape character in strings to be the character with the
supplied ascii code (initially this is {\small\verb%92%},
viz. `{\small\verb%\%}`).
The old escape character is returned.\index{syntax blocks|)}

\section{User-programmable quotation typechecker}

The typechecker for \HOL\ quotations contains a number of arbitrary
design decisions. Several people have suggested changes, e.g. that
full Hindley/Milner type inference be performed. Rather than try to
create a single new improved typechecker, a facility is now provided
that enables the user to write his or her own one and then to install
it in the system.

\index{preterms|(}
The \ML\ abstract type {\small\verb%preterm%} represents the parse trees of \HOL\
terms. A typechecker is a function of type {\small\verb%preterm->term%}. If the flag
\ml{preterm} is set to \ml{true} (the default is \ml{false}), then \HOL\ will use
whatever \ML\ function is currently bound to the name
\ml{preterm\_handler} as
the quotation typechecker. The way this works is that when
\ml{preterm} is true the parser produces a preterm rather than a term,
and then wraps a call of \ml{preterm\_handler}
around the quotation.\index{type checking, in HOL logic@type checking,
in \HOL\ logic!user programmed} Other uses of preterms are possible, for example
as patterns for describing terms.

The definition of the \ML\ type {\small\verb%preterm%} is:

\begin{hol}\index{preterm@\ml{preterm}}\begin{alltt}
   rectype preterm =
     preterm_var      of string                        \({\it Variables}\)
   | preterm_const    of string                        \({\it Constants}\)
   | preterm_comb     of preterm # preterm             \({\it Combinations}\)
   | preterm_abs      of preterm # preterm             \({\it Abstractions}\)
   | preterm_typed    of preterm # type                \({\it Explicit typing}\)
   | preterm_antiquot of term                          \({\it Antiquotation}\)
\end{alltt}\end{hol}

The function:

\begin{holboxed}\index{preterm_to_term@\ml{preterm\_to\_term}}
\begin{verbatim}
   preterm_to_term : preterm -> term
\end{verbatim}\end{holboxed}

\noindent invokes the standard \HOL\ typechecker on a preterm and returns the
resulting typechecked term (or causes the standard error message).\index{preterms|)}



The following is a rather contrived example:

\setcounter{sessioncount}{1}
\begin{session}\index{preterm_handler@\ml{preterm\_handler}}
\begin{verbatim}
#letref p_reg = preterm_var `x`;;
p_reg = preterm_var `x` : preterm

#let preterm_handler p = p_reg:=p;
                         print_string `Typechecking ... `;
                         print_newline();
                         preterm_to_term p;;
preterm_handler = - : (preterm -> term)
\end{verbatim}\end{session}

\begin{session}
\begin{verbatim}
#set_flag(`preterm`,true);;
false : bool

#"x+y";;
Typechecking ...
"x + y" : term

#p_reg;;
preterm_comb((preterm_comb((preterm_const `+`), preterm_var `x`)),
             preterm_var `y`)
: preterm
\end{verbatim}\end{session}

\noindent Different top-level typechecking can be defined by using a
different definition of the function
\ml{preterm\_handler}. Note that typechecking
is purely a `user interface' feature, so changing the typechecker does
not compromise the logical soundness of \HOL.




\section{New conversions}\label{CONVERSIONS}


Many new conversions have been added to the system. Only those likely to be of
general interest are listed here.

\subsection{Generalized beta-reduction}

A new conversion:


\begin{holboxed}\index{PAIRED_BETA_CONV@\ml{PAIRED\_BETA\_CONV}}
\begin{verbatim}
   PAIRED_BETA_CONV : conv
\end{verbatim}\end{holboxed}

\noindent has been added to do
generalized beta-conversion of tupled lambda abstractions applied to
tuples.

Given the term:

\begin{hol}
{\small\verb%   "(\(%}$x_1, \ldots ,x_n${\small\verb%).%}$t${\small\verb%) (%}$t_1, \ldots ,t_n${\small\verb%)"%}
\end{hol}

\noindent \ml{PAIRED\_BETA\_CONV} proves that:

\begin{hol}
{\small\verb%   |- (\(%}$x_1, \ldots ,x_n${\small\verb%). %}$t${\small\verb%[%}$x_1,\ldots,x_n${\small\verb%]) (%}$t_1, \ldots ,t_n${\small\verb%)  =  %}$t${\small\verb%[%}$t_1, \ldots ,t_n${\small\verb%]%}
\end{hol}

The conversion works for arbitrarily nested tuples.  For example:

\setcounter{sessioncount}{1}
\begin{session}\begin{verbatim}
#PAIRED_BETA_CONV "(\((a,b),(c,d)). [a;b;c;d]) ((1,2),(3,4))";;
|- (\((a,b),c,d). [a;b;c;d])((1,2),3,4) = [1;2;3;4]
\end{verbatim}\end{session}

\subsection{Arithmetical conversions}

The conversion:

\begin{holboxed}\index{ADD_CONV@\ml{ADD\_CONV}}
\begin{verbatim}
   ADD_CONV : conv
\end{verbatim}\end{holboxed}

\noindent does addition by formal proof.
If $n$ and $m$ are numerals then
{\small\verb%ADD_CONV "%}$n\ $\ml{+}$\ m$\ml{"}
returns the theorem {\small\verb%|- %}$n\ $\ml{+}$\ m\ $\ml{=}$\  s$,
where $s$ is the numeral denoting the sum of $n$ and $m$.  For example:

\setcounter{sessioncount}{1}
\begin{session}\begin{verbatim}
#ADD_CONV "1 + 2";;
|- 1 + 2 = 3

#ADD_CONV "0 + 1000";;
|- 0 + 1000 = 1000

#ADD_CONV "101 + 102";;
|- 101 + 102 = 203
\end{verbatim}\end{session}



A new conversion has been added for deciding equality of natural number
constants.

\begin{holboxed}\index{num_EQ_CONV@\ml{num\_EQ\_CONV}}
\begin{verbatim}
   num_EQ_CONV : conv
\end{verbatim}\end{holboxed}

\noindent If $n$ and $m$ are terms constructed from numeral constants
and the successor function \ml{SUC}, then:
{\small\verb%num_EQ_CONV "%}$n$\ml{=}$m$\ml{"}
returns:

\begin{hol}
\ml{   |- (\(n\)=\(m\)) = T  }if \(n\) and \(m\) represent the same number,\\
\ml{   |- (\(n\)=\(m\)) = F  }if \(n\) and \(m\) represent different numbers.
\end{hol}

\noindent In addition, {\small\verb%num_EQ_CONV "%}$t\ $\ml{=}$\ t$\ml{"}
returns: {\small\verb%|- (%}$t$\ml{=}$t${\small\verb%) = T%}

\subsection{List processing conversions}

Two new conversions for lists are now built-in:

\begin{holboxed}
\index{LENGTH_CONV@\ml{LENGTH\_CONV}}
\index{list_EQ_CONV@\ml{list\_EQ\_CONV}}
\begin{verbatim}
   LENGTH_CONV : conv
   list_EQ_CONV: conv
\end{verbatim}\end{holboxed}

\ml{LENGTH\_CONV}: computes the length of a list.
A call to:

\begin{hol}\begin{alltt}
   LENGTH_CONV "LENGTH[\(t\sb{1}\);\(\ldots\);\(t\sb{n}\)]"
\end{alltt}\end{hol}

\noindent generates the theorem:

\begin{hol}\begin{alltt}
   |- LENGTH [\(t\sb{1}\);\(\ldots\);\(t\sb{n}\)] = n
\end{alltt}\end{hol}

The other conversion, \ml{list\_EQ\_CONV}, proves or disproves the
equality of two lists, given
a conversion for deciding the equality of elements.
A call to:

\begin{hol}\begin{alltt}
   list_EQ_CONV \(conv\) "[\(u\sb{1}\);\(\ldots\);\(u\sb{n}\)] = [\(v\sb{1}\);\(\ldots\);\(v\sb{m}\)]"
\end{alltt}\end{hol}

\noindent returns: {\small\verb%|- ([%}$u_1$\m{;}$\ldots$\ml{;}$u_n$\ml{]\ =\ [}$v_1$\ml{;}$\ldots$\ml{;}$v_m$\ml{])\ =\ F} if:

\begin{myenumerate}
\item \ml{~($n$=$m$)}  or
\item \ml{$conv$} proves {\small\verb%|- (%}$u_i\ $\ml{=}$\ v_i$\ml{)\ =\ F}
for any $1\leq i \leq m$.
\end{myenumerate}

\noindent {\small\verb%|- ([%}$u_1$\m{;}$\ldots$\ml{;}$u_n$\ml{]\ =\ [}$v_1$\ml{;}$\ldots$\ml{;}$v_m$\ml{])\ =\ T} is returned if:

\begin{myenumerate}

\item \ml{($n$=$m$)} and \ml{$u_i$} is syntactically identical to
\ml{$v_i$} for $1\leq i \leq m$, or
\item \ml{($n$=$m$)} and \ml{$conv$} proves
{\small\verb%|- (%}$u_i$\ml{=}$v_i$\ml{)=T} for $1\leq i \leq n$.
\end{myenumerate}

\subsection{Simplifying {\tt let}-terms}
\index{let-terms, in HOL logic@\ml{let}-terms, in \HOL\ logic!simplification of}

A conversion for reducing {\tt let}-terms is now provided.

\begin{holboxed}\index{let_CONV@\ml{let\_CONV}}
\begin{verbatim}
   let_CONV : conv
\end{verbatim}\end{holboxed}

\noindent Given a term:

\begin{hol}\begin{alltt}
   "let \(v\sb{1}\) = \(t\sb{1}\) and \(\cdots\) and \(v\sb{n}\) = \(t\sb{n}\) in \(t[v\sb{1},\ldots,v\sb{n}]\)"
\end{alltt}\end{hol}

\noindent \ml{let\_CONV} proves that:

\begin{hol}\begin{alltt}
   |- let \(v\sb{1}\) = \(t\sb{1}\) and \(\cdots\) and \(v\sb{n}\) = \(t\sb{n}\) in \(t[v\sb{1},\ldots,v\sb{n}]\)  =  \(t[t\sb{1},\ldots,t\sb{n}]\)
\end{alltt}\end{hol}

\noindent The \ml{$v_i$}'s can take any one of the following forms:

\begin{myenumerate}
\item Variables:    \ml{x} etc.
\item Tuples:   \ml{(x,y)}, \ml{(a,b,c)}, \ml{((a,b),(c,d))} etc.
\item Applications: \ml{f (x,y) z}, \ml{f x} etc.
\end{myenumerate}

\noindent Variables are just substituted for. With tuples, the substitution is
done component-wise, and function applications are effectively
rewritten in the body of the {\tt let}-term.

\setcounter{sessioncount}{1}
\begin{session}\begin{verbatim}
#let_CONV "let x = 1 in x+y";;
|- (let x = 1 in x + y) = 1 + y

#let_CONV "let (x,y) = (1,2) in x+y";;
|- (let (x,y) = 1,2 in x + y) = 1 + 2

#let_CONV "let f x = 1 and f y = 2 in (f 10) + (f 20)";;
|- (let f x = 1 and f y = 2 in (f 10) + (f 20)) = 2 + 2

#let_CONV "let f x = x + 1 and g x = x + 2 in f(g(f(g 0)))";;
|- (let f x = x + 1 and g x = x + 2 in f(g(f(g 0)))) =
   (((0 + 2) + 1) + 2) + 1

#CONV_RULE(DEPTH_CONV ADD_CONV)it;;
|- (let f x = x + 1 and g x = x + 2 in f(g(f(g 0)))) = 6

#let_CONV "let f x y = x+y in f 1";;  % NB: partial application %
|- (let f x y = x + y in f 1) = (\y. 1 + y)
\end{verbatim}\end{session}

\subsection{Skolemization}\index{Skolemization}

Two conversions are provided for a higher-order version of
Skolemization (using existentially quantified function variables
rather than first-order Skolem constants).

The conversion

\begin{holboxed}\index{X_SKOLEM_CONV@\ml{X\_SKOLEM\_CONV}}
\begin{verbatim}
   X_SKOLEM_CONV : term -> conv
\end{verbatim}\end{holboxed}

\noindent takes a variable parameter, \ml{$f$} say, and
proves:

\begin{hol}\begin{alltt}
   |- (!\(x\sb{1}\) \(\ldots\) \(x\sb{n}\). ?\(y\). \(t[x\sb{1},\ldots,x\sb{n},y]\)  =  (?\(f\). !\(x\sb{1}\) \(\ldots\) \(x\sb{n}\). \(t[x\sb{1},\ldots,x\sb{n},f x\sb{1}\ldots x\sb{n}]\)
\end{alltt}\end{hol}

\noindent for any input term
\ml{!$x_1\ \ldots\ x_n$.\ ?$y$. $t[x_1,\ldots,x_n,y]$}.
Note that when $n=\ml{0}$, this
is equivalent to alpha-conversion:

\begin{hol}\begin{alltt}
  |- (?\(y\). \(t[y]\)) = (?\(f\). \(t[f]\))
\end{alltt}\end{hol}

\noindent and that the conversion fails if there is already a free
variable \ml{$f$} of the appropriate type in the input term. For example:

\begin{hol}\begin{alltt}
  X_SKOLEM_CONV "f:num->*" "!n:num. ?x:*. x = (f n)"
\end{alltt}\end{hol}

\noindent will fail.  The conversion \ml{SKOLEM\_CONV} is
like \ml{X\_SKOLEM\_CONV}, except that it
uses a primed variant of the name of the existentially quantified variable
as the name of the skolem function it introduces.  For example:

\begin{hol}\begin{alltt}
  SKOLEM_CONV "!x. ?y. P x y"
\end{alltt}\end{hol}

\noindent proves that:

\begin{hol}\begin{alltt}
  |- ?y. !x. P x (y x)
\end{alltt}\end{hol}


\subsection{Quantifier movement conversions}
\index{quantifiers!conversions for}
\index{conversions!quantifier movement}

A complete and systematically-named set of conversions for moving quantifiers
inwards and outwards through the logical connectives {\small\verb%~%},
{\small\verb%/\%}, {\small\verb%\/%}, and {\small\verb%==>%} has
been added to the system.  The naming scheme is based on the following
atoms:

\begin{hol}\begin{alltt}
   <\(quant\)> := FORALL | EXISTS
   <\(conn\)>  := NOT | AND | OR | IMP
   [\(dir\)]   := LEFT | RIGHT          \({\it (optional)}\)
\end{alltt}\end{hol}



The conversions for moving quantifiers inwards are called:

\begin{hol}\begin{alltt}
   <\(quant\)>_<\(conn\)>_CONV
\end{alltt}\end{hol}

\noindent where the quantifier \ml{<$quant$>} is to be moved inwards
through \ml{<$conn$>}.

The conversions for moving quantifiers outwards are called:

\begin{hol}\begin{alltt}
   [\(dir\)]_<\(conn\)>_<\(quant\)>_CONV
\end{alltt}\end{hol}

\noindent where \ml{<$quant$>} is to be moved outwards
through \ml{<$conn$>}, and the optional
\ml{[$dir$]} identifies which operand (left or right) contains the quantifier.
The complete set is:

\begin{hol}\begin{verbatim}
   NOT_FORALL_CONV    |- ~(!x.P) = ?x.~P
   NOT_EXISTS_CONV    |- ~(?x.P) = !x.~P
   EXISTS_NOT_CONV    |- (?x.~P) = ~!x.P
   FORALL_NOT_CONV    |- (!x.~P) = ~?x.P
\end{verbatim}\end{hol}

\begin{hol}\begin{verbatim}
   FORALL_AND_CONV         |- (!x. P /\ Q) = (!x.P) /\ (!x.Q)
   AND_FORALL_CONV         |- (!x.P) /\ (!x.Q) = (!x. P /\ Q)
   LEFT_AND_FORALL_CONV    |- (!x.P) /\ Q = (!x'. P[x'/x] /\ Q)
   RIGHT_AND_FORALL_CONV   |- P /\ (!x.Q) = (!x'. P /\ Q[x'/x])
\end{verbatim}\end{hol}

\begin{hol}\begin{verbatim}
   EXISTS_OR_CONV          |- (?x. P \/ Q) = (?x.P) \/ (?x.Q)
   OR_EXISTS_CONV          |- (?x.P) \/ (?x.Q) = (?x. P \/ Q)
   LEFT_OR_EXISTS_CONV     |- (?x.P) \/ Q = (?x'. P[x'/x] \/ Q)
   RIGHT_OR_EXISTS_CONV    |- P \/ (?x.Q) = (?x'. P \/ Q[x'/x])
\end{verbatim}\end{hol}

\begin{hol}\begin{verbatim}
   FORALL_OR_CONV
     |- (!x.P \/ Q) = P \/ !x.Q            [x not free in P]
     |- (!x.P \/ Q) = (!x.P) \/ Q          [x not free in Q]
     |- (!x.P \/ Q) = (!x.P) \/ (!x.Q)     [x not free in P or Q]
\end{verbatim}\end{hol}

\begin{hol}\begin{verbatim}
   OR_FORALL_CONV
     |- (!x.P) \/ (!x.Q) = (!x.P \/ Q)     [x not free in P or Q]
\end{verbatim}\end{hol}

\begin{hol}\begin{verbatim}
   LEFT_OR_FORALL_CONV    |- (!x.P) \/ Q = !x'. P[x'/x] \/ Q
   RIGHT_OR_FORALL_CONV   |- P \/ (!x.Q)  = !x'. P \/ Q[x'/x]
\end{verbatim}\end{hol}

\begin{hol}\begin{verbatim}
   EXISTS_AND_CONV
     |- (?x.P /\ Q) = P /\ ?x.Q            [x not free in P]
     |- (?x.P /\ Q) = (?x.P) /\ Q          [x not free in Q]
     |- (?x.P /\ Q) = (?x.P) /\ (?x.Q)     [x not free in P or Q]
\end{verbatim}\end{hol}

\begin{hol}\begin{verbatim}
   AND_EXISTS_CONV
     |- (?x.P) /\ (?x.Q) = (?x.P /\ Q)     [x not free in P or Q]
\end{verbatim}\end{hol}

\begin{hol}\begin{verbatim}
   LEFT_AND_EXISTS_CONV    |- (?x.P) /\ Q = ?x'. P[x'/x] /\ Q
   RIGHT_AND_EXISTS_CONV   |- P /\ (?x.Q)  = ?x'. P /\ Q[x'/x]
\end{verbatim}\end{hol}

\begin{hol}\begin{verbatim}
   FORALL_IMP_CONV
     |- (!x.P ==> Q) = P ==> !x.Q          [x not free in P]
     |- (!x.P ==> Q) = (?x.P) ==> Q        [x not free in Q]
     |- (!x.P ==> Q) = (?x.P) ==> (!x.Q)   [x not free in P or Q]
\end{verbatim}\end{hol}

\begin{hol}\begin{verbatim}
   LEFT_IMP_FORALL_CONV   |- (!x.P) ==> Q = !x'. P[x/'x] ==> Q
   RIGHT_IMP_FORALL_CONV  |- P ==> (!x.Q) = !x'. P ==> Q[x'/x]
\end{verbatim}\end{hol}

\begin{hol}\begin{verbatim}
   EXISTS_IMP_CONV
     |- (?x.P ==> Q) = P ==> ?x.Q          [x not free in P]
     |- (?x.P ==> Q) = (!x.P) ==> Q        [x not free in Q]
     |- (?x.P ==> Q) = (!x.P) ==> (?x.Q)   [x not free in P or Q]
\end{verbatim}\end{hol}

\begin{hol}\begin{verbatim}
   LEFT_IMP_EXISTS_CONV   |- (?x.P) ==> Q = !x'. P[x/'x] ==> Q
   RIGHT_IMP_EXISTS_CONV  |- P ==> (?x.Q) = ?x'. P ==> Q[x'/x]
\end{verbatim}\end{hol}

\section{New built-in theorems}\label{THEOREMS}

Quite a few new theorems have been added to \HOL.  Some are listed
below, others can be found in the libraries and in \ml{contrib}.

\subsection{New arithmetical theorems}

The built-in theory arithmetic has been augmented with a minimal theory of
the functions \ml{MOD} and \ml{DIV} on the natural numbers.  The old definitions of
\ml{MOD} and \ml{DIV}:

\begin{hol}\begin{verbatim}
   MOD         |- MOD k n = @r. ?q. (k = (q * n) + r) /\ r < n
   DIV         |- DIV k n = @q. (k = (q * n) + (k MOD n))
\end{verbatim}\end{hol}

\noindent have been deleted, and replaced by the constant specification:

\begin{hol}\begin{verbatim}
   DIVISION    |- !n. 0 < n ==>
                   (!k. (k = ((k DIV n) * n) + (k MOD n)) /\ (k MOD n) < n)
\end{verbatim}\end{hol}

\noindent the validity of which is justified by the division algorithm:

\begin{hol}\begin{verbatim}
   DA          |- !k n. 0 < n ==> (?r q. (k = (q * n) + r) /\ r < n)
\end{verbatim}\end{hol}

The following theorems are also available:

\begin{hol}\begin{verbatim}
   MOD_ONE      |- !k. k MOD (SUC 0) = 0
   DIV_LESS_EQ  |- !n. 0 < n ==> (!k. (k DIV n) <= k)
   DIV_UNIQUE   |- !n k q. (?r. (k = (q * n) + r) /\ r < n) ==> (k DIV n = q)
   MOD_UNIQUE   |- !n k r. (?q. (k = (q * n) + r) /\ r < n) ==> (k MOD n = r)
   DIV_MULT     |- !n r. r < n ==> (!q. ((q * n) + r) DIV n = q)
   LESS_MOD     |- !n k. k < n ==> (k MOD n = k)
   MOD_EQ_0     |- !n. 0 < n ==> (!k. (k * n) MOD n = 0)
   ZERO_MOD     |- !n. 0 < n ==> (0 MOD n = 0)
   MOD_MULT     |- !n r. r < n ==> (!q. ((q * n) + r) MOD n = r)
   MOD_TIMES    |- !n. 0 < n ==> (!q r. ((q * n) + r) MOD n = r MOD n)
   MOD_PLUS     |- !n. 0 < n ==>
                       (!j k. ((j MOD n) + (k MOD n)) MOD n = (j + k) MOD n)
   MOD_MOD      |- !n. 0 < n ==> (!k. (k MOD n) MOD n = k MOD n)
\end{verbatim}\end{hol}



The following miscellaneous arithmetical theorems
are now pre-proved.

\begin{hol}\begin{verbatim}
   ZERO_LESS_EQ         |- !n. 0 <= n
   LESS_EQ_MONO         |- !n m. (SUC n <= SUC m) = (n <= m)
   LESS_OR_EQ_ADD       |- !n m. n < m \/ ?p. n = p+m
   SUC_NOT              |- !n. ~(0 = SUC n)
   SUB_MONO_EQ          |- !n m. (SUC n) - (SUC m) = n - m
   SUB_PLUS             |- !a b c. a - (b + c) = (a - b) - c
   INV_PRE_LESS         |- !m n. 0 < m /\ 0 < n ==> ((PRE m)<(PRE n) = m<n)
   INV_PRE_LESS_EQ      |- !n. 0 < n ==> !m. ((PRE m)<=(PRE n) = m<=n)
   SUB_LESS_EQ          |- !n m. (n - m) <= n
   SUB_EQ_EQ_0          |- !m n. (m - n = m) = (m = 0) \/ (n = 0)
   SUB_LESS_0           |- !n m. m < n = 0 < (n - m)
   SUB_LESS_OR          |- !m n. n < m ==> n <= (m - 1)
   LESS_SUB_ADD_LESS    |- !n m i. i < (n - m) ==> (i + m) < n
   TIMES2               |- !n. 2 * n = n + n
   LESS_MULT_MONO       |- !m i n. ((SUC n) * m) < ((SUC n) * i) = m < i
   MULT_MONO_EQ         |- !m i n. ((SUC n) * m = (SUC n) * i) = (m = i)
   ADD_SUB              |- !a c. (a + c) - c = a
   LESS_EQ_ADD_SUB      |- !c b. c <= b ==> (!a. (a + b) - c = a + (b - c))
   SUB_EQUAL_0          |- !c. c - c = 0
   LESS_EQ_SUB_LESS     |- !a b. b <= a ==> (!c. (a - b) < c = a < (b + c))
   NOT_SUC_LESS_EQ      |- !n m. ~(SUC n) <= m = m <= n
   SUB_SUB              |- !b c. c <= b ==> (!a. a - (b - c) = (a + c) - b)
   LESS_IMP_LESS_ADD    |- !n m. n < m ==> (!p. n < (m + p))
   LESS_EQ_IMP_LESS_SUC |- !n m. n <= m ==> n < (SUC m)
\end{verbatim}\end{hol}
\begin{hol}\begin{verbatim}
   SUB_LESS_EQ_ADD      |- !m p. m <= p ==> (!n. (p - m) <= n = p <= (m + n))
   SUB_CANCEL           |- !p n m. n <= p /\ m <= p ==>
                                   ((p - n = p - m) = (n = m))
   CANCEL_SUB           |- !p n m. p <= n /\ p <= m ==>
                                   ((n - p = m - p) = (n = m))
   NOT_EXP_0            |- !m n. ~((SUC n) EXP m = 0)
   ZERO_LESS_EXP        |- !m n. 0 < ((SUC n) EXP m)
   ODD_OR_EVEN          |- !n. ?m. (n = (SUC(SUC 0)) * m) \/
                                   (n = ((SUC(SUC 0)) * m) + 1)
   LESS_EXP_SUC_MONO    |- !n m.
                            ((SUC(SUC m)) EXP n)<((SUC(SUC m)) EXP (SUC n))
   LESS_EQUAL_ANTISYM   |- !n m. n <= m /\ m <= n ==> (n = m)
   LESS_ADD_SUC         |- !m n. m < (m + (SUC n))

\end{verbatim}\end{hol}


\subsection{New theorems about lists}

\begin{hol}\begin{verbatim}
   LIST_NOT_EQ: |- !l1 l2. ~(l1 = l2) ==> (!h1 h2. ~(CONS h1 l1 = CONS h2 l2))
   NOT_EQ_LIST: |- !h1 h2. ~(h1 = h2) ==> (!l1 l2. ~(CONS h1 l1 = CONS h2 l2))
   EQ_LIST      |- !h1 h2. (h1 = h2) ==>
                           (!l1 l2. (l1 = l2) ==> (CONS h1 l1 = CONS h2 l2))
\end{verbatim}\end{hol}

\subsection{New propositional theorems}

\begin{hol}\begin{verbatim}
   LEFT_AND_OVER_OR   |- !t1 t2 t3. t1 /\ (t2 \/ t3) = t1 /\ t2 \/ t1 /\ t3
   RIGHT_AND_OVER_OR  |- !t1 t2 t3. (t2 \/ t3) /\ t1 = t2 /\ t1 \/ t3 /\ t1
   LEFT_OR_OVER_AND   |- !t1 t2 t3. t1 \/ t2 /\ t3 = (t1 \/ t2) /\ (t1 \/ t3)
   RIGHT_OR_OVER_AND  |- !t1 t2 t3. t2 /\ t3 \/ t1 = (t2 \/ t1) /\ (t3 \/ t1)
\end{verbatim}\end{hol}
\begin{hol}\begin{verbatim}
   OR_IMP_THM         |- !t1 t2. (t1 = t2 \/ t1) = t2 ==> t1
   NOT_IMP            |- !t1 t2. ~(t1 ==> t2) = t1 /\ ~t2
   COND_ID            |- !b t. (b => t | t) = t
   DISJ_ASSOC         |- !t1 t2 t3. t1 \/ t2 \/ t3 = (t1 \/ t2) \/ t3
\end{verbatim}\end{hol}


\section{Additions and deletions}\label{adddel}

The \ML\ identifiers that have been added and deleted in Version 2.0
are listed here. The reasons for the changes are given in the files:

\begin{hol}\begin{verbatim}
   hol/Version.2.0
   hol/Versions/Version.1.12
   hol/Versions/Versions.pre.1.12
\end{verbatim}\end{hol}

\noindent The file \ml{Version.$m$.$n$} (in the sources) documents \HOL\
Version \ml{$m$.$n$} and may contain details not in \DESCRIPTION\ or
\REFERENCE.

\subsection{Additions}

Here is a list of all the new functions in Version 2.0, the ones that
are not described in this appendix or earlier in \DESCRIPTION\ should be
documented in \REFERENCE.

\begin{hol}\begin{verbatim}
   get_const_type            parity                    is_type
   mk_let                    dest_let                  is_let
   bndvar                    body                      is_axiom
\end{verbatim}\end{hol}
\begin{hol}\begin{verbatim}
   mk_cons                   dest_cons                 is_cons
   mk_list                   dest_list                 is_list
   list_mk_pair              strip_mk_pair
\end{verbatim}\end{hol}
\begin{hol}\begin{verbatim}
   ancestors                 ancestry
\end{verbatim}\end{hol}
\begin{hol}\begin{verbatim}
   prove_rep_fn_one_one      prove_rep_fn_onto
   prove_abs_fn_one_one      prove_abs_fn_onto
\end{verbatim}\end{hol}
\begin{hol}\begin{verbatim}
   define_finite_set_syntax  define_set_abstraction_syntax
   associate_restriction
\end{verbatim}\end{hol}
\begin{hol}\begin{verbatim}
   X_FUN_EQ_CONV             COND_CONV                 PAIRED_BETA_CONV
   BOOL_EQ_CONV              ADD_CONV                  LENGTH_CONV
   list_EQ_CONV              num_EQ_CONV               let_CONV
   EXISTS_LEAST_CONV         X_SKOLEM_CONV             SKOLEM_CONV
   NOT_FORALL_CONV           NOT_EXISTS_CONV
   EXISTS_NOT_CONV           FORALL_NOT_CONV
   FORALL_AND_CONV           AND_FORALL_CONV
   LEFT_AND_FORALL_CONV      RIGHT_AND_FORALL_CONV
   EXISTS_OR_CONV            OR_EXISTS_CONV
   LEFT_OR_EXISTS_CONV       RIGHT_OR_EXISTS_CONV
   FORALL_OR_CONV            OR_FORALL_CONV
   LEFT_OR_FORALL_CONV       RIGHT_OR_FORALL_CONV
   EXISTS_AND_CONV           AND_EXISTS_CONV
   LEFT_AND_EXISTS_CONV      RIGHT_AND_EXISTS_CONV
   FORALL_IMP_CONV           EXISTS_IMP_CONV
   LEFT_IMP_FORALL_CONV      RIGHT_IMP_FORALL_CONV
   LEFT_IMP_EXISTS_CONV      RIGHT_IMP_EXISTS_CONV
\end{verbatim}\end{hol}
\begin{hol}\begin{verbatim}
   last                      butlast
\end{verbatim}\end{hol}
\begin{hol}\begin{verbatim}
   getenv                    library_pathname          is_ml_infix
   set_help_search_path      help_search_path
\end{verbatim}\end{hol}
\begin{hol}\begin{verbatim}
   EQF_INTRO                 ISPEC                     ISPECL
   EQF_ELIM                  EXISTENCE                 IMP_CONJ
   EXIST_IMP
\end{verbatim}\end{hol}
\begin{hol}\begin{verbatim}
   REPEAT_GTCL               AP_THM_TAC
\end{verbatim}\end{hol}


\subsection{Deletions}

The large number of deletions listed below are mostly the result of a ``spring
cleaning'' of the \HOL\ sources. It is hoped that nothing in use has been
deleted by mistake---please let us know if you want anything back.

\begin{hol}\begin{verbatim}
   abbrev_ty_def             EXPAND_TY_DEFINITION
   pp_constants              pp_axioms                 pp_theorem
   pp_delete_theorem         pp_theorems
   new_pp_theory             close_pp_theory
   load_pp_theory            extend_pp_theory          new_pp_predicate
   mk_quant                  mk_bool_comb space        ascii_ize
   dest_quant                dest_bool_comb            truth
   mk_iff                     dest_iff                 is_iff
\end{verbatim}\end{hol}
\begin{hol}\begin{verbatim}
   TOTALLY_AD_HOC_LEMMA
\end{verbatim}\end{hol}
\begin{hol}\begin{verbatim}
   AND_CLAUSE1               AND_CLAUSE2               AND_CLAUSE3
   AND_CLAUSE4               AND_CLAUSE5
   OR_CLAUSE1                OR_CLAUSE2                OR_CLAUSE3
   OR_CLAUSE4                OR_CLAUSE5
   IMP_CLAUSE1               IMP_CLAUSE2               IMP_CLAUSE3
   IMP_CLAUSE4               IMP_CLAUSE5
   EQ_CLAUSE1                EQ_CLAUSE2                EQ_CLAUSE3
   EQ_CLAUSE4
   COND_CLAUSE1              COND_CLAUSE2
\end{verbatim}\end{hol}
\begin{hol}\begin{verbatim}
   paired_mem                paired_tryfind            paired_aconv_term
   paired_map                paired_filter             paired_subst_term
   paired_forall             paired_mapfilter          paired_subst_occs_term
   paired_exists             paired_rev_itlist         paired_term_freein_term
   paired_find               paired_term_match         paired_type_in_term
   paired_type_in_type       paired_set_left           paired_axiom
   paired_inst_type          paired_set_right          paired_inst_term
   paired_cons               paired_variant            paired_eq
   paired_form_match         paired_aconv_form         paired_subst_form
   paired_subst_occs_form    paired_term_freein_form   paired_term_freein_form
   paired_type_in_form       paired_inst_form
\end{verbatim}\end{hol}
\begin{hol}\begin{verbatim}
   new_curried_infix         curried_infixes           paired_infixes
\end{verbatim}\end{hol}
\begin{hol}\begin{verbatim}
   rename_form               rename_term               print_form
   mk_inequiv                dest_inequiv              is_inequiv
   is_equiv                  dest_equiv                mk_equiv
   gen_all                   new_predicate
   make_tuple                list_variant              nil
   diagonal                  eqfst                     eqsnd
   flip                      mk_tuple                  dest_tuple
\end{verbatim}\end{hol}
\begin{hol}\begin{verbatim}
   OLD_IMP_RES_TAC           HOL_IMP_RES_THEN          HOL_STRIP_THM_THEN
   OLD_RES_TAC               HOL_RES_THEN              HOL_STRIP_ASSUME_TAC
   OLD_IMP_RES_THEN          RES_CANON_FUN             HOL_MATCH_MP
   OLD_RES_THEN              MULTI_MP                  HOL_RESOLVE_THEN
   MP_CHAIN                  REDEPTH_CHAIN             SUB_CHAIN
   ONCE_DEPTH_CHAIN          DSPEC                     DSPECL
\end{verbatim}\end{hol}
\begin{hol}\begin{verbatim}
   print_tok_ty              print_tok_thm             apply_type_op
   print_all_tok_thm
   int_of_tok                (replaced by int_of_string)
   tok_of_int                (replaced by string_of_int)
   int_to_term               term_to_int
   form_class                form_frees
   form_vars                 form_tyvars
   term_frees                (equivalent to frees)
   term_vars                 (equivalent to vars)
   term_tyvars               (equivalent to tyvars)
   term_class
\end{verbatim}\end{hol}
\begin{hol}\begin{verbatim}
   seg                       (now in library `eval`)
   word_seg                  (now in library `eval`)
   word_el                   (now in library `eval`)
   truncate                  (now in library `eval`)
   upto
   keyword
\end{verbatim}\end{hol}
\begin{hol}\begin{verbatim}
   LESS_OR                   (replaced by LESS_SUC_EQ)
   GSUBS
\end{verbatim}\end{hol}
\begin{hol}\begin{verbatim}
   DEPTH_EXISTS_CONV         DISJS_CONV                BINOP_CONV
   ANTE_FORALL_CONV
   BOOL_EQ_CONV              (renamed bool_EQ_CONV)
   MOVE_EXISTS_OUT_CONV      (replaced by RIGHT_AND_EXISTS_CONV)
\end{verbatim}\end{hol}
\begin{hol}\begin{verbatim}
   is_string                 is_int                    is_cons
   obj_of_string             obj_of_int                cons
   string_of_obj             int_of_obj
   left                      right
   set_left                  set_right
   eq                        lisp_eval
   make_set                  (renamed to setify)
   thenf                     orelsef                   all_fun
   no_fun                    first_fun                 every_fun
   repeatf                   mapshape
 \end{verbatim}\end{hol}
\begin{hol}\begin{verbatim}
   BETA_RED_CONV             LENGTH_MAP_CONV           make_definitions
   BINDER_CONV               LEN_SIMP_CONV             mk_def
   CHOOSE_RULE               LHS_CONV                  mk_defs
   COND_F                    MAP_ELIM                  mk_fn_ty
   COND_T                    MOVE_FORALL_IN            mk_fun_test
   CONJS_CONV                OR_ELIM_CONV              mk_funct
   DEPTH_FORALL_CONV         OR_IMP                    mk_injs
   DO_ASM                    PROJ                      mk_new_vars
   ELIM_ANTE_EQNS_CONV       RM_LEN_CONV               mk_subset_pred
   ELIM_ANTE_EQN_CONV        SIMP_CONV                 mk_sum_ty
   ELIM_LEN_CONV             SIMP_ISL1                 mk_test
   ELIM_MAP_CONV             SIMP_ISL2                 mk_tl_spec
   EQ_ANTE_ELIM              SIMP_RHS                  mk_tuple_ty
   EXISTS_RULE               TEST_CONV                 pars_ty
\end{verbatim}\end{hol}
\begin{hol}\begin{verbatim}
   E_I_CONV                  clean                     parse_type
   FN_BETA                   cleanup                   proj
   FN_CASES_SIMP             do_abs                    prove_existence_thm
   FN_CASE_SIMP              extract_list              strip_cases
   FORALL_AND_CONV           extract_tuple             strip_tok
   FORALL_CONJ_CONV          gen_names                 strip_val
   FOUR_BINDER_CONV          infix_ty                  strip_vty_tok
   FST_SND_SIMP              is_char                   sub_conv
   HD_TL_SIMP                list_gen_alpha            ty_case
   variant_tyvar             rotate_left               rotate_right
   derive_existence_thm      instantiate_existence_thm
   mk_fn                     closeup
\end{verbatim}\end{hol}
\begin{hol}\begin{verbatim}
   CLOSE_UP                  save_open_thm
   HOLdir                    set_hol_lib_dir
   load_from_lib             loadt_from_lib            loadf_from_lib
   loadx                     theories_dir_pathname
\end{verbatim}\end{hol}
\begin{hol}\begin{verbatim}
   IFF_DEF  I                FF_EQ_THM1                IFF_EQ_THM2
   IFF_EQ                    IFF_EQ_RULE               CONJ_IFF
   IFF_CONJ                  IFF_THEN2                 IFF_THEN
   IFF_TAC                   EXISTS_REFL_TAC
\end{verbatim}\end{hol}
\begin{hol}\begin{verbatim}
   define_new_type_isomorphisms    (renamed define_new_type_bijections)
\end{verbatim}\end{hol}
\begin{hol}\begin{verbatim}
   mk_rewrites               mk_rewritesl              REWRITE_CONV
\end{verbatim}\end{hol}


