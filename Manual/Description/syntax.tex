\chapter{Syntax}
{
 \newcommand{\term}       {\mbox{\it term}}
 \newcommand{\bs}         {\mbox{$\backslash$}}
 \newcommand{\IMP}       {\mbox{\tt ==>}}
 \newcommand{\ALL}       {\mbox{\tt !}}
 \newcommand{\EXISTS}       {\mbox{\tt ?}}
 \newcommand{\CHOOSE}       {\mbox{\tt @}}
 \newcommand{\EXISTSONE}       {\mbox{\tt ?!}}
 \newcommand{\LET}       {\mbox{\tt let}}
 \newcommand{\und}       {\mbox{\tt and}}
 \newcommand{\IN}       {\mbox{\tt in}}
 \newcommand{\CONS}       {\mbox{\tt CONS}}
 \newcommand{\INSERT}    {\mbox{\tt INSERT}}
 \newcommand{\SUC}       {\mbox{\tt SUC}}
 \newcommand{\vstr}       {\mbox{\it vstr}}
 \newcommand{\numeral}    {\mbox{\it numeral}}
 \newcommand{\charseq}    {\mbox{\it charseq}}

The HOL logic is a classical higher-order predicate calculus. Its
syntax enjoys two main differences from the syntax of standard first
order logic.\footnote{We assume the reader is familiar with first
  order logic.}  First, there is no distinction in HOL between terms
and formulas: HOL has only terms. Second, each term has a type: types
are used in order to build well-formed terms. There are two ways to
construct types and terms in HOL: by use of a parser, or by use of the
programmer's interface. In this chapter, we will focus on the concrete
syntax accepted by the parsers, leaving the programmer's interface for
Chapter~\ref{api}.


\section{Types}

A HOL type can be a variable, a constant, or a compound type, which is
a constant of arity $n$ applied to a list of $n$ types.
\[
\begin{array}{rclr}
  \type & ::= & \mbox{\bf '}\ident & \mbox{(type variable)} \\
  & | &  \verb+bool+ & \mbox{(type of truth values)} \\
  & | &  \verb+ind+ & \mbox{(type of individuals)} \\
  & | &  \type\ \verb+->+\ \type & \mbox{(function arrow)} \\
  & | &  \type\ \ident\ \type\ & \mbox{(binary compound type)}\\
  & | &  \ident & \mbox{(nullary type constant)} \\
  & | & \type\; \ident & \mbox{(unary compound type)} \\
  & | & (\type_1,\ldots,\type_n) \ident & \mbox{(compound type)}
\end{array}
\]
Type constants are also known as type operators. They must be
alphanumeric. Type variables are alphanumerics written with a leading
prime ('). In \holn{}, the type constants {\tt bool}, {\tt fun}, and
{\tt ind} are primitive. The introduction of new type constants is
described in Chapter \ref{thy-ops}. {\tt bool} is the two element type
of truth values. The binary operator {\tt fun} is used to denote
function types; it can be written with an infix arrow. The nullary
type constant {\tt ind} denotes an infinite set of individuals; it is
used for a few highly technical developments in the system and can be
ignored by beginners.  Thus
\begin{verbatim}
     'a -> 'b
     (bool -> 'a) -> ind
\end{verbatim}
are both well-formed types. The function arrow is "right associative",
which means that ambiguous uses of the arrow in types are resolved by
adding parentheses in a right-to-left sweep: thus the type expression
\begin{verbatim}
     ind -> ind -> ind -> ind
\end{verbatim}
is identical to
\begin{verbatim}
     ind -> (ind -> (ind -> ind)).
\end{verbatim}
The product (\verb+#+) and sum (\verb!+!) are other infix type
operators, also right associative; however, they are not loaded by
default in \holn{}. How to load in useful logical context is dealt with
in Chapter~\ref{context}.

\section{Terms}

Ultimately, a HOL term can only be a variable, a constant, an
application, or a lambda term.
\[
\begin{array}{rclr}
  \term & ::= & \ident & \mbox{(variable or constant)} \\
  & | &  \term\  \term & \mbox{(combination)} \\
  & | &  \bs\ident.\  \term &
  \mbox{(lambda abstraction)}
\end{array}
\]
In the system, the usual logical operators have already been defined,
including truth (\verb+T+), falsity (\verb+F+), negation (\verb+~+),
equality (\verb+=+), conjunction (\verb+/\+), disjunction (\verb+\/+),
implication (\verb+==>+), universal (\verb+!+) and existential
(\verb+?+) quantification, and an indefinite description operator
(\verb+@+). As well, the basis includes conditional, lambda, and `let'
expressions. Thus the set of terms available is, in general, an
extension of the following grammar:
\[
\begin{array}{rclr}
  \mbox{\it term} & ::= & \term : \type & \mbox{(type constraint)} \\
  & | & \term\ \term & \mbox{(application)} \\
  & | & \verb+~+ \term & \mbox{(negation)} \\
  & | & \term\ =\ \term & \mbox{(equality)} \\
  & | & \term\ \IMP\ \term & \mbox{(implication)} \\
  & | & \term\ \verb+\/+\ \term & \mbox{(disjunction)} \\
  & | & \term\ \verb+/\+\ \term & \mbox{(conjunction)} \\
  & | & \texttt{if}\ \term\ \texttt{then}\ \term\ \texttt{else}\ \term &
  \mbox{(conditional)} \\
  & | & \bs\ident_1\ldots\ident_n.\  \term & \mbox{(lambda abstraction)} \\
  & | & \ALL \ident_1\ldots\ident_n.\ \term & \mbox{(forall)} \\
  & | & \EXISTS \ident_1\ldots\ident_n.\ \term & \mbox{(exists)} \\
  & | & \CHOOSE \ident_1\ldots\ident_n.\ \term & \mbox{(choose)} \\
  & | & \EXISTSONE \ident_1\ldots\ident_n.\ \term &\mbox{(exists-unique)} \\
  & | & \LET\; \ident = \term  & \\
  &   & [\und\ \ident = \term]^{*}\ \IN\ \term & \mbox{(let expression)} \\
  & | & \verb+T+ & \mbox{(truth)} \\
  & | & \verb+F+ & \mbox{(falsity)} \\
  & | & \ident & \mbox{(constant or variable)} \\
  & | & \verb+(+ \term \verb+)+ & \mbox{(parenthesized term)}
\end{array}
\]

Some examples may be found in Table \ref{syntaxExamples}. Term
application can be iterated. Application is left associative so that
$\term\ \term\ \term \ldots \term$ is equivalent in the eyes of the
parser to $(\ldots((\term\ \term)\ \term) \ldots)\ \term$.

The lexical structure for term identifiers is much like that for
ML: identifiers can be alphanumeric or symbolic. Variables must be
alphanumeric. A symbolic identifier is any concatenation of the characters
in the following list:
\begin{verbatim}
    #?+*/\\=<>&%@!,:;_|~-
\end{verbatim}
with the exception of the keywords \verb+\\+, \verb+;+, \verb+=>+,
\verb+|+, and \verb+:+ (colon). Any alphanumeric can be a constant except the
keywords \verb+let+, \verb+in+, \verb+and+, and \verb+of+.

 \begin{table}[h]
\begin{center}
 \begin{tabular}{|r|l|} \hline
 \verb+x = T+ & {\it x is equal to true.} \\
 \verb+!x. Person x ==> Mortal x+ & {\it All persons are mortal.} \\
 \verb+!x y z. (x ==> y) /\ (y ==> z) ==> x ==> z+ & {\it Implication is
 transitive.} \\
 \verb+!x. P x ==> Q x+ & {\it P is a subset of Q} \\
 \verb+S = \f g x. f x (g x)+ & {\it Definition of a famous combinator.} \\ \hline
 \end{tabular}
 \caption{Concrete Syntax Examples}\label{syntaxExamples}
\end{center}
 \end{table}


\subsection{Constants}

The HOL grammar gets extended when a new constant is introduced. The
introduction of new constants will be discussed in section
\ref{thy-ops}. In order to provide some notational flexibility,
constants come in various flavours or {\it fixities}: besides being an
ordinary constant (with a fixity of {\sf Prefix}), constants can also
be {\it binders}, {\it true prefixes}\footnote{The use of the term
  ``true prefix'' is forced upon us by the history of the system,
  which reserved the classification ``prefix'' for terms without any
  special syntactic features.}, {\it suffixes}, {\it infixes}, or {\it
  closefixes}.  More generally, terms can also be represented using
reasonably arbitrary {\it mixfix} specifications.  The degree to which
terms bind their associated arguments is known as precedence.  The
higher this number, the tighter the binding.  For example, when
introduced, \verb-+- has a precedence of 500, while the tighter
binding multiplication (\verb+*+) has a precedence of 600.

\subsubsection{Binders}

A binder is a construct that binds a variable; for example, the
universal quantifier. In HOL, this is represented using a trick that
goes back to Alonzo Church: a binder is a constant that takes a lambda
abstraction as its argument. The lambda binding is used to implement
the binding of the construct. This is an elegant and uniform solution.
Thus the concrete syntax \verb+!v. M+ is represented by the
application of the constant \verb+!+ to the abstraction \verb+(\v. M)+.

The most common binders are \verb+!+, \verb+?+, \verb+?!+, and
\verb+@+. Sometimes one wants to iterate applications of the same
binder, \eg,
\begin{alltt}
  !x. !y. ?p. ?q. ?r. \term.
\end{alltt}
This can instead be rendered
\begin{alltt}
  !x y. ?p q r. \term.
\end{alltt}

\subsubsection{Infixes}

Infix constants can associate in one of three different ways: right,
left or not at all.  (If \verb-+- were non-associative, then {\tt 3 +
  4 + 5} would fail to parse; one would have to write {\tt (3 + 4) +
  5} or {\tt 3 + (4 + 5)} depending on the desired meaning).  The
precedence ordering for the initial set of infixes is \verb+/\+,
\verb+\/+, \verb+==>+, \verb+=+,
 \begin{Large}\verb+,+\end{Large} (comma\footnote{When {\tt pairTheory} has
   been loaded.}). Moreover, all of these constants are right
 associative. Thus
\begin{verbatim}
     X /\ Y ==> C \/ D, P = E, Q
\end{verbatim}
 is equal to
\begin{verbatim}
     ((X /\ Y) ==> (C \/ D)), ((P = E), Q).
\end{verbatim}

\noindent An expression \[\term\ \verb+<infix>+\ \term\] is internally
represented as \[((\verb+<infix>+\ \term)\ \term)\].

\subsubsection{True prefixes}

Where infixes appear between their arguments, true prefixes appear
before theirs.  This might initially appear to be the same thing as
happens with normal function application (is $f$ in $f(x)$ not acting
as a prefix?), but in fact, it is useful to allow for prefixes to have
binding power less than that associated with function application.  An
example of this is \verb+~+, logical negation.  This is a prefix with
lower precedence than function application.  Normally
\[
   f\;x\; y\qquad \mbox{is parsed as}\qquad (f\; x)\; y
\] but \[
  \mbox{\tt \~{}}\; x\; y\qquad\mbox{is parsed as}\qquad
  \mbox{\tt \~{}}\; (x\; y)
\] because the precedence of \verb+~+ is lower than that of function
application.  The unary negation symbol would also typically be
defined as a true prefix, if only to allow one to write \[ {\it
  negop}\,{\it negop}\,3
\] (whatever {\it negop} happened to be) without needing extra parentheses.

\subsubsection{Suffixes}

Suffixes appear after their arguments.  There are no suffixes
introduced into the standard theories available in HOL, but users are
always able to introduce their own if they choose.  Suffixes are
associated with a precedence just as infixes and true prefixes are.
If \verb+p+ is a true prefix, \verb+i+ an infix, and \verb+s+ a
suffix, then there are six possible orderings for the three different
operators based on their precedences, giving five parses for
$\verb+p+\; t_1\; \verb+i+\; t_2\; \verb+s+$ depending on the relative
precedences:
\[
\begin{array}{cl}
\mbox{\begin{tabular}{c}Precedences\\(lowest to highest)\end{tabular}} &
\multicolumn{1}{c}{\mbox{Parses}}\\
\hline
p,\;i,\;s & \verb+p+\;(t_1\;\verb+i+\;(t_2\;\verb+s+))\\
p,\;s,\;i & \verb+p+\;((t_1\;\verb+i+\;t_2)\;\verb+s+)\\
i,\;p,\;s & (\verb+p+\;t_1)\;\verb+i+\;(t_2\;\verb+s+)\\
i,\;s,\;p & (\verb+p+\;t_1)\;\verb+i+\;(t_2\;\verb+s+)\\
s,\;p,\;i & (\verb+p+\;(t_1\;\verb+i+\;t_2))\;\verb+s+\\
s,\;i,\;p & ((\verb+p+\;t_1)\;\verb+i+\;t_2)\;\verb+s+\\
\end{array}
\]

\subsection{Type constraints}

A term can be constrained to be of a certain type.  For example,
\verb+X:bool+ constrains the variable \verb+X+ to have type
\verb+bool+. Similarly, \verb+T:bool+ performs a (vacuous) constraint
of the constant \verb+T+ to \verb+bool+. An attempt to constrain a
term inappropriately will raise an exception: for example,
\begin{verbatim}
  if T then (X:ind) else (Y:bool)
\end{verbatim}
will fail because both branches of a conditional must be of the same
type.  Type constraints can be seen as a suffix that binds more
tightly than everything except function application.  Thus $\term\
\ldots\ \term \ : \type$ is equal to $(\term\ \ldots\ \term)\ :
\type$, but $x < y:\mbox{\tt num}$ is a legitimate (though, again
redundant) constraint on just the variable $y$.

The inclusion of \verb+:+ in the symbolic identifiers means that some
constraints may need to be separated by white space. For example,
\begin{verbatim}
    $=:bool->bool->bool
\end{verbatim}
will be broken up by the HOL lexer as
\begin{verbatim}
    $=: bool -> bool -> bool
\end{verbatim}
and parsed as an application of the symbolic identifier \verb+$=:+ to
the argument list of terms [\verb+bool+, \verb+->+, \verb+bool+,
\verb+->+, \verb+bool+]. A well-placed space will avoid this problem:
\begin{verbatim}
    $= :bool->bool->bool
\end{verbatim}
is parsed as the symbolic identifier "=" constrained by a type.

\subsubsection{Closefixes}

Closefix terms are operators that completely enclose their arguments.
An example one might use in the development of a theory of
denotational semantics is semantic brackets.  Thus, the HOL parsing
facilities can be configured to allow one to write \texttt{denotation x}
as \texttt{[| x |]}.  Closefixes are not associated with precedences
because they can not compete for arguments with other operators.

\subsubsection{Type inference}

Consider the term \verb+x = T+.  Each term (and all of its subterms),
has a type in the HOL logic. Now, \verb+T+ has type \verb+bool+. This
means that the constant \verb+=+ has type \verb+xty -> bool -> bool+,
for some type \verb+xty+. Since the type scheme for \verb+=+ is
\verb+'a -> 'a -> bool+, we know that \verb+xty+ must in fact be
\verb+bool+ in order for the type instance to be well-formed. Knowing
this, we can deduce that the type of `x' must be \verb+bool+.

Ignoring the jargon ("scheme" and "instance") in the previous
paragraph, we have conducted a type assignment to the term structure,
ending up with a well-typed term. It would be very tedious for users
to conduct such argumentation by hand for each term entered to \holn{}.
Thus, \holn{} uses an adaptation of Milner's type inference algorithm
for ML when constructing terms via parsing. At the end of type
inference, unconstrained type variables get assigned by the system.
Usually, this assignment does the right thing. However, at times, the
most general type is not what is desired and the user must add type
constraints to the relevant subterms. For tricky situations, the
global variable \verb+show_types+ can be assigned. When this flag is
set, the prettyprinters for terms and theorems will show how types
have been assigned to subterms. If you do not want the system to
assign type variables for you, the global variable
\verb+guessing_tyvars+ can be set to \verb+false+, in which case the
existence of unassigned type variables at the end of type inference
will raise an exception.

\subsection{Expanded term grammar}

There is some further syntax that is specially treated by the parser.
The theory of pairs introduces the infix pairing operator
(\begin{Large}\verb+,+\end{Large}) as well as the corresponding infix
product (\verb+#+) type operator. The theory of sets introduces
notation for the empty set \verb+{}+ (or \verb+EMPTY+), membership
(the infix \verb+IN+) insertion (the infix \verb+INSERT+), set
comprehension, enumerated sets, and many other defined constants. The
theory of lists introduces the constants \verb+NIL+ (the surface
syntax \verb+[]+ can be used) and \verb+CONS+, as well as notation for
enumerated lists. The theories of (Peano) numbers and strings
introduce the constructors \verb+0+, \verb+SUC+, \verb+""+, and
\verb+STRING+, as well as literals for numbers and strings. If the
theory of restricted quantifiers is present, syntax is provided for
constraining bound variables by predicates.

Thus, if the theories of pairs, sets, numbers, strings, lists, and
restricted quantifiers are loaded, the HOL grammar is an extension of
that in Table \ref{expanded-grammar}.
\begin{table}
  \[
  \begin{array}{rclr}
    \term & ::= & \term : \type & \mbox{(type constraint)} \\
    & | & \term\ \term & \mbox{(application)} \\
    & | & \CONS\ \term \ \term & \mbox{(list builder)} \\
    & | & \INSERT\ \term \ \term & \mbox{(set builder)} \\
    & | & \SUC\ \term & \mbox{(successor)} \\
    & | & \verb+~+ \term & \mbox{(negation)} \\
    & | & \term\ =\ \term & \mbox{(equality)} \\
    & | & \term\ \IMP\ \term & \mbox{(implication)} \\
    & | & \term\ \verb+\/+\ \term & \mbox{(disjunction)} \\
    & | & \term\ \verb+/\+\ \term & \mbox{(conjunction)} \\
    & | & \term\ \verb+<+ \ \term & \mbox{(less-than)} \\
    & | & \term\ \verb!+! \ \term & \mbox{(addition)} \\
    & | & \term\ \verb!*! \ \term & \mbox{(multiplication)} \\
    & | & \term\ \verb!-! \ \term & \mbox{(subtraction)} \\
    & | & \term\ \verb+=>+\ \term\ |\ \term & \mbox{(conditional)} \\
    & | & \bs\vstr_1\ldots\vstr_n.\ \term & \mbox{(lambda abstraction)} \\
    & | & \ALL \vstr_1\ldots\vstr_n.\ \term & \mbox{(forall)} \\
    & | & \EXISTS \vstr_1\ldots\vstr_n.\ \term & \mbox{(exists)} \\
    & | & \CHOOSE \vstr_1\ldots\vstr_n.\ \term & \mbox{(choose)} \\
    & | & \EXISTSONE \vstr_1\ldots\vstr_n.\ \term &\mbox{(exists-unique)} \\
    & | & \LET\; \vstr = \term  & \\
    &   & [\und\ \vstr = \term]^{*}\ \IN\ \term & \mbox{(let expression)} \\
    & | & \verb+T+ & \mbox{(truth)} \\
    & | & \verb+F+ & \mbox{(falsity)} \\
    & | & \verb+[]+ & \mbox{(empty list)} \\
    & | & \verb+{}+ & \mbox{(empty set)} \\
    & | & \verb+(+ \term \verb+,+ \term \verb+)+ & \mbox{(pair)} \\
    & | & \ident & \mbox{(constant or variable)} \\
    & | & \numeral & \mbox{(numeric literal)} \\
    & | & \verb+"+\charseq \verb+"+& \mbox{(string literal)} \\
    & | & \verb+(+ \term \verb+)+ & \mbox{(parenthesized term)} \\
    & | & \verb+[+ \term \verb+;+ \ldots \verb+;+ \term \verb+]+ &
    \mbox{(enumerated list)} \\
    & | & \verb+{+ \term \verb+;+ \ldots \verb+;+ \term \verb+}+ &
    \mbox{(enumerated set)} \\
    & | & \verb+{+ \term \ \verb+|+\  \term \verb+}+ & \mbox{(set comprehension)}
  \end{array}
  \]
  \caption{Expanded Term Grammar} \label{expanded-grammar}
\end{table}

In the table, the varstruct ({\it vstr\/}) construct is used. A
varstruct is (apparently) an arbitrarily nested tuple of variables,
where each variable only occurs once. The translation of varstructs
into the internal abstract syntax trees is complex, so we avoid the
explanation (for this draft).
\[
\begin{array}{rcl}
  \vstr & ::= & \ident : \type \\
  & | & \ident  \\
  & | & \vstr \verb+,+ \vstr \\
  & | & \verb+(+ \vstr \verb+)+\\
  & | & \verb+(+\vstr {\tt ::} \term\verb+)+\\
\end{array}
\] The {\tt ::} syntax is used with restricted quantifiers to allow
arbitrary predicates to restrict binding variables.  Further to the
above, the default grammar also allows restricted quantification of
all of a sequence of binding variables by putting the restriction at
the end of the sequence, thus with a universal quantification: \[
\forall x \, y \, z \, {\tt ::} \; P \, . \; Q(x,y,z)
\] Here the predicate $P$ restricts all of $x$, $y$ and $z$.

Also, in the term grammar a \charseq\ is just a finite sequence of
characters.


\section{Changes from older versions}

This section of the manual documents the (extensive) changes made to
the parsing of HOL terms and types in the Taupo release and beyond
from the point of view of a user who doesn't want to know how to use
the new facilities, but wants to make sure that their old code
continues to work cleanly.

The changes which may cause old terms to fail to parse are:
\begin{itemize}
\newcommand\condexp{\mbox{$p \; {\tt =>} \; q \; {\tt |} \; r$}}
\item The precedence of type annotations has completely changed.  It
  is now a very tight suffix (though with a precedence weaker than
  that associated with function application), instead of a weak one.
  This means that \mbox{\tt (x,y:bool \# bool)} should now be written
  as \mbox{\tt (x,y):bool \# bool}. The previous form will now be
  parsed as a type annotation applying to just the \verb+y+.  This
  change brings the syntax of the logic closer to that of SML and
  should make it generally easier to annotate tuples, as one can now
  write \[ (x\,:\,\tau_1,\;y\,:\,\tau_2,\dots z\,:\,\tau_n)
  \] instead of \[
  (x\,:\,\tau_1, \;(y\,:\,\tau_2, \dots (z\,:\,\tau_n)))
  \] where extra parentheses have had to be added just to allow one to
  write a frequently occurring form of constraint.
\item Most arithmetic operators are now left associative instead of
  right associative.  In particular, $+$, $-$, $*$ and {\tt DIV} are
  all left associative.  Similarly, the analogous operators in other
  numeric theories such as {\tt integer} and {\tt real} are also left
  associative.  This brings the HOL parser in line with standard
  mathematical practice.
\item The binding equality in {\tt let} expressions is treated exactly
  the same way as equalities in other contexts.  In previous versions
  of HOL, equalities in this context have a different, weak binding
  precedence.  This difference can be seen in the following expression
  which parses successfully in the old version:
  \[ {\tt let} \; x \; = \; \condexp \; {\tt
  in} \;Q \] In Taupo releases and later, this expression will not
  parse because the conditional expression binds to the left more
  weakly than the equality binds to the right, and the parser ends up
  believing that the binding between the \verb+let+ and the \verb+in+
  is not an equality after all, as it should be.
\item Old style conditional expressions in the right half of set
  comprehensions have to be parenthesised to avoid confusing the
  parser.  Thus \[
  \{ \; x \; | \; \condexp \; \}
   \qquad\mbox{must be written} \qquad
  \{ \; x \; | \; (\condexp) \; \}
  \] Better yet, {\tt if}-{\tt then}-{\tt else} syntax could be used
  for the conditional expression.
\item Some lexical categories are more strictly policed.  String
  literals (strings inside double quotes) and numerals can't be used
  unless the relevant theories have been loaded.  Nor can these
  literals be used as variables inside binding scopes.
\end{itemize}

\subsection{Error messages}

When complete this subsection will document all of the possibly
confusing error messages that the new parser and lexing code might
generate.

\subsection{Parser tricks and magic}

Here we describe how to achieve some useful effects with the new
parser in \holn{} Taupo-1.

\begin{description}
\item [Mix-fix syntax for {\it if-then-else}:] The first step in
  bringing this about is to look at the general shape of expressions
  of this form.  In this case, it will be:
  \[
  \mbox{\tt if}\;\; \dots \;\;\mbox{\tt then}\;\;\dots\;\;
  \mbox{\tt else}\;\;\dots
  \]
  Because there needs to be a ``dangling'' term to the right, the
  appropriate fixity is \texttt{TruePrefix}.  Knowing that the
  underlying term constant is called \texttt{COND}, the simplest way
  to achieve the desired syntax is:
  \begin{verbatim}
val _ = add_rule{term_name = "COND", fixity = TruePrefix 70,
                 pp_elements = [TOK "if", BreakSpace(1,0), TM,
                                BreakSpace(1,0),
                                TOK "then", BreakSpace(1,0), TM,
                                BreakSpace(1,0),
                                TOK "else", BreakSpace(1,0)],
                 paren_style = Always,
                 block_style =
                   (AroundEachPhrase, (PP.CONSISTENT, 0))};
\end{verbatim}
  \noindent The actual rule is slightly more complicated, and is
  in \texttt{src/bool/boolScript.sml}.


\item[Mix-fix syntax for term substitution:] Here we want to be able
  to write something like:
  \[
  \mbox{\texttt{[}}\,t_1\,\mbox{\texttt{/}}\,t_2\,\mbox{\texttt{]}}\,t_3
  \]
  denoting the substitution of $t_1$ for $t_2$ in $t_3$, perhaps
  translating to \mbox{\texttt{SUB} $t_1$ $t_2$ $t_3$}.  This looks
like it should be another \texttt{TruePrefix}, but the choice of the
square brackets (\texttt{[} and \texttt{]}) as delimiters would
conflict with the concrete syntax for list literals if we did this.
Given that list literals are effectively of the \texttt{CloseFix}
class, we need to make our new syntax the same.  This is easy enough
to do: we set up syntax \[
\mbox{\texttt{[}}\,t_1\,\mbox{\texttt{/}}\,t_2\,\mbox{\texttt{]}}
  \] to map to \mbox{\texttt{SUB} $t_1$ $t_2$} a value of a functional
  type, that when applied to a third argument will look
  right.\footnote{Note that doing the same thing for the
    \textit{if-then-else} example in the previous example would be
    inappropriate, as it would allow one to write \[
    \texttt{if}\,P\,\texttt{then}\,Q\,\texttt{else}
    \] without the trailing argument}
  The rule for this is thus:
  \begin{verbatim}
val _ = add_rule {term_name = "SUB", fixity = Closefix,
                  pp_elements = [TOK "[", TM, TOK "/", TM, TOK "]"],
                  paren_style = OnlyIfNecessary,
                  block_style =
                    (AroundEachPhrase, (PP.INCONSISTENT, 2))};
\end{verbatim}

\item[Aliasing] If one wants a special syntax to be an ``alias'' for a
  normal HOL form, this is easy to achieve; both examples so far have
  effectively done this.  However, if one just wants to have a normal
  one-for-one substitution of one string for another, one can't use
  the grammar/syntax phase of parsing to do this.  Instead, one can
  use the overloading mechanism.  For example, let us alias
  \texttt{MEM} for \texttt{IS\_EL}.  First we should allow for
  overloading on the new name at the exact type of the old.  Thus:
  \begin{verbatim}
val _ = allow_for_overloading_on ("MEM",
                                  Type`:'a -> 'a list -> bool`);
\end{verbatim}
  \noindent The next step is to overload the original constant for the
  new name:
  \begin{verbatim}
val _ = overload_on ("MEM", Term`IS_EL`);
\end{verbatim}

\item[Making addition right associative] If one has a number of old
  scripts that assume addition is right associative because this is
  how HOL used to be, it might be too much pain to convert.  The trick
  is to remove all of the rules at the given level of the grammar, and
  put them back as right associative infixes.  The easiest way to tell
  what rules are in the grammar is by inspection (use
  \texttt{term\_grammar()}).  With just \texttt{arithmeticTheory}
  loaded, the only infixes at level 500 are \texttt{+} and
  \texttt{-}.  So, we remove the rules for them:
  \begin{verbatim}
val _ = app temp_remove_rules_for_term ["+", "-"];
\end{verbatim}
  \noindent And then we put them back with the appropriate
  associativity:
\begin{verbatim}
val _ = app (fn s => temp_add_infix(s, 500, RIGHT)) ["+", "-"];
\end{verbatim}
\noindent Note that we use the \texttt{temp\_} versions of these two
functions so that other theories depending on this one won't be
affected.  Further note that we can't have two infixes at the same
level of precedence with different associativities, so we have to
remove both operators, not just addition.

\end{description}

}

%%% Local Variables:
%%% mode: latex
%%% TeX-master: "description"
%%% End:
