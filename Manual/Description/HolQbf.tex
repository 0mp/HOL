\index{HolQbfLib|(}
\index{QBF solvers|see {HolQbfLib}}
\index{decision procedures!QBF}

\setcounter{sessioncount}{0}

\ml{HolQbfLib} provides a (rudimentary) platform for experimenting
with combinations of theorem proving and Quantified Boolean
Formulae~(QBF) solvers.  \ml{HolQbfLib} was developed as part of a
research project on {\it Expressive Multi-theory Reasoning for
  Interactive Verification} (EPSRC grant EP/F067909/1) from 2008
to~2011.  It is loosely inspired by \ml{HolSatLib}
(Section~\ref{sec:HolSatLib}), and has been described in parts in the
following publication:
\begin{itemize}
\item Tjark Weber: {\it Validating QBF Invalidity in HOL4}.  Published
  at the International Conference on Interactive Theorem Proving (ITP)
  2010.
\end{itemize}
\ml{HolQbfLib} uses an external QBF solver, Squolem, to disprove
invalid Quantified Boolean Formulae.

\subsection{Installing Squolem}

\ml{HolQbfLib} has been tested with Squolem 1.03 (release date
2007-01-28).  Squolem can be obtained for Windows and Linux from
\url{http://www.cprover.org/qbv/download.html}.  After installation,
you must make the executable available as {\tt squolem}, \eg, by
placing it into a folder that is in your {\tt \$PATH}.  This name is
currently hard-coded: there is no configuration option to tell \HOL{}
about the location and name of the Squolem executable.

\subsection{Interface}
\label{qbf-interface}

The library provides a function \ml{disprove} (of type \ml{term ->
  thm}) to invoke Squolem.  This function is defined in the
\ml{HolQbfLib} structure, which is the library's main entry point.

A call \ml{disprove $\phi$} will invoke Squolem on the QBF~$\phi$ to
establish that $\phi$ is invalid.  If this succeeds, \ml{disprove}
will then validate the certificate of invalidity generated by Squolem
in \HOL{} to return a theorem $\phi \vdash \bot$.

\paragraph{Supported subset of higher-order logic}

The argument given to \ml{disprove} must be a QBF in prenex form, \ie,
a term of the form $Q_1 x_1. \, Q_2 x_2. \, \ldots \, Q_n x_n. \,
\phi$, where
\begin{itemize}
\item $n \geq 0$,
\item each $Q_i$ is an (existential or universal) quantifier,
\item $Q_n$ is the existential quantifier,
\item each $x_i$ is a Boolean variable,
\item $\phi$ is a propositional formula in CNF, \ie, a conjunction of
  disjunctions of (possibly negated) Boolean variables,
\item $\phi$ must actually contain each $x_i$,
\item all $x_i$ must be distinct, and
\item $\phi$ does not contain variables other than $x_1$, \dots,
  $x_n$.
\end{itemize}
The behavior is undefined if any of these restrictions are violated.

\begin{session}
\begin{verbatim}
- load "HolQbfLib";
> val it = () : unit

- open HolQbfLib;
> val disprove = fn : term -> thm

- show_assums := true;
> val it = () : unit

- disprove ``?x. x /\ ~x``;
<<HOL message: HolQbfLib: calling external command
  'squolem --save-certificate /tmp/file0Pw2Tg >& /dev/null'>>
> val it =  [?x. x /\ ~x] |- F : thm

- disprove ``!x. ?y. x /\ y``;
<<HOL message: HolQbfLib: calling external command
  'squolem --save-certificate /tmp/fileZAGj4m >& /dev/null'>>
> val it =
     [!x. ?y. x /\ y] |- F : thm
\end{verbatim}
\end{session}

\paragraph{Support for the QDIMACS file format}

The QDIMACS standard defines an input file format for QBF solvers.
\ml{HolQbfLib} provides a structure \ml{QDimacs} that implements
(parts of) the QDIMACS standard, version 1.1 (released on December~21,
2005), as described at \url{http://www.qbflib.org/qdimacs.html}.  The
\ml{QDimacs} structure does not require Squolem (or any other QBF
solver) to be installed.

\ml{QDimacs.write\_qdimacs\_file path $\phi$} creates a QDIMACS file
(with name \ml{path}) that encodes the QBF $\phi$.  $\phi$ must meet
the requirements detailed above.

\ml{QDimacs.read\_qdimacs\_file path} parses an existing QDIMACS file
(with name \ml{path}) and returns the encoded QBF as a \HOL{} term.
Since variables are only given as integers in the QDIMACS format,
variable names in \HOL{} are obtained by prefixing the value of
\ml{QDimacs.var\_prefix} (which is of type \ml{string ref}) to each
integer.  This reference contains \ml{"v"} by default, but can be
changed if desired.

\paragraph{Tracing}

Tracing output can be controlled via \ml{Feedback.set\_trace
  "HolQbfLib"}.  See the source code in \ml{QbfTrace.sml} for possible
values.

Communication between \HOL{} and Squolem is via temporary files.
These files are located in the standard temporary directory, typically
{\tt /tmp} on Unix machines.  The actual file names are generated at
run-time, and can be shown by setting the above tracing variable to a
sufficiently high value.

The default behavior of \ml{HolQbfLib} is to delete temporary files
after successful invocation of Squolem.  This also can be changed via
the above tracing variable.  If there is an error, files are retained
in any case (but note that the operating system may delete temporary
files automatically, \eg, when \HOL{} exits).

\subsection{Wishlist}

The following features have not been implemented yet.  Please submit
additional feature requests (or code contributions) via
\url{http://hol.sf.net}.

\paragraph{Transformation of QBF into prenex form}

\ml{HolQbfLib} at present only supports QBF in prenex form (see the
description of the supported subset of higher-order logic given in
Section~\ref{qbf-interface}).  A transformation (implemented in \HOL)
that converts arbitrary QBF into prenex form would greatly enhance
\ml{HolQbfLib}'s applicability.

\paragraph{Proof reconstruction for valid QBF}

Proof reconstruction is currently limited to invalid QBF (in prenex
form), for which a theorem $\phi \vdash \bot$ is derived.  It would be
nice if the library could derive $\vdash \phi$ when $\phi$ is a valid
QBF.

In principle, this could be done by showing that $\neg \phi$ is
invalid.  However, this (i)~would require the ability to transform
$\neg \phi$ into prenex form (see above), and (ii)~doesn't scale well
in practice.  Instead, Squolem (like various other QBF solvers) uses a
different certificate format for valid QBF, based on Skolem functions.
One should implement checking of this certificate format in \HOL.

\paragraph{Support for other QBF solvers}

So far, Squolem is the only QBF solver that has been integrated with
\HOL.  Several other QBF solvers can produce proofs, and it would be
nice to offer \HOL{} users more choice (also because Squolem's
performance is not necessarily state-of-the-art anymore).

\paragraph{QBF solvers as a web service}

The need to install a QBF solver locally poses an entry barrier.  It
would be much more convenient to have a web server running one (or
several) QBF solvers, roughly similar to the ``System on TPTP''
interface that G.~Sutcliffe provides for first-order theorem provers
(\url{http://www.cs.miami.edu/~tptp/cgi-bin/SystemOnTPTP}).

\index{HolQbfLib|)}

%%% Local Variables:
%%% mode: latex
%%% TeX-master: "description"
%%% End:
