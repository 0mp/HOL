\chapter{Derived Inference Rules}
\label{derived-rules}

The notion of {\it proof\/} is defined abstractly in the manual
\LOGIC: a proof of a sequent $(\Gamma,t)$ from a set of sequents
$\Delta$ (with respect to a deductive system ${\cal D}$) was defined
to be a chain of sequents culminating in $(\Gamma,t)$, such that every
element of the chain either belongs to $\Delta$ or else follows from
$\Delta$ and earlier elements of the chain by deduction.  The notion
of a {\it theorem\/} was also defined in \LOGIC: a theorem of a
deductive system is a sequent that follows from the empty set of
sequents by deduction; \ie, it is the last element of a proof from the
empty set of sequents, in the deductive system.  In this section,
proofs and theorems are made concrete in \HOL.

The deductive system of \HOL\
was sketched in Section~\ref{rules}, where
the eight families of primitive inferences making up the
deductive system were specified by diagrams. It was explained that
these families of inferences are represented in \HOL\ via
\ML\ functions, and that theorems
are represented by an \ML\ abstract type called \ml{thm}.\index{thm@\ml{thm}}
The eight \ML\ functions corresponding to the inferences
are operations of the type \ml{thm}, and each of the eight
returns a value of type \ml{thm}. It was explained that the
type \ml{thm} has primitive destructors, but no primitive
constructor; and that in that way, the logic is protected against
the computation of theorems except by functions representing
primitive inferences, or compositions of these.

Finally, the primitive \HOL\ logic was supplemented by three primitive
constants and four axioms, to form the basic logic.  The primitive
inferences, together with the primitive constants, the five axioms,
and a collection of definitions, give a starting point for
constructing proofs, and hence computing theorems. However, proving
even the simplest theorems from this minimal basis costs considerable
effort. The basis does not immediately provide the transitivity of
equality, for example, or a means of universal quantification; both of
these themselves have to be derived.

\section{Simple derivations}

As an illustration of a proof in \HOL{}, the following chain of
theorems forms a proof (from the empty set, in the \HOL{} deductive
system), for the particular terms \ml{``}$t_1$\ml{``}%''
and \ml{``}$t_2$\ml{``},%''
both of \HOL\ type \ml{``:bool``}:%''

\begin{enumerate}
\item $t_1$\ml{ ==> }$t_2$\ml{ |- }$t_1$\ml{ ==> }$t_2$

\item $t_1$\ml{ |- }$t_1$

\item $t_1$\ml{ ==> }$t_2$\ml{, }$t_1$\ml{ |- }$t_2$
\end{enumerate}

\noindent That is, the third theorem follows from the first and second.

In the session below, the proof is performed in the \HOL\ system,
using the \ML\ functions \ml{ASSUME}\index{ASSUME@\ml{ASSUME}} and
\ml{MP}.

\setcounter{sessioncount}{1}
\begin{session}
\begin{verbatim}
- show_assums := true;
> val it = () : unit

- val th1 = ASSUME ``t1 ==> t2``;
> val th1 = [t1 ==> t2] |- t1 ==> t2 : thm

- val th2 = ASSUME ``t1:bool``
> val th2 = [t1] |- t1 : thm

- MP th1 th2;
> val it = [t1 ==> t2, t1] |- t2 : thm
\end{verbatim}
\end{session}

\noindent More briefly, one could evaluate the following, and `count'\index{counting inferences, in HOL proofs@counting inferences, in \HOL\ proofs} the
invocations of functions representing primitive inferences.

\begin{session}
\begin{verbatim}
#set_flag(`timing`, true);;
false : bool
Run time: 0.0s

#MP(ASSUME "t1 ==> t2")(ASSUME "t1:bool");;
t1 ==> t2, t1 |- t2
Run time: 0.0s
Intermediate theorems generated: 3
\end{verbatim}
\end{session}

\noindent Each of the three inference steps of the abstract proof
corresponds to the application%
%
\index{inferences, in HOL logic@inferences, in \HOL{} logic!as ML function applications@as \ML\ function applications}%
\index{proof steps, as ML function applications@proof steps, as \ML\ function applications}%
\index{proof!the notion of, in HOL system@the notion
  of, in \HOL\ system}% 
%
of an \ML\ function in the performance of the proof in \HOL; and each
of the \ML\ functions corresponds to a primitive inference of the
deductive system.

It is worth emphasising that, in either case, every
primitive inference in the proof chain is made, in the sense
that for each inference, the corresponding \ML\ function is evaluated.  That is,
\HOL\ permits no short-cut around the necessity of performing
complete proofs.  The short-cut provided by derived
\index{inferences, in HOL logic@inferences, in \HOL{} logic!in derived rules}
inference rules (as implemented in \ML) is around the necessity of
\emph{specifying} every step; something that would be impossible for
a proof of any length. It can be seen from this that the derived
rule,%
%
\index{proofs, in HOL logic@proofs, in \HOL{} logic!as ML function applications@as \ML\ function applications}
\index{proofs, in HOL logic@proofs, in \HOL{} logic!as generated by derived rules}
\index{derived rules, in HOL logic@derived rules, in \HOL{} logic!importance of}
%
and its representation as an \ML{} function, is essential to the
\HOL{} methodology; theorem proving would be otherwise impossible.

There are, of course, an infinite number of proofs, of the `form'
shown in the example, that can be conducted in \HOL: one for every
pair of \ml{``:bool``}-typed terms. %''
Moreover, every time a theorem of the form

$$t_1 \ \imp \ t_2, \ t_1 \ \vdash \ t_2$$

\noindent is required, its proof must be constructed anew. To capture the
general pattern of inference, an \ML\ function can be written to
implement an inference rule as a derivation from the primitive inferences.
Abstractly, a \emph{derived inference rule}
\index{derived rules, in HOL logic@derived rules, in \HOL{} logic!justification of}
%
is a rule that can be justified on the basis of the primitive
inference rules (and/or the axioms).  In the present case, the rule
required `undischarges' assumptions.  It is specified for \HOL{} by

\bigskip

\begin{center}
\begin{tabular}{c}
$\Gamma${\small\verb% |- %}$t_1${\small\verb% ==> %}$t_2$\\ \hline
$\Gamma\cup\{t_1\}${\small\verb% |- %}$t_2$
\end{tabular}
\end{center}

\bigskip

\noindent This general rule is valid because from a \HOL\ theorem of the
form $\Gamma${\small\verb% |- %}$t_1${\small\verb%==>%}$t_2$, the theorem
$\Gamma\cup\{t_1\}${\small\verb% |- %}$t_2$ can be derived
as for the specific instance above.
The rule can be implemented in \ML\ as a function (\ml{UNDISCH},
say)\index{UNDISCH@\ml{UNDISCH}} that calls the
appropriate sequence of primitive
inferences. The \ML\ definition of \ml{UNDISCH} is simply

\begin{session}
\begin{verbatim}
- val UNDISCH th = MP th (ASSUME(fst(dest_imp(concl th))));;
> val UNDISCH = fn : thm -> thm
\end{verbatim}
\end{session}

\noindent This provides a function that maps a theorem to a theorem;
that is, performs proofs in \HOL.
The following session illustrates the use of the derived rule, on
a consequence of the axiom \ml{IMP\_ANTISYM\_AX}. (The inferences are
counted.%
\index{counting inferences, in HOL proofs@counting inferences, in \HOL\ proofs}%
%
) Assume that the printing of theorems has been adjusted as above and
\ml{th} is bound as shown below:

\setcounter{sessioncount}{1}
\begin{session}
\begin{verbatim}
#th;;
|- (t1 ==> t2) ==> (t2 ==> t1) ==> (t1 = t2)
Run time: 0.0s

#set_flag(`timing`,true);;
true : bool
Run time: 0.0s

#UNDISCH th;;
t1 ==> t2 |- (t2 ==> t1) ==> (t1 = t2)
Run time: 0.1s
Intermediate theorems generated: 2

#UNDISCH it;;
t1 ==> t2, t2 ==> t1 |- t1 = t2
Run time: 0.0s
Intermediate theorems generated: 2
\end{verbatim}
\end{session}

\noindent Each successful application of {\small\verb%UNDISCH%}
to a theorem invokes an application
of {\small\verb%ASSUME%}, followed by an application of {\small\verb%MP%};
\ml{UNDISCH} constructs the
2-step proof for any given theorem (of appropriate form).
As can be
seen, it relies on the class of \ML\ functions that access \HOL\ syntax:
in particular, \ml{concl} to produce the conclusion
of the theorem, \ml{dest\_imp} to separate the implication, and the
selector \ml{fst} to choose the antecedent.

This particular example is very simple, but a derived inference rule
can perform proofs of arbitrary length.  It can also make use of
previously defined rules.  In this way, the normal inference patterns
can be developed much more quickly and easily; transitivity,
generalization, and so on, support the familiar patterns of inference.

A number of derived inference rules are pre-defined when the \HOL\
system is entered (of which \ml{UNDISCH} is one of the first).  In
Section~\ref{avra_standard}, the abstract derivations are given for
the pre-defined rules that reflect the more usual inference patterns
of the predicate (and lambda) calculi.  Like those shown, some of the
pre-defined derived rules in \HOL\ generate relatively short proofs.
Others invoke thousands of primitive inferences, and clearly save a
great deal of effort. Furthermore, rules can be defined by the user to
make still larger steps, or to implement more specialized patterns.

All of the pre-defined derived rules in \HOL\ are described
in \REFERENCE.

\section{Rewriting}
\label{avra_rewrite}

\index{rewriting!rules for|(}
\index{REWRITE_RULE@\ml{REWRITE\_RULE}|(}
Included in the set of derived inferences that are pre-defined in
\HOL\ is a group of rules with complex definitions that do a limited
amount of `automatic' theorem-proving in the form of rewriting.  The
ideas and implementation were originally developed by
Milner\index{Milner, R.}  and Wadsworth\index{Wadsworth, C.} for
Edinburgh \LCF,\index{LCF@\LCF!Edinburgh} and were later implemented
more flexibly and efficiently by Paulson\index{Paulson, L.} and
Huet\index{Huet, G.} for Cambridge \LCF.\index{LCF@\LCF!Cambridge}
They appear in \HOL{} in the Cambridge form. The basic rewriting rule
is \ml{REWRITE\_RULE}.  All of the rewriting rules are described in
detail in \REFERENCE.

\ml{REWRITE\_RULE} uses a list of equational theorems
\index{equational theorems, in HOL logic@equational theorems, in \HOL\ logic!use of in rewriting}
\index{theorems, in HOL logic@theorems, in \HOL{} logic!equational}
(theorems whose conclusions can be regarded as having the form
$t_1${\small\verb% = %}$t_2$) to replace
any subterms of an object theorem that `match' $t_1$ by the
corresponding instance of $t_2$. The rule matches recursively and to any depth,
until no more replacements can be made,
using internally defined search, matching and
instantiation\index{type instantiation, in HOL logic@type instantiation, in \HOL{} logic!in rewriting rule} algorithms.  The validity
 of \ml{REWRITE\_RULE} rests
ultimately on the primitive rules \ml{SUBST} (for making the substitutions);
\ml{INST\_TYPE}\index{INST_TYPE@\ml{INST\_TYPE}} (for instantiating types); and the derived rules for
generalization and specialization (see Sections~\ref{avra_gen}
and \ref{avra_spec}) for instantiating terms.  The definition
of \ml{REWRITE\_RULE} in \ML\
also relies on a large number of general and \HOL-oriented
\ML\ functions. 
%The implementation is partly described in Chapter~\ref{avra_conv}.

In practice, the derived rule \ml{REWRITE\_RULE} plays a central role
in proofs, because it takes over a very large number of inferences
which may happen in a complex and unpredictable order. It is unlike
any other primitive or pre-defined rule, first because of the number
of inferences it generates\footnote{The number of inferences performed
  by this rule is generally `inflated'; \ie\ is generally greater than
  the length of the proof itself, if the proof could be `seen'.  This
  is because, in the current implementation, some inference is done
  during the search phase that is not necessarily in support of
  successful replacements.}; and second because its outcome is often
unexpected. Its power is increased by the fact that any existing
equational theorem can be supplied as a `rewrite rule', including a
standard \HOL\ set of pre-proved tautologies; and these rewrite rules
can interact with each other in the rewriting process to transform the
original theorem.

The application of \ml{REWRITE\_RULE}, in the session below,
illustrates that replacements are made at all levels of the
structure of a term.
The example is numerical;
the infixes {\small\verb%"$>"%} and {\small\verb%"$<"%} are
the usual `greater than'  and `less than' relations, respectively,
and \ml{"SUC"}, the
usual successor function.
Use is made of the pre-existing definition of {\small\verb%"$>"%}:
\ml{GREATER} (see \REFERENCE).
The timing\index{counting inferences, in HOL proofs@counting inferences, in \HOL\ proofs} facility is used again, for interest, and the printing
of theorems is adjusted as above.


\setcounter{sessioncount}{1}
\begin{session}
\begin{verbatim}
#top_print print_all_thm;;
- : (thm -> void)

#set_flag(`timing`,true);;
false : bool
Run time: 0.0s

#REWRITE_RULE
 [GREATER]
 (ASSUME "SUC 4 > 0 = (SUC 3 > 0 = (SUC 2 > 0 = (SUC 1 > 0 = SUC 0 > 0)))");;
##Definition GREATER autoloaded from theory `arithmetic`.
GREATER = |- !m n. m > n = n < m
Run time: 1.5s
Intermediate theorems generated: 1

(SUC 4) > 0 =
((SUC 3) > 0 = ((SUC 2) > 0 = ((SUC 1) > 0 = (SUC 0) > 0)))
|- 0 < (SUC 4) =
   (0 < (SUC 3) = (0 < (SUC 2) = (0 < (SUC 1) = 0 < (SUC 0))))
Run time: 0.3s
Intermediate theorems generated: 23
\end{verbatim}
\end{session}

\noindent Notice that rewriting
equations can be extracted from
universally quantified theorems.
To construct the
proof step-wise, with all of the instantiations,
substitutions, and uses of transitivity, \etc,
would be a lengthy process. The rewriting rules make it easy,
and do so whilst still generating the entire chain of inferences.
\index{REWRITE_RULE@\ml{REWRITE\_RULE}|)}
\index{rewriting!rules for|)}

\section{Derivation of the standard rules}
\label{avra_standard}

\index{derived rules, in HOL logic@derived rules, in \HOL{} logic!pre-defined|(}
%
The \HOL{} system provides all the standard introduction and
elimination rules of the predicate calculus pre-defined as derived
inferences.  It is these derived rules, rather than the primitive
rules, that one normally uses in practice.  In this section, the
derivations of some of the standard rules are given, in sequence.
These derivations only use the axioms and definitions in the theory
\theoryimp{bool} (see Section~\ref{boolfull}), the eight primitive
inferences of the \HOL\ logic, and inferences defined earlier in the
sequence.

Theorems,%
%
\index{theorems, in HOL logic@theorems, in \HOL{} logic!as inference rules}%
%
in accordance with the definition given at the beginning of this
chapter, are treated as rules without hypotheses; thus the derivation
of a theorem resembles the derivation of a rule except in not having
hypotheses. (The derivation of \ml{TRUTH}, Section~\ref{avra_T}, is
the only example given of this, but there are several others in \HOL.)
There are also some rules that are intrinsically more general than
theorems.  For example, for any two terms $t_1$ and $t_2$, the theorem
$\vdash(\lquant{x}t_1)t_2 = t_1[t_2/x]$ follows by the primitive rule
\rul{BETA\_CONV}. The rule \ml{BETA\_CONV} returns a theorem for each
pair of terms $t_1$ and $t_2$, and is therefore equivalent to an
infinite family%
%
\index{families of inferences, in HOL logic@families of inferences, in \HOL\ logic}%
%
of theorems. No single theorem can be expressed in the \HOL{} logic
that is equivalent to \rul{BETA\_CONV}.%
\index{theorems, in HOL logic@theorems, in \HOL{} logic!rules inexpressible as}
\index{beta-conversion, in HOL logic@beta-conversion, in \HOL\ logic!not expressible as a theorem}%
%
%(See Chapter~\ref{avra_conv} for further discussion of this point.)
(\ml{UNDISCH} is not a rule of this sort, as it can, in fact, be
expressed as a theorem.)

For each derivation given below, there is an \ML\ function definition
in the \HOL\ system that implements the derived rule as a procedure in
\ML. The actual implementation in the \HOL\ system differs in some
cases from the derivations given here, since the system code has been
optimised for improved performance.

In addition, for reasons that are mostly historical, not all the
inferences that are derived in terms of the abstract logic are
actually derived in the current version of the \HOL\ system.  That is,
there are currently about forty rules that are installed in the system
on an `axiomatic' basis, all of which should be derived by explicit
inference.  Although the current status of these rules is not
satisfactory, and it is planned, as a high priority, to derive them
properly in a future version, their current status does not actually
compromise the consistency of the logic.  In effect, the existing
\HOL\ system has a deductive system more comprehensive than the one
presented abstractly, but the model outlined in \LOGIC{} would easily
extend to cover it.%
%
\index{derived rules, in HOL logic@derived rules, in \HOL{} logic!pre-defined|)}

For reference, in \HOL\ Version 2.0 the following rules that should be
derived
%
\index{inference rules, of HOL logic@inference rules, of \HOL{} logic!some not properly derived|(}
\index{rules in HOL logic, some not properly derived@rules in \HOL{} logic, some not properly derived|(}
%
are not derived, but (for efficiency) are implemented as
primitives. The list includes some conversions and conversion-valued
functions. % (conversions are discussed in Chapter~\ref{avra_conv}).
\vfill \newpage

\begin{hol}
\index{derived rules, in HOL logic@derived rules, in \HOL{} logic!list of axiomatic}
\index{AP_TERM@\ml{AP\_TERM}}
\index{AP_THM@\ml{AP\_THM}}
\index{CCONTR@\ml{CCONTR}}
\index{CHOOSE@\ml{CHOOSE}}
\index{CONJ@\ml{CONJ}}
\index{CONJUNCT1@\ml{CONJUNCT1}}
\index{CONJUNCT2@\ml{CONJUNCT2}}
\index{DISJ_CASES@\ml{DISJ\_CASES}}
\index{DISJ1@\ml{DISJ1}}
\index{DISJ2@\ml{DISJ2}}
\index{EQ_IMP_RULE@\ml{EQ\_IMP\_RULE}}
\index{EQ_MP@\ml{EQ\_MP}}
\index{EQT_INTRO@\ml{EQT\_INTRO}}
\index{ETA_CONV@\ml{ETA\_CONV}}
\index{EXISTS@\ml{EXISTS}}
\index{EXT@\ml{EXT}}
\index{GEN@\ml{GEN}}
\index{MK_ABS@\ml{MK\_ABS}}
\index{MK_COMB@\ml{MK\_COMB}}
\index{num_CONV@\ml{num\_CONV}}
\index{SPEC@\ml{SPEC}}
\index{SUBS@\ml{SUBS}}
\index{SUBS_OCCS@\ml{SUBS\_OCCS}}
\index{SUBST_CONV@\ml{SUBST\_CONV}}
\index{SYM@\ml{SYM}}
\index{TRANS@\ml{TRANS}}
\begin{verbatim}
   ADD_ASSUM              CONTR                  IMP_ANTISYM_RULE
   ALPHA                  DEF_EXISTS_RULE        IMP_TRANS
   AP_TERM                DISJ_CASES             INST
   AP_THM                 DISJ1                  MK_ABS
   SUBS                   DISJ2                  MK_COMB
   SUBS_OCCS              EQ_IMP_RULE            MK_EXISTS
   CCONTR                 EQ_MP                  NOT_ELIM
   CHOOSE                 EQT_INTRO              NOT_INTRO
   CONJ                   ETA_CONV               num_CONV
   EXISTS                 SPEC                   TRANS
   EXT                    SUBST_CONV             CONJUNCT1
   GEN                    SYM                    CONJUNCT2
\end{verbatim}\end{hol}
\index{inference rules, of HOL logic@inference rules, of \HOL{} logic!some not properly derived|)}
\index{rules in HOL logic, some not properly derived@rules in \HOL{} logic, some not properly derived|)}

\index{inference rules, of HOL logic@inference rules, of \HOL{} logic!derived|(}
The derivations that follow consist of sequences of numbered steps each of
which
\begin{enumerate}
\item is an axiom, or
\item is a hypothesis of the rule being derived, or
\item follows from preceding steps by a rule of inference (either primitive
or previously derived).
\end{enumerate}

\noindent Note that the abbreviation \ml{conv} (standing for
`conversion') is used for the \ML\ type \ml{term ->
  thm}.%
% \footnote{This stands for `conversion', as explained in
% Chapter~\ref{avra_conv}.}

\subsection{Adding an assumption}
\index{derived rules, in HOL logic@derived rules, in \HOL{} logic!list and derivations of some|(}

\begin{holboxed}
\index{ADD_ASSUM@\ml{ADD\_ASSUM}|pin}
\begin{verbatim}
   ADD_ASSUM : term -> thm -> thm
\end{verbatim}\end{holboxed}


\vspace{12pt plus2pt minus1pt}

$$\Gamma\turn t\over \Gamma,\ t'\turn t$$

\vspace{12pt plus2pt minus1pt}

\begin{proof}
\item $t'\turn t'$ \hfill [\rul{ASSUME}]
\item $\Gamma\turn t$ \hfill [Hypothesis]
\item $\Gamma\turn t'\imp t$ \hfill [\rul{DISCH} 2]
\item $\Gamma,\ t'\turn t$ \hfill [\rul{MP} 3,1]
\end{proof}



%\subsection{Undischarging [\rul{UNDISCH}]}
\subsection{Undischarging}


\begin{holboxed}
\index{implication, in HOL logic@implication, in \HOL\ logic!inference rules for}
\index{UNDISCH@\ml{UNDISCH}|pin}
\begin{verbatim}
   UNDISCH : thm -> thm
\end{verbatim}\end{holboxed}


\vspace{12pt plus2pt minus1pt}

$$\Gamma\turn t_1\imp t_2 \over\Gamma,\ t_1\turn t_2$$

\vspace{12pt plus2pt minus1pt}

\begin{proof}
\item $t_1\turn t_1$ \hfill [\rul{ASSUME}]
\item $\Gamma\turn t_1\imp t_2$ \hfill [Hypothesis]
\item $\Gamma,\ t_1\turn t_2$ \hfill [\rul{MP} 2,1]
\end{proof}




\subsection{Symmetry of equality}


\begin{holboxed}
\index{SYM@\ml{SYM}|pin}
\index{symmetry of equality rule, in HOL logic@symmetry of equality rule, in \HOL\ logic}
\index{equality, in HOL logic@equality, in \HOL\ logic!symmetry rule for}
\begin{verbatim}
   SYM : thm -> thm
\end{verbatim}\end{holboxed}



\vspace{12pt plus2pt minus1pt}

$$\Gamma\turn t_1 = t_2\over \Gamma\turn t_2 = t_1$$

\vspace{12pt plus2pt minus1pt}

\begin{proof}
\item $\Gamma\turn t_1=t_2$\hfill [Hypothesis]
\item $\turn t_1=t_1$ \hfill [\rul{REFL}]
\item $\Gamma\turn t_2=t_1$\hfill [\rul{SUBST} 1,2]
\end{proof}



\subsection{Transitivity of equality}


\begin{holboxed}
\index{transitivity of equality rule, in HOL logic@transitivity of equality rule, in \HOL\ logic}
\index{equality, in HOL logic@equality, in \HOL\ logic!transitivity rule for}
\index{TRANS@\ml{TRANS}|pin}
\begin{verbatim}
   TRANS : thm -> thm -> thm
\end{verbatim}\end{holboxed}

\vspace{12pt plus2pt minus1pt}

$$\Gamma_1\turn t_1=t_2\qquad\qquad\qquad \Gamma_2\turn t_2=t_3 \over
\Gamma_1\cup\Gamma_2\turn t_1=t_3$$

\vspace{12pt plus2pt minus1pt}

\begin{proof}
\item $\Gamma_2\turn t_2=t_3$\hfill [Hypothesis]
\item $\Gamma_1\turn t_1=t_2$\hfill [Hypothesis]
\item $\Gamma_1\cup\Gamma_2\turn t_1=t_3$\hfill [\rul{SUBST} 1,2]
\end{proof}



\subsection{Application of a term to a theorem}%
\index{function application, in HOL logic@function application, in \HOL{} logic!inference rules for}

\begin{holboxed}
\index{AP_TERM@\ml{AP\_TERM}|pin}
\begin{verbatim}
   AP_TERM : term -> thm -> thm
\end{verbatim}\end{holboxed}

\vspace{12pt plus2pt minus1pt}

$$\Gamma\turn t_1=t_2\over\Gamma\turn t\ t_1 = t\ t_2$$

\vspace{12pt plus2pt minus1pt}

\begin{proof}
\item $\Gamma\turn t_1=t_2$\hfill [Hypothesis]
\item $\turn t\ t_1 = t\ t_1$ \hfill [\rul{REFL}]
\item $\Gamma\turn t\ t_1 = t\ t_2$ \hfill [\rul{SUBST} 1,2]
\end{proof}



\subsection{Application of a theorem to a term}

\begin{holboxed}
\index{AP_THM@\ml{AP\_THM}|pin}
\begin{verbatim}
   AP_THM : thm -> conv
\end{verbatim}\end{holboxed}

\vspace{12pt plus2pt minus1pt}

$$\Gamma\turn t_1=t_2\over \Gamma\turn t_1\ t = t_2\ t$$

\vspace{12pt plus2pt minus1pt}

\begin{proof}
\item $\Gamma\turn t_1=t_2$\hfill [Hypothesis]
\item$\turn t_1\ t = t_1\ t$\hfill [\rul{REFL}]
\item $\Gamma\turn t_1\ t = t_2\ t$\hfill [\rul{SUBST} 1,2]
\end{proof}



\subsection{Modus Ponens for equality}
\label{avra_eq_mp}

\begin{holboxed}
\index{EQ_MP@\ml{EQ\_MP}|pin}
\index{equality, in HOL logic@equality, in \HOL\ logic!MP rule for@\ml{MP} rule for}
\begin{verbatim}
   EQ_MP : thm -> thm -> thm
\end{verbatim}\end{holboxed}

\vspace{12pt plus2pt minus1pt}

$$\Gamma_1\turn t_1=t_2\qquad\qquad\qquad \Gamma_2\turn t_1\over
\Gamma_1\cup\Gamma_2\turn t_2$$

\vspace{12pt plus2pt minus1pt}

\begin{proof}
\item $\Gamma_1\turn t_1=t_2$ \hfill [Hypothesis]
\item $\Gamma_2\turn t_1$ \hfill [Hypothesis]
\item $\Gamma_1\cup\Gamma_2\turn t_2$ \hfill [\rul{SUBST} 1,2]
\end{proof}




\subsection{Implication from equality}
\index{equality, in HOL logic@equality, in \HOL\ logic!other rules for|(}
\index{implication, in HOL logic@implication, in \HOL\ logic!inference rules for}
\begin{holboxed}
\index{EQ_IMP_RULE@\ml{EQ\_IMP\_RULE}|pin}
\begin{verbatim}
   EQ_IMP_RULE : thm -> (thm # thm)
\end{verbatim}\end{holboxed}

\vspace{12pt plus2pt minus1pt}

$$\Gamma\turn t_1=t_2\over
\Gamma\turn t_1\imp t_2 \qquad\qquad\qquad \Gamma\turn t_2\imp t_1$$

\vspace{12pt plus2pt minus1pt}

\begin{proof}
\item $\Gamma\turn t_1=t_2$ \hfill [Hypothesis]
\item $t_1\turn t_1$ \hfill [\rul{ASSUME}]
\item $\Gamma,\ t_1\turn t_2$ \hfill [\rul{EQ\_MP} 1,2]
\item $\Gamma\turn t_1\imp t_2$ \hfill [\rul{DISCH} 3]
\item $\Gamma\turn t_2=t_1$ \hfill [\rul{SYM} 1]
\item $t_2\turn t_2$ \hfill [\rul{ASSUME}]
\item $\Gamma,\ t_2\turn t_1$ \hfill [\rul{EQ\_MP} 5,6]
\item $\Gamma\turn t_2\imp t_1$ \hfill [\rul{DISCH} 7]
\item $\Gamma\turn t_1\imp t_2$ and $\Gamma\turn t_2\imp t_1$\hfill [4,8]
\end{proof}



\subsection{\T-Introduction}
\label{avra_T}

\index{T@\holtxt{T}!rules of inference for|(}
\begin{hol}
\begin{verbatim}
   TRUTH
\end{verbatim}
\end{hol}

\vspace{12pt plus2pt minus1pt}

$$\turn\T$$

\vspace{12pt plus2pt minus1pt}

\begin{proof}
\item $\turn \T = ((\lquant{x}x)=(\lquant{x}x))$\hfill [Definition of \T]
\item $\turn ((\lquant{x}x)=(\lquant{x}x)) = \T$\hfill [\rul{SYM} 1]
\item $\turn (\lquant{x}x)=(\lquant{x}x)$\hfill [\rul{REFL}]
\item $\turn\T$ \hfill [\rul{EQ\_MP} 2,3]
\end{proof}




\subsection{Equality-with-\T\ elimination}

\begin{holboxed}
\index{EQT_ELIM@\ml{EQT\_ELIM}|pin}
\begin{verbatim}
   EQT_ELIM : thm -> thm
\end{verbatim}
\end{holboxed}

\vspace{12pt plus2pt minus1pt}

$$\Gamma\turn t = \T\over \Gamma\turn t$$

\vspace{12pt plus2pt minus1pt}

\begin{proof}
\item $\Gamma\turn t = \T$\hfill [Hypothesis]
\item $\Gamma\turn \T = t$\hfill [\rul{SYM} 1]
\item $\turn \T$\hfill [\rul{TRUTH}]
\item $\Gamma\turn t$\hfill [\rul{EQ\_MP} 2,3]
\end{proof}


\subsection{Specialization ($\forall$-elimination)}

\begin{holboxed}
\index{SPEC@\ml{SPEC}|pin}
\index{specialization rule, in HOL logic@specialization rule, in \HOL\ logic}
\begin{verbatim}
   SPEC : term -> thm -> thm
\end{verbatim}
\end{holboxed}

\label{avra_spec}

\vspace{12pt plus2pt minus1pt}

$$\Gamma\turn \uquant{x}t\over \Gamma\turn t[t'/x]$$
\begin{itemize}
\item $t[t'/x]$ denotes the result of substituting $t'$ for free\index{free variables, in HOL logic@free variables, in \HOL\ logic}
occurrences of $x$ in $t$, with the restriction that no free variables in $t'$
become bound after substitution.
\end{itemize}

\vspace{12pt plus2pt minus1pt}

\begin{proof}
\item $\turn \forall = (\lquant{P}P = (\lquant{x}\T))$ \hfill
[\rul{INST\_TYPE} applied to the definition of $\forall$]
\item $\Gamma\turn \forall(\lquant{x}t)$\hfill [Hypothesis]
\item $\Gamma\turn (\lquant{P}P=(\lquant{x}\T))(\lquant{x}t)$\hfill
[\rul{SUBST} 1,2]
\item $\turn  (\lquant{P}P=(\lquant{x}\T))(\lquant{x}t) =
((\lquant{x}t)=(\lquant{x}\T))$\hfill [\rul{BETA\_CONV}]
\item $\Gamma\turn (\lquant{x}t)=(\lquant{x}\T)$\hfill [\rul{EQ\_MP} 4,3]
\item $\Gamma\turn (\lquant{x}t)\ t' = (\lquant{x}\T)\ t'$ \hfill
[\rul{AP\_THM} 5]
\item $\turn (\lquant{x}t)\ t' = t[t'/x]$ \hfill [\rul{BETA\_CONV}]
\item $\Gamma\turn t[t'/x] = (\lquant{x}t)\ t'$ \hfill [\rul{SYM} 7]
\item $\Gamma\turn t[t'/x] = (\lquant{x}\T)\ t'$ \hfill [\rul{TRANS} 8,6]
\item $\turn (\lquant{x}\T)\ t' = \T$ \hfill [\rul{BETA\_CONV}]
\item $\Gamma\turn t[t'/x] = \T$ \hfill [\rul{TRANS} 9,10]
\item $\Gamma\turn t[t'/x]$ \hfill [\rul{EQT\_ELIM} 11]
\end{proof}




\subsection{Equality-with-\T\ introduction}

\begin{holboxed}
\index{EQT_INTRO@\ml{EQT\_INTRO}|pin}
\begin{verbatim}
   EQT_INTRO : thm -> thm
\end{verbatim}
\end{holboxed}


\vspace{12pt plus2pt minus1pt}

$$\Gamma\turn t\over\Gamma\turn t=\T$$

\vspace{12pt plus2pt minus1pt}

\begin{proof}
\item $\turn\uquant{b_1\ b_2}(b_1\imp b_2)\imp(b_2\imp b_1)\imp(b_1=b_2)$
\hfill [Axiom]
\item $\turn\uquant{b_2}(t\imp b_2)\imp(b_2\imp t)\imp(t=b_2)$
\hfill [\rul{SPEC} 1]
\item $\turn(t\imp\T)\imp(\T\imp t)\imp(t=\T)$\hfill [\rul{SPEC} 2]
\item $\turn\T$\hfill [\rul{TRUTH}]
\item $\turn t\imp\T$\hfill [\rul{DISCH} 4]
\item $\turn(\T\imp t)\imp(t=\T)$\hfill [\rul{MP} 3,5]
\item $\Gamma \turn t$\hfill [Hypothesis]
\item $\Gamma\turn\T\imp t$\hfill [\rul{DISCH} 7]
\item $\Gamma\turn t=\T$\hfill [\rul{MP} 6,8]
\end{proof}
\index{equality, in HOL logic@equality, in \HOL\ logic!other rules for|)}
\index{T@\holtxt{T}!rules of inference for|)}


\subsection{Generalization ($\forall$-introduction)}%
\index{universal quantifier, in HOL logic@universal quantifier, in \HOL{} logic!inference rules for}


\begin{holboxed}
\index{GEN@\ml{GEN}|pin}
\index{generalization rule, in HOL logic@generalization rule, in \HOL\ logic}
\begin{verbatim}
   GEN : term -> thm -> thm
\end{verbatim}
\end{holboxed}

\label{avra_gen}

\vspace{12pt plus2pt minus1pt}

$$\Gamma\turn t\over\Gamma\turn\uquant{x} t$$
\begin{itemize}
\item Where $x$ is not free in $\Gamma$.
\end{itemize}

\vspace{12pt plus2pt minus1pt}

\begin{proof}
\item $\Gamma\turn t$\hfill [Hypothesis]
\item $\Gamma\turn t = \T$\hfill [\rul{EQT\_INTRO} 1]
\item $\Gamma\turn(\lquant{x}t)=(\lquant{x}\T)$\hfill [\rul{ABS} 2]
\item $\turn \forall(\lquant{x}t) = \forall(\lquant{x}t)$\hfill [\rul{REFL}]
\item $\turn \forall = (\lquant{P} P =(\lquant{x}\T))$\hfill
[\rul{INST\_TYPE} applied to the definition of $\forall$]
\item $\turn\forall(\lquant{x}t)=(\lquant{P} P=(\lquant{x}\T))(\lquant{x}t)$
\hfill [\rul{SUBST} 5,4]
\item $\turn(\lquant{P} P=(\lquant{x}\T))(\lquant{x}t)=((\lquant{x}t)
=(\lquant{x}\T))$\hfill [\rul{BETA\_CONV}]
\item $\turn\forall(\lquant{x}t) = ((\lquant{x}t)=(\lquant{x}\T))$
\hfill [\rul{TRANS} 6,7]
\item $\turn((\lquant{x}t)=(\lquant{x}\T)) = \forall(\lquant{x}\T)$
\hfill [\rul{SYM} 8]
\item $\Gamma\turn\forall(\lquant{x}t)$\hfill [\rul{EQ\_MP} 9,3]
\end{proof}



\subsection{Simple $\alpha$-conversion}

\begin{holboxed}
\begin{verbatim}
   SIMPLE_ALPHA
\end{verbatim}
\end{holboxed}

\vspace{12pt plus2pt minus1pt}

$$\turn(\lquant{x_1}t\ x_1) = (\lquant{x_2}t\ x_2)$$
\begin{itemize}
\item Where neither $x_1$ nor $x_2$ occurs free in $t$.\footnote{\ml{SIMPLE\_ALPHA} is
included here because it is
used in a subsequent derivation, but it is not actually in the
\HOL\ system, as it is subsumed by other functions.}
\end{itemize}

\vspace{12pt plus2pt minus1pt}

\begin{proof}
\item$\turn(\lquant{x_1}t\ x_1)\ x = t\ x$\hfill [\rul{BETA\_CONV}]
\item$\turn(\lquant{x_2}t\ x_2)\ x = t\ x$\hfill [\rul{BETA\_CONV}]
\item $\turn t\ x = (\lquant{x_2}t\ x_2)\ x$\hfill [\rul{SYM} 2]
\item $\turn (\lquant{x_1}t\ x_1)\ x = (\lquant{x_2}t\ x_2)\ x$
\hfill [\rul{TRANS} 1,3]
\item $\turn(\lquant{x}(\lquant{x_1}t\ x_1)\ x) =
(\lquant{x}(\lquant{x_2}t\ x_2)\ x)$\hfill [\rul{ABS} 4]
\item $\turn\uquant{f}(\lquant{x}f\ x) = f$\hfill
[Appropriately type-instantiated axiom]
\item $\turn(\lquant{x}(\lquant{x_1}t\ x_1)x) = \lquant{x_1}t\ x_1$
\hfill [\rul{SPEC} 6]
\item $\turn(\lquant{x}(\lquant{x_2}t\ x_2)x) = \lquant{x_2}t\ x_2$
\hfill [\rul{SPEC} 6]
\item $\turn (\lquant{x_1}t\ x_1) = (\lquant{x}(\lquant{x_1}t\ x_1)x)$
\hfill [\rul{SYM} 7]
\item $\turn (\lquant{x_1}t\ x_1) = (\lquant{x}(\lquant{x_2}t\ x_2)x)$
\hfill [\rul{TRANS} 9,5]
\item $\turn(\lquant{x_1}t\ x_1)=(\lquant{x_2}t\ x_2)$\hfill
[\rul{TRANS} 10,8]
\end{proof}




\subsection{$\eta$-conversion}

\begin{holboxed}
\index{ETA_CONV@\ml{ETA\_CONV}|pin}
\begin{verbatim}
   ETA_CONV : conv
\end{verbatim}
\end{holboxed}
\vspace{12pt plus2pt minus1pt}

$$\turn(\lquant{x'}t\ x') = t$$
\begin{itemize}
\item Where $x'$ does not occur free\index{free variables, in HOL logic@free variables, in \HOL\ logic} in $t$ (we use $x'$ rather than just $x$
to motivate the use of \rul{SIMPLE\_ALPHA} in the derivation below).
\end{itemize}

\vspace{12pt plus2pt minus1pt}

\begin{proof}
\item $\turn\uquant{f}(\lquant{x}f\ x) = f$\hfill
[Appropriately type-instantiated axiom]
\item  $\turn(\lquant{x}t\ x) = t$\hfill [\rul{SPEC} 1]
\item $\turn(\lquant{x'}t\ x')=(\lquant{x}t\ x)$\hfill [\rul{SIMPLE\_ALPHA}]
\item $\turn(\lquant{x'}t\ x')=t$\hfill [\rul{TRANS} 3,2]
\end{proof}



\subsection{Extensionality}
\index{universal quantifier, in HOL logic@universal quantifier, in \HOL{} logic!inference rules for}

\begin{holboxed}
\index{EXT@\ml{EXT}|pin}
\index{extensionality rule, in HOL logic@extensionality rule, in \HOL{} logic}
\begin{verbatim}
   EXT : thm -> thm
\end{verbatim}
\end{holboxed}

\vspace{12pt plus2pt minus1pt}

$$\Gamma\turn\uquant{x} t_1\ x = t_2\ x\over\Gamma\turn t_1=t_2$$
\begin{itemize}
\item Where $x$ is not free\index{free variables, in HOL logic@free variables, in \HOL{} logic} in $t_1$ or $t_2$.
\end{itemize}

\vspace{12pt plus2pt minus1pt}

\begin{proof}
\item $\Gamma\turn\uquant{x}t_1\ x=t_2\ x$\hfill [Hypothesis]
\item $\Gamma\turn t_1\ x'=t_2\ x'$\hfill [\rul{SPEC} 1 ($x'$ is a fresh)]
\item $\Gamma\turn(\lquant{x'}t_1\ x') = (\lquant{x'}t_2\ x')$\hfill
        [\rul{ABS} 2]
\item $\turn(\lquant{x'}t_1\ x') = t_1$\hfill [\rul{ETA\_CONV}]
\item $\turn t_1 = (\lquant{x'}t_1\ x')$\hfill [\rul{SYM} 4]
\item $\Gamma\turn t_1 = (\lquant{x'}t_2\ x')$\hfill [\rul{TRANS} 5,3]
\item $\turn(\lquant{x'}t_2\ x') = t_2$\hfill [\rul{ETA\_CONV}]
\item $\Gamma\turn t_1=t_2$\hfill [\rul{TRANS} 6,7]
\end{proof}




\subsection{$\hilbert$-introduction}

\begin{holboxed}
\index{choice operator, in HOL logic@choice operator, in \HOL{} logic!inference rules for}
\index{SELECT_INTRO@\ml{SELECT\_INTRO}|pin}
\begin{verbatim}
   SELECT_INTRO : thm -> thm
\end{verbatim}\end{holboxed}

\vspace{12pt plus2pt minus1pt}

$$\Gamma\turn t_1\ t_2\over\Gamma\turn t_1(\hilbert\ t_1)$$

\vspace{12pt plus2pt minus1pt}

\begin{proof}
\item $\turn\uquant{P\ x}P\ x\imp P(\hilbert\ P)$\hfill [Suitably
type-instantiated axiom]
\item $\turn t_1\ t_2 \imp t_1(\hilbert\ t_1)$\hfill [\rul{SPEC} 1 (twice)]
\item $\Gamma\turn t_1\ t_2$\hfill [Hypothesis]
\item $\Gamma\turn t_1(\hilbert\ t_1)$\hfill [\rul{MP} 2,3]
\end{proof}




\subsection{$\hilbert$-elimination}

\begin{holboxed}
\index{choice operator, in HOL logic@choice operator, in \HOL{} logic!inference rules for}
\index{SELECT_ELIM@\ml{SELECT\_ELIM}|pin}
\begin{verbatim}
   SELECT_ELIM : thm -> (term # thm) -> thm
\end{verbatim}\end{holboxed}

\vspace{12pt plus2pt minus1pt}

$$\Gamma_1\turn t_1(\hilbert\ t_1)\qquad\qquad\qquad\Gamma_2,\ t_1\ v\turn t
\over \Gamma_1\cup\Gamma_2\turn t$$
\begin{itemize}
\item Where $v$ occurs nowhere except in the assumption $t_1\ v$ of the second
hypothesis.
\end{itemize}

\vspace{12pt plus2pt minus1pt}

\begin{proof}
\item $\Gamma_2,\ t_1\ v\turn t$ \hfill [Hypothesis]
\item $\Gamma_2\turn t_1\ v\imp t$\hfill [\rul{DISCH} 1]
\item $\Gamma_2\turn\uquant{v}t_1\ v\imp t$\hfill [\rul{GEN} 2]
\item $\Gamma_2\turn t_1(\hilbert\ t_1)\imp t$\hfill [\rul{SPEC} 3]
\item $\Gamma_1\turn t_1(\hilbert\ t_1)$\hfill [Hypothesis]
\item $\Gamma_1\cup\Gamma_2\turn t$\hfill [\rul{MP} 4,5]
\end{proof}




\subsection{$\exists$-introduction}
\index{existential quantifier, in HOL logic@existential quantifier, in \HOL{} logic!inference rules for|(}

\begin{holboxed}
\index{EXISTS@\ml{EXISTS}|pin}
\begin{verbatim}
   EXISTS : (term # term) -> thm -> thm
\end{verbatim}\end{holboxed}

\vspace{12pt plus2pt minus1pt}

$$\Gamma\turn t_1[t_2]\over \Gamma\turn \equant{x}t_1[x]$$
\begin{itemize}
\item Where $t_1[t_2]$ denotes a term $t_1$ with some free\index{free variables, in HOL logic@free variables, in \HOL{} logic}
occurrences of $t_2$
singled out, and $t_1[x]$ denotes the result of replacing these
occurrences of $t_1$ by $x$, subject to the restriction that $x$
doesn't become bound after substitution.
\end{itemize}

\vspace{12pt plus2pt minus1pt}

\begin{proof}
\item $\turn(\lquant{x}t_1[x])t_2= t_1[t_2]$\hfill [\rul{BETA\_CONV}]
\item $\turn t_1[t_2] = (\lquant{x}t_1[x])t_2$\hfill [\rul{SYM} 1]
\item $\Gamma\turn t_1[t_2]$\hfill [Hypothesis]
\item $\Gamma\turn(\lquant{x}t_1[x])t_2$\hfill [\rul{EQ\_MP} 2,3]
\item $\Gamma\turn(\lquant{x}t_1[x])(\hilbert(\lquant{x}t_1[x]))$\hfill
[\rul{SELECT\_INTRO} 4]
\item $\turn \exists = \lquant{P} P(\hilbert\ P)$\hfill
[\rul{INST\_TYPE} applied to the definition of $\exists$]
\item $\turn\exists(\lquant{x}t_1[x]) =
(\lquant{P}P(\hilbert\ P))(\lquant{x}t_1[x])$\hfill [\rul{AP\_THM} 6]
\item $\turn(\lquant{P}P(\hilbert\ P))(\lquant{x}t_1[x]) =
(\lquant{x}t_1[x])(\hilbert(\lquant{x}t_1[x]))$\hfill [\rul{BETA\_CONV}]
\item $\turn\exists(\lquant{x}t_1[x]) =
(\lquant{x}t_1[x])(\hilbert(\lquant{x}t_1[x]))$\hfill [\rul{TRANS} 7,8]
\item $\turn(\lquant{x}t_1[x])(\hilbert(\lquant{x}t_1[x])) =
\exists(\lquant{x}t_1[x])$\hfill [\rul{SYM} 9]
\item $\Gamma\turn\exists(\lquant{x}t_1[x])$\hfill [\rul{EQ\_MP} 10,5]
\end{proof}



\subsection{$\exists$-elimination}

\begin{holboxed}
\index{CHOOSE@\ml{CHOOSE}|pin}
\begin{verbatim}
   CHOOSE : (term # thm) -> thm -> thm
\end{verbatim}\end{holboxed}

\vspace{12pt plus2pt minus1pt}

$$\Gamma_1\turn\equant{x}t[x]\qquad\qquad\qquad \Gamma_2,\ t[v]\turn t'
\over \Gamma_1\cup\Gamma_2\turn t'$$
\begin{itemize}
\item Where $t[v]$ denotes a term $t$ with some free\index{free variables, in HOL logic@free variables, in \HOL{} logic}
occurrences of the variable $v$
singled out, and $t[x]$ denotes the result of replacing these
occurrences of $v$ by $x$, subject to the restriction that $x$ doesn't become
bound after substitution.
\end{itemize}

\vspace{12pt plus2pt minus1pt}

\begin{proof}
\item $\turn \exists = \lquant{P} P(\hilbert\ P)$\hfill
[\rul{INST\_TYPE} applied to the definition of $\exists$]
\item $\turn\exists(\lquant{x}t[x]) =
(\lquant{P}P(\hilbert\ P))(\lquant{x}t[x])$\hfill [\rul{AP\_THM} 1]
\item $\Gamma_1\turn\exists(\lquant{x}t[x])$\hfill [Hypothesis]
\item $\Gamma_1\turn (\lquant{P}P(\hilbert\ P))(\lquant{x}t[x])$
\hfill [\rul{EQ\_MP} 2,3]
\item $\turn(\lquant{P}P(\hilbert\ P))(\lquant{x}t[x]) =
(\lquant{x}t[x])(\hilbert(\lquant{x}t[x]))$\hfill [\rul{BETA\_CONV}]
\item $\Gamma_1\turn(\lquant{x}t[x])(\hilbert(\lquant{x}t[x])$\hfill
[\rul{EQ\_MP} 5,4]
\item $\turn(\lquant{x}t[x])v = t[v]$\hfill [\rul{BETA\_CONV}]
\item $\turn t[v] =(\lquant{x}t[x])v$\hfill [\rul{SYM} 7]
\item $\Gamma_2,\ t[v]\turn t'$\hfill [Hypothesis]
\item $\Gamma_2\turn t[v]\imp t'$\hfill [\rul{DISCH} 9]
\item $\Gamma_2\turn(\lquant{x}t[x])v\imp t'$\hfill [\rul{SUBST} 8,10]
\item $\Gamma_2,\ (\lquant{x}t[x])v\turn t'$\hfill [\rul{UNDISCH} 11]
\item $\Gamma_1\cup\Gamma_2\turn t'$\hfill [\rul{SELECT\_ELIM} 6,12]
\end{proof}
\index{existential quantifier, in HOL logic@existential quantifier, in \HOL{} logic!inference rules for|)}

\subsection{Use of a definition}

\begin{holboxed}
\index{RIGHT_BETA@\ml{RIGHT\_BETA}|pin}
\begin{verbatim}
   RIGHT_BETA : thm -> thm
\end{verbatim}\end{holboxed}

\vspace{12pt plus2pt minus1pt}

$$\Gamma\turn t = \lquant{x}t'[x]
\over \Gamma\turn t\ t = t'[t]$$
\begin{itemize}
\item Where  $t$ does not contain $x$.
\end{itemize}

\vspace{12pt plus2pt minus1pt}

\begin{proof}
\item $\Gamma\turn t = \lquant{x} t'[x]$\hfill
[Suitably type-instantiated hypothesis]
\item $\Gamma\turn t\ t =
(\lquant{x}t'[x])\ t$\hfill
[\rul{AP\_THM} 1 ]
\item $\turn(\lquant{x}t'[x])\ t =
t'[t]$\hfill [\rul{BETA\_CONV}]
\item $\Gamma\turn t\ t = t'[t]$\hfill
[\rul{TRANS} 2,3]
\end{proof}


\subsection{Use of a definition}

\begin{holboxed}
\index{RIGHT_LIST_BETA@\ml{RIGHT\_LIST\_BETA}|pin}
\begin{verbatim}
   RIGHT_LIST_BETA : thm -> thm
\end{verbatim}\end{holboxed}

\vspace{12pt plus2pt minus1pt}

$$\Gamma\turn t = \lquant{x_1\cdots x_n}t'[x_1,\ldots,x_n]
\over \Gamma\turn t\ t_1\cdots t_n = t'[t_1,\ldots,t_n]$$
\begin{itemize}
\item Where none of the $t_i$ contain any of the $x_i$.
\end{itemize}

\vspace{12pt plus2pt minus1pt}

\begin{proof}
\item $\Gamma\turn t = \lquant{x_1\cdots x_n} t'[x_1,\ldots,x_n]$\hfill
[Suitably type-instantiated hypothesis]
\item $\Gamma\turn t\ t_1\cdots t_n =
(\lquant{x_1\cdots x_n}t'[x_1,\ldots,x_n])\ t_1\cdots t_n$\hfill
[\rul{AP\_THM} 1 (n times)]
\item $\turn(\lquant{x_1\cdots x_n}t'[x_1,\ldots,x_n])\ t_1\cdots t_n =
t'[t_1,\ldots,t_n]$\hfill [\rul{BETA\_CONV} (n times)]
\item $\Gamma\turn t\ t_1\cdots t_n = t'[t_1,\ldots,t_n]$\hfill
[\rul{TRANS} 2,3]
\end{proof}





\subsection{$\wedge$-introduction}
\label{avra_conj}


\begin{holboxed}
\index{CONJ@\ml{CONJ}|pin}
\index{conjunction, in HOL logic@conjunction, in \HOL{} logic!inference rule for}
\begin{verbatim}
   CONJ : thm -> thm -> thm
\end{verbatim}\end{holboxed}

\vspace{12pt plus2pt minus1pt}

$$\Gamma_1\turn t_1\qquad\qquad\qquad\Gamma_2\turn t_2\over
\Gamma_1\cup\Gamma_2 \turn t_1\conj t_2$$

\vspace{12pt plus2pt minus1pt}

\begin{proof}
\item $\turn \conj = \lquant{b_1\ b_2}\uquant{b}(b_1\imp(b_2\imp b))\imp b$
\hfill [Definition of $\conj$]
\item $\turn t_1\conj t_2 = \uquant{b}(t_1\imp(t_2\imp b))\imp b$\hfill
[\rul{RIGHT\_LIST\_BETA} 1]
\item $t_1\imp(t_2\imp b)\turn t_1\imp(t_2\imp b)$\hfill [\rul{ASSUME}]
\item $\Gamma_1\turn t_1$\hfill [Hypothesis]
\item $\Gamma_1,\ t_1\imp(t_2\imp b)\turn t_2\imp b$\hfill [\rul{MP} 3,4]
\item $\Gamma_2\turn t_2$\hfill [Hypothesis]
\item $\Gamma_1\cup\Gamma_2,\ t_1\imp(t_2\imp b)\turn b$\hfill [\rul{MP} 5,6]
\item $\Gamma_1\cup \Gamma_2\turn(t_1\imp(t_2\imp b))\imp b$\hfill
[\rul{DISCH} 7]
\item $\Gamma_1\cup \Gamma_2\turn \uquant{b}(t_1\imp(t_2\imp b))\imp b$\hfill
[\rul{GEN} 8]
\item $\Gamma_1\cup \Gamma_2\turn t_1\conj t_2$\hfill
[\rul{EQ\_MP} (\rul{SYM} 2),9]
\end{proof}




\subsection{$\wedge$-elimination}


\begin{holboxed}
\index{CONJUNCT1@\ml{CONJUNCT1}|pin}
\index{CONJUNCT2@\ml{CONJUNCT2}|pin}
\begin{verbatim}
   CONJUNCT1 : thm -> thm, CONJUNCT2 : thm -> thm
\end{verbatim}\end{holboxed}

\vspace{12pt plus2pt minus1pt}

$$\Gamma\turn t_1\conj t_2\over
\Gamma\turn t_1\qquad\qquad\qquad \Gamma\turn t_2$$

\vspace{12pt plus2pt minus1pt}

\begin{proof}
\item $\turn \conj = \lquant{b_1\ b_2}\uquant{b}(b_1\imp(b_2\imp b))\imp b$
\hfill [Definition of $\conj$]
\item $\turn t_1\conj t_2 = \uquant{b}(t_1\imp(t_2\imp b))\imp b$\hfill
[\rul{RIGHT\_LIST\_BETA} 1]
\item $\Gamma\turn t_1\conj t_2$\hfill [Hypothesis]
\item $\Gamma\turn \uquant{b}(t_1\imp(t_2\imp b))\imp b$\hfill
[\rul{EQ\_MP} 2,3]
\item $\Gamma\turn (t_1\imp(t_2\imp t_1))\imp t_1$\hfill [\rul{SPEC} 4]
\item $t_1\turn t_1$\hfill [\rul{ASSUME}]
\item $t_1 \turn t_2\imp t_1$\hfill [\rul{DISCH} 6]
\item $\turn t_1\imp(t_2\imp t_1)$\hfill [\rul{DISCH} 7]
\item $\Gamma\turn t_1$\hfill [\rul{MP} 5,8]
\item $\Gamma\turn (t_1\imp(t_2\imp t_2))\imp t_2$\hfill [\rul{SPEC} 4]
\item $t_2\turn t_2$\hfill [\rul{ASSUME}]
\item $\turn t_2\imp t_2$\hfill [\rul{DISCH} 11]
\item $\turn t_1\imp(t_2\imp t_2)$\hfill [\rul{DISCH} 12]
\item $\Gamma\turn t_2$\hfill [\rul{MP} 10,13]
\item $\Gamma\turn t_1$ and $\Gamma\turn t_2$\hfill [9,14]
\end{proof}




\subsection{Right $\vee$-introduction}\index{disjunction, in HOL logic@disjunction, in \HOL{} logic!inference rule for|(}

\begin{holboxed}
\index{DISJ1@\ml{DISJ1}|pin}
\begin{verbatim}
   DISJ1 : thm -> conv
\end{verbatim}\end{holboxed}

\vspace{12pt plus2pt minus1pt}

$$\Gamma\turn t_1\over \Gamma\turn t_1\disj t_2$$

\vspace{12pt plus2pt minus1pt}

\begin{proof}
\item $\turn \disj =
\lquant{b_1\ b_2}\uquant{b}(b_1\imp b)\imp(b_2\imp b)\imp b$
\hfill [Definition of $\disj$]
\item $\turn t_1\disj t_2 = \uquant{b}(t_1\imp b)\imp(t_2\imp b)\imp b$
\hfill [\rul{RIGHT\_LIST\_BETA} 1]
\item $\Gamma\turn t_1$\hfill [Hypothesis]
\item $t_1\imp b\turn t_1\imp b$\hfill [\rul{ASSUME}]
\item $\Gamma,\ t_1\imp b\turn b$\hfill [\rul{MP} 4,3]
\item $\Gamma,\ t_1\imp b\turn(t_2\imp b)\imp b$\hfill [\rul{DISCH} 5]
\item $\Gamma\turn (t_1\imp b)\imp(t_2\imp b)\imp b$\hfill [\rul{DISCH} 6]
\item $\Gamma\turn \uquant{b}(t_1\imp b)
\imp(t_2\imp b)\imp b$\hfill [\rul{GEN} 7]
\item $\Gamma\turn t_1\disj t_2$\hfill [\rul{EQ\_MP} (\rul{SYM} 2),8]
\end{proof}




\subsection{Left $\vee$-introduction}


\begin{holboxed}
\index{DISJ2@\ml{DISJ2}|pin}
\begin{verbatim}
   DISJ2 : term -> thm -> thm
\end{verbatim}\end{holboxed}


\vspace{12pt plus2pt minus1pt}

$$\Gamma\turn t_2\over \Gamma\turn t_1\disj t_2$$

\vspace{12pt plus2pt minus1pt}

\begin{proof}
\item $\turn \disj =
\lquant{b_1\ b_2}\uquant{b}(b_1\imp b)\imp(b_2\imp b)\imp b$
\hfill [Definition of $\disj$]
\item $\turn t_1\disj t_2 = \uquant{b}(t_1\imp b)\imp(t_2\imp b)\imp b$
\hfill [\rul{RIGHT\_LIST\_BETA} 1]
\item $\Gamma\turn t_2$\hfill [Hypothesis]
\item $t_2\imp b\turn t_2\imp b$\hfill [\rul{ASSUME}]
\item $\Gamma,\ t_2\imp b\turn b$\hfill [\rul{MP} 4,3]
\item $\Gamma\turn(t_2\imp b)\imp b$\hfill [\rul{DISCH} 5]
\item $\Gamma\turn (t_1\imp b)\imp(t_2\imp b)\imp b$\hfill [\rul{DISCH} 6]
\item $\Gamma\turn \uquant{b}(t_1\imp b)
\imp(t_2\imp b)\imp b$\hfill [\rul{GEN} 7]
\item $\Gamma\turn t_1\disj t_2$\hfill [\rul{EQ\_MP} (\rul{SYM} 2),8]
\end{proof}


\subsection{$\vee$-elimination}

\begin{holboxed}
\index{DISJ_CASES@\ml{DISJ\_CASES}|pin}
\begin{verbatim}
   DISJ_CASES : thm -> thm -> thm -> thm
\end{verbatim}\end{holboxed}

\vspace{12pt plus2pt minus1pt}

$$\Gamma\turn t_1\disj t_2\qquad\qquad\qquad\Gamma_1,\ t_1\turn t
\qquad\qquad\qquad \Gamma_2,\ t_2\turn t\over
\Gamma\cup\Gamma_1\cup\Gamma_2\turn t$$

\vspace{12pt plus2pt minus1pt}

\begin{proof}
\item $\turn \disj =
\lquant{b_1\ b_2}\uquant{b}(b_1\imp b)\imp(b_2\imp b)\imp b$
\hfill [Definition of $\disj$]
\item $\turn t_1\disj t_2 = \uquant{b}(t_1\imp b)\imp(t_2\imp b)\imp b$
\hfill [\rul{RIGHT\_LIST\_BETA} 1]
\item $\Gamma\turn t_1\disj t_2$\hfill [Hypothesis]
\item $\Gamma\turn\uquant{b}(t_1\imp b)\imp(t_2\imp b)\imp b$\hfill
[\rul{EQ\_MP} 2,3]
\item $\Gamma\turn(t_1\imp t)\imp(t_2\imp t)\imp t$\hfill [\rul{SPEC} 4]
\item $\Gamma_1,\ t_1\turn t$\hfill [Hypothesis]
\item $\Gamma_1\turn t_1\imp t$\hfill [\rul{DISCH} 6]
\item $\Gamma\cup \Gamma_1\turn (t_2\imp t)\imp t$\hfill [\rul{MP} 5,7]
\item $\Gamma_2,\ t_2\turn t$\hfill [Hypothesis]
\item $\Gamma_2\turn t_2\imp t$\hfill [\rul{DISCH} 9]
\item $\Gamma\cup \Gamma_1\cup \Gamma_2\turn t$\hfill [\rul{MP} 8,10]
\end{proof}
\index{disjunction, in HOL logic@disjunction, in \HOL{} logic!inference rule for|)}




\subsection{Classical contradiction rule}
\index{F (falsity), the HOL constant@\holtxt{F} (falsity), the HOL{} constant!rules of inference for}

\begin{holboxed}
\index{CCONTR@\ml{CCONTR}|pin}
\index{contradiction rule, in HOL logic@contradiction rule, in \HOL{} logic}
\begin{verbatim}
   CCONTR : term -> thm -> thm
\end{verbatim}\end{holboxed}

\vspace{12pt plus2pt minus1pt}

$$\Gamma,\ \neg t\turn \F\over \Gamma\turn t$$

\vspace{12pt plus2pt minus1pt}

\begin{proof}
\item $\turn \neg = \lquant{b}b\imp\F$\hfill [Definition of $\neg$]
\item $\turn \neg t = t\imp\F$\hfill [\rul{RIGHT\_LIST\_BETA} 1]
\item $\Gamma,\ \neg t\turn\F$\hfill [Hypothesis]
\item $\Gamma\turn \neg t\imp\F$\hfill  [\rul{DISCH} 3]
\item $\Gamma\turn (t\imp\F)\imp\F$\hfill [\rul{SUBST} 2,4]
\item $t = \F\turn t = \F$\hfill [\rul{ASSUME}]
\item $\Gamma,\ t=\F\turn (\F\imp\F)\imp\F$\hfill [\rul{SUBST} 6,5]
\item $\F\turn\F$\hfill [\rul{ASSUME}]
\item $\turn \F\imp\F$\hfill [\rul{DISCH} 8]
\item $\Gamma,\ t=\F\turn\F$\hfill [\rul{MP} 7,9]
\item $\turn \F = \uquant{b}b$\hfill [Definition of $\F$]
\item $\Gamma,\ t=\F\turn \uquant{b}b$\hfill [\rul{SUBST} 11,10]
\item $\Gamma,\ t=\F\turn t$\hfill [\rul{SPEC} 12]
\item $\turn \uquant{b} (b = \T)\disj(b = \F)$\hfill [Axiom]
\item $\turn (t = \T)\disj(t = \F)$\hfill [\rul{SPEC} 14]
\item $t=\T\turn t=\T$\hfill [\rul{ASSUME}]
\item $t=\T\turn t$\hfill [\rul{EQT\_ELIM} 16]
\item $\Gamma\turn t$\hfill [\rul{DISJ\_CASES} 15,17,13]
\end{proof}
\index{derived rules, in HOL logic@derived rules, in \HOL{} logic!list and derivations of some|)}
\index{inference rules, of HOL logic@inference rules, of \HOL{} logic!derived|)}



%%% Local Variables:
%%% mode: latex
%%% TeX-master: "description"
%%% End:
