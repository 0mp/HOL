\chapter*{Prefazione}\markboth{Prefazione}{Prefazione}
\label{intro}

Questo volume contiene la descrizione del sistema \HOL.
E' uno di quattro volumi che compongono la documentazione per \HOL:

\begin{myenumerate}
\item \LOGIC: una descrizione formale della logica di ordine superiore
  implementata dal sistema \HOL{}.
\item \TUTORIAL: un tutorial d'introduzione a \HOL, con dei casi di studio.
\item \DESCRIPTION: una guida utente dettagliata per il sistema \HOL{}.
\item \REFERENCE: il manuale di riferimento per \HOL.
\end{myenumerate}

\noindent Si far� riferimento a questi quattro documenti con i nomi abbreviati (in
maiuscoletto inclinato) dati di sopra.

Questo documento, \DESCRIPTION, � una guida avanzata per utenti con qualche
esperienza precedente del sistema. I principianti dovrebbero iniziare con il
documento di accompagnamento \TUTORIAL.

Il sistema \HOL\ � sviluppato per supportare la dimostrazione interattiva di teoremi nella
logica di ordine superiore (da qui l'acronimo `\HOL' [Higher Order Logic (ndt)]). A questo scopo, la logica
formale � interfacciata a un linguaggio di programmazione di scopo generale (\ML, per
meta-linguaggio) in cui possono essere denotati i termini e i teoremi della logica,
possono essere espresse ed applicate le strategie di dimostrazione, e sviluppate le teorie logiche.
La versione della logica di ordine superiore usata in \HOL\ � il calcolo dei predicati
con termini dal lambda calcolo tipizzato (cio� la teoria dei
tipi semplici). Questo fu inizialmente sviluppato come una fondazione per la matematica
\cite{Church}. La principale area di applicazione di \HOL\ fu inizialmente
intesa essere la specifica e la verifica dei modelli hardware.
(L'uso della logica di ordine superiore a questo scopo fu inizialmente difesa da
Keith Hanna \cite{Hanna-Daeche}.) Tuttavia, la logica non limita
le applicazione all'hardware; \HOL\ � stato applicato a molte altre aree.

Questo documento presenta la logica \HOL\ nella sua forma \ML{}, e
spiega i mezzi con cui le funzioni del meta-linguaggio possono essere usate per
generare dimostrazioni nella logica. Cos�. descrive come il sistema
astratto di \LOGIC{} � di fatto implementato nel linguaggio di programmazione
\ML{}, fornendo una descrizione complessiva delle principali caratteristiche
del sistema.

L'approccio di meccanizzare le dimostrazioni formali usato in \HOL\ � dovuto a Robin
Milner \cite{Edinburgh-LCF}, che ha anche diretto il team che ha progettato
e implementato il linguaggio \ML. Quel lavoro era centrato su un sistema
chiamato \LCF\ (logica per le funzioni computabili), che era inteso per
il ragionamento interattivo automatizzato circa funzioni di ordine superiore definite
ricorsivamente. L'interfaccia della logica al meta-linguaggio era reso
esplicito, usando la struttura di tipi dell'\ML, con l'intenzione che alla fine
altre logiche sarebbe state provate al posto della logica originaria. Il
sistema \HOL\ � un discendente diretto di \LCF; questo � riflesso in
ogni cosa dalla sua struttura e aspetto alla sua incorporazione dell'\ML,
e persino a parti della sua implementazione. Cos� \HOL\ soddisfa il
piano iniziale di applicare la metodologia \LCF\ ad altre logiche.

L'\LCF\ originario fu implementato ad Edimburgo nei primi anni 1970, ed ora ci si
riferisce ad esso come all'`\LCF di Edimburgo'. Il suo codice fu portato dallo Stanford Lisp
al Franz Lisp da G\'erard Huet presso l'{\small INRIA}, e fu usato in un progetto di ricerca
francese chiamato `Formel'. La versione Franz Lisp dell'\LCF\ di Huet fu
ulteriormente sviluppata a Cambridge da Larry Paulson, e divenne conosciuta come il
`Cambridge \LCF'. Il sistema \HOL\ � implementato sulla base di una prima versione del Cambridge
\LCF\ e di conseguenza molte delle caratteristiche sia dell'Edinburgh che del Cambridge \LCF\ furono
ereditate da \HOL. Per esempio, l'assiomatizzazione della logica di ordine superiore utilizzata
non � quella classica dovuta a Church, ma una formulazione equivalente
influenzata da \LCF.

Una versione avanzata e razionalizzata di \HOL, chiamata \HOL 88, fu
rilasciata (nel 1988), dopo che il sistema \HOL\ originario era stato in uso
per molti anni. \HOL 90 (rilasciato nel 1990) fu un porting di \HOL 88
in SML \cite{sml} da parte di Konrad Slind presso l'Universit� di Calgary. Esso � stato
ulteriormente sviluppato nel corso degli anni 1990. \HOL{} 4 � l'ultima
versione di \HOL, e anch'esso � implementato in SML; � dotato di un numero
di novit� rispetto ai suoi predecessori. \HOL{} 4 � anche la
versione del sistema supportata per la comunit� internazionale \HOL.

Abbiamo deciso di numerare le implementazioni di \HOL{} retroattivamente nel
modo seguente
\begin{enumerate}
\item \HOL88 e precedenti: implementazioni basate su un substrato Lisp,
	con l'ML Classico.
\item \HOL90: implementazioni in Standard ML, che usano principalmente
	l'implementazione SML/NJ.
\item \HOL98 (distribuzioni Athabasca e Taupo): implementazioni che usano
	il Moscow ML, e con una nuova libreria e un nuovo meccanismo di teoria.
\item \HOL{} (distribuzioni Kananaskis)
\end{enumerate}
Di conseguenza, con \HOL{}~4, facciamo a meno dell'abitudine di associare
le implementazioni con i numeri degli anni. Le distribuzioni individuali all'interno
di \HOL{}~4 conservano lo schema di nome \textit{lake-number}.

In questo documento, l'acronimo `\HOL' si riferisce sia al sistema di dimostrazione
interattiva di teoremi sia alla versione di logica di ordine superiore che il sistema
supporta; dove c'� una seria ambiguit�, il primo � chiamato `il sistema
\HOL' e la seconda `la logica \HOL'.


%%% Local Variables:
%%% mode: latex
%%% TeX-master: "description"
%%% End:
