\chapter{Ottenere e Installare \HOL{}}
\label{install}

Questo capitolo descrive come ottenere il sistema \HOL{} e come installarlo.
Si assume in generale che si stia usando un qualche sistema Unix,
ma le istruzioni che seguono dovrebbero applicarsi {\it mutatis
  mutandis\/} ad altre piattaforme. Unix non � un prerequisito per
usare il sistema. \HOL{} pu� essere lanciato su PC che utilizzano sistemi
operativi Windows da Windows~NT in avanti (cio� anche Windows~2000, XP e Vista
sono supportati), esattamente come Macintosh che utilizzano MacOS~X.

\section{Ottenere \HOL{}}

Il sistema \HOL{} pu� essere scaricato da
\url{http://hol.sourceforge.net}. Lo schema di denominazione per le release di
\holn{} � $\langle${\it nome}$\rangle$-$\langle${\it
  numero}$\rangle$; la release descritta qui � \holnversion.

\section{La mailing list {\tt hol-info}}

La mailing list \texttt{hol-info} serve come un forum per discutere su
\HOL{} e diffondere notizie che lo riguardano. Se si desidera essere in questa
lista (cosa che si raccomanda a tutti gli utenti di \HOL), si visiti
\url{http://lists.sourceforge.net/lists/listinfo/hol-info}. Questa
pagina web pu� anche essere usata per annullare l'iscrizione alla mailing list.

\section{Installare \HOL{}}

Si assume che si siano ottenuti i sorgenti di \HOL{} e che il
file \texttt{tar} sia stato estratto in una directory \ml{hol}\footnote{Si pu�
scegliere un altro nome se lo si desidera; non � importante.}. E' probabile che
il contenuto di questa directory cambi nel tempo, ma dovrebbe
contenere i seguenti:

\begin{center}
\begin{tabular}{|l|l|l|} \hline
\multicolumn{3}{|c|}{ } \\
\multicolumn{3}{|c|}{\bf File Principali nella Directory di Distribuzione di HOL} \\
\multicolumn{3}{|c|}{ } \\
{\it Nome File} & {\it Descrizione} & {\it Tipo di File}  \\ \hline
{\tt README} & Descrizione della directory {\tt hol} & Testo\\
{\tt COPYRIGHT}& Un avviso di copyright & Testo\\
{\tt INSTALL} & Istruzioni d'installazione & Testo\\
{\tt tools} & Il codice sorgente per la costruzione del sistema & Directory\\
{\tt bin} & Directory per gli eseguibili di HOL & Directory\\
{\tt sigobj} & Directory per i file object \ML{} & Directory\\
{\tt src} & Sorgenti \ML{} di \HOL & Directory\\
{\tt help} & File di help per il sistema \HOL{} & Directory\\
{\tt examples} & File sorgenti di esempio & Directory\\
\hline
\end{tabular}
\end{center}

La sessione nel box di sotto mostra una directory di distribuzione tipica.
La distribuzione \HOL{} � stata messa su un PC che esegue Linux nella
directory {\small\tt /home/mn200/hol/}.

Tutte le sessioni in questa documentazione saranno mostrate in box con un
numero nell'angolo destro in alto. Questo numero indica se
la sessione � una nuova (quando il numero sar� {\small\sl 1}) o
la continuazione di una sessione iniziata in un box precedente.
Si assume che box numerati consecutivamente siano parte di una singola
sessione continua. Il prompt Unix per la sessione �
\texttt{\small \dol}, cos� le linee che iniziano con questo prompt sono state digitate
dall'utente. Dopo essere entrato nel sistema \HOL{} (si veda di sotto), all'utente � presentato
il prompt {\small\verb|-|} per inserire un'espressione o un comando del
meta-linguaggio \ML\ di \HOL{}; le linee che cominciano con questo prompt sono dunque espressioni o
dichiarazioni \ML. Le linee che non cominciano con \texttt{\small
  \$} o {\small\verb|-|} sono output del sistema. Saltuariamente, l'output
di sistema sar� rimpiazzato con una linea contenente {\small\verb|...|} quando
� di interesse minimo. Il meta-linguaggio \ML{} � introdotto nel
Capitolo~\ref{ML}.

\setcounter{sessioncount}{0}
\begin{session}
\begin{verbatim}
$ pwd
/home/mn200/hol
$ ls -F
COPYRIGHT  bin/  examples/  INSTALL  src/
README     doc/  help/      sigobj/  tools/
\end{verbatim}
\end{session}

Ora sar� necessario ricostruire \HOL{} dai sorgenti\footnote{ E'
  possibile che sistemi pre-costruiti siano disponibili a breve dalla
	pagina web menzionata di sopra.}.

Prima di iniziare � necessario avere una versione attuale di Moscow~ML o
Poly/ML\footnote{Poly/ML non pu� essere usato con HOL su Windows.}. Nel
caso di Moscow~ML, bisogna avere la versione 2.01. Moscow~ML � disponibile
sul web all'indirizzo \url{http://www.itu.dk/~sestoft/mosml.html}.
Poly/ML � disponibile all'indirizzo \url{http://polyml.org}.

Quando si lavora con Poly/ML, l'installazione deve assicurare che il caricamento dinamico delle librerie (tipicamente fatto settando la variabile d'ambiente \texttt{LD\_LIBRARY\_PATH}) trovi \texttt{libpolyml.so} e \texttt{libpolymain.so}.
Se questi file sono in \texttt{/usr/lib}, non c'� bisogno di cambiare nulla, ma altre posizioni possono richiedere un'ulteriore configurazione di sistema.
Un esempio di comando di inizializzazione di \texttt{LD\_LIBRARY\_PATH}  (in un file come \texttt{.bashrc}) potrebbe essere
\begin{verbatim}
   declare -x LD_LIBRARY_PATH=/usr/local/lib:$HOME/lib $
\end{verbatim}

Quando il proprio sistema ML � installato, e ci si trova nella directory radice della distribuzione, il passo successivo � eseguire \texttt{smart-configure}.
Con Moscow~ML, questo equivale a:

\begin{session}
\begin{alltt}
\dol mosml < tools/smart-configure.sml
Moscow ML version 2.01 (January 2004)
Enter `quit();' to quit.
- [opening file "tools/smart-configure-mosml.sml"]

HOL smart configuration.

Determining configuration parameters: OS mosmldir holdir
OS:                 linux
mosmldir:           /home/mn200/mosml/bin
holdir:             /home/mn200/hol
dynlib_available:   true

Configuration will begin with above values.  If they are wrong
press Control-C.
\end{alltt}
\end{session}

Se si usa Poly/ML, allora si scrive
\begin{verbatim}
   poly < tools/smart-configure.sml
\end{verbatim}
al suo posto.

Assumendo che il processo di configurazione non venga interrotto, questo
costruir� i programmi \texttt{Holmake} e \texttt{build}, e li sposter�
nella directory \texttt{hol/bin}. Se qualcosa va storto in  questo
passo, si consulti la Sezione~\ref{sec:editting-configure} di sotto.

Il passo successivo � eseguire il programma \texttt{build}. Questo dovrebbe
risultare in una grande quantit� di output mentre il codice del sistema viene compilato
e le teorie costruite. Alla fine � prodotto un sistema \HOL{}\footnote{Sono
  prodotti quattro eseguibili \HOL{}: \textsf{hol}, \textsf{hol.noquote},
  \textsf{hol.bare} e \textsf{hol.bare.noquote}. Il primo di questi
	sar� usato per la maggior parte degli esempi nel \TUTORIAL{}.} nella
directory \texttt{bin/}.

\begin{session}
\begin{alltt}
\dol bin/build
  ...
  ...
Uploading files to /home/mn200/hol/sigobj

Hol built successfully.
\dol
\end{alltt}
\end{session}


\subsection{Sovrascrivere \texttt{smart-configure}}
\label{sec:editting-configure}

Se \texttt{smart-configure} non � in rado di indovinare i valori corretti per i
vari parametri (\texttt{holdir}, \texttt{OS} \etc) allora si pu�
creare un file richiamato per fornire i valori corretti. Con Moscow~ML, questo
dovrebbe essere \texttt{config-override} nella directory radice della distribuzione
HOL. Con Poly/ML, questo dovrebbe essere \texttt{poly-includes.ML}
nella directory \texttt{tools-poly}. In questo file, si specifichi il
valore corretto per il parametro appropriato fornendo un binding ML
per esso. A tutte le variabili eccetto \texttt{dynlib\_available} deve essere data
una stringa come valore possibile, mentre \texttt{dynlib\_available} deve essere
\texttt{true} o \texttt{false}, Cos�, si potrebbe scrivere

\begin{session}
\begin{verbatim}
val OS = "unix";
val holdir = "/local/scratch/myholdir";
val dynlib_available = false;
\end{verbatim}
\end{session}

Il file \texttt{config-override} ha bisogno di fornire soli i valori per quelle
variabili che devono essere sovra scritte.

Con questo file nel posto corretto, il programma \texttt{smart-configure} user�
i valori li specificati piuttosto che quelli che tenta di calcolare
da solo. Il valore dato per la variabile \texttt{OS} deve essere uno tra
\texttt{"unix"}, \texttt{"linux"}, \texttt{"solaris"},
\texttt{"macosx"} or \texttt{"winNT"}\footnote{La stringa
  \texttt{"winNT"} � usata per sistemi operativi Microsoft Windows
	recenti almeno quanto Windows~NT. Questi includono
	Windows~2000, XP and Vista.}.

In circostanze estreme � possibile modificare il file
\texttt{tools/configure.sml} da soli per impostare le variabili di configurazione
direttamente. (Se si sta usando Poly/ML, bisogna invece
modificare \texttt{tools-poly/configure.sml}.) All'inizio di questo file
sono presenti varie dichiarazioni SML incomplete, ma commentate.
Sar� necessario decommentare questa sezione (rimuovendo i segni \texttt{(*} e
\texttt{*)}), e fornire valori appropriati. Tutte le stringhe devono
essere incluse tra doppie virgolette.

Il valore \texttt{holdir} deve essere il nome della directory principale
elencata nella prima sessione di sopra. Il valore \texttt{OS} dovrebbe essere
una delle stringhe specificate nel commento che accompagna quella sezione.

Quando si lavora con Moscow~ML, il valore \texttt{mosmldir} deve essere il
nome della directory che contiene i binari di Moscow~ML
(\texttt{mosmlc}, \texttt{mosml}, \texttt{mosmllex} ecc.). Quando
si lavora con Poly/ML, la stringa \texttt{poly} deve essere il percorso
all'eseguibile \texttt{poly} che inizia una sessione \ML{} interattiva.
Il valore \texttt{polymllibdir} deve essere un percorso a una directory che contiene
il file \texttt{libpolymain.a}.

I valori successivi (\texttt{CC} e \texttt{GNUMAKE}) sono necessari per
componenti del sistema ``opzionali''. La prima da una stringa
appropriata per invocare il compilatore C di sistema, e la seconda
specifica un programma \textsf{make}.

Dopo la modifica, le righe di \texttt{tools/configure.sml} di sopra saranno
qualcosa di simile a :

\begin{session}
\begin{alltt}
\dol more configure.sml
  ...
val mosmldir = "/home/mn200/mosml";
val holdir   = "/home/mn200/hol";
val OS       = "linux"       (* Operating system; choices are:
                                "linux", "solaris", "unix", "winNT" *)

val CC       = "gcc";     (* C compiler (for building quote filter)        *)
val GNUMAKE  = "gnumake"; (* for robdd library                             *)
  ...
\dol
\end{alltt}
\end{session}

\noindent Ora, a questo livello (nella directory \texttt{tools} o
\texttt{tools-poly}) o al livello di sopra, bisogna passare lo script
\texttt{configure.sml} all'interprete \ML{} (\ie,
\texttt{mosml} o \texttt{poly}).  Per esempio,

\begin{session}
\begin{alltt}
\dol mosml < tools/configure.sml
Moscow ML version 2.01 (January 2004)
Enter `quit();' to quit.
- > val mosmldir = "/home/mn200/mosml" : string
  val holdir = "/home/mn200/hol" : string
  val OS = "linux" : string
- > val CC = "gcc" : string
  ...
Beginning configuration.
- Making bin/Holmake.
  ...
Making bin/build.
- Making hol98-mode.el (for Emacs)
- Setting up the standard prelude.
- Setting up src/0/Globals.sml.
- Generating bin/hol.
- Generating bin/hol.noquote.
- Attempting to compile quote filter ... successful.
- Setting up the muddy library Makefile.
- Setting up the help Makefile.
-
Finished configuration!
-
\dol
\end{alltt}
\end{session}



%%% Local Variables:
%%% mode: latex
%%% TeX-master: "tutorial"
%%% End:
