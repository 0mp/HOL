\chapter{Introduction to ML}
\label{ML}

This chapter is a brief introduction to the meta-language \ML.  The
aim is just to give a feel for what it is like to interact with the
language.  A more detailed introduction can be found in numerous
textbooks and web-pages; see for example the list of resources on the
\url{http://www.dina.kvl.dk/\~{}sestoft/mosml.html}{MoscowML home-page},
or the
\url{http://src.doc.ic.ac.uk/usenet/news-faqs/comp.lang.ml/}{\texttt{comp.lang.ml} FAQ}.

\section{How to interact with ML}

\ML{} is an interactive programming language like Lisp. At top level
one can evaluate expressions and perform declarations. The former
results in the expression's value and type being printed, the latter
in a value being bound to a name.

A standard way to interact with \ML{} is to configure the workstation
screen so that there are two windows:
\begin{myenumerate}
\item An editor window into which \ML{} commands are initially typed
  and recorded.
\item A shell window (or non-Unix equivalent) which is used to
  evaluate the commands.
\end{myenumerate}

\noindent
A common way to achieve this is to work inside \ml{Emacs} with a text
window and a shell window.

After typing a command into the edit (text) window it can be
transferred to the shell and evaluated in \HOL{} by `cut-and-paste'. In
\ml{Emacs} this is done by copying the text into a buffer and then
`yanking' it into the shell. The advantage of working via an editor is
that if the command has an error, then the text can simply be edited
and used again; it also records the commands in a file which can then
be used again (via a batch load) later. In \ml{Emacs}, the shell
window also records the session, including both input from the user
and the system's response. The sessions in this tutorial were produced
this way. These sessions are split into segments displayed in boxes
with a number in their top right hand corner (to indicate their
position in the complete session).

The interactions in these boxes should be understood as occurring in
sequence.  For example, variable bindings made in earlier boxes are
assumed to persist to later ones.  To enter the \HOL{} system one types
{\small\verb|hol|} or {\small\verb|hol.unquote|} to Unix, possibly
  preceded by path information if the \HOL{} system's \texttt{bin}
  directory is not in one's path.  The \HOL{} system then prints a
  sign-on message and puts one into \ML.  The \ML{} prompt is
  {\small\verb|-|}, so lines beginning with {\small\verb|-|} are typed
  by the user and other lines are the system's responses.

  Here, as elsewhere in the \TUTORIAL{}, we will be assuming use of
  {\small\verb|hol.unquote|}.

\setcounter{sessioncount}{0}
\begin{session}\begin{alltt}
\$ bin/hol.unquote
[loading HOL power tools ************* ]
-----------------------------------------------------------------
       HOL98 [\holnsversion (built Fri Apr 12 15:34:35 2002)]

       For introductory HOL help, type: help "hol";
-----------------------------------------------------------------

[closing file "/local/scratch/mn200/Work/hol98/tools/end-init-boss.sml"]
- 1 :: [2,3,4,5];
> val it = [1, 2, 3, 4, 5] : int list
\end{alltt}
\end{session}

The \ML{} expression {\small\verb|1 :: [2,3,4,5]|} has the form $e_1\
op\ e_2$ where $e_1$ is the expression {\small\verb|1|} (whose value
is the integer $1$), $e_2$ is the expression {\small\verb|[2,3,4,5]|}
(whose value is a list of four integers) and $op$ is the infixed
operator `{\small\verb|::|}' which is like Lisp's {\it cons} function.
Other list processing functions include {\small\verb|hd|} ($car$ in
Lisp), {\small\verb|tl|} ($cdr$ in Lisp) and {\small\verb|null|}
($null$ in Lisp).  The semicolon `{\small\verb|;|}' terminates a
top-level phrase.  The system's response is shown on the line starting
with the {\small\verb|>|} prompt.  It consists of the value of the
expression followed, after a colon, by its type. The \ML{} type checker
infers the type of expressions using methods invented by Robin Milner
\cite{Milner-types}. The type {\small\verb|int list|} is the type of
`lists of integers'; {\small\verb|list|} is a unary type operator.
The type system of \ML{} is very similar to the type system of the
\HOL{} logic which is explained in Chapter~\ref{HOLlogic}.

The value of the last expression evaluated at top-level in \ML{} is always
remembered in a variable called {\small\verb|it|}.

\begin{session}
\begin{verbatim}
- val l = it;
> val l = [1, 2, 3, 4, 5] : int list

- tl l;
> val it = [2, 3, 4, 5] : int list

- hd it;
> val it = 2 : int

- tl(tl(tl(tl(tl l))));
> val it = [] : int list
\end{verbatim}
\end{session}

Following standard $\lambda$-calculus usage, the application of a
function $f$ to an argument $x$ can be written without brackets as $f\
x$ (although the more conventional
$f${\small\verb|(|}$x${\small\verb|)|} is also allowed).  The
expression $f\ x_1\ x_2\ \cdots\ x_n$ abbreviates the less
intelligible expression {\small\verb|(|}$\cdots${\small\verb|((|}$f\ x_1$%
{\small\verb|)|}$x_2${\small\verb|)|}$\cdots${\small\verb|)|}$x_n$
(function application is left associative).

Declarations have the form {\small\verb|val |}$x_1${\small\verb|=|}$e_1${\small\verb| and |}$\cdots
${\small\verb| and |}$x_n${\small\verb|=|}$e_n$ and result in the value of
each expression $e_i$ being bound to the name $x_i$.

\begin{session}
\begin{verbatim}
- val l1 = [1,2,3] and l2 = ["a","b","c"];
> val l1 = [1, 2, 3] : int list
  val l2 = ["a", "b", "c"] : string list
\end{verbatim}
\end{session}

\ML{} expressions like {\small\verb|"a"|}, {\small\verb|"b"|},
{\small\verb|"foo"|} \etc\ are {\it strings\/} and have type
{\small\verb|string|}. Any sequence of {\small ASCII} characters can
be written between the quotes.\footnote{Newlines must be written as
  \ml{$\backslash$n}, and quotes as \ml{$\backslash$"}.} The function
{\small\verb|explode|} splits a string into a list of single
characters, which are written like single character strings, with a
{\small\verb|#|} character prepended.

\begin{session}
\begin{verbatim}
- explode "a b c";
> val it = [#"a", #" ", #"b", #" ", #"c"] : char list
\end{verbatim}
\end{session}

An expression of the form
{\small\verb|(|}$e_1${\small\verb|,|}$e_2${\small\verb|)|} evaluates
to a pair of the values of $e_1$ and $e_2$. If $e_1$ has type
$\sigma_1$ and $e_2$ has type $\sigma_2$ then
{\small\verb|(|}$e_1${\small\verb|,|}$e_2${\small\verb|)|} has type
$\sigma_1${\small\verb|*|}$\sigma_2$.  The first and second components
of a pair can be extracted with the \ML{} functions {\small\verb|#1|}
and {\small\verb|#2|} respectively.  If a tuple has more than two
components, its $n$-th component can be extracted with a function
{\small\verb|#|$n$}.

The values {\small\verb|(1,2,3)|}, {\small\verb|(1,(2,3))|} and
{\small\verb|((1,2), 3)|} are all distinct and have types
\linebreak{} {\small\verb|int * int * int|}, {\small\verb|int * (int * int)|} and
{\small\verb|(int * int) * int|} respectively.

\begin{session}
\begin{verbatim}
- val triple1 = (1,true,"abc");
> val triple1 = (1, true, "abc") : int * bool * string
- #2 triple1;
> val it = true : bool

- val triple2 = (1, (true, "abc"));
> val triple2 = (1, (true, "abc")) : int * (bool * string)
- #2 triple2;;
> val it = (true, "abc") : bool * string
\end{verbatim}
\end{session}

\noindent The \ML{} expressions {\small\verb|true|} and {\small\verb|false|}
denote the two truth values of type {\small\verb|bool|}.

\ML{} types can contain the {\it type variables\/} {\small\verb|'a|},
{\small\verb|'b|}, {\small\verb|'c|}, \etc\ Such types are called {\it
polymorphic\/}. A function with a polymorphic type should be thought of as
possessing all the types obtainable by replacing type variables by types.
This is illustrated below with the function {\small\verb|zip|}.

Functions are defined with declarations of the form {\small\verb|fun|}$\ f\
v_1\ \ldots\ v_n$ \ml{=} $e$ where each $v_i$ is either a variable or a pattern
built out of variables.

The function {\small\verb|zip|}, below, converts a pair of lists
{\small\verb|([|}$x_1${\small\verb|,|}$\ldots${\small\verb|,|}$x_n$%
{\small\verb|], [|}$y_1${\small\verb|,|}$\ldots${\small\verb|,|}$y_n$%
{\small\verb|])|} to a list of pairs
{\small\verb|[(|}$x_1${\small\verb|,|}$y_1${\small\verb|),|}$\ldots$%
{\small\verb|,(|}$x_n${\small\verb|,|}$y_n${\small\verb|)]|}.

\begin{session}
\begin{verbatim}
- fun zip(l1,l2) =
    if null l1 orelse null l2 then []
    else (hd l1,hd l2) :: zip(tl l1,tl l2);
> val zip = fn : 'a list * 'b list -> ('a * 'b) list

- zip([1,2,3],["a","b","c"]);
> val it = [(1, "a"), (2, "b"), (3, "c")] : (int * string) list
\end{verbatim}
\end{session}

Functions may be {\it curried\/}, \ie\ take their arguments `one at a time'
instead of as a tuple.  This is illustrated with the function
{\small\verb|curried_zip|} below:

\begin{session}
\begin{verbatim}
- fun curried_zip l1 l2 = zip(l1,l2);
> val curried_zip = fn : 'a list -> 'b list -> ('a * 'b) list

- fun zip_num l2 = curried_zip [0,1,2] l2;
> val zip_num = fn : 'a list -> (int * 'a) list

- zip_num ["a","b","c"];
> val it = [(0, "a"), (1, "b"), (2, "c")] : (int * string) list
\end{verbatim}
\end{session}

The evaluation of an expression either {\it succeeds\/} or {\it
  fails\/}.  In the former case, the evaluation returns a value; in
the latter case the evaluation is aborted and an \emph{exception} is
raised.  This exception passed to whatever invoked the evaluation.
This context can either propagate the failure (this is the default) or
it can {\it trap\/} it. These two possibilities are illustrated below.
An exception trap is an expression of the form
$e_1${\small\verb| handle _ => |}$e_2$. An expression of this form is
evaluated by first evaluating $e_1$. If the evaluation succeeds (\ie\
doesn't fail) then the value of the whole expression is the value of
$e_1$.  If the evaluation of $e_1$ raises an exception, then the value
of the whole is obtained by evaluating $e_2$.\footnote{This
  description of exception handling is actually a gross simplification
  of the way exceptions can be handled in \ML{}; consult a proper text
  for a better explanation.}

\begin{session}
\begin{verbatim}
- 3 div 0;
! Uncaught exception:
! Div

- 3 div 0 handle _ => 0;
> val it = 0 : int
\end{verbatim}
\end{session}

The sessions above are enough to give a feel for \ML.  In the next
chapter, the logic supported by the \HOL{} system (higher order logic)
will be introduced, together with the tools in \ML{} for manipulating
it.

%%% Local Variables:
%%% mode: latex
%%% TeX-master: "tutorial"
%%% End:
