\chapter{Proof Tools: Propositional Logic}
\label{chap:proof-tools}

\newcommand{\naive}{na\"\i{}ve}

Users of \HOL{} can create their own theorem proving tools by
combining predefined rules and tactics. The \ML{} type-discipline
ensures that only logically sound methods can be used to create values
of type \ml{thm}.  In this chapter, a real example is described.

Two implementations of the tool are given to illustrate various styles
of proof programming. The first implementation is the obvious one, but
is inefficient because of the `brute force' method used. The second
implementation attempts to be a great deal more intelligent.
Extensions to the tools to allow more general applicability are also
discussed.

The problem to be solved is that of deciding the truth of a closed
formula of propositional logic.  Such a formula has the general form
\[
\begin{array}{ccl}
\varphi & ::= & v \;|\;\neg\varphi\;|\;\varphi
\land \varphi \;|\; \varphi \lor \varphi \;|\; \varphi \Rightarrow
\varphi\;|\;\varphi = \varphi
\\[1ex]
\mathit{formula} &::= & \forall \vec{v}. \;\varphi
\end{array}
\]
where the variables $v$ are all of boolean type, and where the
universal quantification at the outermost level captures all of the
free variables.

\section{Method 1: Truth Tables}

\setcounter{sessioncount}{0}

The first method to be implemented is the brute force method of trying
all possible boolean combinations.  This approach's only real virtue
is that it is exceptionally easy to implement.  First we will prove
the motivating theorem:
\begin{hol}
\begin{verbatim}
   val FORALL_BOOL = prove(
     ``(!v. P v) = P T /\ P F``,
     SRW_TAC [][EQ_IMP_THM] THEN Cases_on `v` THEN SRW_TAC [][]);
\end{verbatim}
\end{hol}
The proof proceeds by splitting the goal into two halves, showing
\[
(\forall v. \;P(v))\Rightarrow P(\top) \land P(\bot)
\]
(which goal is automatically shown by the simplifier), and
\[
P(\top) \land P(\bot) \Rightarrow P(v)
\]
for an arbitrary boolean variable $v$.  After case-splitting on $v$,
the assumptions are then enough to show the goal.  (This theorem is
actually already proved in the theory \theoryimp{bool}.)

The next, and final, step is to rewrite with this theorem:
\begin{hol}
\begin{verbatim}
   val tautDP = SIMP_CONV bool_ss [FORALL_BOOL]
\end{verbatim}
\end{hol}

This enables the following

\begin{session}
\begin{verbatim}
- tautDP ``!p q. p /\ q /\ ~p``;
> val it = |- (!p q. p /\ q /\ ~p) = F : thm

- tautDP ``!p. p \/ ~p``
> val it = |- (!p. p \/ ~p) = T : thm
\end{verbatim}
\end{session}
and even the marginally more intimidating
\begin{session}
\begin{verbatim}
- time tautDP
     ``!p q c a. ~(((~a \/ p /\ ~q \/ ~p /\ q) /\
                    (~(p /\ ~q \/ ~p /\ q) \/ a)) /\
                   (~c \/ p /\ q) /\ (~(p /\ q) \/ c)) \/
                 ~(p /\ q) \/ c /\ ~a``;
runtime: 0.147s,    gctime: 0.012s,     systime: 0.000s.
> val it =
    |- (!p q c a.
          ~(((~a \/ p /\ ~q \/ ~p /\ q) /\ (~(p /\ ~q \/ ~p /\ q) \/ a)) /\
            (~c \/ p /\ q) /\ (~(p /\ q) \/ c)) \/ ~(p /\ q) \/ c /\ ~a) =
       T : thm
\end{verbatim}
\end{session}

This is a dreadful algorithm for solving this problem.  The system's
built-in function, \ml{tautLib.TAUT\_CONV} solves the problem above
ten times faster, even though it is also using a truth-table
technique.\footnote{The main difference is that \ml{TAUT\_CONV}
  simplifies the body of the universally quantified formula after each
  case-split.  Our function does all of the case-splits and then
  simplifies.  The simplifier, which works over a term top-down, can't
  implement \ml{TAUT\_CONV}'s algorithm directly.}  The only real
merit in this solution is that it took one line to write.  This is a
general illustration of the truth that \HOL{}'s high-level tools,
particularly the simplifier, can provide fast prototypes for a variety
of proof tasks.

\section{Method 2: the DPLL Algorithm}

The Davis-Putnam-Loveland-Logemann method~\cite{DPLL-paper} for
deciding the satisfiability of propositional formulas in CNF
(Conjunctive Normal Form) is a powerful technique, still used in
state-of-the-art solvers today.  If we strip the universal quantifiers
from our input formulas, our task can be seen as determining the
validity of a propositional formula.  Testing the negation of such a
formula for satisfiability is a test for validity: if the formula's
negation is satisfiable, then it is not valid (the satisfying
assignment will make the original false); if the formula's negation is
unsatisfiable, then the formula is valid (no assignment can make it
false).

\smallskip
\noindent
(The source code for this example is available in the file \texttt{examples/dpll.sml}.)

\subsection*{Preliminaries}

To begin, assume that we have code already to convert arbitrary
formulas into CNF{}, and to then decide the satisfiability of these
formulas.  Assume further that if the input to the latter procedure is
unsatisfiable, then it will return with a theorem of the form
\[ \vdash \varphi = \holtxt{F}
\]
or if it is satisfiable, then it will return a satisfying assignment,
a map from variables to booleans.   This map will be a function from
\HOL{} variables to one of the \HOL{} terms \holtxt{T} or \holtxt{F}.
Thus, we will assume
\begin{hol}
\begin{verbatim}
   datatype result = Unsat of thm | Sat of term -> term
   val toCNF : term -> thm
   val DPLL : term -> result
\end{verbatim}
\end{hol}
(The theorem returned by \ml{toCNF} will equate the input term to
another in CNF{}.)

\smallskip
\noindent
Before looking into implementing these functions, we will need to
consider
\begin{itemize}
\item how to transform our inputs to suit the function; and
\item how to use the outputs from the functions to produce our desired
  results
\end{itemize}

We are assuming our input is a universally quantified formula.  Both
the CNF and DPLL procedures expect formulas without quantifiers.  We
also want to pass these procedures the negation of the original
formula.  Both of the required term manipulations required can be done
by functions found in the structure \ml{boolSyntax}.  (In general,
important theories (such as \theoryimp{bool}) are accompanied by
\ml{Syntax} modules containing functions for manipulating the
term-forms associated with that theory.)

In this case we need the functions
\begin{hol}
\begin{verbatim}
   strip_forall : term -> term list * term
   mk_neg       : term -> term
\end{verbatim}
\end{hol}
The function \ml{strip\_forall} strips a term of all its outermost
universal quantifications, returning the list of variables stripped
and the body of the quantification.  The function \ml{mk\_neg} takes a
term of type \holtxt{bool} and returns the term corresponding to its
negation.

Using these functions, it is easy to see how we will be able to take
$\forall\vec{v}.\;\varphi$ as input, and pass the term $\neg\varphi$
to the function \ml{toCNF}.  A more significant question is how to
use the results of these calls.   The call to \ml{toCNF} will return a
theorem
\[
\vdash \neg\varphi = \varphi'
\]
The formula $\varphi'$ is what will then be passed to \ml{DPLL}.  (We
can extract it by using the \ml{concl} and \ml{rhs} functions.) If
\ml{DPLL} returns the theorem $\vdash \varphi' = \holtxt{F}$, an
application of \ml{TRANS} to this and the theorem displayed above will
derive the formula $\vdash \neg\varphi = F$.  In order to derive the
final result, we will need to turn this into $\vdash\varphi$.  This is
best done by proving a bespoke theorem embodying the equality (there
isn't one such already in the system):
\begin{hol}
\begin{verbatim}
   val NEG_EQ_F = prove(``(~p = F) = p``, REWRITE_TAC []);
\end{verbatim}
\end{hol}
To turn $\vdash \varphi$ into $\vdash (\forall
\vec{v}.\;\varphi) = \holtxt{T}$, we will perform the following proof:
\[
\infer[\texttt{\scriptsize EQT\_INTRO}]{
  \vdash (\forall \vec{v}.\;\varphi) = \holtxt{T}}{
  \infer[\texttt{\scriptsize GENL}(\vec{v})]{\vdash \forall \vec{v}.\;\varphi}{
    \vdash \varphi
  }
}
\]
The other possibility is that \ml{DPLL} will return a satisfying
assignment demonstrating that $\varphi'$ is satisfiable.  If this is
the case, we want to show that $\forall\vec{v}.\;\varphi$ is false.
We can do this by assuming this formula, and then specialising the
universally quantified variables in line with the provided map.  In
this way, it will be possible to produce the theorem
\[
\forall \vec{v}.\;\varphi \vdash \varphi[\vec{v} := \mbox{\emph{satisfying
  assignment}}]
\]
Because there are no free variables in $\forall\vec{v}.\;\varphi$, the
substitution will produce a completely ground boolean formula.  This
will straightforwardly rewrite to \holtxt{F} (if the assignment
makes $\neg\varphi$ true, it must make $\varphi$ false).  Turning
$\phi\vdash \holtxt{F}$ into $\vdash \phi = \holtxt{F}$ is a matter of
calling \ml{DISCH} and then rewriting with the built-in theorem
\ml{IMP\_F\_EQ\_F}:
\[
\vdash \forall t.\;t \Rightarrow \holtxt{F} = (t = \holtxt{F})
\]
Putting all of the above together, we can write our wrapper function,
which we will call \ml{DPLL\_UNIV}, with the \ml{UNIV} suffix
reminding us that the input must be universally quantified.
\begin{hol}
\begin{verbatim}
   fun DPLL_UNIV t = let
     val (vs, phi) = strip_forall t
     val cnf_eqn = toCNF (mk_neg phi)
     val phi' = rhs (concl cnf_eqn)
   in
     case DPLL phi' of
       Unsat phi'_eq_F => let
         val negphi_eq_F = TRANS cnf_eqn phi'_eq_F
         val phi_thm = CONV_RULE (REWR_CONV NEG_EQ_F) negphi_eq_F
       in
         EQT_INTRO (GENL vs phi_thm)
       end
     | Sat f => let
         val t_assumed = ASSUME t
         fun spec th =
             spec (SPEC (f (#1 (dest_forall (concl th)))) th)
             handle HOL_ERR _ => REWRITE_RULE [] th
       in
         CONV_RULE (REWR_CONV IMP_F_EQ_F) (DISCH t (spec t_assumed))
       end
   end
\end{verbatim}
\end{hol}

The auxiliary function \ml{spec} that is used in the second case
relies on the fact that \ml{dest\_forall} will raise a \ml{HOL\_ERR}
exception if the term it is applied to is not universally quantified.
When \ml{spec}'s argument is not universally quantified, this means
that the recursion has bottomed out, and all of the original formula's
universal variables have been specialised.  Then the resulting formula
can be rewritten to false (\ml{REWRITE\_RULE}'s built-in rewrites will
handle all of the necessary cases).

The \ml{DPLL\_UNIV} function also uses \ml{REWR\_CONV} in two places.
The \ml{REWR\_CONV} function applies a single (first-order) rewrite at
the top of a term.  These uses of \ml{REWR\_CONV} are done within
calls to the \ml{CONV\_RULE} function.  This lifts a conversion $c$ (a
function taking a term $t$ and producing a theorem $\vdash t = t'$),
so that $\ml{CONV\_RULE}\;c$ takes the theorem $\vdash t$ to $\vdash t'$.


\subsection{Conversion to Conjunctive Normal Form}
\label{sec:conv-conj-norm}

A formula in Conjunctive Normal Form is a conjunction of disjunctions
of literals (either variables, or negated variables).  It is possible
to convert formulas of the form we are expecting into CNF by simply
rewriting with the following theorems
\begin{eqnarray*}
\neg (\phi \land \psi) &=& \neg\phi \lor \neg\psi\\
\neg (\phi \lor \psi) &=& \neg\phi \land \neg\psi\\
\phi \lor (\psi \land \xi) &=& (\phi \lor \psi) \land (\phi \lor \xi)\\
(\psi \land \xi)\lor\phi \ &=& (\phi \lor \psi) \land (\phi \lor
\xi)\\[1ex]
\phi \Rightarrow\psi &=& \neg\phi \lor \psi\\
(\phi = \psi) &=& (\phi \Rightarrow \psi) \land (\psi \Rightarrow
\phi)
\end{eqnarray*}
Unfortunately, using these theorems as rewrites can result in an
exponential increase in the size of a formula.  (Consider using them
to convert an input in Disjunctive Normal Form, a disjunction
of conjunctions of literals, into CNF{}.)

A better approach is to convert to what is known as ``definitional
CNF''. \HOL{} includes functions to do this in the structure
\ml{defCNF}.  Unfortunately, this approach adds extra, existential,
quantifiers to the formula.  For example
\begin{session}
\begin{verbatim}
- defCNF.DEF_CNF_CONV ``p \/ (q /\ r)``;
> val it =
    |- p \/ q /\ r =
       ?x. (x \/ ~q \/ ~r) /\ (r \/ ~x) /\ (q \/ ~x) /\ (p \/ x) : thm
\end{verbatim}
\end{session}
Under the existentially-bound \holtxt{x}, the code has produced a
formula in CNF{}.  With an example this small, the formula is actually
bigger than that produced by the \naive{} translation, but with more
realistic examples, the difference quickly becomes significant.  The
last example used with \ml{tautDP} is 20 times bigger when translated
\naive{}ly than when using \ml{defCNF}, and the translation takes 150
times longer to perform.

But what of these extra existentially quantified variables?  In fact,
we can ignore the quantification when calling the core DPLL procedure.
If we pass the unquantified body to DPLL, we will either get back an
unsatisfiable verdict of the form $\vdash \varphi' = \holtxt{F}$, or a
satisfying assignment for all of the free variables.  If the latter
occurs, the same satisfying assignment will also satisfy the
original.  If the former, we will perform the following proof
\[
\infer{\vdash (\exists \vec{x}.\;\varphi') = \holtxt{F}}{
  \infer{\vdash (\exists \vec{x}.\;\varphi') \Rightarrow \holtxt{F}}{
    \infer{\vdash\forall \vec{x}.\;\varphi' \Rightarrow \holtxt{F}}{
      \infer{\vdash\varphi' \Rightarrow \holtxt{F}}{
        \vdash \varphi' = \holtxt{F}
      }
    }
  }
}
\]
producing a theorem of the form expected by our \ml{wrapper}
function.

In fact, there is an alternative function in the \ml{defCNF} API that
we will use in preference to \ml{DEF\_CNF\_CONV}.   The problem with
\ml{DEF\_CNF\_CONV} is that it can produce a big quantification,
involving lots of variables.  We will rather use
\ml{DEF\_CNF\_VECTOR\_CONV}.  Instead of output of the form
\[
\vdash \varphi = (\exists \vec{x}.\; \varphi')
\]
this second function produces
\[
\vdash \varphi = (\exists (v : \textsf{num} \rightarrow
\textsf{bool}).\; \varphi')
\]
where the individual variables $x_i$ of the first formula are replaced
by calls to the $v$ function $v(i)$, and there is just one quantified
variable, $v$.  This variation will not affect the operation of the
proof sketched above.  And as long as we don't require literals to be
variables or their negations, but also allow them to be terms of
the form $v(i)$ and $\neg v(i)$ as well, then the action of the DPLL
procedure on the formula $\varphi'$ won't be affected either.

Unfortunately for uniformity, in simple cases, the definitional CNF
conversion functions may not result in any existential
quantifications at all.  This makes our implementation of \ml{DPLL}
somewhat more complicated.  We calculate a \ml{body} variable that
will be passed onto the \ml{CoreDPLL} function, as well as a
\ml{transform} function that will transform an unsatisfiability result
into something of the desired form.  If the result of conversion to
CNF produces an existential quantification, we use the proof sketched
above.  Otherwise, the transformation can be the identity function,
\ml{I}:
\begin{hol}
\begin{verbatim}
   fun DPLL t = let
     val (transform, body) = let
       val (vector, body) = dest_exists t
       fun transform body_eq_F = let
         val body_imp_F = CONV_RULE (REWR_CONV (GSYM IMP_F_EQ_F)) body_eq_F
         val fa_body_imp_F = GEN vector body_imp_F
         val ex_body_imp_F = CONV_RULE FORALL_IMP_CONV fa_body_imp_F
       in
         CONV_RULE (REWR_CONV IMP_F_EQ_F) ex_body_imp_F
       end
     in
       (transform, body)
     end handle HOL_ERR _ => (I, t)
   in
     case CoreDPLL body of
       Unsat body_eq_F => Unsat (transform body_eq_F)
     | x => x
   end
\end{verbatim}
\end{hol}
where we have still to implement the core DPLL procedure (called
\ml{CoreDPLL} above).  The above code uses \ml{REWR\_CONV} with the
\ml{IMP\_F\_EQ\_F} theorem to affect two of the proof's
transformations.  The \ml{GSYM} function is used to flip the
orientation a theorem's top-level equalities.  Finally, the
\ml{FORALL\_IMP\_CONV} conversion takes a term of the form
\[
\forall x.\;P(x) \Rightarrow Q
\]
and returns the theorem
\[
\vdash (\forall x.\;P(x) \Rightarrow Q) = ((\exists
x.\;P(x))\Rightarrow Q)
\]




\subsection{The Core DPLL Procedure}
\label{sec:dpll-procedure}

The DPLL procedure can be seen as a slight variation on the basic
``truth table'' technique we have already seen.  As with that
procedure, the core operation is a case-split on a boolean variable.
There are two significant differences though: DPLL can be seen as a
search for a satisfying assignment, so that if picking a variable to
have a particular value results in a satisfying assignment, we do not
need to also check what happens if the same variable is given the
opposite truth-value.  Secondly, DPLL takes some care to pick good
variables to split on.  In particular, \emph{unit propagation} is used
to eliminate variables that will not cause branching in the
search-space.

Our implementation of the core DPLL procedure is a function that takes
a term and returns a value of type \ml{result}: either a theorem
equating the original term to false, or a satisfying assignment (in
the form of a function from terms to terms).  As the DPLL search for a
satisfying assignment proceeds, an assignment is incrementally
constructed.  This suggests that the recursive core of our function
will need to take a term (the current formula) and a context (the
current assignment) as parameters.  The assignment can be naturally
represented as a set of equations, where each equation is either $v =
\holtxt{T}$ or $v = \holtxt{F}$.

This suggests that a natural representation for our program state is a
theorem: the hypotheses will represent the assignment, and the
conclusion can be the current formula.  Of course, \HOL{} theorems
can't just be wished into existence.  In this case, we can make
everything sound by also assuming the initial formula.  Thus, when we
begin our initial state will be $\phi\vdash\phi$.  After splitting on
variable $v$, we will generate two new states
$\phi,(v\!=\!\holtxt{T})\vdash \phi_1$, and
$\phi,(v\!=\!\holtxt{F})\vdash \phi_2$, where the $\phi_i$ are the
result of simplifying $\phi$ under the additional assumption
constraining $v$.

The easiest way to add an assumption to a theorem is to use the
derived rule \ml{ADD\_ASSUM}.  But in this situation, we also want to
simplify the conclusion of the theorem with the same assumption.  This
means that it will be enough to rewrite with the theorem
$\psi\vdash\psi$, where $\psi$ is the new assumption.  The action of
rewriting with such a theorem will cause the new assumption to appear
among the assumptions of the result.

The \ml{casesplit} function is thus:
\begin{hol}
\begin{verbatim}
   fun casesplit v th = let
     val eqT = ASSUME (mk_eq(v, boolSyntax.T))
     val eqF = ASSUME (mk_eq(v, boolSyntax.F))
   in
     (REWRITE_RULE [eqT] th, REWRITE_RULE [eqF] th)
   end
\end{verbatim}
\end{hol}

A case-split can result in a formula that has been rewritten all the
way to true or false.  These are the recursion's base cases. If the
formula has been rewritten to true, then we have found a satisfying
assignment, one that is now stored for us in the hypotheses of the
theorem itself.  The following function, \ml{mk\_satmap}, extracts
those hypotheses into a finite-map, and then returns the lookup
function for that finite-map:
\begin{hol}
\begin{verbatim}
   fun mk_satmap th = let
     val hyps = hypset th
     fun foldthis (t,acc) = let
       val (l,r) = dest_eq t
     in
       Binarymap.insert(acc,l,r)
     end handle HOL_ERR _ => acc
     val fmap = HOLset.foldl foldthis (Binarymap.mkDict Term.compare) hyps
   in
     Sat (fn v => Binarymap.find(fmap,v)
                  handle Binarymap.NotFound => boolSyntax.T)
   end
\end{verbatim}
\end{hol}
The \ml{foldthis} function above adds the equations that are stored as
hypotheses into the finite-map.  The exception handler in
\ml{foldthis} is necessary because one of the hypotheses will be the
original formula.  The exception handler in the function that looks up
variable bindings is necessary because a formula may be reduced to
true without every variable being assigned a value at all.  In this
case, it is irrelevant what value we give to the variable, so we
arbitrarily map such variables to \holtxt{T}.

If the formula has been rewritten to false, then we can just return
this theorem directly.  Such a theorem is not quite in the right form
for the external caller, which is expecting an equation, so if the
final result is of the form $\phi\vdash \holtxt{F}$, we will have to
transform this to $\vdash \phi = \holtxt{F}$.

The next question to address is what to do with the results of
recursive calls.  If a case-split returns a satisfying assignment this
can be returned unchanged.  But if a recursive call returns a theorem
equating the input to false, more needs to be done.  If this is the
first call, then the other branch needs to be checked.  If this also
returns that the theorem is unsatisfiable, we will have two theorems:
\[
\phi_0,\Delta,(v\!=\!\holtxt{T})\vdash \holtxt{F} \qquad
\phi_0,\Delta,(v\!=\!\holtxt{F})\vdash \holtxt{F}
\] where $\phi_0$ is the original formula, $\Delta$ is the rest of the
current assignment, and $v$ is the variable on which a split has just
been performed.  To turn these two theorems into the desired
\[
\phi_0,\Delta\vdash \holtxt{F}
\]
we will use the rule of inference \ml{DISJ\_CASES}:
\[
\infer{\Gamma \cup \Delta_1 \cup \Delta_2 \vdash \phi}{
  \Gamma \vdash \psi \lor \xi &
  \Delta_1 \cup \{\psi\}\vdash \phi &
  \Delta_2 \cup \{\xi\} \vdash \phi
}
\]
and the theorem \ml{BOOL\_CASES\_AX}:
\[
\vdash \forall t.\;(t = \holtxt{T}) \lor (t = \holtxt{F})
\]

We can put these fragments together and write the top-level
\ml{CoreDPLL} function, in Figure~\ref{fig:coredpll}.
\begin{figure}[htbp]
\begin{boxed}
\begin{hol}
\begin{verbatim}
fun CoreDPLL form = let
  val initial_th = ASSUME form
  fun recurse th = let
    val c = concl th
  in
    if c = boolSyntax.T then
      mk_satmap th
    else if c = boolSyntax.F then
      Unsat th
    else let
        val v = find_splitting_var c
        val (l,r) = casesplit v th
      in
        case recurse l of
          Unsat l_false => let
          in
            case recurse r of
              Unsat r_false =>
                Unsat (DISJ_CASES (SPEC v BOOL_CASES_AX) l_false r_false)
            | x => x
          end
        | x => x
      end
  end
in
  case (recurse initial_th) of
    Unsat th => Unsat (CONV_RULE (REWR_CONV IMP_F_EQ_F) (DISCH form th))
  | x => x
end
\end{verbatim}
\end{hol}
\end{boxed}
\caption{The core of the DPLL function}
\label{fig:coredpll}
\end{figure}


All that remains to be done is to figure out which variable to
case-split on.  The most important variables to split on are those
that appear in what are called ``unit clauses'', a clause containing
just one literal.  If there is a unit clause in a formula then it is
of the form
\[
\phi \land v \land \phi'
\]
or
\[
\phi \land \neg v \land \phi'
\]
In either situation, splitting on $v$ will always result in a branch
that evaluates directly to false.  We thus eliminate a variable
without increasing the size of the problem.  The process of
eliminating unit clauses is usually called ``unit propagation''.
Unit propagation is not usually thought of as a case-splitting
operation, but doing it this way makes our code simpler.

If a formula does not include a unit clause, then choice of the next
variable to split on is much more of a black art.  Here we will
implement a very simple choice: to split on the variable that occurs
most often.  Our function \ml{find\_splitting\_var} takes a formula
and returns the variable to split on.
\begin{hol}
\begin{verbatim}
   fun find_splitting_var phi = let
     fun recurse acc [] = getBiggest acc
       | recurse acc (c::cs) = let
           val ds = strip_disj c
         in
           case ds of
             [lit] => (dest_neg lit handle HOL_ERR _ => lit)
           | _ => recurse (count_vars ds acc) cs
         end
   in
     recurse (Binarymap.mkDict Term.compare) (strip_conj phi)
   end
\end{verbatim}
\end{hol}
This function works by handing a list of clauses to the inner
\ml{recurse} function.  This strips each clause apart in turn.  If a
clause has only one disjunct it is a unit-clause and the variable can
be returned directly.  Otherwise, the variables in the clause are
counted and added to the accumulating map by \ml{count\_vars}, and the
recursion can continue.

The \ml{count\_vars} function has the following implementation:
\begin{hol}
\begin{verbatim}
   fun count_vars ds acc =
     case ds of
       [] => acc
     | lit::lits => let
         val v = dest_neg lit handle HOL_ERR _ => lit
       in
         case Binarymap.peek (acc, v) of
           NONE => count_vars lits (Binarymap.insert(acc,v,1))
         | SOME n => count_vars lits (Binarymap.insert(acc,v,n + 1))
       end
\end{verbatim}
\end{hol}

The use of a binary tree to store variable data makes it efficient to
update the data as it is being collected.  Extracting the variable
with the largest count is then a linear scan of the tree, which we can
do with the \ml{foldl} function:
\begin{hol}
\begin{verbatim}
   fun getBiggest acc =
     #1 (Binarymap.foldl(fn (v,cnt,a as (bestv,bestcnt)) =>
                            if cnt > bestcnt then (v,cnt) else a)
                        (boolSyntax.T, 0)
                        acc
\end{verbatim}
\end{hol}

\subsection{Performance}
\label{sec:dpll-performance}

Once inputs get even a little beyond the clearly trivial, the function
we have written (at the top-level, \ml{DPLL\_UNIV}) performs considerably
better than the truth table implementation.  For example, the
generalisation of the following term, with 29 variables, takes
\ml{wrapper} two and a half minutes to demonstrate as a tautology:
\begin{hol}
\begin{verbatim}
   (s0_0 = (x_0 = ~y_0)) /\ (c0_1 = x_0 /\ y_0) /\
   (s0_1 = ((x_1 = ~y_1) = ~c0_1)) /\
   (c0_2 = x_1 /\ y_1 \/ (x_1 \/ y_1) /\ c0_1) /\
   (s0_2 = ((x_2 = ~y_2) = ~c0_2)) /\
   (c0_3 = x_2 /\ y_2 \/ (x_2 \/ y_2) /\ c0_2) /\
   (s1_0 = ~(x_0 = ~y_0)) /\ (c1_1 = x_0 /\ y_0 \/ x_0 \/ y_0) /\
   (s1_1 = ((x_1 = ~y_1) = ~c1_1)) /\
   (c1_2 = x_1 /\ y_1 \/ (x_1 \/ y_1) /\ c1_1) /\
   (s1_2 = ((x_2 = ~y_2) = ~c1_2)) /\
   (c1_3 = x_2 /\ y_2 \/ (x_2 \/ y_2) /\ c1_2) /\
   (c_3 = ~c_0 /\ c0_3 \/ c_0 /\ c1_3) /\
   (s_0 = ~c_0 /\ s0_0 \/ c_0 /\ s1_0) /\
   (s_1 = ~c_0 /\ s0_1 \/ c_0 /\ s1_1) /\
   (s_2 = ~c_0 /\ s0_2 \/ c_0 /\ s1_2) /\ ~c_0 /\
   (s2_0 = (x_0 = ~y_0)) /\ (c2_1 = x_0 /\ y_0) /\
   (s2_1 = ((x_1 = ~y_1) = ~c2_1)) /\
   (c2_2 = x_1 /\ y_1 \/ (x_1 \/ y_1) /\ c2_1) /\
   (s2_2 = ((x_2 = ~y_2) = ~c2_2)) /\
   (c2_3 = x_2 /\ y_2 \/ (x_2 \/ y_2) /\ c2_2) ==>
   (c_3 = c2_3) /\ (s_0 = s2_0) /\ (s_1 = s2_1) /\ (s_2 = s2_2)
\end{verbatim}
\end{hol}
The truth table implementation in \ml{tautLib} takes over 100 times as
long to prove the tautology.  (But if you want real speed, the
\ml{SAT\_PROVE} function in the \ml{HolSatLib} library does the
above in less than a second, by using an external tool to generate the
proof of unsatisfiability, and then translating that proof back into
HOL.)

\section{Extending our Procedure's Applicability}
\label{sec:dpll-applicability-extension}

The function \ml{DPLL\_UNIV} requires its input to be universally
quantified, with all free variables bound, and for each literal to be
a variable or the negation of a variable.  This makes \ml{DPLL\_UNIV}
a little unfriendly when it comes to using it as part of the proof of
a goal.  In this section, we will write one further ``wrapper''
layer to wrap around \ml{DPLL\_UNIV}, producing a tool that can be
applied to many more goals.

\paragraph{Relaxing the Quantification Requirement}  The first step is
to allow formulas that are not closed.  In order to hand on a formula
that \emph{is} closed to \ml{DPLL\_UNIV}, we can simply generalise
over the formula's free variables.   If \ml{DPLL\_UNIV} then says that
the new, ground formula is true, then so too will be the original.  On
the other hand, if \ml{DPLL\_UNIV} says that the ground formula is
false, then we can't conclude anything further and will have to raise
an exception.

Code implementing this is shown below:
\begin{hol}
\begin{verbatim}
   fun nonuniv_wrap t = let
     val fvs = free_vars t
     val gen_t = list_mk_forall(fvs, t)
     val gen_t_eq = DPLL_UNIV gen_t
   in
     if rhs (concl gen_t_eq) = boolSyntax.T then let
         val gen_th = EQT_ELIM gen_t_eq
       in
         EQT_INTRO (SPECL fvs gen_th)
       end
     else
       raise mk_HOL_ERR "dpll" "nonuniv_wrap" "No conclusion"
   end
\end{verbatim}
\end{hol}

\paragraph{Allowing Non-Literal Leaves}
We can do better than \ml{nonuniv\_wrap}: rather than quantifying over
just the free variables (which we have conveniently assumed will only
be boolean), we can turn any leaf part of the term that is not a
variable or a negated variable into a fresh variable.  We first
extract those boolean-valued leaves that are not the constants true or
false.
\begin{hol}
\begin{verbatim}
   fun var_leaves acc t = let
     val (l,r) = dest_conj t handle HOL_ERR _ =>
                 dest_disj t handle HOL_ERR _ =>
                 dest_imp t handle HOL_ERR _ =>
                 dest_bool_eq t
   in
     var_leaves (var_leaves acc l) r
   end handle HOL_ERR _ =>
     if type_of t <> bool then
       raise mk_HOL_ERR "dpll" "var_leaves" "Term not boolean"
     else if t = boolSyntax.T then acc
     else if t = boolSyntax.F then acc
     else HOLset.add(acc, t)
\end{verbatim}
\end{hol}
Note that we haven't explicitly attempted to pull apart boolean
negations (which one might do with \ml{dest\_neg}).  This is because
\ml{dest\_imp} also destructs terms \holtxt{\~{}p}, returning
\holtxt{p} and \holtxt{F} as the antecedent and conclusion.  We have
also used a function \ml{dest\_bool\_eq} designed to pull apart only
those equalities which are over boolean values.  Its definition is
\begin{hol}
\begin{verbatim}
   fun dest_bool_eq t = let
     val (l,r) = dest_eq t
     val _ = type_of l = bool orelse
             raise mk_HOL_ERR "dpll" "dest_bool_eq" "Eq not on bools"
   in
     (l,r)
   end
\end{verbatim}
\end{hol}

Now we can finally write our final \ml{DPLL\_TAUT} function:
\begin{hol}
\begin{verbatim}
   fun DPLL_TAUT tm =
     let val (univs,tm') = strip_forall tm
         val insts = HOLset.listItems (var_leaves empty_tmset tm')
         val vars = map (fn t => genvar bool) insts
         val theta = map2 (curry (op |->)) insts vars
         val tm'' = list_mk_forall (vars,subst theta tm')
     in
         EQT_INTRO (GENL univs
                      (SPECL insts (EQT_ELIM (DPLL_UNIV tm''))))
     end
\end{verbatim}
\end{hol}
Note how this code first pulls off all external universal
quantifications (with \ml{strip\_forall}), and then re-generalises
(with \ml{list\_mk\_forall}).  The calls to \ml{GENL} and \ml{SPECL}
undo these manipulations, but at the level of theorems.  This produces
a theorem equating the original input to true.  (If the input term is
not an instance of a valid propositional formula, the call to
\ml{EQT\_ELIM} will raise an exception.)

\section*{Exercises}

\begin{enumerate}
\item Extend the procedure so that it handles conditional expressions
  (both arms of the terms must be of boolean type).
\end{enumerate}


%%% Local Variables:
%%% mode: latex
%%% TeX-master: "tutorial"
%%% End:
