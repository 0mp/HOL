
\chapter{Example: Packages}\label{chap:package}

This chapter explains existential types and 
packages, which are used for information hiding and modularity in proofs. 
These are very well described in chapter 24 of {\it Types and Programming Languages\/} 
by Benjamin C.~Pierce (MIT Press, 2002), 
and this chapter will draw significantly from Professor Pierce's work.
Packages are a new variety of term, and
just like the new type abstraction terms, require a new
variety of type to serve as their types. In particular, 
where a type abstraction term has a universal type, 
a package term has a new type called an {\it existential type}.

Packages are somewhat similar to objects in 
object-oriented languages like Java, in the sense that the 
internal details of how the object's data is represented are
to some degree hidden from the users of the object.
This is very useful in practice, since it allows the actual
representation of the data within the object to be changed at
some later time, perhaps 
%to a more efficient representation,
for efficiency concerns,
while not disturbing the 
object's appearance to 
external code that uses the object
as a black box.
Packages do not include all of the features or
flexibility of Java objects, omitting for example inheritance
of methods and dynamic dispatch. 
But they are a first step toward organizing
the data structures of a system in an object-oriented way.

There is also some overlap in purpose between packages
and the abstract data types described in chapter \ref{chap:adt};
both shield and abstract away some of the details of internal
data structures, for the purposes of information hiding and modularity.
The main difference between these two approaches
is that abstract data types are actual new
full-weight types introduced into the \HOLW{} logic 
whose properties are truly only partially determined, 
whereas packages are more
light-weight data structures that may be constructed or deconstructed
on the fly as first-class values, and whose internal details are fixed
and real but intentionally obscured, just as an object's signature only reveals
the general patterns of access to the object's internal data,
not the precise shape of how it is actually laid out.
Thus abstract data types are truly abstract; the internal
representation is not simply unknown at present but in fact
is completely unknowable. By contrast, the internal representation
of a package is secret but determined and fixed. 
The particular representation might possibly disclose itself 
through occasions of the use of the package to compute values.

%Packages are terms, whose types are 
%a new variety of type called 
As mentioned earlier, the types of package terms are
existential types. These types are distinct from any other types
that we have described before; for example, there is no overlap between
universal and existential types.
Superficially, these two types look almost identical, except that where 
the universal types use a universal quantification symbol,  
existential types use an existential quantification symbol.
However, despite these superficial parallels, universal and
existential types are not really duals in the close sense that 
normal universal
and existential quantification are duals of each other.
In particular, the ways that packages are first created and then
taken apart and used are more heavy-weight than the ways that
type abstraction terms are created and then used. Not only is
the syntax more cumbersome, the ideas behind existential types
are also somewhat more difficult to grasp at first glance than 
those behind universal types. 
Therefore we take some more time in this chapter to 
%explain them.
%, and 
ease their introduction.

%There are three intentions:
%There are three intended accomplishments:
%The intention is to accomplish three things:
The goals of this chapter are:

\begin{myenumerate}
\item To present how existential types are created and what they mean,
\item To show how packages, terms of existential type, can be created and used,
\item To show how packages can be applied in an object-oriented way.
%\item To practically demonstrate the usefulness of the new ideas of \HOLW{}.
\end{myenumerate}

To some extent packages and existential types have already been demonstrated
in the Appetizers chapter \ref{chap:appetizers}. In this chapter we will examine
them more closely, 
and illustrate some of their interesting aspects.

\section{Existential Types}

An existential type is written as $\mbox{\tt ?}\alpha.\sigma$ or
$\exists\alpha.\sigma$, where the type
variable $\alpha$ may appear freely within the type $\sigma$. 
Such occurrences of $\alpha$ are
considered bound by the existential quantification. 
Therefore, although other type variables
that are free within $\sigma$ are also free type variables of
$\exists\alpha.\sigma$, the type variable $\alpha$ is never a free type
variable of $\exists\alpha.\sigma$.

Notably, $\alpha$ may be of any kind or rank, but $\sigma$ must have a
base kind, just as for universal types.
The kind of $\exists\alpha.\sigma$ itself will be a base kind, where its
rank is determined the same as for universal types:
if $\alpha$ has rank $r_\alpha$ and $\sigma$ has rank $r_\sigma$, then
the rank of $\exists\alpha.\sigma$ will be the maximum of $r_\alpha + 1$
and $r_\sigma$.

To get an intuition as to the meaning of universal and existential types,
a universal type $\forall\alpha.\sigma$
may be thought of as 
%an infinite collection
%of values which are indexed by types of the kind of $\alpha$; 
%for {\it all\/} index types $\sigma_\alpha$, the collection
%contains all values of type $\sigma[\sigma_\alpha / \alpha]$. 
%Alternatively, the universal
%type may be thought of as 
%containing 
%the set of 
%all
%having values which are 
a type whose values are
functions that map any type $\sigma_\alpha$ 
%of the same kind as $\alpha$ 
%(or where 
such that
the kind of $\alpha$ is \texttt{:>=:} the kind of $\sigma_\alpha$
to a term value of the type $\sigma[\sigma_\alpha / \alpha]$.
These are special functions in that 
1) they map from types to term values, not from a value to a value, and
2) the type of the resulting value
depends on the actual type $\sigma_\alpha$ that is input.
Because of this, these functions are called {\it dependent functions}.
%These functions can be understood as {\it dependent functions},
%since the type of their results, $\sigma[\sigma_\alpha / \alpha]$,
%depends on the actual value of their input, $\sigma_\alpha$.

In comparison to this, the meaning of an existential type 
$\exists\alpha.\sigma$
may be thought of as 
%the type containing all values
%of type $\sigma[\sigma_\alpha / \alpha]$ for {\it some\/} type
%$\sigma_\alpha$ 
%of the same kind as $\alpha$.
%Alternatively, the existential type may be thought of as
%the collection of all 
%having values which are
a type whose values are
pairs of a type and a term, 
%$(\sigma_\alpha,\ t : \sigma[\sigma_\alpha / \alpha])$
e.g.~$(\sigma_\alpha, t)$,
where $\sigma_\alpha$ is a type
%with the same kind as $\alpha$ 
%(or where 
such that
the kind of $\alpha$ is \texttt{:>=:} the kind of $\sigma_\alpha$,
and 
%where 
$t$ is a term with the type $\sigma[\sigma_\alpha / \alpha]$.
%for some possible type $\sigma_\alpha$ 
%of the same kind as $\alpha$.
These are special pairs in that 
1) they join a type and a term in a pair, not two terms, and
2) the type of the second element of the pair $t$
depends on the actual type $\sigma_\alpha$ that is the first element
of the pair.
Because of this, these pairs are called {\it dependent pairs}.
%These pairs can be understood as {\it dependent pairs},
%since the type of the term $t$ which is the second element of the pair
%depends on the actual type $\sigma_\alpha$ which is the first element of the pair.
\setcounter{sessioncount}{0}
\begin{session}
\begin{verbatim}
- set_trace "Unicode" 0;
> val it = () : unit
- new_theory "package";
<<HOL message: Created theory "package">>
> val it = () : unit
\end{verbatim}
\end{session}

Syntactically, existential types look very much like universal types, 
except that the {\tt !}~symbol is replaced by 
{\tt ?}~(or $\forall$ by $\exists$).
Just as for universal types, in $\exists\alpha.\sigma$ the
bound variable $\alpha$ has scope over the body, $\sigma$.
The free type variables of an existential type are the free type
variables of the body, minus the bound type variable.
\begin{session}
\begin{verbatim}
- val ety1 = ``:?'a. 'a -> 'a``;
> val ety1 = ``:?'a. 'a -> 'a`` :
  hol_type
  
- val ety1_vars = type_vars ety1;
> val ety1_vars = [] : hol_type list

- val ety2 = ``:?'a. 'a -> 'b``;
> val ety2 = ``:?'a. 'a -> 'b`` :
  hol_type
  
- val ety2_vars = type_vars ety2;
> val ety2_vars = [``:'b``] : hol_type list
\end{verbatim}
\end{session}

\ML{} functions {\tt is\_exist\_type}, {\tt mk\_exist\_type}, {\tt dest\_exist\_type},
etc.~are provided to test, create, or take apart 
%existential 
these types.
\begin{session}
\begin{verbatim}
- is_exist_type ety2;
> val it = true : bool

- val ety2' = mk_exist_type(gamma, gamma --> beta);
> val ety2' = ``:?'c. 'c -> 'b`` :
  hol_type
  
- val check = eq_ty ety2 ety2';
> val check = true : bool

- val (bvar,body) = dest_exist_type ety2;
> val bvar = ``:'a`` : hol_type
  val body = ``:'a -> 'b`` : hol_type
\end{verbatim}
\end{session}

Similarly, {\tt list\_mk\_exist\_type} and {\tt strip\_exist\_type} create or take
apart multiple instances of existential types.
\begin{session}
\begin{verbatim}
- val ety3 = list_mk_exist_type ([alpha,gamma], alpha --> beta --> gamma);
> val ety3 =
    ``:?'a 'c. 'a -> 'b -> 'c``
     : hol_type
     
- val (ety3_bvars,ety3_body) = strip_exist_type ety3;
> val ety3_bvars =
    [``:'a``, ``:'c``] :
  hol_type list
  val ety3_body =
    ``:'a -> 'b -> 'c``
     : hol_type
\end{verbatim}
\end{session}


\section{Packages}

Terms with existential types are called {\it packages}. 
With respect to such terms, there are two key issues: 
how are they created, and how are they used.
Therefore there is a need for a new form to create such pairs, 
and another new form to take them apart.
Remember that intuitively 
a value of an existential type 
(i.e., a package)
is
%we are dealing with
a special 
kind of 
pair of a type and a term,
%Since intuitively we are dealing a special kind of pairs of a type and a term, 
where the type of the term depends upon the type which is the first element of the pair.

In \HOLW{}, a package is constructed using the syntax $$\mathbf{pack}({:}\sigma, t)\ ,$$ 
where {\bf pack} is a reserved keyword, $\sigma$ is a type, and $t$ is a term. 
%Here $\sigma$ may appear freely within the term $t$.
Note the presence of a colon ($:$) before the type $\sigma$ 
to indicate the presence of a type, rather than a term.

In general, this may not be enough information to 
determine 
%the type of 
the package uniquely. In such cases, it suffices to
add a type annotation to the package syntax, as for example 
$$\mathbf{pack}({:}\sigma, t) : \exists\alpha.\sigma'\ ,$$
where the type annotation is an existential type $\exists\alpha.\sigma'$
such that the type of the body $t$ is $\sigma'[\sigma / \alpha]$.
This annotated version will always determine the package uniquely,
so it is advisable to generally include the type annotation if there is any
uncertainty.
%Note that the type $\sigma$ and the type variable $\alpha$ must
%have the same kind.

The following examples are taken from Pierce's book, pages 364-365.
The first two show how the same package text, made from the same
type and term, can have two different existential types,
depending on whether or not a type annotation is provided.
\begin{session}
\begin{verbatim}
- val pkg1 = ``pack (:num, (5, \x:num. SUC x))``;
> val pkg1 = ``pack (:num,(5,(\x. SUC x)))`` :
  term
- val pkg1_ty = type_of pkg1;
> val pkg1_ty =
    ``:?'x. 'x # ('x -> 'x)``
     : hol_type
     
- val pkg2 = ``pack (:num, (5, \x:num. SUC x)) : ?'x. 'x # ('x -> num)``;
> val pkg2 = ``pack (:num,(5,(\x. SUC x)))`` :
  term
- val pkg2_ty = type_of pkg2;
> val pkg2_ty =
    ``:?'x :(ty:1). 'x # ('x -> num)``
     : hol_type
\end{verbatim}
\end{session}

The type of a package $\mathbf{pack}({:}\sigma, t)$
is always an existential type $\exists\alpha.\sigma'$
%where we require 
such
that the type of the body $t$ is $\sigma'[\sigma / \alpha]$.
Remember the kind of $\alpha$ must be \texttt{:>=:} the kind~of~$\sigma$.
In the above two cases, the 
%$\sigma'$s 
two existential types
are either 
{\tt $\exists$'x.'x~\#~('x~->~'x)} or
{\tt $\exists$'x.'x~\#~('x~->~num)}, but in both cases the type of the 
body is $\sigma'[\sigma / \alpha]$ which is {\tt num~\#~(num~->~num)}.
This shows that in general it is a good idea to provide the type annotation,
even though it might not be always necessary.
%Also, every term of existential type is a package constructed
%from some type $\sigma$ and body $t$.

Also, it is entirely possible to make different packages which
have the exact same type. In fact, this is part of the point of 
having packages, because we wish to hide certain information,
in particular the exact representation type which is the
first element of the pair that is the package value.
Here are two different packages with the same type.
\begin{session}
\begin{verbatim}
- val pkg3 = ``pack (:num, 0) : ?'x. 'x``;
> val pkg3 =
    ``pack (:num,(0 :num))`` :
  term
- val pkg3_ty = type_of pkg3;
> val pkg3_ty = ``:?'x. 'x`` :
  hol_type

- val pkg4 = ``pack (:bool, T) : ?'x. 'x``;
> val pkg4 = ``pack (:bool,T)`` : term
- val pkg4_ty = type_of pkg4;
> val pkg4_ty = ``:?'x. 'x`` :
  hol_type

- val check = eq_ty pkg3_ty pkg4_ty;
> val check = true : bool
\end{verbatim}
\end{session}

Here is another example of two packages which are different
internally, but have the same existential type.
We will see an actual use for these packages in what follows.
\begin{session}
\begin{verbatim}
- val pkg5 =
    ``pack (:num, (0, \x:num. SUC x)) : ?'x. 'x # ('x -> num)``;
> val pkg5 =
    ``pack (:num,((0 :num),(\(x :num). SUC x)))``
     : term
- val pkg5_ty = type_of pkg5;
> val pkg5_ty =
    ``:?'x. 'x # ('x -> num)``
     : hol_type

- val pkg6 =
    ``pack (:bool, (T, \x:bool. 0)) : ?'x. 'x # ('x -> num)``;
> val pkg6 =
    ``pack (:bool,(T,(\(x :bool). (0 :num))))``
     : term
- val pkg6_ty = type_of pkg6;
> val pkg6_ty =
    ``:?'x. 'x # ('x -> num)``
     : hol_type

- val check = eq_ty pkg5_ty pkg6_ty;
> val check = true : bool
\end{verbatim}
\end{session}
This has shown how we construct packages. Now we will see how we take them apart.

In \HOLW{}, packages are taken apart (deconstructed) 
%(that is, taken apart and used) 
via the special syntax
$$\mathbf{let}\ ({:}\alpha,x) = p\ \mathbf{in}\ s\ .$$
Here $\alpha$ is a type variable, $x$ is a term variable whose type may mention
$\alpha$, $p$ is a term yielding a package, 
and $s$, the body of the {\bf let}, is a term.
The idea here is that the package $p$ is opened up, and
the pair that is inside the package is bound to 
the pair of $\alpha$ and $x$, 
and then $\alpha$ and $x$ are usable within the body $s$.
Both $\alpha$ and $x$ are bound by this syntax.
The scope of $\alpha$ is $x$ and $s$, while the scope of $x$ is $s$.
We require that the 
types of the free variables of $s$ do not mention the type variable $\alpha$,
and likewise that the type of $s$ itself does not mention the type variable $\alpha$.
This is only sensible, as the scope of $\alpha$ is only the variable $x$
and the body $s$, and
%the type variable 
$\alpha$ has no meaning outside that scope.

This syntax for deconstructing a package is distinguished from the normal
{\bf let} 
syntax for 
%deconstructing a pair, 
pairs,
$\mathbf{let}\ (x,y) = e\ \mathbf{in}\ e'$,
by the presence of the colon just after the left parenthesis.

Here is an example of deconstructing \texttt{pkg5} from above.
\begin{session}
\begin{verbatim}
- val unpkg5 = ``let (:'x, t:'x # ('x -> num)) = ^pkg5
                 in (SND t) (FST t)``;
> val unpkg5 =
    ``let (:'x,(t :'x # ('x -> num))) =
            (pack (:num,((0 :num),(\(x :num). SUC x))))
      in
        SND t (FST t)``
     : term
\end{verbatim}
\end{session}

So now that we can write down how to take a package apart, what does it mean?
The effect of deconstructing a package is given by the following \HOLW{} theorem:
$$ \vdash (\mathbf{let}\ ({:}\alpha,x) = 
                         \mathbf{pack}({:}\sigma, t) % : \exists\alpha.\sigma'
           \ \mathbf{in}\ s)\ =\ 
          (\lambda{:}\alpha. \lambda x. s)\ [{:}\sigma{:}]\ t
$$
The body of the {\bf let}, $s$, is made into a function that expects first a
type argument $\sigma$, which is  bound to the formal type parameter $\alpha$,
and secondly a term argument $t$, which is  bound to the formal term
parameter $x$. Then the body $s$ is executed in this context.
This is called {\it package reduction}, by analogy with beta reduction:
$(\lambda x.s)\ t = s[t / x]$.

To show the meaning of such uses of packages, we define a
simple evaluation tool using the \ML{} functions \texttt{SIMP\_CONV}
and \texttt{srw\_ss}
from the simplification library.
\begin{session}
\begin{verbatim}
- fun eval ths tm = QCONV (SIMP_CONV (srw_ss()) ths) tm;
> val eval = fn : thm list -> term -> thm

- val unpkg5_res = eval [] unpkg5;
<<HOL message: Initialising SRW simpset ... done>>
> val unpkg5_res =
    |- (let (:'x,(t :'x # ('x -> num))) =
              (pack (:num,((0 :num),(\(x :num). SUC x))))
        in
          SND t (FST t)) =
       (1 :
       num) : thm
\end{verbatim}
\end{session}
The simplifier knows how to reduce such deconstructions of packages, and 
how to evaluate the \texttt{SUC} function as well, even though that
was hidden within \texttt{pkg5}.

Here is another example, using the same {\bf let} but with \texttt{pkg6} from above.
\begin{session}
\begin{verbatim}
- val unpkg6 = ``let (:'x, t:'x # ('x -> num)) = ^pkg6
                 in (SND t) (FST t)``;
> val unpkg6 =
    ``let (:'x,(t :'x # ('x -> num))) =
            (pack (:bool,(T,(\(x :bool). (0 :num)))))
      in
        SND t (FST t)``
     : term

- val unpkg6_res = eval [] unpkg6;
> val unpkg6_res =
    |- (let (:'x,(t :'x # ('x -> num))) =
              (pack (:bool,(T,(\(x :bool). (0 :num)))))
        in
          SND t (FST t)) =
       (0 :
       num) : thm
\end{verbatim}
\end{session}
Note that the resulting value is different (although the type is the same),
because although the body of the {\bf let} is the same,
the package that is being unpacked is different.

%The following shows how the same package can be deconstructed in different
%ways, yet return the same answer.
%\begin{session}
%\begin{verbatim}
%- val unpkg5a = ``let (:'x, t:'x # ('x -> num)) = ^pkg5 in (\y:'x. (SND t) y) (FST t)``;
%> val unpkg5a =
%    ``let (:'x,(t :'x # ('x -> num))) =
%            (pack (:num,((0 :num),(\(x :num). SUC x))))
%      in
%        (\(y :'x). SND t y) (FST t)``
%     : term
%
%- val unpkg5a_res = eval [] unpkg5a;
%> val unpkg5a_res =
%    |- (let (:'x,(t :'x # ('x -> num))) =
%              (pack (:num,((0 :num),(\(x :num). SUC x))))
%        in
%          (\(y :'x). SND t y) (FST t)) =
%       (1 :
%       num) : thm
%\end{verbatim}
%\end{session}

\section{Underlying Implementation of Packages}

One can use the {\bf pack} and {\bf let} syntax as described above
to construct and deconstruct packages. In the following, we describe the
implementation of this syntax, which can be skipped by the casual reader without loss.
But for those who are interested, the syntax as given so far for both these new forms
to construct and deconstruct packages
is actually syntactic sugar. Underneath, the actual syntax rests on two new 
fundamental term constants in the \HOLW{} logic, {\tt PACK} and {\tt UNPACK}.
These are part of the theory {\tt bool}, with the following primal types:
$$
\begin{array}{rcl}
\mathtt{PACK} & : & \ \,\forall \psi{:}\kappa.\ 
                    \psi\ (\alpha : \kappa \Rightarrow \mathbf{ty}{:}1)
                    \ \mbox{\tt ->}\ 
                    (\exists \phi{:}\kappa.\ \phi\ \alpha) \\
\mathtt{UNPACK} & : & (\forall \psi{:}\kappa.\ 
                    \psi\ (\alpha : \kappa \Rightarrow \mathbf{ty}{:}1)
                    \ \mbox{\tt ->}\ 
                    (\beta {:}\mathbf{ty}{:}1))
                    \ \mbox{\tt ->}\ 
                    (\exists \phi{:}\kappa.\ \phi\ \alpha)
                    \ \mbox{\tt ->}\ 
                    \beta
\end{array}
$$
Note that the kind variable $\kappa$ and the type variables $\alpha$ and
$\beta$ are free in the above types, which provide all the needed flexibility
for instances of the constants {\tt PACK} and {\tt UNPACK}
to use packages in every appropriate situation.
In particular we use the 
%higher (arrow) kinded 
free type variable $\alpha$
in combination with its argument, the type variable $\phi$,
to represent any possible type expression with free type variable $\phi$,
and thus any possible existential type $\exists \phi{:}\kappa.\ \phi\ \alpha$.

The 
surface syntax
%syntactic sugar
for packages given previously is actually translated 
by the parser into
uses of {\tt PACK} and {\tt UNPACK}, as follows.
$$
\begin{array}{lcl}
\mathbf{pack}({:}\sigma, t) % : \exists\alpha.\sigma'
& \ \mapsto \ & \mathtt{PACK}\ [{:}\sigma{:}]\ t \\
\mathbf{let}\ ({:}\alpha,x) = p\ \mathbf{in}\ s
& \mapsto & \mathtt{UNPACK}\ (\lambda{:}\alpha. \lambda x. s)\ p
\end{array}
$$
\subsection{Package Axioms}

Using {\tt PACK} and {\tt UNPACK}, the reduction of packages is 
fundamentally expressed
by the new \HOLW{} axiom {\tt UNPACK\_PACK\_AX}:
%$$ \vdash \mathtt{UNPACK}\ f\ (\mathtt{PACK}\ [{:}\sigma{:}]\ t) \ =\ 
%          f\ [{:}\sigma{:}]\ t
%$$
\\

\begin{tabular}{p{4.8in}p{1.0in}}
\makebox[4.8in][c]{
\begin{tabular}{c}
$\begin{array}{rl}
\vdash & 
\forall{:} (\alpha : \kappa \Rightarrow \mathbf{ty}{:}1) \ 
           (\beta  : \mathbf{ty}{:}1) \ 
           (\phi : \kappa). \\
& \ \ \ \forall (f : \forall\psi{:}\kappa.\ \psi\ \alpha
              \ \mbox{\tt ->} \ \beta) \ 
         (t : \phi\ \alpha). \\
& \ \ \ \ \ \ {\tt UNPACK} \ 
f \ (\mathtt{PACK}\ [{:} \, \phi \, {:}] \ t)
    = f\ [{:} \, \phi \, {:}]\ t
\end{array}$ \\
\end{tabular}} &
\makebox[1.0in][r]{({\tt UNPACK\_PACK\_AX})} \\
\end{tabular}
\\

\noindent
This axiom is included in the basic simplification set {\tt bool\_ss},
so almost every invocation of the simplifier will attempt this reduction
of packages by higher-order rewriting with {\tt UNPACK\_PACK\_AX},
along with {\tt TY\_BETA\_CONV} and {\tt BETA\_CONV}
to resolve $(\lambda{:}\alpha. \lambda x. s)\ [{:}\sigma{:}]\ t$.

\HOLW{} also includes the new axiom {\tt PACK\_ONTO\_AX}:
\\

\begin{tabular}{p{4.8in}p{1.0in}}
\makebox[4.8in][c]{
\begin{tabular}{rl}
$\vdash$ & 
$\forall{:} (\alpha : \kappa \Rightarrow \mathbf{ty}{:}1).\ 
    \forall (p : \exists\psi{:}\kappa.\ \psi\ \alpha).$ \\
& \ \ \ \ \ \ $\exists{:} (\phi : \kappa).\ 
               \exists (t : \phi\ \alpha).$ \\
& \ \ \ \ \ \ \ \ \ $p = \mathtt{PACK}\ [{:} \, \phi \, {:}] \ t$ \\
\end{tabular}} &
\makebox[1.0in][r]{({\tt PACK\_ONTO\_AX})} \\
\end{tabular}
\\

\noindent
{\tt PACK\_ONTO\_AX} expresses the idea that every package was
created using {\tt PACK} from some type and term. Thus {\tt PACK}
is the sole constructor of values of existential type.

\newpage
\section{Example: Counters}

To see how packages can be used to simulate the idea of an object,
we consider a counter, as an object for which there are three methods;
a {\tt new} method to create a counter with an initialized count, 
a {\tt get} method to obtain the counter's current value, and
an {\tt inc} method to increment the counter's value.

The first step is to create a record type to hold these three methods.
The methods will need to be parameterized on the actual type 
of the counter's contents,
since that data structure is what the object is hiding.
Eventually we will form a package with an existential type,
where the actual representation of the counter's data
will be hidden.

When designing the record type, we need to consider 
that we are working in a logic that does not have all 
of the features of a sophisticated programming language.
Accordingly, the {\tt get} method needs to take that internal
data structure as input in order to compute the count that is output.
In Java, an object can update its private data fields, but in a
functional logic like \HOLW{}, any change must be represented by
making a new copy of the original object, with updated contents.
Also, whereas in Java a new object can be constructed by a
routine that establishes its initial value, here we need
to use the {\tt new} method to obtain that initial value,
since it must be of the unknown 
and parameterized 
type.
%In addition, in this functional logic, the {\tt inc} method will 
%actually create a new copy of the counter object, rather than
%updating the existing counter in place.

The new record type
can be introduced to the \HOLW{} logic by the following.
% command.
\begin{session}
\begin{verbatim}
- val _ = Hol_datatype
       `counter_recd1 =
                     <| new : 'a;
                        get : 'a -> num;
                        inc : 'a -> 'a
                      |>`;
<<HOL message: Defined type: "counter_recd1">>
- val counter_kind = kind_of ``:counter_recd1``;
> val counter_kind = ``::ty => ty`` : kind
\end{verbatim}
\end{session}
This introduces a new type constant in the \HOLW{} logic
named {\tt counter\_recd1}, along with term constants for
constructing, accessing, and updating 
the fields of values of this record type.
The type constant created
%, {\tt counter\_recd1}, 
has the arrow kind {\bf ty} {\tt =>} {\bf ty}, so {\tt counter\_recd1}
is a type operator that expects a type argument of kind {\bf ty}.

The aim now is to create packages as values of this abstract data type,
as values of type $\exists\psi.\ \psi\ \mathtt{counter\_recd1}$.
One way is to simply use a natural number as the hidden data
structure, so that the increment function is just the successor function {\tt SUC}.
\begin{session}
\begin{verbatim}
- val counterADT =
       ``pack ( :num,
                <| new := 1;
                   get := \i:num. i;
                   inc := \i:num. SUC i
                |> ) : ?'a. 'a counter_recd1``;
> val counterADT =
    ``pack
        (:num,
          <|new := (1 :num); get := (\(i :num). i);
            inc := (\(i :num). SUC i)|>)``
     : term
- val counterADT_type = type_of counterADT;
> val counterADT_type =
    ``:?'x. 'x counter_recd1``
     : hol_type
\end{verbatim}
\end{session}
This yields a value with the desired existential type.

Next we wish to test this package by using it. The following
code creates a new counter, increments it, and then returns its value.
\begin{session}
\begin{verbatim}
- val counter_ex1 =
  ``let (:'Counter,counter) = ^counterADT in
    counter.get (counter.inc counter.new)``;
> val counter_ex1 =
    ``let (:'Counter,(counter :'Counter counter_recd1)) =
            (pack
               (:num,
                 <|new := (1 :num); get := (\(i :num). i);
                   inc := (\(i :num). SUC i)|>))
      in
        counter.get (counter.inc counter.new)``
     : term
\end{verbatim}
\end{session}

We can test to see how this works by using the {\tt eval} tool we
defined earlier. This provides a theorem where this program is
simplified by evaluation.
\begin{session}
\begin{verbatim}
- val ex1_res = eval[] counter_ex1;
> val ex1_res =
    |- (let (:'Counter,(counter :'Counter counter_recd1)) =
              (pack
                 (:num,
                   <|new := (1 :num); get := (\(i :num). i);
                     inc := (\(i :num). SUC i)|>))
        in
          counter.get (counter.inc counter.new)) =
       (2 :
       num) : thm
\end{verbatim}
\end{session}

Here is another example, where we define a routine to increment
the counter three times, and then return the result.
\begin{session}
\begin{verbatim}
- val counter_ex2 =
  ``let (:'Counter,counter) = ^counterADT in
    let add3 = \c:'Counter. counter.inc (counter.inc (counter.inc c)) in
    counter.get (add3 counter.new)``;
> val counter_ex2 =
    ``let (:'Counter,(counter :'Counter counter_recd1)) =
            (pack
               (:num,
                 <|new := (1 :num); get := (\(i :num). i);
                   inc := (\(i :num). SUC i)|>))
      in
        let (add3 :'Counter -> 'Counter) (c :'Counter) =
              counter.inc (counter.inc (counter.inc c))
        in
          counter.get (add3 counter.new)``
     : term
\end{verbatim}
\end{session}

We can use the {\tt eval} tool as before to evaluate this expression.
%It is necessary 
We need
to supply {\tt LET\_DEF}, 
the definition of the constant {\tt LET} which is used to create
the syntactic sugar {\bf let..in} form, 
so that it can be reduced as well. 
(For some reason this is not included in
the automatic set of simplifications.)
\begin{session}
\begin{verbatim}
- LET_DEF;
> val it =
    |- (LET :('a -> 'b) -> 'a -> 'b) = (\(f :'a -> 'b) (x :'a). f x)
     : thm
     
- val ex2_res = eval[LET_DEF] counter_ex2;
> val ex2_res =
    |- (let (:'Counter,(counter :'Counter counter_recd1)) =
              (pack
                 (:num,
                   <|new := (1 :num); get := (\(i :num). i);
                     inc := (\(i :num). SUC i)|>))
        in
          let (add3 :'Counter -> 'Counter) (c :'Counter) =
                counter.inc (counter.inc (counter.inc c))
          in
            counter.get (add3 counter.new)) =
       (4 :
       num) : thm
\end{verbatim}
\end{session}


\section{Example: Scheduling Queues}

To demonstrate the usefulness of packages, we now will develop a theory
of scheduling queues for an operating system.

Consider an operating system in which there are multiple processes,
some of which are from time to time suspended in an inactive state,
while others are running on the available processors.

The processes which are suspended are remembered in some kind of
a data structure. When a new process is created, or when 
a currently running process becomes suspended,
it is added to this data structure. When a process finishes its task
and ends, its processor is then assigned to work on one of the
other processes in the data structure.

Exactly which process is chosen is a matter of policy. One simple
choice is that it should be the oldest process residing in the data
structure. This leads to a last-in-first-out policy, which is 
accomplished by using a queue for the data structure.

However, this is not the only legitimate choice. There may be
processes with higher priority, or there may be some more subtle
measure of utility which ranks different processes more suitable
to be assigned to the new processor.

One could even use a first-in-first-out policy, which would be 
accomplished by using a stack for the data structure. While this 
would not possess certain desirable properties like fairness, 
that each suspended process will eventually run, it might be 
suitable for certain restricted applications.

Even for a given policy, there may be several possible implementations
which may vary in their pragmatics. We may begin, say, with a
very simple and clean implementation of queues, and later move
to a more complex representation for faster performance.

What is interesting here is that these different policy choices can
be modularized by simply making them part of the data structure.
Each such policy choice 
would be implemented by a different data structure.
As long as each data structure obeys certain general properties,
we can swap any of a variety of data structures for the one used in
the implementation of the scheduling algorithm. We can even delay
the choice of which policy to use until late in the overall development,
or even switch the policy on the fly during runtime, simply by changing
the data structure, as long as each of the family of data structures used 
all obey the same general properties.

The rest of the software development can then rely on the data structure
to obey these general properties, but cannot rely on any
other special properties of any individual data structure or the
policy it represents. This enforces a modularity that is a vital
feature of good system design, where design choices are isolated
and hidden from the rest of the system, so that later changes have
a minimum ripple effect.

Packages are critical to performing this necessary information hiding.
This example demonstrates how packages can be used to accomplish
the hiding of the actual data structure used. This supports 
proper modularization,
so that the software that uses the scheduling data structure is
isolated from the details of its implementation.

%\subsection{Scheduling operations}

To begin, we will consider data structures that collect a number
of elements. The types of these data structures will be modeled 
as a type operator $\beta$ of kind $\mathbf{ty} \Rightarrow \mathbf{ty}$
that maps an element of type $\alpha$ into a data structure of
type $\alpha\ \beta$.

The data structures themselves may have many different definitions,
but here we want to concentrate on their fundamental operations and
general properties. As a beginning, we want the scheduling
data structures to support the following operations:
%
\begin{center}
\begin{tabular}{|lcl|}
\hline
emptyq & : & $\forall\beta.\ \beta\ \alpha$ \\
insert & : & $\forall\beta.\ \beta \rightarrow \beta\ \alpha \rightarrow \beta\ \alpha$ \\
remove & : & $\forall\beta.\ \beta\ \alpha \rightarrow \beta\ \mbox{\tt \#}\ \beta\ \alpha$ \\
count  & : & $\forall\beta.\ \beta \rightarrow \beta\ \alpha \rightarrow \mathtt{num}$ \\
\hline
\end{tabular}
\end{center}
%
Note that each operation is polymorphic in the element type $\beta$.

The meaning of the emptyq operation is to be the empty version of the queue.
Likewise, the insert operation takes an element and a queue, and inserts
the element into the queue, returning the new, increased queue.
The remove operation takes a queue, selects some element of it,
removing that element from the queue, and returns a pair of the 
selected element and the new, diminished queue.
The count operation takes an element and a queue, and returns
the number of times (possibly zero) that that element appears
in the queue.

These operations can be assembled into a record of related operations
that are meant to work together, as follows.
\begin{session}
\begin{verbatim}
- val _ = Hol_datatype
         `sched_q_opers = <| emptyq : !'b. 'b 'a;
                             insert : !'b. 'b -> 'b 'a -> 'b 'a;
                             remove : !'b. 'b 'a -> 'b # 'b 'a;
                             count  : !'b. 'b -> 'b 'a -> num  |>`;
<<HOL message: Defined type: "sched_q_opers">>
\end{verbatim}
\end{session}
This datatype definition in the \HOLW{} logic not only creates the 
%new
type {\tt sched\_q\_opers}, but also introduces term constants so that
we can index the fields of any record of this type using the familiar
notation {\tt rcd.emptyq}, {\tt rcd.insert}, etc.

Next we specify the properties that we wish to be true of these operations
to constitute a valid and proper scheduling queue. These should be broad
enough to encompass all the possible implementations we might wish to use,
but narrow enough to be a base on which to build the rest of the program
that uses this data structure.

The properties we will choose for this example are the following.
\begin{enumerate}
\item The count of any item in the empty queue is zero.
\item For the queue resulting from the insertion of an element $y$ 
into a queue $q$, the count of any element $x$ should be the
count of $x$ in the original queue $q$ plus one if $x = y$,
but otherwise simply the count of $x$ in $q$.
\item If a queue $q$ has at least one element (that is, $q$ is not
empty), then the result of removing an element from $q$ is a pair
$(y,\ q')$ which has the following property.
The count of any element $x$ in the original queue $q$ is equal 
to the count of $x$ in the result of inserting $y$ into $q'$.
\end{enumerate}

These properties can be represented in the following definition.
\begin{session}
\begin{verbatim}
- val is_scheduling_q_def = Define
     `is_scheduling_q (ops:'a sched_q_opers) =
        (!:'b. !(x:'b). ops.count x (ops.emptyq[:'b:]) = 0) /\
        (!:'b. !(q:'b 'a) (x:'b) (y:'b).
                  ops.count x (ops.insert y q) =
                    if x = y then ops.count x q + 1
                             else ops.count x q) /\
        (!:'b. !q:'b 'a.
                if (!x:'b. ops.count x q = 0) then T else
                  let (y, q') = ops.remove q in
                  !x:'b. ops.count x q = ops.count x (ops.insert y q'))`;
Definition has been stored under "is_scheduling_q_def"
\end{verbatim}
\end{session}
The system will respond with the theorem of the definition, which we omit here.

This covers the properties of a scheduling data structure. We will represent
the data structure in an object-oriented way, by forming a record that contains
both the current value of the data structure, and also all of the operations
that can be performed on it. First, we create the type of this record.
\begin{session}
\begin{verbatim}
- val _ = Hol_datatype
         `sched_q = <| this : 'b 'a;
                       ops  : 'a sched_q_opers  |>`;
<<HOL message: Defined type: "sched_q">>
\end{verbatim}
\end{session}

For our first scheduling queue implementation, we choose to use
a simple list, where new elements are added to the list at the front,
and when an element is removed it is taken from the back of the list.
For simplicity, instead of a complex type representing processes,
we choose to use just a simple natural number.
We intend to build a record of scheduling operations as follows.

\begin{verbatim}
val reference_q_def = Define
   `reference_q =
      <| this := [] : num list;
         ops  := <| emptyq := \:'b. [] : 'b list;
                    insert := \:'b. \(x:'b) xs. CONS x xs;
                    remove := \:'b. \xs:'b list. (LAST xs, FRONT xs);
                    count  := \:'b. \(x:'b) xs. COUNT x xs
                  |>
       |>`;
\end{verbatim}
Fortunately, we have predefined operations in the {\tt list}
library for most of this. Both {\tt []} and {\tt CONS} are commonly used,
and {\tt FRONT} and {\tt LAST} are predefined in the {\tt list} library.
{\tt LAST} returns the last element of a list, while 
{\tt FRONT} returns all of the list except for the last element.
Both of these are undefined if they are applied to an empty list.
%
What is not available in the \HOL{} libraries is the {\tt COUNT} function,
but this is easy enough to create.
\begin{session}
\begin{verbatim}
- open listTheory;
. . .
- val COUNT_DEF = Define
    `(COUNT (x:'a) [] = 0) /\
     (COUNT x (y::ys) = if x = y then COUNT x ys + 1
                                 else COUNT x ys)`;
Definition has been stored under "COUNT_def"
> val COUNT_DEF =
    |- (!(x :'a). COUNT x ([] :'a list) = (0 :num)) /\
       !(x :'a) (y :'a) (ys :'a list).
         COUNT x (y::ys) =
         if x = y then COUNT x ys + (1 :num) else COUNT x ys
     : thm
\end{verbatim}
\end{session}

Then the following properties of {\tt COUNT} are proven by straightforward means.
\begin{session}
\begin{verbatim}
> val ALL_COUNT_ZERO =
    |- !(xs :'a list).
         (!(x :'a). COUNT x xs = (0 :num)) <=> (xs = ([] :'a list))
     : thm

> val ALL_COUNT_ZERO_2 =
    |- !(xs :'a list) (ys :'a list).
         (!(x :'a). (COUNT x xs = (0 :num)) /\ (COUNT x ys = (0 :num))) <=>
         (xs = ([] :'a list)) /\ (ys = ([] :'a list))
     : thm

> val COUNT_FRONT =
    |- !(xs :'a list) (x :'a).
         xs <> ([] :'a list) ==>
         x <> LAST xs ==>
         (COUNT x (FRONT xs) = COUNT x xs)
     : thm

> val COUNT_LAST =
    |- !(xs :'a list).
         xs <> ([] :'a list) ==>
         (COUNT (LAST xs) xs = COUNT (LAST xs) (FRONT xs) + (1 :num))
     : thm
     
> val COUNT_APPEND =
    |- !(xs :'a list) (ys :'a list) (x :'a).
         COUNT x ((xs ++ ys) :'a list) = COUNT x xs + COUNT x ys
     : thm
\end{verbatim}
\end{session}

\begin{session}
\begin{verbatim}
> val COUNT_REVERSE =
    |- !(xs :'a list) (x :'a). COUNT x (REVERSE xs) = COUNT x xs
     : thm
\end{verbatim}
\end{session}
The tactics to prove each are provided in the {\tt packageScript.sml} file in
the directory {\tt examples/HolOmega}, but not covered further here.

The definition of {\tt COUNT} now allows us to create our reference
implementation of scheduling queues, as a very simple and clean implementation
which is not necessarily efficient, but for which the necessary properties 
should be easy to prove.
\begin{session}
\begin{verbatim}
- val reference_q_def = Define
     `reference_q =
        <| this := [] : num list;
           ops  := <| emptyq := \:'b. [] : 'b list;
                      insert := \:'b. \(x:'b) xs. CONS x xs;
                      remove := \:'b. \xs:'b list. (LAST xs, FRONT xs);
                      count  := \:'b. \(x:'b) xs. COUNT x xs
                    |>
         |>`;
Definition has been stored under "reference_q_def"
> val reference_q_def =
    |- (reference_q :(list, num) sched_q) =
       <|this := ([] :num list);
         ops :=
           <|emptyq := (\:'b. ([] :'b list));
             insert := (\:'b. (\(x :'b) (xs :'b list). x::xs));
             remove := (\:'b. (\(xs :'b list). (LAST xs,FRONT xs)));
             count := (\:'b. (\(x :'b) (xs :'b list). COUNT x xs))|> |>
     : thm
\end{verbatim}
\end{session}

Now we can prove that the operations of this reference scheduling
queue satisfies the properties required to be a scheduling queue.
\begin{session}
\begin{verbatim}
- val reference_q_is_scheduling_q = store_thm(
   "reference_q_is_scheduling_q",
   ``is_scheduling_q reference_q.ops``,
   SRW_TAC [ARITH_ss] [reference_q_def,is_scheduling_q_def,COUNT_DEF]
   THEN REWRITE_TAC [ALL_COUNT_ZERO]
   THEN STRIP_ASSUME_TAC (ISPEC ``q:'b list`` list_CASES)
   THEN ASM_REWRITE_TAC [NOT_CONS_NIL]
   THEN GEN_TAC
   THEN SIMP_TAC list_ss [GSYM COUNT_LAST]
   THEN COND_CASES_TAC
   THEN SRW_TAC [] [COUNT_FRONT]
  );
> val reference_q_is_scheduling_q =
    |- is_scheduling_q (reference_q :(list, num) sched_q).ops
     : thm
\end{verbatim}
\end{session}

Our intention here is to eventually show that this is one possible
implementation of scheduling queues, but in the process to hide the
actual data structure being used. In this case the data structure is
lists, but we don't wish this to be visible. What we would like is
to use existential types to hide this, through the use of packages,
like the following. 
\begin{session}
\begin{verbatim}
- val sched_queue_ty = ``:?'a:ty => ty. ('a,num)sched_q``;
> val sched_queue_ty =
    ``:?'a :ty => ty. ('a, num) sched_q``
     : hol_type

- val sched_queue_ty' = ty_antiq sched_queue_ty;
> val sched_queue_ty' =
    ``(ty_antiq( `:?'a :ty => ty. ('a, num) sched_q `))``
     : term
\end{verbatim}
\end{session}
Here we create the desired type ({\tt sched\_queue\_ty})
as an existential type, wrapping around the type
{\tt ('a,num)sched\_q}, but hiding the actual type operator {\tt 'a}
by the existential type quantification. 
(The type antiquotation is required to use the type inside terms
that are being parsed.)

Then a term of this existential type can be created using packages,
taking our reference queue implementation and abstracting away
from the actual {\tt list} datatype.
\begin{session}
\begin{verbatim}
- val reference_q_pkg = ``pack(:list, reference_q) : ^sched_queue_ty'``;
> val reference_q_pkg =
    ``pack (:list,(reference_q :(list, num) sched_q))``
     : term
\end{verbatim}
\end{session}
We now create another possible implementation of priority queues. This
one will use a pair of lists, where new additions
to the queue are added to the first list at its front, and removals
from the queue are taken from the second list at its front.
In the case when the second list is empty, the first list is
reversed and then replaces the second list.
%, and then the head is taken.
%Our intention is to create a definition like the following.
\begin{verbatim}
val efficient_q_def = Define
   `efficient_q =
      <| this := ([] : num list, [] : num list);
         ops  := <| emptyq := \:'b. ([] : 'b list, [] : 'b list);
                    insert := \:'b. \x (xs,ys). (CONS x xs, ys);
                    remove := \:'b. \(xs,ys). REMOVE xs ys;
                    count  := \:'b. \x (xs,ys). COUNT x xs + COUNT x ys
                  |>
       |>`;
\end{verbatim}
\noindent
The operation of removing an element from this data structure is
clearly more complex than before, and we express this using a subsidiary
operator {\tt REMOVE} that we intend to define. But {\tt REMOVE}
is nontrivial to define, as seen if we try the normal {\tt Define}
tool.
\begin{session}
\begin{verbatim}
- val REMOVE_def = Define
    `(REMOVE (xs:'a list) ([]:'a list) = REMOVE [] (REVERSE xs)) /\
     (REMOVE xs (y::ys) = (y, (xs,ys)))`;
Initial goal:


?(R :'a list # 'a list -> 'a list # 'a list -> bool).
  WF R /\
  !(xs :'a list). R (([] :'a list),REVERSE xs) (xs,([] :'a list))


Exception raised at TotalDefn.Define:
between line 141, character 3 and line 142, character 36:
at TotalDefn.defnDefine:

Unable to prove termination!

Try using "TotalDefn.tDefine <name> <quotation> <tac>".
The termination goal has been set up using Defn.tgoal <defn>.
! Uncaught exception: 
! HOL_ERR
\end{verbatim}
\end{session}
\noindent
The normal {\tt Define} machinery is not able to automatically prove 
the termination of this definition, so the system does not accept this
as a valid definition.

Instead, the definition package is directing us to Konrad Slind's excellent
total recursive function definition package, which we will now use.
%The reader is encouraged to read the corresponding sections in the
We begin by reforming our definition using
the {\tt Hol\_defn} tool as described in the \DESCRIPTION{} manual.
\begin{session}
\begin{verbatim}
- val REMOVE_defn = Hol_defn "REMOVE"
  `(REMOVE [] [] = (ARB, ([],[]))) /\
   (REMOVE (xs:'a list) ([]:'a list) = REMOVE [] (REVERSE xs)) /\
   (REMOVE xs (y::ys) = (y, (xs,ys)))`;
<<HOL message: mk_functional: 
  pattern completion has added 1 clause to the original specification.>>
> val REMOVE_defn =
    HOL function definition (recursive)
    
    Equation(s) :
     [..]
    |- REMOVE ([] :'a list) ([] :'a list) =
       ((ARB :'a),([] :'a list),([] :'a list))
     [..]
    |- REMOVE ((v2 :'a)::(v3 :'a list)) ([] :'a list) =
       REMOVE ([] :'a list) (REVERSE (v2::v3))
     [..]
    |- REMOVE ([] :'a list) ((y :'a)::(ys :'a list)) = (y,([] :'a list),ys)
     [..]
    |- REMOVE ((v4 :'a)::(v5 :'a list)) ((y :'a)::(ys :'a list)) =
       (y,v4::v5,ys)
\end{verbatim}
\end{session}
   
\begin{session}
\begin{verbatim}
    Induction :
     [..]
    |- !(P :'a list -> 'a list -> bool).
         P ([] :'a list) ([] :'a list) /\
         (!(v2 :'a) (v3 :'a list).
            P ([] :'a list) (REVERSE (v2::v3)) ==>
            P (v2::v3) ([] :'a list)) /\
         (!(y :'a) (ys :'a list). P ([] :'a list) (y::ys)) /\
         (!(v4 :'a) (v5 :'a list) (y :'a) (ys :'a list).
            P (v4::v5) (y::ys)) ==>
         !(v :'a list) (v1 :'a list). P v v1
    
    Termination conditions :
      0. !(v3 :'a list) (v2 :'a).
           (R :'a list # 'a list -> 'a list # 'a list -> bool)
             (([] :'a list),REVERSE (v2::v3)) (v2::v3,([] :'a list))
      1. WF (R :'a list # 'a list -> 'a list # 'a list -> bool)
     : defn
\end{verbatim}
\end{session}
This has created an \ML{} data structure (here stored in {\tt REMOVE\_defn})
which contains the elements of a provisional recursive function definition.
In particular, it contains both the equations which are the 
reflections of our original specification, as well as an induction
principle which can be used to induct over this particular pattern
of recursion.

But none of this is available for us to use until we prove the
termination of the definition. This we will have to do by hand.
Fortunately, Slind's package provides a number of useful tools
to aid our task.

First, we set up the termination property as a goal to be proved.
\begin{session}
\begin{verbatim}
- Defn.tgoal REMOVE_defn;
> val it =
    Proof manager status: 1 proof.
    1. Incomplete goalstack:
         Initial goal:
    
         ?(R :'a list # 'a list -> 'a list # 'a list -> bool).
           WF R /\
           !(v3 :'a list) (v2 :'a).
             R (([] :'a list),REVERSE (v2::v3)) (v2::v3,([] :'a list))
    
    
     : proofs
\end{verbatim}
\end{session}
We need to supply a well-founded relation as a witness for {\tt R},
that strictly decreases for each call. A natural choice is
{\tt measure(\verb|\|(xs,ys).~LENGTH~xs)}. This uses the
constant {\tt measure} which transforms a function of type
{\tt 'a~->~num} into a relation on {\tt 'a}.
\begin{session}
\begin{verbatim}
- e (WF_REL_TAC `measure (\(xs,ys). LENGTH xs)`);
OK..
1 subgoal:
> val it =
    
    !(v3 :'a list) (v2 :'a). LENGTH ([] :'a list) < LENGTH (v2::v3)
    
     : proof
\end{verbatim}
\end{session}
This is easily solved by simplification.
\begin{session}
\begin{verbatim}
- e (SIMP_TAC list_ss []);
OK..

Goal proved.
|- !(v3 :'a list) (v2 :'a). LENGTH ([] :'a list) < LENGTH (v2::v3)

> val it =
    Initial goal proved.
    |- ((REMOVE ([] :'a list) ([] :'a list) =
         ((ARB :'a),([] :'a list),([] :'a list))) /\
        (REMOVE ((v2 :'a)::(v3 :'a list)) ([] :'a list) =
         REMOVE ([] :'a list) (REVERSE (v2::v3))) /\
        (REMOVE ([] :'a list) ((y :'a)::(ys :'a list)) =
         (y,([] :'a list),ys)) /\
        (REMOVE ((v4 :'a)::(v5 :'a list)) (y::ys) = (y,v4::v5,ys))) /\
       !(P :'a list -> 'a list -> bool).
         P ([] :'a list) ([] :'a list) /\
         (!(v2 :'a) (v3 :'a list).
            P ([] :'a list) (REVERSE (v2::v3)) ==>
            P (v2::v3) ([] :'a list)) /\
         (!(y :'a) (ys :'a list). P ([] :'a list) (y::ys)) /\
         (!(v4 :'a) (v5 :'a list) (y :'a) (ys :'a list).
            P (v4::v5) (y::ys)) ==>
         !(v :'a list) (v1 :'a list). P v v1
     : proof
\end{verbatim}
\end{session}

Now that we know the tactics to use, we can convert 
{\tt REMOVE\_defn} into a real definition 
by a single call to {\tt Defn.tprove}.
\begin{session}
\begin{verbatim}
- val (REMOVE_def,REMOVE_ind) =
  Defn.tprove (REMOVE_defn,
    WF_REL_TAC `measure (\(xs,ys). LENGTH xs)`
    THEN SIMP_TAC list_ss []
  );

> val REMOVE_def =
    |- (REMOVE ([] :'a list) ([] :'a list) =
        ((ARB :'a),([] :'a list),([] :'a list))) /\
       (REMOVE ((v2 :'a)::(v3 :'a list)) ([] :'a list) =
        REMOVE ([] :'a list) (REVERSE (v2::v3))) /\
       (REMOVE ([] :'a list) ((y :'a)::(ys :'a list)) =
        (y,([] :'a list),ys)) /\
       (REMOVE ((v4 :'a)::(v5 :'a list)) (y::ys) = (y,v4::v5,ys))
     : thm
  val REMOVE_ind =
    |- !(P :'a list -> 'a list -> bool).
         P ([] :'a list) ([] :'a list) /\
         (!(v2 :'a) (v3 :'a list).
            P ([] :'a list) (REVERSE (v2::v3)) ==>
            P (v2::v3) ([] :'a list)) /\
         (!(y :'a) (ys :'a list). P ([] :'a list) (y::ys)) /\
         (!(v4 :'a) (v5 :'a list) (y :'a) (ys :'a list).
            P (v4::v5) (y::ys)) ==>
         !(v :'a list) (v1 :'a list). P v v1
     : thm
\end{verbatim}
\end{session}

Even better, we can wrap both the definition and
its proof of termination all up in one piece
by a single call to {\tt tDefine}.
\begin{session}
\begin{verbatim}
- val REMOVE_def =
  tDefine "REMOVE"
   `(REMOVE [] [] = (ARB, ([],[]))) /\
    (REMOVE (xs:'a list) ([]:'a list) = REMOVE [] (REVERSE xs)) /\
    (REMOVE xs (y::ys) = (y, (xs,ys)))`
  (WF_REL_TAC `measure (\(xs,ys). LENGTH xs)`
   THEN SIMP_TAC list_ss []);
<<HOL message: mk_functional: 
  pattern completion has added 1 clause to the original specification.>>
Equations stored under "REMOVE_def".
Induction stored under "REMOVE_ind".
> val REMOVE_def =
    |- (REMOVE ([] :'a list) ([] :'a list) =
        ((ARB :'a),([] :'a list),([] :'a list))) /\
       (REMOVE ((v2 :'a)::(v3 :'a list)) ([] :'a list) =
        REMOVE ([] :'a list) (REVERSE (v2::v3))) /\
       (REMOVE ([] :'a list) ((y :'a)::(ys :'a list)) =
        (y,([] :'a list),ys)) /\
       (REMOVE ((v4 :'a)::(v5 :'a list)) (y::ys) = (y,v4::v5,ys))
     : thm
\end{verbatim}
\end{session}

The induction theorem is not returned directly, but we can obtain
it from the current theory using the \ML{} {\tt theorem} command.
\begin{session}
\begin{verbatim}
- val REMOVE_ind = theorem "REMOVE_ind";
> val REMOVE_ind =
    |- !(P :'a list -> 'a list -> bool).
         P ([] :'a list) ([] :'a list) /\
         (!(v2 :'a) (v3 :'a list).
            P ([] :'a list) (REVERSE (v2::v3)) ==>
            P (v2::v3) ([] :'a list)) /\
         (!(y :'a) (ys :'a list). P ([] :'a list) (y::ys)) /\
         (!(v4 :'a) (v5 :'a list) (y :'a) (ys :'a list).
            P (v4::v5) (y::ys)) ==>
         !(v :'a list) (v1 :'a list). P v v1
     : thm
\end{verbatim}
\end{session}

Now we prove a number of elementary facts about {\tt REMOVE}.
\begin{session}
\begin{verbatim}
- val REMOVE_CONS = store_thm(
   "REMOVE_CONS",
   ``!xs ys (y:'a). REMOVE xs (y::ys) = (y,(xs,ys))``,
   Cases
   THEN REWRITE_TAC [REMOVE_def]
  );
> val REMOVE_CONS =
    |- !(xs :'a list) (ys :'a list) (y :'a). REMOVE xs (y::ys) = (y,xs,ys)
     : thm

- val REVERSE_CONS_NOT_NIL = store_thm(
   "REVERSE_CONS_NOT_NIL",
   ``!xs (x:'a). ~(REVERSE (x::xs) = [])``,
   SIMP_TAC list_ss []
  );
> val REVERSE_CONS_NOT_NIL =
    |- !(xs :'a list) (x :'a). REVERSE (x::xs) <> ([] :'a list)
     : thm
\end{verbatim}
\end{session}

The interaction of {\tt REMOVE} with {\tt COUNT} is more
interesting, and we prove the following lemma for use later.
It begins with a use of the {\tt REMOVE} induction principle
to follow the same recursion structure that {\tt REMOVE} itself does.
\begin{session}
\begin{verbatim}
- val REMOVE_COUNT = store_thm(
   "REMOVE_COUNT",
   ``!xs ys (u:'a) us vs.
        (REMOVE xs ys = (u,(us,vs))) ==>
        ~((xs = []) /\ (ys = [])) ==>
          !z. COUNT z xs + COUNT z ys =
              if z = u then COUNT z us + COUNT z vs + 1
                       else COUNT z us + COUNT z vs``,
   HO_MATCH_MP_TAC REMOVE_ind
   THEN REPEAT CONJ_TAC
   THEN REPEAT GEN_TAC
   THEN SIMP_TAC list_ss [REMOVE_def,COUNT_DEF,COUNT_APPEND,COUNT_REVERSE]
   THEN CONV_TAC (RATOR_CONV (ONCE_DEPTH_CONV SYM_CONV))
   THEN STRIP_TAC
   THEN GEN_TAC
   THEN COND_CASES_TAC
   THEN ASM_SIMP_TAC arith_ss [COUNT_DEF]
  );
> val REMOVE_COUNT =
    |- !(xs :'a list) (ys :'a list) (u :'a) (us :'a list) (vs :'a list).
         (REMOVE xs ys = (u,us,vs)) ==>
         ~((xs = ([] :'a list)) /\ (ys = ([] :'a list))) ==>
         !(z :'a).
           COUNT z xs + COUNT z ys =
           if z = u then
             COUNT z us + COUNT z vs + (1 :num)
           else
             COUNT z us + COUNT z vs
     : thm
\end{verbatim}
\end{session}

Now we are ready to create our new scheduling queue implementation.
We will call this the ``efficient'' implementation, because it is
faster than the reference implementation, and would be a better
candidate for eventual deployment.
\begin{session}
\begin{verbatim}
- val efficient_q_def = Define
     `efficient_q =
        <| this := ([] : num list, [] : num list);
           ops  := <| emptyq := \:'b. ([] : 'b list, [] : 'b list);
                      insert := \:'b. \x (xs,ys). (CONS x xs, ys);
                      remove := \:'b. \(xs,ys). REMOVE xs ys;
                      count  := \:'b. \x (xs,ys). COUNT x xs + COUNT x ys
                    |>
         |>`;
\end{verbatim}
\end{session}

\begin{session}
\begin{verbatim}
Definition has been stored under "efficient_q_def"
> val efficient_q_def =
    |- (efficient_q :('b list prod o list, num) sched_q) =
       <|this := (([] :num list),([] :num list));
         ops :=
           <|emptyq := (\:'b. (([] :'b list),([] :'b list)));
             insert :=
               (\:'b. (\(x :'b) ((xs :'b list),(ys :'b list)). (x::xs,ys)));
             remove :=
               (\:'b. (\((xs :'b list),(ys :'b list)). REMOVE xs ys));
             count :=
               (\:'b.
                  (\(x :'b) ((xs :'b list),(ys :'b list)).
                     COUNT x xs + COUNT x ys))|> |>
     : thm
\end{verbatim}
\end{session}

The efficient implementation can be proven to satisfy the conditions 
to be a scheduling queue.

Before we start, we need to import the pair library, to be able to 
easily simplify expressions involving pairs, such as the pair of lists
involved in this implementation.
\begin{session}
\begin{verbatim}
- local open pairLib in end;
\end{verbatim}
\end{session}

Here is the proof that the efficient queue is a scheduling queue.
It uses the lemmas we have proven before, 
and a good deal of simplification.
\begin{session}
\begin{verbatim}
- val efficient_q_is_scheduling_q = store_thm(
   "efficient_q_is_scheduling_q",
   ``is_scheduling_q efficient_q.ops``,
   SRW_TAC [ARITH_ss] [efficient_q_def,is_scheduling_q_def,COUNT_DEF]
   THEN REPEAT (POP_ASSUM MP_TAC)
   THEN PairCases_on `q`
   THEN SRW_TAC [ARITH_ss] [COUNT_DEF]
   THEN REWRITE_TAC [ALL_COUNT_ZERO_2]
   THEN POP_ASSUM MP_TAC
   THEN Cases_on `q0`
   THEN Cases_on `q1`
   THEN PairCases_on `q'`
   THEN DISCH_TAC
   THEN IMP_RES_THEN MP_TAC REMOVE_COUNT
   THEN SRW_TAC [] [COUNT_DEF]
   THEN COND_CASES_TAC
   THEN ASM_SIMP_TAC arith_ss []
  );
> val efficient_q_is_scheduling_q =
    |- is_scheduling_q (efficient_q :('b list prod o list, num) sched_q).ops
     : thm
\end{verbatim}
\end{session}

Similar to before, we can form a package from this scheduling queue,
hiding the actual data structure.
\begin{session}
\begin{verbatim}
- val efficient_q_pkg = ``pack(:\'a. 'a list # 'a list, efficient_q)
                          : ^sched_queue_ty'``;
> val efficient_q_pkg =
    ``pack
        (:'a list prod o list,
          (efficient_q :('b list prod o list, num) sched_q))``
     : term
\end{verbatim}
\end{session}

We can even check to see if the two package terms for the two
different implementations have the same type.
\begin{session}
\begin{verbatim}
- val check = eq_ty (type_of reference_q_pkg) (type_of efficient_q_pkg);
> val check = true : bool
\end{verbatim}
\end{session}

So we have now shown two different implementations of priority queues,
that both satisfy the required properties, as specified in 
{\tt is\_scheduling\_q}. Both of these implementations, despite
their differences, are actually functionally equivalent;
they both implement queues.

But this is not actually necessary to satisfy the properties of
{\tt is\_scheduling\_q}. We could implement a variety of other
kinds of data structures that do not operate as pure queues, and
as long as they satisfy {\tt is\_scheduling\_q}, they will be
acceptable. As an example, let's take an extreme variant.
Instead of a first-in-first-out queue, we can implement a
last-in-first-out stack. This will not have one very desirable
property that queues do, of fairness, which guarantees that
each element entered into the queue is eventually removed.
But since that is not mentioned in {\tt is\_scheduling\_q},
a stack is nevertheless an acceptable option.
%
\begin{session}
\begin{verbatim}
- val stack_q_def = Define
     `stack_q =
        <| this := [] : num list;
           ops  := <| emptyq := \:'b. [] : 'b list;
                      insert := \:'b. \x xs. CONS x xs;
                      remove := \:'b. \xs. (HD xs,TL xs);
                      count  := \:'b. \x xs. COUNT x xs
                    |>
         |>`;
Definition has been stored under "stack_q_def"

> val stack_q_def =
    |- (stack_q :(list, num) sched_q) =
       <|this := ([] :num list);
         ops :=
           <|emptyq := (\:'b. ([] :'b list));
             insert := (\:'b. (\(x :'b) (xs :'b list). x::xs));
             remove := (\:'b. (\(xs :'b list). (HD xs,TL xs)));
             count := (\:'b. (\(x :'b) (xs :'b list). COUNT x xs))|> |>
     : thm
\end{verbatim}
\end{session}
\noindent
This is a simple implementation, so it is not hard to prove
that it is a scheduling queue.
\begin{session}
\begin{verbatim}
- val stack_q_is_scheduling_q = store_thm(
   "stack_q_is_scheduling_q",
   ``is_scheduling_q stack_q.ops``,
   SRW_TAC [] [stack_q_def,is_scheduling_q_def,COUNT_DEF]
   THEN Cases_on `q`
   THEN SRW_TAC [] [COUNT_DEF]
  );
> val stack_q_is_scheduling_q =
    |- is_scheduling_q (stack_q :(list, num) sched_q).ops
     : thm
\end{verbatim}
\end{session}

As for the other implementations, the scheduling stack can be
wrapped up in a package, hiding its implementation type ({\tt list}).
\begin{session}
\begin{verbatim}
- val stack_q_pkg = ``pack(:list, stack_q) : ^sched_queue_ty'``;
> val stack_q_pkg =
    ``pack (:list,(stack_q :(list, num) sched_q))``
     : term
\end{verbatim}
\end{session}

The purpose of the scheduling stack is not to suggest this as a
suitable data structure to actually schedule processes, but just to
show how different implementations may have significantly different
functional behavior. Whatever the properties we specify, those
are what any suitable candidate implementation must meet.
Furthermore, those are also the properties, and only the 
properties, which the rest of the program can expect any
scheduling queue to meet. Thus this choice of the properties
of the data structure helps to form a boundary, isolating
information between parts of the program, which contributes
to good system design and modularity.

The advantage of wrapping these implementations up as packages
is that we hide their implementation types, so that we can write
common, general routines that make use of any of them, 
interchangably.

\newpage

%\noindent
The following function counts how many elements in a scheduling package
are equal to a given element.
\begin{session}
\begin{verbatim}
- val countp_def = Define
   `countp i (p:^sched_queue_ty') =
      let (:'a,q) = p in
      q.ops.count i q.this`;
Definition has been stored under "countp_def"
> val countp_def =
    |- !(i :num) (p :?'a :ty => ty. ('a, num) sched_q).
         countp i p =
         (let (:'a :ty => ty,(q :('a, num) sched_q)) = p
          in
            q.ops.count [:num:] i q.this)
     : thm
\end{verbatim}
\end{session}

\noindent
Here is a function to take a package and create an empty version 
of that same kind of package.
\begin{session}
\begin{verbatim}
- val emptyp_def = Define
   `emptyp (p:^sched_queue_ty') =
      let (:'a,q) = p in
        pack(:'a, <| this := q.ops.emptyq [:num:];
                     ops  := q.ops |> )`;
Definition has been stored under "emptyp_def"
> val emptyp_def =
    |- !(p :?'a :ty => ty. ('a, num) sched_q).
         emptyp p =
         (let (:'a :ty => ty,(q :('a, num) sched_q)) = p
          in
            pack
              (:'a :ty => ty,
                <|this := q.ops.emptyq [:num:]; ops := q.ops|>))
     : thm
\end{verbatim}
\end{session}

\noindent
Here is a function to insert an element into a scheduling package.
\begin{session}
\begin{verbatim}
- val insertp_def = Define
   `insertp i (p:^sched_queue_ty') =
      let (:'a,q) = p in
        pack(:'a, <| this := q.ops.insert i q.this;
                     ops  := q.ops |> )`;
Definition has been stored under "insertp_def"
> val insertp_def =
    |- !(i :num) (p :?'a :ty => ty. ('a, num) sched_q).
         insertp i p =
         (let (:'a :ty => ty,(q :('a, num) sched_q)) = p
          in
            pack
              (:'a :ty => ty,
                <|this := q.ops.insert [:num:] i q.this; ops := q.ops|>))
     : thm
\end{verbatim}
\end{session}

\noindent
This function removes an element of the scheduling package and returns it,
paired with the diminished scheduling package.
\begin{session}
\begin{verbatim}
- val removep_def = Define
   `removep (p:^sched_queue_ty') =
      let (:'a,q) = p in
      let (x,this') = q.ops.remove q.this in
        (x, pack(:'a, <| this := this';
                         ops  := q.ops |>))`;
Definition has been stored under "removep_def"
> val removep_def =
    |- !(p :?'a :ty => ty. ('a, num) sched_q).
         removep p =
         (let (:'a :ty => ty,(q :('a, num) sched_q)) = p
          in
            let ((x :num),(this' :num 'a)) = q.ops.remove [:num:] q.this
            in
              (x,(pack (:'a :ty => ty,<|this := this'; ops := q.ops|>))))
     : thm
\end{verbatim}
\end{session}

\noindent
Similarly, we can lift the definition of the properties
of a scheduling queue to packages.
\begin{session}
\begin{verbatim}
- val is_scheduling_p_def = Define
   `is_scheduling_p (p : ?'b. ('b,'a)sched_q) =
      let (:'a,q) = p in
      is_scheduling_q q.ops`;
Definition has been stored under "is_scheduling_p_def"
> val is_scheduling_p_def =
    |- !(p :?'b :ty => ty. ('b, 'a) sched_q).
         is_scheduling_p p <=>
         (let (:'b :ty => ty,(q :('b, 'a) sched_q)) = p
          in
            is_scheduling_q q.ops) : thm
\end{verbatim}
\end{session}

The type checking of packages ensures that the representation
type of the package is not disclosed outside of the
{\bf let} \ldots {\bf in} form. If we try to do this,
say by returning the internal data structure that holds
the elements of the queue,
\begin{session}
\begin{verbatim}
- val thisp_def = Define
   `thisp (p:^sched_queue_ty') =
      let (:'a,q) = p in
        q.this`;
\end{verbatim}
\end{session}

\noindent
we would see an error message like the following.
\begin{session}
\begin{verbatim}
Exception raised at Preterm.typecheck:
roughly on line 326, characters 8-13:

Type inference failure: unable to infer a type for the application of

(UNPACK :(!('x :ty => ty). ('x, num) sched_q -> num ('a :ty => ty)) ->
         (?('y :ty => ty). ('y, num) sched_q) -> num 'a)

roughly on line 325, characters 10-20

to

\:'a :ty => ty. (\(q :('a, num) sched_q). q.this)

roughly on line 326, characters 8-13

which has type

:!'a :ty => ty. ('a, num) sched_q -> num 'a

unification failure message: unify failed

! Uncaught exception: 
! HOL_ERR
\end{verbatim}
\end{session}

Now we can take each of our implementations, define their
packaged versions, and show that the packaged version satisfies
the predicate {\tt is\_scheduling\_p}.

Here we define the packaged version of the reference scheduling
queue.
\begin{session}
\begin{verbatim}
- val reference_p_def = Define
     `reference_p = pack(:list, reference_q) : ^sched_queue_ty'`;
Definition has been stored under "reference_p_def"
> val reference_p_def =
    |- (reference_p :?'a :ty => ty. ('a, num) sched_q) =
       (pack (:list,(reference_q :(list, num) sched_q)))
     : thm

- val reference_p_is_scheduling_p = store_thm(
     "reference_p_is_scheduling_p",
     ``is_scheduling_p reference_p``,
     SIMP_TAC bool_ss [is_scheduling_p_def,reference_p_def,
                       reference_q_is_scheduling_q]
  );
> val reference_p_is_scheduling_p =
    |- is_scheduling_p (reference_p :?'a :ty => ty. ('a, num) sched_q)
     : thm
\end{verbatim}
\end{session}

This defines the packaged version of the efficient scheduling
queue.
\begin{session}
\begin{verbatim}
- val efficient_p_def = Define
     `efficient_p = pack(:\'a. 'a list # 'a list, efficient_q)
                     : ^sched_queue_ty'`;
Definition has been stored under "efficient_p_def"
> val efficient_p_def =
    |- (efficient_p :?'a :ty => ty. ('a, num) sched_q) =
       (pack
          (:'a list prod o list,
            (efficient_q :('b list prod o list, num) sched_q)))
     : thm
                     
- val efficient_p_is_scheduling_p = store_thm(
     "efficient_p_is_scheduling_p",
     ``is_scheduling_p efficient_p``,
     SIMP_TAC bool_ss [is_scheduling_p_def,efficient_p_def,
                       efficient_q_is_scheduling_q]
  );
> val efficient_p_is_scheduling_p =
    |- is_scheduling_p (efficient_p :?'a :ty => ty. ('a, num) sched_q)
     : thm
\end{verbatim}
\end{session}

And finally, this defines the packaged version of the stack
scheduling queue.
\begin{session}
\begin{verbatim}
- val stack_p_def = Define
     `stack_p = pack(:list, stack_q) : ^sched_queue_ty'`;
Definition has been stored under "stack_p_def"
> val stack_p_def =
    |- (stack_p :?'a :ty => ty. ('a, num) sched_q) =
       (pack (:list,(stack_q :(list, num) sched_q)))
     : thm

- val stack_p_is_scheduling_p = store_thm(
     "stack_p_is_scheduling_p",
     ``is_scheduling_p stack_p``,
     SIMP_TAC bool_ss [is_scheduling_p_def,stack_p_def,
                       stack_q_is_scheduling_q]
  );
> val stack_p_is_scheduling_p =
    |- is_scheduling_p (stack_p :?'a :ty => ty. ('a, num) sched_q)
     : thm
\end{verbatim}
\end{session}

Programmers on a project using these scheduling queues will need
to be able to access the desired properties of the queue, even if 
they do not know exactly what the implementation is. 
Furthermore, since they will not have access to the internal
record structure of the implementation, we should express these 
properties in terms of the functions they do have access to,
that is, the functions defined above that work on packages, 
namely {\tt countp}, {\tt emptyp}, {\tt insertp}, and {\tt removep}.

To make the properties of a scheduling package
easily available, we will first prove that any package satisfying
{\tt is\_scheduling\_p} will be guaranteed of fulfilling the following
properties.
\begin{session}
\begin{verbatim}
- val scheduling_p_props = store_thm(
   "scheduling_p_props",
   ``!(p: ^sched_queue_ty').
        is_scheduling_p p ==>
        (!x. countp x (emptyp p) = 0) /\
        (!x y. countp x (insertp y p) =
               if x = y then countp x p + 1
                        else countp x p) /\
        (~(!x. countp x p = 0) ==>
               let (y,p') = removep p
               in !x. countp x p = countp x (insertp y p'))``,
   GEN_TAC
   THEN REWRITE_TAC [is_scheduling_p_def]
   THEN STRIP_ASSUME_TAC (ISPEC ``p:^sched_queue_ty'`` PACK_ONTO_AX)
   THEN ASM_REWRITE_TAC []
   THEN SIMP_TAC bool_ss []
   THEN REWRITE_TAC [is_scheduling_q_def]
   THEN REPEAT STRIP_TAC
   THENL
     [ SRW_TAC [] [countp_def,emptyp_def],

       SRW_TAC [] [countp_def,insertp_def],

       POP_ASSUM MP_TAC
       THEN SRW_TAC [] [countp_def,removep_def,insertp_def]
       THEN FIRST_ASSUM (MP_TAC o Q.SPEC `t.this` o TY_SPEC ``:num``)
       THEN COND_CASES_TAC
       THENL
         [ POP_ASSUM (STRIP_ASSUME_TAC o SPEC ``x:num``),

           SRW_TAC [] [LET_DEF]
         ]
     ]
  );
> val scheduling_p_props =
    |- !(p :?'a :ty => ty. ('a, num) sched_q).
         is_scheduling_p p ==>
         (!(x :num). countp x (emptyp p) = (0 :num)) /\
         (!(x :num) (y :num).
            countp x (insertp y p) =
            if x = y then countp x p + (1 :num) else countp x p) /\
         (~(!(x :num). countp x p = (0 :num)) ==>
          (let ((y :num),(p' :?'a :ty => ty. ('a, num) sched_q)) = removep p
           in
             !(x :num). countp x p = countp x (insertp y p')))
     : thm
\end{verbatim}
\end{session}

For convenience, we can break this up into three theorems for each of the
properties, named by the operation they concentrate on.
\begin{session}
\begin{verbatim}
- val emptyp_prop = store_thm(
   "emptyp_prop",
   ``!(p: ^sched_queue_ty').
        is_scheduling_p p ==>
        !x. countp x (emptyp p) = 0``,
   SIMP_TAC bool_ss [scheduling_p_props]
  );
> val emptyp_prop =
    |- !(p :?'a :ty => ty. ('a, num) sched_q).
         is_scheduling_p p ==> !(x :num). countp x (emptyp p) = (0 :num)
     : thm

- val insertp_prop = store_thm(
   "insertp_prop",
   ``!(p: ^sched_queue_ty').
        is_scheduling_p p ==>
        !x y. countp x (insertp y p) =
               if x = y then countp x p + 1
                        else countp x p``,
   SIMP_TAC bool_ss [scheduling_p_props]
  );
> val insertp_prop =
    |- !(p :?'a :ty => ty. ('a, num) sched_q).
         is_scheduling_p p ==>
         !(x :num) (y :num).
           countp x (insertp y p) =
           if x = y then countp x p + (1 :num) else countp x p
     : thm

- val removep_prop = store_thm(
   "removep_prop",
   ``!(p: ^sched_queue_ty').
        is_scheduling_p p ==>
        ~(!x. countp x p = 0) ==>
               let (y,p') = removep p
               in !x. countp x p = countp x (insertp y p')``,
   SIMP_TAC bool_ss [scheduling_p_props]
  );
> val removep_prop =
    |- !(p :?'a :ty => ty. ('a, num) sched_q).
         is_scheduling_p p ==>
         ~(!(x :num). countp x p = (0 :num)) ==>
         (let ((y :num),(p' :?'a :ty => ty. ('a, num) sched_q)) = removep p
          in
            !(x :num). countp x p = countp x (insertp y p'))
     : thm
\end{verbatim}
\end{session}
These properties hold about any scheduling queue package
%that we can define 
that satisfies {\tt is\_scheduling\_p}.
In particular, the properties hold for the example implementations
defined earlier.

\begin{session}
\begin{verbatim}
- val _ = set_trace "types" 0;
- val reference_p_props = save_thm(
   "reference_p_props",
   MATCH_MP scheduling_p_props reference_p_is_scheduling_p);
> val reference_p_props =
    |- (!x. countp x (emptyp reference_p) = 0) /\
       (!x y.
          countp x (insertp y reference_p) =
          if x = y then
            countp x reference_p + 1
          else
            countp x reference_p) /\
       (~(!x. countp x reference_p = 0) ==>
        (let (y,p') = removep reference_p
         in
           !x. countp x reference_p = countp x (insertp y p')))
     : thm
- val efficient_p_props = save_thm(
   "efficient_p_props",
   MATCH_MP scheduling_p_props efficient_p_is_scheduling_p);
> val efficient_p_props =
    |- (!x. countp x (emptyp efficient_p) = 0) /\
       (!x y.
          countp x (insertp y efficient_p) =
          if x = y then
            countp x efficient_p + 1
          else
            countp x efficient_p) /\
       (~(!x. countp x efficient_p = 0) ==>
        (let (y,p') = removep efficient_p
         in
           !x. countp x efficient_p = countp x (insertp y p')))
     : thm
- val stack_p_props = save_thm(
   "stack_p_props",
   MATCH_MP scheduling_p_props stack_p_is_scheduling_p);
> val stack_p_props =
    |- (!x. countp x (emptyp stack_p) = 0) /\
       (!x y.
          countp x (insertp y stack_p) =
          if x = y then countp x stack_p + 1 else countp x stack_p) /\
       (~(!x. countp x stack_p = 0) ==>
        (let (y,p') = removep stack_p
         in
           !x. countp x stack_p = countp x (insertp y p')))
     : thm
\end{verbatim}
\end{session}

This concludes the exercise on scheduling queues. The purpose of this
exercise was to show how a modular boundary can be established
between the implementation of a data structure and the rest of the
program that uses that data structure. The specific properties
are freely chosen by the system design team, and form a good
documentation of the interface at this modular boundary.

In this fashion, packages and existential types promote good
programming practices, and this also contributes to a cleaner
overall system, more resilient under future changes, and a
considerable lessening of the proof effort required to
revalidate the system when such changes occur.






%The meaning of a package {\bf pack}$({:}\sigma, t)$ is a special kind
%of pair, consisting of the type $\sigma$ and the term $t$.
%The meaning of the existential type $\exists\alpha.\sigma'$ 
%is a set of such pairs,
%such that for each such pair ($\sigma_i$, $t_i$) we have that
%the type of $t_i$ is $\sigma'[\sigma_i / \alpha]$.
%
%Existential types like $\exists\alpha.\sigma'$ bind the type variable
%$\alpha$ over the body of the type, $\sigma'$. 
%Existential types are identified up to 
%alpha-equivalence, so that $\exists\alpha.\sigma'$ and 
%$\exists\beta.(\sigma'[\beta / \alpha])$ are the same type.


%%% Local Variables:
%%% mode: latex
%%% TeX-master: "tutorial"
%%% End:
