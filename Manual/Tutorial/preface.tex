\newcommand\holn{\textsf{hol98}}

\chapter*{Preface}\markboth{Preface}{Preface}
\label{intro}

This volume contains a tutorial on the \HOL{} system.  It is one of three
documents making up the documentation for \HOL:

\begin{myenumerate}
\item \TUTORIAL: a tutorial introduction to \HOL.
\item \DESCRIPTION: a description of higher order logic, the \ML\
  programming language, and theorem proving methods in the \HOL\
  system;
\item \REFERENCE: the reference documentation of the tools available
  in \HOL.
\end{myenumerate}

\noindent These three documents will be referred to by the short names (in
small slanted capitals) given above.

This document, \TUTORIAL, is intended to be the first item read by new
users of \HOL.  It provides a self-study introduction to the structure
and use of the system.  The tutorial is intended to give a `hands-on'
feel for the way \HOL{} is used, but it does not systematically explain
all the underlying principles (\DESCRIPTION, explains these).  After
working through \TUTORIAL\ the reader should be capable of using \HOL\
for simple tasks, and should also be in a position to consult the
other two documents.

\section*{Getting started}

Chapter~\ref{install} explains how to get and install \HOL.  Once this
is done, the potential \HOL{} user should become familiar with the
following subjects:

\begin{enumerate}
\item The programming meta-language \ML, and how to interact with it
  through an editor.
\item The formal logic supported by the \HOL{} system (higher order
  logic) and its manipulation via \ML.
\item Forward proof and derived rules of inference.
\item Goal directed proof, tactics and tacticals.
\end{enumerate}

Chapters 1--3 introduce the first two of these topics.
Chapter~\ref{chap:euclid} then develops an extended example (Euclid's
proof of the infinitude of primes) to demonstrate how \HOL{} is used
to prove theorems.  This example is intended to demonstrate \HOL{}'s
capabilities and to explain some of the issues at a high level.
Chapters~\ref{proof} and~\ref{tactics} then describe forward and goal
directed proof in much greater detail.

Chapter~\ref{parity} consists of a worked example: the specification
and verification of a simple sequential parity checker.  The intention
is to accomplish two things: (i) to present a complete piece of work
with \HOL; and (ii) to give an idea of what it is like to use the
\HOL{} system for a tricky proof.

Chapter~\ref{chap:more-examples} briefly discusses some of the
examples distributed with \holn{} in the \ml{examples} directory.

%\item Chapter~\ref{tool} shows how a special purpose proof tool (a
%  conjunction normaliser) can be implemented and optimised. It
%  illustrates methods for `tuning' proof generating programs and
%  discusses trade-offs between generality and efficiency.

%\item Chapter~\ref{binomial} is a proof of the Binomial Theorem in a
%  ring.  It is a medium sized worked example whose subject matter is
%  probably more widely known than any specific piece of hardware or
%  software. The small amount of algebra and mathematical notation
%  needed to state and prove the Binomial Theorem is presented; the
%  notation is expressed in \HOL{}, and the structure of the proof is
%  outlined.

%\end{itemize}

\vspace{1cm}

\noindent \TUTORIAL{} has been kept short so that new users of \HOL{} can get
going as fast as possible. Sometimes details have been simplified. It
is recommended that as soon as a topic in \TUTORIAL\ has been
digested, the relevant bits of \DESCRIPTION\ and \REFERENCE\ be
studied.

%%% Local Variables:
%%% mode: latex
%%% TeX-master: "tutorial"
%%% End:
