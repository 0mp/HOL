\chapter*{Preface}\markboth{Preface}{Preface}
\label{intro}

This volume contains a tutorial on the \HOLW{} system.  It is one of four
documents making up the documentation for \HOLW:

\begin{myenumerate}
\item \LOGIC: a formal description of the higher order logic
  implemented by the \HOLW{} system.
\item \TUTORIAL: a tutorial introduction to \HOLW, with case studies.
\item \DESCRIPTION: a detailed user's guide for the \HOLW{} system;
\item \REFERENCE: the reference manual for \HOLW.
\end{myenumerate}

\noindent These four documents will be referred to by the short names (in
small slanted capitals) given above.

This document, \TUTORIAL, is intended to be the first item read by new
users of \HOLW.  It provides a self-study introduction to the structure
and use of the system.  The tutorial is intended to give a `hands-on'
feel for the way \HOLW{} is used, but it does not systematically
explain all the underlying principles (\DESCRIPTION{} and \LOGIC{}
explain these).  After working through \TUTORIAL\ the reader should be
capable of using \HOLW\ for simple tasks, and should also be in a
position to consult the other documents.

\section*{Getting started}

Experienced \HOL{} users who are eager to get started with \HOLW{}
will want to read chapters 1, 4, 5, and
\ref{chap:adt} through \ref{chap:package}.
Others who are not familiar with \HOL{} should read chapters 1 through 5
and then select chapters with examples that match their interests.
Chapter~\ref{install} explains how to get and install \HOLW.  Once this
is done, the potential \HOLW{} user should become familiar with the
following subjects:
%
\begin{enumerate}
\item The programming meta-language \ML, and how to interact with it.
\item The formal logic supported by the \HOLW{} system
  and its manipulation via \ML.
\item Forward proof and derived rules of inference.
\item Goal directed proof, tactics, and tacticals.
\end{enumerate}
%
Chapters 2 and 3 introduce these topics. Then the new, additional
concepts and features of the \HOLW{} logic (higher order logic 
extended with System {\it F}, kinds, and ranks)
are first casually demonstrated, as an appetizer, and then
discussed in more detail, in chapters \ref{chap:appetizers} and \ref{chap:HOLWlogic}. 
After this, a series of examples are presented using the \HOLW{} system.
Chapters \ref{chap:euclid}~through~\ref{chap:proof-tools} use only the
classical higher order logic subset of \HOLW, and are
the same as for the existing \HOL{} theorem prover.
The new, extended features of \HOLW{} are demonstrated in
examples in chapters \ref{chap:adt} through \ref{chap:simple-cat}.

Chapter~\ref{chap:euclid} develops an extended example --- Euclid's
proof of the infinitude of primes---to illustrate how \HOLW{} is used
to prove theorems.

%Chapter~\ref{proof} then describes forward and goal
%directed proof in much greater detail.

Chapter~\ref{parity} features another worked example: the specification
and verification of a simple sequential parity checker.  The intention
is to accomplish two things: (i) to present another complete piece of
work with \HOLW; and (ii) to give an idea of what it is like to use the
\HOLW{} system for a tricky proof.  Chapter~\ref{chap:combin} is a more
extensive example: the proof of confluence for combinatory
logic.  Again, the aim is to present a complete piece of non-trivial
work.

Chapter~\ref{chap:proof-tools} gives an example of implementing a
proof tool of one's own.  This demonstrates the programmability of
\HOLW: the way in which technology for solving specific problems can be
implemented on top of the underlying kernel.  With high-powered tools
to draw on, it is possible to write prototypes very quickly.

Chapter~\ref{chap:adt} shows how one can construct an abstract data
type in the \HOLW{} logic, where a new type representing an abstract
algebra can be created where the only thing known about it is its
defining property.  This property might not completely specify its
behavior in all circumstances, which allows for under-specification
of the new type. This is a means for information
hiding and proper modularization of large proofs.

Chapter~\ref{chap:simple-cat} exercises the kind system of the
new logic, exploiting the ability to have type variables which
represent type operators, as well as the ability to quantify
type variables over expressions, to explore some of the concepts
of category theory, defining functors and natural transformations
as a shallow embedding in the \HOLW{} logic, and showing how they 
may be combined in a fluid way.

Chapter~\ref{chap:package} gives a thorough discussion and
exercise of existential types and packages. It shows how these can
be used to support in a practical and principled way good
software engineering practices such as modularity and information
hiding.

Chapter~\ref{chap:more-examples} briefly discusses some of the
examples distributed with \holnw{} in the \ml{examples} directory.

%\item Chapter~\ref{tool} shows how a special purpose proof tool (a
%  conjunction normaliser) can be implemented and optimised. It
%  illustrates methods for `tuning' proof generating programs and
%  discusses trade-offs between generality and efficiency.

%\item Chapter~\ref{binomial} is a proof of the Binomial Theorem in a
%  ring.  It is a medium sized worked example whose subject matter is
%  probably more widely known than any specific piece of hardware or
%  software. The small amount of algebra and mathematical notation
%  needed to state and prove the Binomial Theorem is presented; the
%  notation is expressed in \HOLW{}, and the structure of the proof is
%  outlined.

%\end{itemize}

\vspace{1cm}

\noindent \TUTORIAL{} has been kept short so that new users of \HOLW{} can get
going as fast as possible. Sometimes details have been simplified. It
is recommended that as soon as a topic in \TUTORIAL\ has been
digested, the relevant parts of \DESCRIPTION\ and \REFERENCE\ be
studied.

%%% Local Variables:
%%% mode: latex
%%% TeX-master: "tutorial"
%%% End:
