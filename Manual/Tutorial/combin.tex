\documentclass[12pt]{article}
%poor man's infer
\newcommand{\infer}[2]{\begin{array}{c}#2\\ \hline #1\end{array}}
\newcommand{\eos}{\hfill{}$\cdots\diamond\cdots$\hfill{}\vspace{5mm}}

\usepackage{pstricks}
\usepackage{pst-node}
\usepackage{alltt}
% =====================================================================
%
% Macros for typesetting the HOL system manual
%
% =====================================================================

% ---------------------------------------------------------------------
% Abbreviations for words and phrases
% ---------------------------------------------------------------------
\newcommand\VERSION{{\small\tt 2.0}}
\newcommand\TUTORIAL{{\footnotesize\sl TUTORIAL}}
\newcommand\DESCRIPTION{{\footnotesize\sl DESCRIPTION}}
\newcommand\REFERENCE{{\footnotesize\sl REFERENCE}}
\newcommand\LIBRARIES{{\footnotesize\sl LIBRARIES}}

\def\HOL{{\small HOL}}
\def\LCF{{\small LCF}}
\def\LCFLSM{{\small LCF{\kern-.2em}{\normalsize\_}{\kern0.1em}LSM}}
\def\PPL{{\small PP}{\kern-.095em}$\lambda$}
\def\PPLAMBDA{{\small PPLAMBDA}}
\def\ML{{\small ML}}

\newcommand\ie{\mbox{i{.}e{.}}}
\newcommand\eg{\mbox{e{.}g{.}}}
\newcommand\viz{\mbox{viz{.}}}
\newcommand\adhoc{\mbox{\it ad hoc}}
\newcommand\etal{{\it et al.\/}}
\newcommand\etc{\mbox{etc{.}}}

% ---------------------------------------------------------------------
% Simple abbreviations and macros for mathematical typesetting
% ---------------------------------------------------------------------

\newcommand\fun{{\to}}
\newcommand\prd{{\times}}

\newcommand\conj{\ \wedge\ }
\newcommand\disj{\ \vee\ }
\newcommand\imp{ \Rightarrow }
\newcommand\eqv{\ \equiv\ }
\newcommand\cond{\rightarrow}
\newcommand\vbar{\mid}
\newcommand\turn{\ \vdash\ }
\newcommand\hilbert{\varepsilon}
\newcommand\eqdef{\ \equiv\ }

\newcommand\natnums{\mbox{${\sf N}\!\!\!\!{\sf N}$}}
\newcommand\bools{\mbox{${\sf T}\!\!\!\!{\sf T}$}}

\newcommand\p{$\prime$}
\newcommand\f{$\forall$\ }
\newcommand\e{$\exists$\ }

\newcommand\orr{$\vee$\ }
\newcommand\negg{$\neg$\ }

\newcommand\arrr{$\rightarrow$}
\newcommand\hex{$\sharp $}

\newcommand{\uquant}[1]{\forall #1.\ }
\newcommand{\equant}[1]{\exists #1.\ }
\newcommand{\hquant}[1]{\hilbert #1.\ }
\newcommand{\iquant}[1]{\exists ! #1.\ }
\newcommand{\lquant}[1]{\lambda #1.\ }

\newcommand{\leave}[1]{\\[#1]\noindent}
\newcommand\entails{\mbox{\rule{.3mm}{4mm}\rule[2mm]{.2in}{.3mm}}}

% ---------------------------------------------------------------------
% Font-changing commands
% ---------------------------------------------------------------------

\newcommand{\theory}[1]{\hbox{{\small\tt #1}}}

\newcommand{\con}[1]{{\sf #1}}
\newcommand{\rul}[1]{{\tt #1}}
\newcommand{\ty}[1]{{\sl #1}}

\newcommand{\ml}[1]{\mbox{{\def\_{\char'137}\small\tt #1}}}
\newcommand\ms{\tt}
\newcommand{\s}[1]{{\small #1}}

\newcommand{\pin}[1]{{\bf #1}}
\def\m#1{\mbox{\normalsize$#1$}}

% ---------------------------------------------------------------------
% Abbreviations for particular mathematical constants etc.
% ---------------------------------------------------------------------

\newcommand\T{\con{T}}
\newcommand\F{\con{F}}
\newcommand\OneOne{\con{One\_One}}
\newcommand\OntoSubset{\con{Onto\_Subset}}
\newcommand\Onto{\con{Onto}}
\newcommand\TyDef{\con{Type\_Definition}}
\newcommand\Inv{\con{Inv}}
\newcommand\com{\con{o}}
\newcommand\Id{\con{I}}
\newcommand\MkPair{\con{Mk\_Pair}}
\newcommand\IsPair{\con{Is\_Pair}}
\newcommand\Fst{\con{Fst}}
\newcommand\Snd{\con{Snd}}
\newcommand\Suc{\con{Suc}}
\newcommand\Nil{\con{Nil}}
\newcommand\Cons{\con{Cons}}
\newcommand\Hd{\con{Hd}}
\newcommand\Tl{\con{Tl}}
\newcommand\Null{\con{Null}}
\newcommand\ListPrimRec{\con{List\_Prim\_Rec}}


\newcommand\SimpRec{\con{Simp\_Rec}}
\newcommand\SimpRecRel{\con{Simp\_Rec\_Rel}}
\newcommand\SimpRecFun{\con{Simp\_Rec\_Fun}}
\newcommand\PrimRec{\con{Prim\_Rec}}
\newcommand\PrimRecRel{\con{Prim\_Rec\_Rel}}
\newcommand\PrimRecFun{\con{Prim\_Rec\_Fun}}

\newcommand\bool{\ty{bool}}
\newcommand\num{\ty{num}}
\newcommand\ind{\ty{ind}}
\newcommand\lst{\ty{list}}

% ---------------------------------------------------------------------
% \minipagewidth = \textwidth minus 1.02 em
% ---------------------------------------------------------------------

\newlength{\minipagewidth}
\setlength{\minipagewidth}{\textwidth}
\addtolength{\minipagewidth}{-1.02em}

% ---------------------------------------------------------------------
% Environment for the items on the title page of a case study
% ---------------------------------------------------------------------

\newenvironment{inset}[1]{\noindent{\large\bf #1}\begin{list}%
{}{\setlength{\leftmargin}{\parindent}%
\setlength{\topsep}{-.1in}}\item }{\end{list}\vskip .4in}

% ---------------------------------------------------------------------
% Macros for little HOL sessions displayed in boxes.
%
% Usage: (1) \setcounter{sessioncount}{1} resets the session counter
%
%        (2) \begin{session}\begin{verbatim}
%             .
%              < lines from hol session >
%             .
%            \end{verbatim}\end{session}
%
%            typesets the session in a numbered box.
% ---------------------------------------------------------------------

\newlength{\hsbw}
\setlength{\hsbw}{\textwidth}
\addtolength{\hsbw}{-\arrayrulewidth}
\addtolength{\hsbw}{-\tabcolsep}
\newcommand\HOLSpacing{13pt}

\newcounter{sessioncount}
\setcounter{sessioncount}{1}

\newenvironment{session}{\begin{flushleft}
 \begin{tabular}{@{}|c@{}|@{}}\hline
 \begin{minipage}[b]{\hsbw}
 \vspace*{-.5pt}
 \begin{flushright}
 \rule{0.01in}{.15in}\rule{0.3in}{0.01in}\hspace{-0.35in}
 \raisebox{0.04in}{\makebox[0.3in][c]{\footnotesize\sl \thesessioncount}}
 \end{flushright}
 \vspace*{-.55in}
 \begingroup\small\baselineskip\HOLSpacing}{\endgroup\end{minipage}\\ \hline
 \end{tabular}
 \end{flushleft}
 \stepcounter{sessioncount}}

% ---------------------------------------------------------------------
% Macro for boxed ML functions, etc.
%
% Usage: (1) \begin{boxed}\begin{verbatim}
%               .
%               < lines giving names and types of mk functions >
%               .
%            \end{verbatim}\end{boxed}
%
%            typesets the given lines in a box.
%
%            Conventions: lines are left-aligned under the "g" of begin,
%            and used to highlight primary reference for the ml function(s)
%            that appear in the box.
% ---------------------------------------------------------------------

\newenvironment{boxed}{\begin{flushleft}
 \begin{tabular}{@{}|c@{}|@{}}\hline
 \begin{minipage}[b]{\hsbw}
% \vspace*{-.55in}
 \vspace*{.06in}
 \begingroup\small\baselineskip\HOLSpacing}{\endgroup\end{minipage}\\ \hline
 \end{tabular}
 \end{flushleft}}

% ---------------------------------------------------------------------
% Macro for unboxed ML functions, etc.
%
% Usage: (1) \begin{hol}\begin{verbatim}
%               .
%               < lines giving names and types of mk functions >
%               .
%            \end{verbatim}\end{hol}
%
%            typesets the given lines exactly like {boxed}, except there's
%            no box.
%
%            Conventions: lines are left-aligned under the "g" of begin,
%            and used to display ML code in verbatim, left aligned.
% ---------------------------------------------------------------------

\newenvironment{hol}{\begin{flushleft}
 \begin{tabular}{c@{}@{}}
 \begin{minipage}[b]{\hsbw}
% \vspace*{-.55in}
 \vspace*{.06in}
 \begingroup\small\baselineskip\HOLSpacing}{\endgroup\end{minipage}\\
 \end{tabular}
 \end{flushleft}}

% ---------------------------------------------------------------------
% Emphatic brackets
% ---------------------------------------------------------------------

\newcommand\leb{\lbrack\!\lbrack}
\newcommand\reb{\rbrack\!\rbrack}


%These macros were included by jac1: they are used in two of the index entries

\def\per{\ml{\%}}
\def\pes{\ml{\%<}}
\def\pee{\ml{>\%}}


%These macros were included by ap; they are used in Chapters 9 and 10
%of the HOL DESCRIPTION

\newcommand{\inds}%standard infinite set
 {\mbox{\rm I}}

\newcommand{\ch}%standard choice function
 {\mbox{\rm ch}}

\newcommand{\den}[1]%denotational brackets
 {[\![#1]\!]}

\newcommand{\two}%standard 2-element set
 {\mbox{\rm 2}}














\title{Combinatory Logic}
\author{Michael Norrish}
\date{}

\newcommand{\KC}{\con{K}}
\newcommand{\SC}{\con{S}}

\begin{document}
\maketitle

\begin{abstract}
A formalisation of combinatory logic, building principally on a
development done by Tom Melham.
\end{abstract}


\section{Introduction}
\label{sec:Introduction}

This small case study is a formalisation of combinatory logic.  This
logic is of foundational importance in theoretical computer science,
and has a very rich theory.  We aim to put HOL through its paces more
than we intend to explore what is such a big field.  Nonetheless, we
hope that as the formalisation proceeds, the nature of the object of
study is reasonably clear.


\section{The Type of Combinators}
\label{sec:Type-Combinators}

The first thing we need to do is define the type of
\emph{combinators}.  There are just two of these, \KC{} and \SC, but
we also need to be able to \emph{combine} them, and for this we need
to introduce the notion of application.  For lack of a better ASCII
symbol, we will use the hash (\#) to represent this in the logic:
\begin{session}\begin{verbatim}
- Hol_datatype `cl = K | S | # of cl => cl`;
> val it = () : unit
\end{verbatim}\end{session}

We also want the \# to be an infix, so we set its fixity to be a tight
left-associative infix:
\begin{session}\begin{verbatim}
- set_fixity ("#", Infixl 1100);
> val it = () : unit
\end{verbatim}\end{session}

% C = S (S (K S) (S (K K) S)) (K K)


\section{Combinator Reductions}
\label{sec:Comb-Reduct}

Combinatory logic is the study of how values of this type can evolve
given various rules describing how they change.  Therefore, our next
step is to define the reductions that combinators can undergo.  There
are two basic rules:
\[\begin{array}{l@{\;\;\rightarrow\;\;}l}
\KC\;x\;y & x\\
\SC\;f\;g\;x & (f x)(g x)
\end{array}\]
Here, in our description outside of HOL, we use juxtaposition instead
of the \#.  Further, juxtaposition is also left-associative, so that
$\con{K}\;x\;y$ should be read as $\con{K}\;\#\;x\;\#\;y$ which is in
turn $(\con{K}\;\#\;x)\;\#\;y$.

Given a term in the logic, we want these reductions to be able to fire
at any point, not just at the top level, so we need two further
congruence rules:\[
\begin{array}{l}
\infer{x\;y\;\;\rightarrow\;\;x'\;y}{x\;\;\rightarrow\;\;x'}\\[5mm]
\infer{x\;y\;\;\rightarrow\;\;x\;y'}{y\;\;\rightarrow\;\;y'}
\end{array}\]
In HOL, we can capture this relation with an inductive definition.
First we set our arrow symbol up as an infix to make everything that
bit prettier:
\begin{session}\begin{verbatim}
- set_fixity ("-->", Infix(NONASSOC, 510));
> val it = () : unit
\end{verbatim}\end{session}
    (By choosing to make our arrow symbol non-associative, we make it
    a parse error to write \verb!x --> y --> z!. It would be nice to
    be able to write this and have it mean \verb!x --> y /\ y --> z!,
    but this is not presently possible with the HOL parser.)

Our next step is to actually define the relation.  The function for
doing this returns three separate theorems, so we bind each
separately:
\begin{session}\begin{verbatim}
val (redn_rules, redn_ind, redn_cases) =
  IndDefLib.Hol_reln
    `(!x y f. x --> y   ==>    f # x --> f # y) /\
     (!f g x. f --> g   ==>    f # x --> g # x) /\
     (!x y.   K # x # y --> x) /\
     (!f g x. S # f # g # x --> (f # x) # (g # x))`;
> val redn_rules =
    |- (!x y f. x --> y ==> f # x --> f # y) /\
       (!f g x. f --> g ==> f # x --> g # x) /\
       (!x y. K # x # y --> x) /\
       !f g x. S # f # g # x --> f # x # (g # x) : thm
  val redn_ind =
    |- !-->'.
         (!x y f. -->' x y ==> -->' (f # x) (f # y)) /\
         (!f g x. -->' f g ==> -->' (f # x) (g # x)) /\
         (!x y. -->' (K # x # y) x) /\
         (!f g x. -->' (S # f # g # x) (f # x # (g # x))) ==>
         !a0 a1. a0 --> a1 ==> -->' a0 a1 : thm
  val redn_cases =
    |- !a0 a1.
         a0 --> a1 =
         (?x y f. (a0 = f # x) /\ (a1 = f # y) /\ x --> y) \/
         (?f g x. (a0 = f # x) /\ (a1 = g # x) /\ f --> g) \/
         (?y. a0 = K # a1 # y) \/
         ?f g x. (a0 = S # f # g # x) /\ (a1 = f # x # (g # x))
    : thm
\end{verbatim}\end{session}
The theorem induction theorem \texttt{redn\_ind} looks a little
strange because the induction predicate is given the name
\texttt{-->'}.  We can change the name to make things prettier with
the function \texttt{RENAME\_VARS\_CONV}, a conversion:
\begin{session}\begin{verbatim}
- val redn_ind = CONV_RULE (RENAME_VARS_CONV ["P"]) redn_ind;
> val redn_ind =
    |- !P.
         (!x y f. P x y ==> P (f # x) (f # y)) /\
         (!f g x. P f g ==> P (f # x) (g # x)) /\
         (!x y. P (K # x # y) x) /\
         (!f g x. P (S # f # g # x) (f # x # (g # x))) ==>
         !a0 a1. a0 --> a1 ==> P a0 a1 : thm
\end{verbatim}\end{session}

Now, using our theorem \texttt{redn\_rules} we can demonstrate single
steps of our reduction relation:
\begin{session}\begin{verbatim}
- PROVE [redn_rules] ``S # (K # x # x) --> S # x``;
Meson search level: ...
> val it = |- S # (K # x # x) --> S # x : thm
\end{verbatim}\end{session}
    The system we have just defined is as powerful as the
    $\lambda$-calculus, Turing machines, and all the other standard
    models of computation.

    One useful result about the combinatory logic is that it is
    \emph{confluent}.  Consider the term $\SC\;z\;(\KC\;\KC)\;(\KC\;
    y\; x)$.  It can make two reductions, to $\SC\;z\;(\KC\;\KC)\;y$
    and also to $(z\;(\KC\;y\;x))\,(\KC\;\KC\;(\KC\;y\;x))$.  Do these
    two choices of reduction mean that from this point on the terms
    have two completely separate histories?  Roughly speaking, to be
    confluent means that the answer to this question is \emph{no}.


\section{Transitive Closure and Confluence}
\label{sec:Transitive-Clos-Conf}

A notion crucial to that of confluence is that of \emph{transitive
  closure}.  We have defined a system that evolves by specifying how
an algebraic value can evolve into possible successor values in one
step.  The natural next question is to ask for a characterisation of
evolution over one or more steps of the $\rightarrow$ relation.

In fact, we will define a relation that holds between two values if
the second can be reached from the first in zero or more steps.  This
is the \emph{reflexive, transitive closure} of our original relation.
However, rather than tie our new definition to our original relation,
we will develop this notion independently and prove a variety of
results that are true of any system, not just our system of
combinatory logic.

So, we begin our abstract digression with another inductive
definition.  Our new constant is \con{RTC}, such that
$\con{RTC}\;R\;x\;y$ is true if it is possible to get from $x$ to $y$
with zero or more ``steps'' of the $R$ relation.  We can express this
idea with just two rules.  The first \[
\infer{\con{RTC}\;R\;x\;x}{} \]
says that it's always possible to get from $x$ to $x$ in zero or more
steps.  The second \[
\infer{\con{RTC}\;R\;x\;z}{R\;x\;y\qquad\con{RTC}\;R\;y\;z}
\] says that if you can take a single step from $x$ to $y$, and then
take zero or more steps to get $y$ to $z$, then it's possible to take
zero or more steps to get between $x$ and $z$.  The realisation of
these rules in HOL is again straightforward:
\begin{session}\begin{verbatim}
val (RTC_rules, RTC_ind, RTC_cases) =
  IndDefLib.Hol_reln `
    (!x.     RTC R x x) /\
    (!x y z. R x y /\ RTC R y z ==> RTC R x z)`;
<<HOL message: inventing new type variable names: 'a>>
> val RTC_rules =
    |- !R. (!x. RTC R x x) /\
           !x y z. R x y /\ RTC R y z ==> RTC R x z : thm
  val RTC_ind =
    |- !R RTC'.
         (!x. RTC' x x) /\
         (!x y z. R x y /\ RTC' y z ==> RTC' x z) ==>
         !a0 a1. RTC R a0 a1 ==> RTC' a0 a1 : thm
  val RTC_cases =
    |- !R a0 a1. RTC R a0 a1 = (a1 = a0) \/
                               ?y. R a0 y /\ RTC R y a1 : thm
\end{verbatim}\end{session}
Now let us go back to the notion of confluence.  We want this to mean
something like: ``though a system may take different paths in the
short-term, it will always end up in the same place''.  Of course, a
notion like ``short-term'' is hopelessly vague.   On the other hand,
we can pin down a reasonable meaning for ``long-term'': let us worry
about just those situations where the values can take no further
steps.  This suggests our next definition (again parameterised by an
arbitrary relation):
\begin{session}\begin{verbatim}
- val stopped_def = Define`stopped R x = !y. ~(R x y)`;
<<HOL message: inventing new type variable names: 'a, 'b>>
Definition has been stored under "stopped_def".
> val stopped_def = |- !R x. stopped R x = !y. ~R x y : thm
\end{verbatim}\end{session}
In other words, a system is $R$-stopped at $x$ if there are no
connections via $R$ to any other values.  (We could have written
\verb!~?y. R x y! as our RHS for the definition above.)

Now we can define confluence.  We will say that a system is confluent
if it always ends up at the same end-point on starting from the same
start-point:
\begin{session}\begin{verbatim}
val confluent_def = Define
  `confluent R =
    !x y z. RTC R x y /\ stopped R y /\
            RTC R x z /\ stopped R z ==>
            (y = z)`;
<<HOL message: inventing new type variable names: 'a>>
Definition has been stored under "confluent_def".
> val confluent_def =
    |- !R.
         confluent R =
         !x y z.
           RTC R x y /\ stopped R y /\
           RTC R x z /\ stopped R z ==> (y = z) : thm
\end{verbatim}\end{session}
    Clearly confluence is a nice property for a system to have.  Note
    though that it does not imply that a system has no infinite
    paths.  Rather it just says that any one starting point has no
    more than one stopping point.

    How might we establish confluence for a system?  One way is to
    show this stronger property:\[
    \forall x\, y\, z.\; \con{RTC}\;R\;x\;y \land
    \con{RTC}\;R\;x\;z\land (y \neq z)\supset
    \exists u.\;\con{RTC}\;R\;y\;u \land \con{RTC}\;R\;z\;u
    \]  This property states of $R$ that we can ``complete the diamond'';
    if we have \[
    \begin{array}{ccc}
      & \rnode{x}{x}\\[3mm]
      \rnode{y}{y} && \rnode{z}{z}
    \end{array}
    \ncline[nodesep=1mm]{->}{x}{y}\ncline[nodesep=1mm]{->}{x}{z}
    \] then there must be a $u$ such that \[
    \begin{array}{ccc}
      & \rnode{x2}{x}\\[3mm]
      \rnode{y2}{y} && \rnode{z2}{z}\\[3mm]
      & \rnode{u}{u}
      \ncline[nodesep=1mm]{->}{x2}{y2}\ncline[nodesep=1mm]{->}{x2}{z2}
      \ncline[nodesepB=1mm,linestyle=dotted]{->}{y2}{u}
      \ncline[nodesepB=1mm,linestyle=dotted]{->}{z2}{u}
    \end{array}
    \]
    Our condition above states that $\con{RTC}\;R$ has this property,
    but this is clearly a property that might be true of any relation.
    In fact, it is known as the Church-Rosser property, so we define
    \con{CR} in the logic thus:
\begin{session}\begin{verbatim}
- val CR_def = Define
    `CR R = !x y z. R x y /\ R x z /\ ~(y = z) ==>
                    ?u. R y u /\ R z u`;
<<HOL message: inventing new type variable names: 'a>>
Definition has been stored under "CR_def".
> val CR_def = |- !R. CR R = !x y z. R x y /\ R x z ==>
                                     ?u. R y u /\ R z u : thm
\end{verbatim}\end{session}
    Now, let us prove our claim about Church-Rosser, that if a
    reflexive and transitive closure is Church-Rosser, then the
    relation is confluent.  We set up the goal, and rewrite with the
    definitions of what it is to be Church-Rosser:
\begin{session}\begin{verbatim}
- g `!R. CR (RTC R) ==> confluent R`;
<<HOL message: inventing new type variable names: 'a>>
> val it =
    Proof manager status: 1 proof.
    1. Incomplete:
         Initial goal:
         !R. CR (RTC R) ==> confluent R

- e (RW_TAC std_ss [CR_def, confluent_def])
OK..
1 subgoal:
> val it =
    y = z
    ------------------------------------
      0.  !x y z. RTC R x y /\ RTC R x z /\ ~(y = z) ==>
                  ?u. RTC R y u /\ RTC R z u
      1.  RTC R x y
      2.  stopped R y
      3.  RTC R x z
      4.  stopped R z
\end{verbatim}\end{session}
    Our Church-Rosser property is now assumption 0, and we can use it
    to infer that there is either a $u$ at the base of the diamond, or
    that $y = z$, which second possibility is what we're after in any
    case.
\begin{session}\begin{verbatim}
- e (`(y = z) \/ ?u. RTC R y u /\ RTC R z u` by PROVE_TAC []);
OK..
Meson search level: .........
1 subgoal:
> val it =
    y = z
    ------------------------------------
      0.  !x y z. RTC R x y /\ RTC R x z /\ ~(y = z) ==>
                  ?u. RTC R y u /\ RTC R z u
      1.  RTC R x y
      2.  stopped R y
      3.  RTC R x z
      4.  stopped R z
      5.  RTC R y u
      6.  RTC R z u
     : goalstack
\end{verbatim}\end{session}
So, from $y$ we can take zero or more steps to get to $u$ and
similarly from $z$.  But, we also know that we're $R$-stopped in both
$y$ and $z$.  We can't take any steps at all from these values.  We
don't have a contradiction, but we can conclude both that $u = y$ and
$u = z$, but this means that $y = z$, which is our goal.  So we can
finish with
\begin{session}\begin{verbatim}
- e (PROVE_TAC [stopped_def, RTC_cases]);
OK..
Meson search level: ..........

Goal proved. [...]
> val it =
    Initial goal proved.
    |- !R. CR (RTC R) ==> confluent R : goalstack
\end{verbatim}\end{session}
Packaged up so as to remove the sub-goal package commands, we can
prove and save the theorem for future use by:
\begin{session}\begin{verbatim}
val CR_RTC_confluent = store_thm(
  "CR_RTC_confluent",
  ``!R. CR (RTC R) ==> confluent R``,
  RW_TAC std_ss [CR_def, confluent_def] THEN
  `(y = z) \/ ?u. RTC R y u /\ RTC R z u` by PROVE_TAC [] THEN
  PROVE_TAC [stopped_def, RTC_cases]);
\end{verbatim}\end{session}
\eos{}

So far so good.  How then do we show Church-Rosser (the ``diamond
property'') for $\con{RTC}\;R$?  The answer that leaps to mind is to
hope that if the original relation is Church-Rosser, then maybe the
reflexive and transitive closure is too.  The theorem we want is \[
\con{CR}\;R \supset \con{CR}\,(\con{RTC}\;R)\]
Graphically, this is hoping that from \[
\begin{array}{ccc}
& \rnode{x1}{x}\\[3mm]
\rnode{y1}{y} && \rnode{z1}{z}\\[3mm]
& \rnode{u1}{u}
\end{array}
\ncline[nodesep=1mm]{->}{x1}{y1}
\ncline[nodesep=1mm]{->}{x1}{z1}
\ncline[nodesepB=1mm,linestyle=dotted]{->}{z1}{u1}
\ncline[nodesepB=1mm,linestyle=dotted]{->}{y1}{u1}
\] we will be able to conclude \[
\begin{array}{ccccc}
& & \rnode{x2}{x}\\[3mm]
& \rnode{y2}{y} && \rnode{z2}{z}\\[3mm]
& & \rnode{u2}{u}\\[3mm]
 \;\;\;\;\rnode{p}{p}\;\;\;\; &&&& \;\;\;\;\rnode{q}{q}\;\;\;\;\\[15mm]
&& \rnode{r}{r}
\end{array}
\ncline[nodesep=1mm]{->}{x2}{y2}
\ncline[nodesep=1mm]{->}{x2}{z2}
\ncline[nodesepB=1mm,linestyle=dotted]{->}{z2}{u2}
\ncline[nodesepB=1mm,linestyle=dotted]{->}{y2}{u2}
\ncline[nodesep=1mm,linestyle=dashed]{->}{y2}{p}
\ncline[nodesep=1mm,linestyle=dashed]{->}{z2}{q}
\ncline[nodesep=1mm,linestyle=dotted]{->}{p}{r}
\ncline[nodesep=1mm,linestyle=dotted]{->}{q}{r}
\] where the dashed lines indicate that these steps (from $x$ to $p$,
for example) are using $\con{RTC}\;R$.  The presence of two instances
of $\con{RTC}\;R$ is an indication that this proof will require two
inductions.  With the first we will prove
\[
\begin{array}{ccccc}
& & & \rnode{x3}{x}\\[3mm]
& & \rnode{y3}{y} && \rnode{z3}{z}\\[3mm]
& & & \rnode{u3}{u}\\[3mm]
 \rnode{p2}{p}\\[3mm]
& \rnode{r2}{r}
\end{array}
\ncline[nodesep=1mm]{->}{x3}{y3}
\ncline[nodesep=1mm]{->}{x3}{z3}
\ncline[nodesepB=1mm,linestyle=dotted]{->}{z3}{u3}
\ncline[nodesepB=1mm,linestyle=dotted]{->}{y3}{u3}
\ncline[nodesep=1mm,linestyle=dashed]{->}{y3}{p2}
\ncline[nodesepB=1mm,linestyle=dotted]{->}{u3}{r2}
\ncline[nodesepB=1mm,linestyle=dotted]{->}{p2}{r2}
\]
In other words, we want to show that if we take one step in one
direction (to $z$) and many steps in another (to $p$), then the
Church-Rosser property for $R$ will guarantee us the existence of $r$,
to which will we be able to take many steps from both $p$ and $z$.

We take some care to state the goal so that after stripping away the
outermost assumption (that $R$ is Church-Rosser), it will match the
induction principle for \con{RTC}.\footnote{In this and subsequent
  proofs using the sub-goal package, we will present the proof manager
  as if the goal to be proved is the first ever on this stack.  In
  other words, we have done a \texttt{dropn 1;} after every successful
  proof to remove the evidence of the old goal.  In practice, there is
  no harm in leaving these goals on the proof manager's stack.}
Further, bitter experience has taught us that it is not necesssary to
worry about the possibility that $p$ and $z$ might be the same, so
there is no need to include that as an assumption.  If we can prove
our result with by assuming less, then it will go through alright when
we come to state our final result.
\begin{session}\begin{verbatim}
- g `!R. CR R ==>
         !x p. RTC R x p ==>
               !z. R x z ==>
                   ?u. RTC R p u /\ RTC R z u`;
<<HOL message: inventing new type variable names: 'a>>
> val it =
    Proof manager status: 1 proof.
    1. Incomplete:
         Initial goal:
         !R.
           CR R ==>
           !x p. RTC R x p ==> !z. R x z ==>
                                   ?u. RTC R p u /\ RTC R z u
\end{verbatim}\end{session}
First, we strip away the Church-Rosser assumption (two things need to
be stripped: the outermost universal quantifier and the antecedent of
the implication):
\begin{session}\begin{verbatim}
- e (GEN_TAC THEN STRIP_TAC);
OK..
1 subgoal:
> val it =
    !x p. RTC R x p ==> !z. R x z ==> ?u. RTC R p u /\ RTC R z u
    ------------------------------------
      CR R
\end{verbatim}\end{session}
Now we can use the induction principle.  We use the higher-order
backward chaining rule, \texttt{HO\_MATCH\_MP\_TAC}, which takes a
theorem of the form $\vdash\;P\supset Q$, tries to instantiate it to
make it $\vdash\;P'\supset Q'$, such that $Q'$ is the same as the goal
to be proved, and then requires the user to prove $P'$.
\begin{session}\begin{verbatim}
- e (HO_MATCH_MP_TAC RTC_ind);
OK..
1 subgoal:
> val it =
    (!x z. R x z ==> ?u. RTC R x u /\ RTC R z u) /\
    !x y z.
      R x y /\ (!z'. R y z' ==> ?u. RTC R z u /\ RTC R z' u) ==>
      !z'. R x z' ==> ?u. RTC R z u /\ RTC R z' u
    ------------------------------------
      CR R
\end{verbatim}\end{session}
Let's strip the goal as much as possible with the aim of making what
remains to be proved easier to see:
\begin{session}\begin{verbatim}
- e (REPEAT STRIP_TAC);
OK..
2 subgoals:
> val it =
    ?u. RTC R z u /\ RTC R z' u
    ------------------------------------
      0.  CR R
      1.  R x y
      2.  !z'. R y z' ==> ?u. RTC R z u /\ RTC R z' u
      3.  R x z'

    ?u. RTC R x u /\ RTC R z u
    ------------------------------------
      0.  CR R
      1.  R x z
\end{verbatim}\end{session}
This first goal is easy.  It corresponds to the case where the many
steps from $x$ to $p$ are actually no steps at all, and $p$ and $x$
are actually the same place.  In the other direction, $x$ has taken
one step to $z$, and we need to find somewhere reachable in zero or
more steps from both $x$ and $z$.  Given what we know so far, the only
candidate is $z$ itself.  In fact, we don't even need to provide this
witness explicitly, \texttt{PROVE\_TAC} will find it for us, as long
as we tell it what the rules governing \con{RTC} are:
\begin{session}\begin{verbatim}
- e (PROVE_TAC [RTC_rules]);
OK..
Meson search level: .....

Goal proved. [...]
Remaining subgoals:
> val it =
    ?u. RTC R z u /\ RTC R z' u
    ------------------------------------
      0.  CR R
      1.  R x y
      2.  !z'. R y z' ==> ?u. RTC R z u /\ RTC R z' u
      3.  R x z'
\end{verbatim}\end{session}
    And what of this remaining goal?  Assumptions one and three
    between them are the top of an $R$-diamond, as long as $y$ and
    $z'$ are different.  Let's use the fact that $R$ is Church-Rosser
    and infer that either they are the same, or that there exists a
    $v$ to which $y$ and $z'$ can both take single steps:
\begin{session}\begin{verbatim}
- e (`(y = z') \/ ?v. R y v /\ R z' v` by PROVE_TAC [CR_def]);
OK..
Meson search level: ............
2 subgoals:
> val it =
    ?u. RTC R z u /\ RTC R z' u
    ------------------------------------
      0.  CR R
      1.  R x y
      2.  !z'. R y z' ==> ?u. RTC R z u /\ RTC R z' u
      3.  R x z'
      4.  R y v
      5.  R z' v

    ?u. RTC R z u /\ RTC R z' u
    ------------------------------------
      0.  CR R
      1.  R x y
      2.  !z'. R y z' ==> ?u. RTC R z u /\ RTC R z' u
      3.  R x z'
      4.  y = z'
\end{verbatim}\end{session}
Now we are really stuck.  All our assumptions allow is the existence
of a single step from $x$ to $y$.  We can't apply our inductive
hypothesis because we don't have proof of any step out of $y$.  Surely
though, there are steps between $y$ and $z$.  Why do we not have
$\con{RTC}\;R\;y\;z$ as an assumption?  The problem is the form of the
induction principle:
\begin{session}\begin{verbatim}
- RTC_ind;
> val it =
    |- !R RTC'.
         (!x. RTC' x x) /\
         (!x y z. R x y /\ RTC' y z ==> RTC' x z) ==>
         !a0 a1. RTC R a0 a1 ==> RTC' a0 a1 : thm
\end{verbatim}\end{session}
    It says that when we are proving the ``step'' or inductive case,
    we can assume the predicate we are trying to prove for $y$ and $z$
    (the \texttt{RTC'}) but doesn't allow us to assume
    $\con{RTC}\;R\;y\;z$.  Such a rule would be valid, and is commonly
    known as the \emph{strong induction principle}.  The strong
    induction principle is something that we could derive
    automatically, but let us strengthen the inductive hypothesis by
    hand.  Inductive proofs often need strengthening in ways similar
    to this, so this will be illustrative.

We restate our goal as
\newcommand{\bk}{\char'134}
\begin{session}\begin{alltt}
- g `!R. CR R ==>
         !x p. RTC R x p ==>
               \psframebox[linestyle=dashed]{RTC R x p /\bk} !z. R x z ==>
                   ?u. RTC R p u /\bk{} RTC R z u`;
[..]
\end{alltt}\end{session}
We apply the same tactics as before, and generate four (!) sub-goals:
\begin{session}\begin{verbatim}
- e (GEN_TAC THEN STRIP_TAC THEN HO_MATCH_MP_TAC RTC_ind THEN
     REPEAT STRIP_TAC);
OK..
4 subgoals:
> val it =
    ?u. RTC R z u /\ RTC R z' u
    ------------------------------------
      0.  CR R
      1.  R x y
      2.  RTC R y z
      3.  !z'. R y z' ==> ?u. RTC R z u /\ RTC R z' u
      4.  R x z'

    RTC R x z
    ------------------------------------
      0.  CR R
      1.  R x y
      2.  RTC R y z
      3.  !z'. R y z' ==> ?u. RTC R z u /\ RTC R z' u

    ?u. RTC R x u /\ RTC R z u
    ------------------------------------
      0.  CR R
      1.  R x z

    RTC R x x
    ------------------------------------
      CR R
\end{verbatim}\end{session}
The first and third goals are new, and represent the need to prove
the inductive cases for the goal
$\con{RTC}\;R\;x\;y\;\;\supset\;\;\con{RTC}\;R\;x\;y$.  They and the
second goal can all be proved with the one tactic (as before):
\begin{session}\begin{verbatim}
- e (PROVE_TAC [RTC_rules]);
\end{verbatim}\end{session}
After three applications of the above, we are left with our last
sub-goal:
\begin{session}\begin{verbatim}
Remaining subgoals:
> val it =
    ?u. RTC R z u /\ RTC R z' u
    ------------------------------------
      0.  CR R
      1.  R x y
      2.  RTC R y z
      3.  !z'. R y z' ==> ?u. RTC R z u /\ RTC R z' u
      4.  R x z'
\end{verbatim}\end{session}
Now we do have that extra assumption about steps of $R$ between $y$
and $z$.  We again appeal to our Church-Rosser assumption:
\begin{session}\begin{verbatim}
- e (`(y = z') \/ ?v. R y v /\ R z' v` by PROVE_TAC [CR_def]);
OK..
Meson search level: ...........
2 subgoals:
> val it =
    ?u. RTC R z u /\ RTC R z' u
    ------------------------------------
      0.  CR R
      1.  R x y
      2.  RTC R y z
      3.  !z'. R y z' ==> ?u. RTC R z u /\ RTC R z' u
      4.  R x z'
      5.  R y v
      6.  R z' v

    ?u. RTC R z u /\ RTC R z' u
    ------------------------------------
      0.  CR R
      1.  R x y
      2.  RTC R y z
      3.  !z'. R y z' ==> ?u. RTC R z u /\ RTC R z' u
      4.  R x z'
      5.  y = z'
\end{verbatim}\end{session}
In this case, the $u$ we must exhibit is simply $z$, and
\texttt{PROVE\_TAC} can figure this out as well as we can:
\begin{session}\begin{verbatim}
- e (PROVE_TAC [RTC_rules]);
OK..
Meson search level: ........

Goal proved. [...]
Remaining subgoals:
> val it =
    ?u. RTC R z u /\ RTC R z' u
    ------------------------------------
      0.  CR R
      1.  R x y
      2.  RTC R y z
      3.  !z'. R y z' ==> ?u. RTC R z u /\ RTC R z' u
      4.  R x z'
      5.  R y v
      6.  R z' v
\end{verbatim}\end{session}
Now we can apply our induction hypothesis (assumption 2) to complete
the long, lop-sided strip of the diamond.  We will conclude that there
is a $u$ such that $\con{RTC}\;R\;z\;u$ and $\con{RTC}\;R\;v\;u$.  We
actually need a $u$ such that $\con{RTC}\;R\;z'\;u$, but because there
is a single $R$-step between $z'$ and $v$ we have that as well.  All
we need to provide \texttt{PROVE\_TAC} is the rules for \con{RTC}:
\begin{session}\begin{verbatim}
- e (PROVE_TAC [RTC_rules]);
OK..
Meson search level: .......

Goal proved. [...]
> val it =
    Initial goal proved.
    |- !R.
         CR R ==> !x p. RTC R x p ==>
                        RTC R x p /\
                        !z. R x z ==> ?u. RTC R p u /\ RTC R z u
\end{verbatim}\end{session}
    Again we can (and should) package up the lemma, avoiding the
    sub-goal package commands:
\begin{session}\begin{verbatim}
val R_RTC_CR_lemma = prove(
  ``!R. CR R ==>
         !x p. RTC R x p ==>
               RTC R x p /\
               !z. R x z ==>
                   ?u. RTC R p u /\ RTC R z u``,
  GEN_TAC THEN STRIP_TAC THEN HO_MATCH_MP_TAC RTC_ind THEN
  REPEAT STRIP_TAC THENL [
    PROVE_TAC [RTC_rules],
    PROVE_TAC [RTC_rules],
    PROVE_TAC [RTC_rules],
    `(y = z') \/ ?v. R y v /\ R z' v` by PROVE_TAC [CR_def] THEN
    PROVE_TAC [RTC_rules]
  ]);
\end{verbatim}\end{session}
    Finally, it's very easy to turn this into the result we really
    want by getting rid of the spurious \texttt{RTC R x p}; we
    simplify the theorem and contextual rewriting eliminates it
    because it's a consequence of the antecendent of the implication:
\begin{session}\begin{verbatim}
val R_RTC_CR = save_thm(
  "R_RTC_CR",
  simpLib.SIMP_RULE std_ss [] R_RTC_CR_lemma);
> val R_RTC_CR =
    |- !R.
         CR R ==> !x p. RTC R x p ==>
                        !z. R x z ==> ?u. RTC R p u /\ RTC R z u
\end{verbatim}\end{session}
\eos{}

Now we can move on to proving that if $R$ is Church-Rosser, so too is
$\con{RTC}\;R$.  We want to prove this by induction again.  It's very
tempting to state the goal as the obvious \[
\con{CR}\;R\supset\con{CR}\,(\con{RTC}\;R)
\] but doing so will actually make it harder to apply the induction
principle when the time is right.  Better to start out with a
statement of the goal that is very near in form to the induction
princple.  So, we manually expand the meaning of \con{CR} and state
our next goal thus:
\begin{session}\begin{verbatim}
- g `!R. CR R ==> !x y. RTC R x y ==>
                        !z. RTC R x z ==>
                            ?u. RTC R y u /\ RTC R z u`;
<<HOL message: inventing new type variable names: 'a>>
> val it =
    Proof manager status: 1 proof.
    1. Incomplete:
         Initial goal:
         !R.
           CR R ==>
           !x y. RTC R x y ==> !z. RTC R x z ==>
                                   ?u. RTC R y u /\ RTC R z u
\end{verbatim}\end{session}
    Again we can do without the condition that $y$ and $z$ be unequal,
    and just as before, we strip the \con{CR} assumption, apply the
    induction principle, and strip repeatedly:
\begin{session}\begin{verbatim}
- e (GEN_TAC THEN STRIP_TAC THEN HO_MATCH_MP_TAC RTC_ind THEN
     REPEAT STRIP_TAC);
OK..
2 subgoals:
> val it =
    ?u. RTC R z u /\ RTC R z' u
    ------------------------------------
      0.  CR R
      1.  R x y
      2.  !z'. RTC R y z' ==> ?u. RTC R z u /\ RTC R z' u
      3.  RTC R x z'

    ?u. RTC R x u /\ RTC R z u
    ------------------------------------
      0.  CR R
      1.  RTC R x z
\end{verbatim}\end{session}
The first goal is again an easy one, corresponding to the case where
the trip from $x$ to $y$ has been one of no steps whatsoever.
\begin{session}\begin{verbatim}
- e (PROVE_TAC [RTC_rules]);
OK..
Meson search level: ...

Goal proved. [...]

Remaining subgoals:
> val it =
    ?u. RTC R z u /\ RTC R z' u
    ------------------------------------
      0.  CR R
      1.  R x y
      2.  !z'. RTC R y z' ==> ?u. RTC R z u /\ RTC R z' u
      3.  RTC R x z'
\end{verbatim}\end{session}
This goal is very similar to the one we saw earlier.  We have the top
of a (``lop-sided'') diamond in assumptions 1 and 3, so we can infer
the existence of a common destination for $y$ and $z'$:
\begin{session}\begin{verbatim}
- e (`?v. RTC R y v /\ RTC R z' v` by PROVE_TAC [R_RTC_CR]);
OK..
Meson search level: ............
1 subgoal:
> val it =
    ?u. RTC R z u /\ RTC R z' u
    ------------------------------------
      0.  CR R
      1.  R x y
      2.  !z'. RTC R y z' ==> ?u. RTC R z u /\ RTC R z' u
      3.  RTC R x z'
      4.  RTC R y v
      5.  RTC R z' v
\end{verbatim}\end{session}
    At this point in the last proof we were able to finish it all off
    by just appealing to the rules for \con{RTC}.  This time it is not
    quite so straightforward.  When we use the induction hypothesis
    (assumption 2), we can conclude that there is a $u$ to which both
    $z$ and $v$ can connect in zero or more steps, but in order to
    show that this $u$ is reachable from $z'$, we need to be able to
    conclude $\con{RTC}\;R\;z'\;u$ when we know that
    $\con{RTC}\;R\;z'\;v$ (assumption 5 above) and
    $\con{RTC}\;R\;v\;u$ (our consequence of the inductive
    hypothesis).  We leave the proof of this general result as an
    exercise, and here assume that it is already proved as the theorem
    \texttt{RTC\_RTC}.
\begin{session}\begin{verbatim}
- e (PROVE_TAC [RTC_rules, RTC_RTC]);
Meson search level: .......

Goal proved. [...]
> val it =
    Initial goal proved.
    |- !R.
         CR R ==>
         !x y. RTC R x y ==> !z. RTC R x z ==>
                                 ?u. RTC R y u /\ RTC R z u
\end{verbatim}\end{session}
We can package this result up as a lemma and then prove the prettier
version directly:
\begin{session}\begin{verbatim}
val CR_RTC_lemma = prove(
  ``!R.
       CR R ==>
       !x y. RTC R x y ==> !z. RTC R x z ==>
                               ?u. RTC R y u /\ RTC R z u``,
  GEN_TAC THEN STRIP_TAC THEN HO_MATCH_MP_TAC RTC_ind THEN
  REPEAT STRIP_TAC THENL [
    PROVE_TAC [RTC_rules],
    `?v. RTC R y v /\ RTC R z' v` by PROVE_TAC [R_RTC_CR] THEN
    PROVE_TAC [RTC_RTC, RTC_rules]
  ]);
val CR_RTC = store_thm(
  "CR_RTC",
  ``!R. CR R ==> CR (RTC R)``,
  PROVE_TAC [CR_def,CR_RTC_lemma]);
\end{verbatim}\end{session}


\subsection{Return to the Land of the Combinators}
\label{sec:Return-to-Land}

Now, surely we are in a position to return to the real object of study
and prove confluence for combinatory logic.  We have done an abstract
development and established that\[
\begin{array}{ccccc}
\con{CR}\;R & \supset & \con{CR}\,(\con{RTC}\;R)\\
& & \land\\
& & \con{CR}\,(\con{RTC}\;R) & \supset & \con{confluent}\;R\\
\end{array}
\]  (We have also established a couple of other useful results along
the way.)

\newcommand{\topk}{\KC\;\SC\;(\KC\;\KC\;\KC)} Sadly, it just isn't
the case that $\rightarrow$ our one-step relation for combinators is
Church-Rosser.  A counter-example is $\topk$.  Its possible evolution
can be described graphically: \[
\begin{array}{ccc}
& \rnode{top}{\topk}\\[5mm]
\rnode{left}{\SC} & & \rnode{right}{\KC\;\SC\;\KC}\\[5mm]
& \rnode{bottom}{\SC}
\end{array}
\psset{nodesep=1mm}
\ncline{->}{top}{left}
\ncline{->}{top}{right}
\ncline{->}{right}{bottom}
\]
If we had Church-Rosser, it should be possible to find a common
destination for $\KC\;\SC\;\KC$ and $\SC$.  However, \SC{} doesn't
admit any reductions whatsoever, so there isn't a common destination.

If we don't have Church-Rosser, what do we have?  More significantly,
what do we have that implies confluence?  If we examine the diagram
above once more, we see that we almost seem to have a diamond property
of a sort.  More importantly, this ``almost diamond'' does give us the
right property.  We can prove:\[
\begin{array}{c}
  \left(
    \begin{array}{ll}
      \forall x\,y\,z. &
      R\;x\;y\;\;\land\;\; R\;x\;z \;\;\;\supset\\
      & \exists u.\;\con{RTC}\;R\;y\;u \;\;\land\;\;\con{RTC}\;R\;z\;u
      \end{array}\right)\\
    \supset\\
    \left(
      \begin{array}{ll}
        \forall x\,y\,z. &
        \con{RTC}\;R\;x\;y\;\;\land\;\;\con{RTC}\;R\;x\;z
        \;\;\;\supset\\
        & \exists u.\;\con{RTC}\;R\;y\;u
        \;\;\land\;\;\con{RTC}\;R\;z\;u
      \end{array}
    \right)
  \end{array}
  \]
Though a large formula, the proof is actually fairly straightforward,
and we again leave it as an exercise.  We will assume in the remainder
of the development that

\section{Exercises}
\label{sec:Exercises}

\begin{enumerate}
\item Prove that \[\con{RTC}\;R \;x\; y \;\;\land \;\;
  \con{RTC}\;R\;y\;z\;\;\;\supset\;\;\; \con{RTC}\;R\;x\;z
\] You will need to prove the goal by induction, and will probably
  need to massage it slightly first to get it to match the appropriate
  induction principle.  Store the theorem under the name
  \texttt{RTC\_RTC}.
\end{enumerate}

\end{document}


%%% Local Variables:
%%% mode: latex
%%% TeX-master: t
%%% End:
