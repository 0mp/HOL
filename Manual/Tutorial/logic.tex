\chapter{The HOL Logic}
\label{HOLlogic}

The \HOL\  system  supports {\it  higher order  logic}.   This is  a version of
predicate calculus with three main extensions:

\begin{itemize}
\item Variables can range over functions and predicates
(hence `higher order').
\item The logic is {\it typed}.
\item There is no separate syntactic category of {\it formulae\/}
(terms of type \ml{bool} fulfill their role).
\end{itemize}

\section{Overview of higher order logic}

It is assumed the reader is familiar with predicate logic.  The syntax
and semantics of the particular logical system supported by \HOL\ is
described in detail in \DESCRIPTION.  The table below summarizes the
notation used.

\begin{center}
\begin{tabular}{|l|l|l|l|} \hline
\multicolumn{4}{|c|}{ } \\
\multicolumn{4}{|c|}{\bf Terms of the HOL Logic} \\
\multicolumn{4}{|c|}{ } \\
{\it Kind of term} & {\it \HOL\ notation} &
{\it Standard notation} &
{\it Description} \\ \hline
 & & & \\
Truth & {\small\verb|T|} & $\top$ & {\it true}\\ \hline
Falsity & {\small\verb|F|} & $\bot$ & {\it false}\\ \hline
Negation & {\small\verb|~|}$t$ & $\neg t$ & {\it not}$\ t$\\ \hline
Disjunction & $t_1${\small\verb|\/|}$t_2$ & $t_1\vee t_2$ &
$t_1\ ${\it or}$\ t_2$ \\ \hline
Conjunction & $t_1${\small\verb|/\|}$t_2$ & $t_1\wedge t_2$ &
$t_1\ ${\it and}$\ t_2$ \\ \hline
Implication & $t_1${\small\verb|==>|}$t_2$ & $t_1\imp t_2$ &
$t_1\ ${\it implies}$\ t_2$ \\ \hline
Equality & $t_1${\small\verb|=|}$t_2$ & $t_1 = t_2$ &
$t_1\ ${\it equals}$\ t_2$ \\ \hline
$\forall$-quantification & {\small\verb|!|}$x${\small\verb|.|}$t$ &
$\uquant{x}t$ & {\it for\ all\ }$x: t$ \\ \hline
$\exists$-quantification & {\small\verb|?|}$x${\small\verb|.|}$t$ &
$\equant{x}\ t$ & {\it for\ some\ }$x: t$ \\ \hline
$\hilbert$-term & {\small\verb|@|}$x${\small\verb|.|}$t$ &
$\hquant{x}t$ & {\it an}$\ x\ ${\it such\ that:}$\ t$ \\ \hline
Conditional & {\small\verb|(if|} $t$ {\small\verb|then|} $t_1$
              {\small\verb|else|} $t_2${\small\verb|)|} &
$(t\rightarrow t_1, t_2)$ & {\it if\ }$t${\it \ then\ }$t_1${\it\ else\ }$t_2$
 \\ \hline
\end{tabular}
\end{center}\label{logic-table}

\paragraph{Note on HOL example sessions}
All of the examples below assume that you are running
\texttt{hol.unquote}, the executable for which is in the \texttt{bin/}
directory along with that for \texttt{hol}.  Further, you need to
execute the following commands before starting the sessions below:
\setcounter{sessioncount}{0}
\begin{session}
\begin{verbatim}
- app load ["arithmeticTheory", "pairTheory", "Psyntax"];
> val it = () : unit
- open Psyntax;
\end{verbatim}
\end{session}

\bigskip

Terms of the \HOL\ logic are represented in \ML\ by an {\it abstract
  type\/}\footnote{Abstract types appear to the user as primitive
  types with a collection of operations; they are described in
  \DESCRIPTION} called {\small\verb|term|}. They are normally input
between double back-quote marks.  For example, the expression
{\small\verb|``x /\ y ==> z``|} evaluates in \ML\ to a term representing
{\small\verb|x|}$\wedge${\small\verb|y|}$\Rightarrow${\small\verb|z|}.
Terms can be manipulated by various built-in \ML\ functions. For
example, the \ML\ function \ml{dest\_imp} with \ML\ type
{\small\verb|term -> term * term|} splits an implication into a pair
of terms consisting of its antecedent and consequent, and the \ML\
function \ml{dest\_conj} of type {\small\verb|term -> term * term|}
splits a conjunction into its two conjuncts.


\setcounter{sessioncount}{1}
\begin{session}
\begin{verbatim}
- ``x /\ y ==> z``;
> val it = `x /\ y ==> z` : term

- dest_imp it;
> val it = (`x /\ y`, `z`) : term * term

- dest_conj(#1 it);
> val it = (`x`, `y`) : term * term
\end{verbatim}
\end{session}

Terms of the \HOL\ logic are quite similar to \ML\ expressions, and
this can at first be confusing.  Indeed, terms of the logic have types
similar to those of \ML\ expressions.  For example,
{\small\verb|``(1,2)``|} is an \ML\ expression with \ML\ type
{\small\verb|term|}.  The \HOL\ type of this term is
{\small\verb|num # num|}.  By contrast, the \ML\ expression
{\small\verb|(``1``, ``2``)|} has type {\small\verb|term * term|}.

The types of \HOL\ terms form an \ML\ type called
{\small\verb|hol_type|}.  Types are usually input by applying the
parsing functionExpressions having the form
{\small\verb|``: |}$\cdots${\small\verb| ``|} evaluate to logical
types.
The built-in function {\small\verb|type_of|} has \ML\ type
{\small\verb|term->type|} and returns the logical type of a term.

\begin{session}
\begin{verbatim}
- ``(1,2)``;
> val it = `(1,2)` : term

- type_of it;
> val it = `:num # num` : hol_type

- (``1``, ``2``);
> val it = (`1`, `2`) : term * term

- type_of(#1 it);
> val it = `:num` : hol_type
\end{verbatim}
\end{session}

To try to minimise confusion between the logical types of \HOL\ terms and
the \ML\ types of \ML\ expressions, the former will be referred to as {\it object
language types\/} and the latter as {\it meta-language types\/}.  For example,
{\small\verb|``(1,T)``|} is an \ML\ expression that has meta-language type
{\small\verb|term|} and evaluates to a term with object language type
{\small\verb|``:num#bool``|}.


\begin{session}
\begin{verbatim}
- ``(1,T)``;
> val it = `(1,T)` : term

- type_of it;
> val it = `:num * bool` : hol_type
\end{verbatim}
\end{session}

\HOL\ terms can be input using explicit {\it quotation\/}, as above, or
they can be constructed using \ML\ constructor functions. The function
{\small\verb|mk_var|} constructs a variable from a string and a type.  In
the example below, three variables of type {\small\verb|bool|} are
constructed.  These are used later.

\begin{session}
\begin{verbatim}
- val x = mk_var("x", ``:bool``)
  and y = mk_var("y", ``:bool``)
  and z = mk_var("z", ``:bool``);
> val x = `x` : term
  val y = `y` : term
  val z = `z` : term
\end{verbatim}
\end{session}

The constructors {\small\verb|mk_conj|} and {\small\verb|mk_imp|} construct
conjunctions and implications respectively.

\begin{session}
\begin{verbatim}
- val t = mk_imp(mk_conj(x,y),z);
> val t = `x /\ y ==> z` : term
\end{verbatim}
\end{session}

\section{Terms}

There are only four different kinds of terms:
\begin{enumerate}
\item Variables.
\item Constants.
\item Function applications: \ml{``$t_1$\ $t_2$``}.
\item $\lambda$-abstractions: {\small\verb|``\|}$x$\ml{.}$t$\ml{``}.
\end{enumerate}

Both variables and constants have a name and a type; the difference is
that constants cannot be bound by quantifiers, and their type is fixed
when they are declared (see below). The type checking algorithm uses
the types of constants to infer the types of variables in the same
quotation. If there is not enough type information type variables will
be guessed:

\begin{session}\begin{verbatim}
- ``~x``;
val it = `~x` : term

- ``x``;
<<HOL message: inventing new type variable names: 'a.>>
> val it = `x` : Term.term
- type_of it;
> val it = `:'a` : hol_type
\end{verbatim}\end{session}

    In the first case, the \HOL\ type checker used the known type
    \ml{bool->bool} of {\small\verb|~|} to deduce that the variable
    \ml{x} must have type \ml{bool}.  In the second case, it cannot
    deduce the type of \ml{x}.  The default `scope' of type
    information for type checking is a single quotation, so a type in
    one quotation cannot affect the type-checking of another.  If
    there is not enough contextually-determined type information to
    resolve the types of all variables in a quotation, then the system
    will guess different type variables for all the unconstrained
    variables.  Alternatively, it is possible to explicitly indicate
    the required types by using \ml{``$term$:$type$``} as illustrated
    below.

\begin{session}\begin{verbatim}
- ``(x,y)``;
<<HOL message: inventing new type variable names: 'a, 'b.>>
> val it = `(x,y)` : term
- type_of it;
> val it = `:'a # 'b` : hol_type

- ``x:num``;
> val it = `x` : term
- type_of it;
> val it = `:num` : hol_type
\end{verbatim}\end{session}

    Functions have types of the form \ml{$\sigma_1$->$\sigma_2$},
    where $\sigma_1$ and $\sigma_2$ are the types of the domain and
    range of the function, respectively.

\begin{session}\begin{verbatim}
- type_of ``$==>``;
> val it = `:bool -> bool -> bool` : hol_type

- type_of ``$+``;
> val it = `:num -> num -> num` : hol_type
\end{verbatim}\end{session}

\noindent Both \ml{+} and \ml{==>} are infixes, so their use in
contexts where they are not being used as such requires their
prefixing by the \texttt{\$}-sign.  This is analogous to the way in
which \texttt{op} is used in \ML. The session below illustrates the
use of these constants as infixes; it also illustrates object language
versus meta-language types.

\begin{session}\begin{verbatim}
- ``(x + 1, t1 ==> t2)``;
> val it = `(x + 1,t1 ==> t2)` : term

- type_of it;
> val it = `:num # bool` : hol_type

- (``x=1``, ``t1==>t2``);
> val it = (`x = 1`, `t1 ==> t2`) : term * term

- (type_of (#1 it), type_of (#2 it));
> val it = (`:bool`, `:bool`) : hol_type * hol_type
\end{verbatim}\end{session}

\noindent The types of constants are declared in {\it theories}; this is
described in Section~\ref{theories}.

An application $t_1\ t_2$ is badly typed if $t_1$ is not a function:

\begin{session}\begin{verbatim}
- ``1 2``;

Type inference failure: unable to infer a type for the application of

(1 :num)

to

(2 :num)

unification failure message: unify failed
! Uncaught exception:
! HOL_ERR <poly>
\end{verbatim}\end{session}

\noindent or if it is a function, but $t_2$ is not in its range:

\begin{session}\begin{verbatim}
- ``~1``;

Type inference failure: unable to infer a type for the application of

$~

to

(1 :num)

unification failure message: unify failed
! Uncaught exception:
! HOL_ERR <poly>
\end{verbatim}\end{session}

    As before, the dollar in front of {\small\verb|~|} indicates that
    the constant has a special syntactic status (in this case a
    non-standard precedence). Putting {\small\verb|$|} in front of any
    symbol causes the parser to ignore any special syntactic status
    (like being an infix) it might have.

\begin{session}\begin{verbatim}
- ``$==> t1 t2``;
> val it = `t1 ==> t2` : term
\end{verbatim}\end{session}

Lambda-terms, or $\lambda$-terms, denote functions. The
symbol `{\small\verb|\|}'
is used as an {\small ASCII} approximation to $\lambda$.
Thus `{\small\verb|\|}$x$\ml{.}$t$' should be read
as `$\lquant{x}t$'. For example,
{\small\verb|"\x. x+1"|} is a term that denotes the function
$n\mapsto n{+}1$.

\begin{session}\begin{verbatim}
- ``\x. x + 1``;
> val it = `\x. x + 1` : term

- type_of it;
> val it = `:num -> num` : hol_type
\end{verbatim}\end{session}

\section{Theories}
\label{theories}

The result of a session with the \HOL\ system is an object called a
{\it theory\/}.  This object is closely related to what a logician
would call a theory, but there are some differences arising from the
needs of mechanical proof.  A \HOL\ theory, like a logician's theory,
contains sets of types, constants, definitions and axioms.  In
addition, however, a \HOL\ theory contains an explicit list of
theorems that have been proved from the axioms and definitions.
Logicians normally do not need to distinguish theorems that have
actually been proved from those that could be proved, hence they do
not normally consider sets of proven theorems as part of a theory;
rather, they take the theorems of a theory to be the (often infinite)
set of all consequences of the axioms and definitions.  Another
difference between logicians' theories and \HOL\ theories is that, for
logicians, theories are relatively static objects, but in \HOL\ they
can be thought of as potentially extendable. For example, the \HOL\
system provides tools for adding to theories and combining theories.
A typical interaction with \HOL\ consists in combining some existing
theories, making some definitions, proving some theorems and then
saving the resulting new theory.

The purpose of the \HOL\ system is to provide tools to enable
well-formed theories to be constructed.  All the theorems of such
theories are logical consequences of the definitions and axioms of the
theory.  The \HOL\ system ensures that only well-formed theories can
be constructed by allowing theorems to be created by {\it formal
  proof\/} only.

A theory is represented in the \HOL\ system as a collection of files,
called theory files.  Each file has a name of the form $name$\ml{.th},
where $name$ is a string supplied by the user.

Theory files are structured hierarchically to represent sequences of
extensions of an initial theory called \ml{HOL}.  Each theory file
making up a theory records some types, constants, axioms and theorems,
together with pointers to other theory files called its {\it
  parents\/}.  This collection of reachable files is called the {\it
  ancestry\/} of the theory file. Axioms, definitions and theorems are
named in the \HOL\ system by two strings: the name of the theory file
where they are stored, together with a name within that file supplied
by the user.  Specifically, axioms, definitions and theorems are named
by a pair of strings $\langle thy,name\rangle$ where $thy$ is the name
of the theory current when the item was declared and $name$ is a
specific name supplied by the user (see the functions \ml{new\_axiom},
\ml{new\_definition} \etc\ below).

A typical piece of work with the \HOL\ system consists in a number of
sessions. In the first of these a new theory, ${\cal T}$ say, is
created by extending existing theories with a number of definitions.
The concrete result of the session will be a theory file ${\cal
  T}$\ml{.th} whose contents are created during the session and whose
ancestry represents the desired logical theory.  In subsequent
sessions this theory is extended by proving new theorems which will be
stored in the file ${\cal T}$\ml{.th}. The logical meaning of these
sessions is that a new extension to ${\cal T}$ is created which
replaces the old version.  Subsequent pieces of work can build on
(\ie\ extend) the definitions and theorems of ${\cal T}$ by making it
a parent of new theories.

There are two modes of working with \HOL: {\it draft mode\/} and {\it
  proof mode\/}.  In draft mode, inconsistencies can be introduced by
asserting inconsistent axioms, but in proof mode only consistency
preserving actions (namely valid proof) can be done.  Draft mode is
analogous to `super user mode' in Unix, in that it gives access to
dangerous facilities.  Everything that can be done in proof mode can
be done in draft mode, but not vice versa.

There is always a {\it current theory\/}, whose name is given by the
function \ml{current\_theory}.  This function maps the value void,
written `\ml{()}' in \ML, to a string giving the name of the current
theory.  Thus the \ML\ expression \ml{current\_theory ()} evaluates to
a string giving the name of the current theory.  Initially \HOL\ is in
proof mode with current theory called \ml{HOL}.  So evaluating
\ml{current\_theory ()} immediately after initiating a \HOL\ session
gives the value \ml{`HOL`}, as can be seen in the session shown below:


\setcounter{sessioncount}{1}
\begin{session}\begin{verbatim}
% hol

          _  _    __    _      __    __
   |___   |__|   |  |   |     |__|  |__|
   |      |  |   |__|   |__   |__|  |__|

          Version 2.0, built on Sep 1 1991

#current_theory();;
`HOL` : string

#
\end{verbatim}\end{session}

    Executing \ml{new\_theory`$thy$`} creates a new theory called
    $thy$; it fails if there already exists a file $thy$\ml{.th} in
    the current working directory (or in any directory on the current
    \HOL\ {\it search path\/}, for a description of which, see the
    appropriate section of \DESCRIPTION).

\begin{session}\begin{verbatim}
#new_theory `Peano`;;
() : void
\end{verbatim}\end{session}

\noindent This starts a theory called \ml{Peano},  which is  to be  made into a
theory containing Peano's postulates as axioms for the natural
numbers.\footnote{An elaborate theory of numbers is built into \HOL,
  in which Peano's postulates are in fact derived theorems rather than
  postulated as axioms.}  These postulates, stated informally, are:

\begin{list}{{\small\bf P\arabic{Peano}}}{\usecounter{Peano}
\setlength{\leftmargin}{12mm}
\setlength{\rightmargin}{7mm}
\setlength{\labelwidth}{6mm}
\setlength{\labelsep}{2mm}
\setlength{\listparindent}{0mm}
\setlength{\itemsep}{14pt plus1pt minus1pt}
\setlength{\topsep}{3mm}
\setlength{\parsep}{0mm}}

\item There is a number $0$.
\item There is a function \Suc\ called the successor function such that
if $n$ is a number then $\Suc \ n$ is a number.
\item $0$ is not the successor of any number.
\item If two numbers have the same successor then they are equal.
\item If a property holds of $0$, and if whenever it holds of a number then it
also holds of the successor of the number, then the property holds of all
numbers. This postulate is called {\it Mathematical Induction}.
\end{list}

To fomalize this in \HOL\ a new type is introduced called \ml{nat}:

\begin{session}\begin{verbatim}
#new_type 0 `nat`;;
() : void
\end{verbatim}\end{session}

    In general \ml{new\_type}$\ n\ \ml{`\ty{op}`}$ makes \ty{op} a new
    $n$-ary type operator in the current theory.  Constant types are
    regarded as degenerate type operators with no arguments, thus the
    new type \ml{nat} is declared to be a $0$-ary type operator. The
    type operator \ml{list} is an example of a $1$-ary operator, types
    built with \ml{list} are written as \ml{$\sigma\ $list}. The type
    operator {\small\verb|prod|} is $2$-ary, types using it are
    written as \ml{($\sigma_1$,$\sigma_2$)prod} (which can, in fact,
    be written as $\sigma_1${\small\verb|#|}$\sigma_2$).

    The axioms {\small\bf P1} and {\small\bf P2} can now be formalized
    by declaring two new constants of type \ml{nat} to represent $0$
    and \Suc.

    Evaluating \ml{new\_constant(`\con{c}`,}$\sigma$\ml{)} makes
    $\con{c}$ a new constant of type $\sigma$ in the current theory.
    This fails if:

\begin{myenumerate}
\item not in draft mode;
\item there already exists a constant named $\con{c}$ in the current
theory.
\end{myenumerate}

\begin{session}\begin{verbatim}
#new_constant(`0`, ":nat");;
evaluation failed     new_constant -- 0 clashes with existing constant
\end{verbatim}\end{session}


The symbol \ml{0}
is already a constant of the built-in theory \ml{num}, and this prevents
us using it as a constant of our new theory. The symbol \ml{O} will be used
instead.

\begin{session}\begin{verbatim}
#new_constant(`O`,":nat");;
() : void

#new_constant(`Suc`, ":nat->nat");;
() : void
\end{verbatim}\end{session}



The \HOL\ type checker ensures that {\small\bf P1} and {\small\bf P2} hold.
{\small\bf P3} is now asserted as an axiom:

\begin{session}\begin{verbatim}
#new_axiom(`P3`, "!n. ~(O = Suc n)");;
|- !n. ~(O = Suc n)
\end{verbatim}\end{session}

\noindent This creates an axiom in the current theory (\ie\ in \ml{Peano}) called
\ml{P3}. Axiom {\small\bf P4} can be declared similarly:


\begin{session}\begin{verbatim}
#new_axiom(`P4`, "!m n. (Suc m = Suc n) ==> (m = n)");;
|- !m n. (Suc m = Suc n) ==> (m = n)
\end{verbatim}\end{session}

The final Peano axiom is Mathematical Induction:

\begin{session}\begin{verbatim}
#new_axiom(`P5`,"!P. P O /\ (!n. P n ==> P(Suc n)) ==> (!n. P n)");;
|- !P. P O /\ (!n. P n ==> P(Suc n)) ==> (!n. P n)
\end{verbatim}\end{session}

To inspect the theory, the function \ml{print\_theory} can be used:

\begin{session}\begin{verbatim}
#print_theory `Peano`;;
The Theory Peano
Parents --  HOL
Types --  ":nat"
Constants --  O ":nat"     Suc ":nat -> nat"
Axioms --
  P3  |- !n. ~(O = Suc n)
  P4  |- !m n. (Suc m = Suc n) ==> (m = n)
  P5  |- !P. P O /\ (!n. P n ==> P(Suc n)) ==> (!n. P n)

******************** Peano ********************
#\end{verbatim}\end{session}

To end the session and write out the various declarations into the theory file
\ml{Peano.th} the function \ml{close\_theory} is used.

\begin{session}\begin{verbatim}
#close_theory();;
() : void

#quit();;
% ls Peano.th
Peano.th
%
\end{verbatim}\end{session}

\noindent The function \ml{quit} exits from the \HOL\ system.

The preceding session set up a theory called \ml{Peano}. It is usual
to include in `Peano arithmetic' axioms defining addition and multiplication.
To do this a new session can be started and the theory extended.

\begin{session}\begin{verbatim}
% hol

          _  _    __    _      __    __
   |___   |__|   |  |   |     |__|  |__|
   |      |  |   |__|   |__   |__|  |__|

          Version 2.0 (Sun3/Franz), built on Sep 1 1991

#extend_theory `Peano`;;
Theory Peano loaded
() : void
\end{verbatim}\end{session}

\noindent The two new axioms can now be added, but first constants \ml{add} and
\ml{mult} must  be  declared  to  represent  addition  and  multiplication (the
symbols \ml{+}  and  \ml{*}  are  already in  use).   These can  be declared as
infixes using \ml{new\_infix} instead of \ml{new\_constant}.

\begin{session}\begin{verbatim}
#new_infix(`add`,":nat->nat->nat");;
() : void

#new_infix(`mult`,":nat->(nat->nat)");;
() : void
\end{verbatim}\end{session}

\noindent Constants declared with \ml{new\_infix}
must have a type of the form
\ml{$\sigma$->($\sigma$->$\sigma$)}.

Axioms defining \ml{add} and \ml{mult} can now be defined.

\begin{session}\begin{verbatim}
#new_axiom
# (`add_def`,
#  "(!n. O add n = n) /\ (!m n. (Suc m) add n = Suc(m add n))");;
|- (!n. O add n = n) /\ (!m n. (Suc m) add n = Suc(m add n))

#new_axiom
# (`mult_def`,
#  "(!n. O mult n = O) /\ (!m n. (Suc m) mult n = (m mult n) add n)");;
|- (!n. O mult n = O) /\ (!m n. (Suc m) mult n = (m mult n) add n)
\end{verbatim}\end{session}

\noindent The theory \ml{Peano} has now been extended to contain the new
definitions.

This example shows how a theory is set up. How to prove consequences of axioms and
definitions is described later. The \HOL\ system contains a built-in
theory of numbers called \ml{num} which contains Peano's postulates and a
(descendent)  theory \ml{arithmetic} containing the definitions of addition
(\ml{+}) and multiplication (\ml{*}) and other things.
In fact, Peano's
postulates are theorems not axioms in the theory \ml{num}. The primitive constants
\ml{0} and \ml{SUC} (corresponding to \ml{O} and \ml{Suc} in \ml{Peano}) are
{\it defined\/} in terms of purely logical notions.
In \HOL, {\it definitions\/} are a special kind of axiom that are
guaranteed to be consistent. The commonest (but not only) form of a definition is:

\[f\ x_1\ \ldots\ x_n = t\]

\noindent where $f$ is declared to be a new constant satisfying this
equation (and $t$ is a term whose free variables are included in the set
$\{x_1,\ldots,x_n\}$).  Such definitions cannot be recursive because, for
example:

\[ f\ x = (f\ x)+1 \]

\noindent would imply $0=1$ (subtract $f\ x$ from both sides)
and is therefore inconsistent. An example of a definition is:

\begin{session}\begin{verbatim}
#new_definition(`Double_def`, "Double x = x add x");;
|- !x. Double x = x add x
\end{verbatim}\end{session}

\noindent This definition both declares \ml{Double} as a new constant of the
appropriate type and asserts the defining equation as a definitional axiom.

\begin{session}\begin{verbatim}
#print_theory `Peano`;;
The Theory Peano
Parents --  HOL
Types --  ":nat"
Constants --
  add ":nat -> (nat -> nat)"     mult ":nat -> (nat -> nat)"
  O ":nat"     Suc ":nat -> nat"     Double ":nat -> nat"
Infixes --
  add ":nat -> (nat -> nat)"     mult ":nat -> (nat -> nat)"
Axioms --
  P3  |- !n. ~(O = Suc n)
  P4  |- !m n. (Suc m = Suc n) ==> (m = n)
  P5  |- !P. P O /\ (!n. P n ==> P(Suc n)) ==> (!n. P n)
  add_def  |- (!n. O add n = n) /\ (!m n. (Suc m) add n = Suc(m add n))
  mult_def
    |- (!n. O mult n = O) /\ (!m n. (Suc m) mult n = (m mult n) add n)

Definitions --  Double_def  |- !x. Double x = x add x
******************** Peano ********************
\end{verbatim}\end{session}


The use of axioms, as illustrated here, carries considerable  danger in general
because it is very easy to assert inconsistent axioms.  It is thus safer to use
only definitions.  At first sight this might appear impossible, but in fact all
of ordinary  mathematics  can  be  developed  from  logic  by definition alone.
Showing this was  the achievement  of Russell  and Whitehead  in their treatise
{\sl Principia  Mathematica}  \cite{Principia}.    A   theory  containing  only
definitions is called a {\it definitional theory\/}.   Many useful definitional
theories are built into the \HOL\ system, or available as libraries.  Examples
include theories of numbers  (both natural  numbers and  integers), sets, bags,
finite trees, group theory, properties of fixed points and more.  Libraries are
constantly being  added  to \HOL\  and users  are encouraged  to contribute new
theories to the library when they develop them.

\newpage % PBHACK

The theory of numbers built into \HOL\ is a definitional theory that defines
numbers logically. Peano's postulates are proved from the definitions of
the type \ml{num} and the constants \ml{0} and \ml{SUC}.
It follows from Peano's postulates that certain kinds of recursion equations are
equivalent to non-recursive definitions. There is a built-in theory called
\ml{prim\_rec} (for `primitive recursion') that supports this,
together with tools for automatically transforming recursion equations into
definitions. This is illustrated later but, for example, here are the built-in
definitions of \ml{+} and \ml{*}:

\begin{session}\begin{verbatim}
#ADD;;
Definition ADD autoloaded from theory `arithmetic`.
ADD = |- (!n. 0 + n = n) /\ (!m n. (SUC m) + n = SUC(m + n))

|- (!n. 0 + n = n) /\ (!m n. (SUC m) + n = SUC(m + n))

#MULT;;
Definition MULT autoloaded from theory `arithmetic`.
MULT = |- (!n. 0 * n = 0) /\ (!m n. (SUC m) * n = (m * n) + n)

|- (!n. 0 * n = 0) /\ (!m n. (SUC m) * n = (m * n) + n)
\end{verbatim}\end{session}

\noindent This also illustrates how certain built-in definitions (and theorems)
are `autoloaded' from the theories where they are defined (\ml{arithmetic} in this
case).

\newpage % PBHACK

The theory  \ml{arithmetic} contains  too many  pre-proved theorems  to show in
full, but  here are  the constants,  definitions, and  the first  two (out of
113) theorems.

\begin{session}\begin{verbatim}
#print_theory `arithmetic`;;
The Theory arithmetic
Parents --  BASIC-HOL     prim_rec
Constants --
  + ":num -> (num -> num)"     - ":num -> (num -> num)"
  * ":num -> (num -> num)"     EXP ":num -> (num -> num)"
  > ":num -> (num -> bool)"     <= ":num -> (num -> bool)"
  >= ":num -> (num -> bool)"     MOD ":num -> (num -> num)"
  DIV ":num -> (num -> num)"
Infixes --
  + ":num -> (num -> num)"     - ":num -> (num -> num)"
  * ":num -> (num -> num)"     EXP ":num -> (num -> num)"
  > ":num -> (num -> bool)"     <= ":num -> (num -> bool)"
  >= ":num -> (num -> bool)"     MOD ":num -> (num -> num)"
  DIV ":num -> (num -> num)"
Definitions --
  ADD  |- (!n. 0 + n = n) /\ (!m n. (SUC m) + n = SUC(m + n))
  SUB
    |- (!m. 0 - m = 0) /\
       (!m n. (SUC m) - n = (m < n => 0 | SUC(m - n)))
  MULT  |- (!n. 0 * n = 0) /\ (!m n. (SUC m) * n = (m * n) + n)
  EXP  |- (!m. m EXP 0 = 1) /\ (!m n. m EXP (SUC n) = m * (m EXP n))
  GREATER  |- !m n. m > n = n < m
  LESS_OR_EQ  |- !m n. m <= n = m < n \/ (m = n)
  GREATER_OR_EQ  |- !m n. m >= n = m > n \/ (m = n)
  DIVISION
    |- !n.
        0 < n ==>
        (!k. (k = ((k DIV n) * n) + (k MOD n)) /\ (k MOD n) < n)

Theorems --
  SUC_NOT  |- !n. ~(0 = SUC n)
  ADD_0  |- !m. m + 0 = m
  .
  .
  .
\end{verbatim}\end{session}

\noindent Note that \ml{DIV} is defined by a constant specification. See
\DESCRIPTION\ for further explanation.
















%%% Local Variables:
%%% mode: latex
%%% TeX-master: "tutorial"
%%% End:
