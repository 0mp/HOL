
\chapter{Epilogue}
\label{chap:teaser_epilogue}

This small sample of the use of the \HOLW{} theorem prover is only intended
to give a taste of what kinds of things are possible. A more complete tutorial
is expected to be released soon.  When it is, there will be an announcement
on the \texttt{hol-info} mailing list, \texttt{hol-info@lists.sourceforge.net}.

In addition to the examples covered in this text, the full
tutorial contains a more complete description of the elements of the 
\HOLW{} logic, and provides a variety of instructive worked examples.
%
The code for these examples is currently present in the 
\texttt{./examples/HolOmega/} directory.
There the following examples (among others) are to be found:

\begin{description}

\item [\tt functorScript.sml]

  This example shows how a simple version of category theory can
  be nicely realized as a shallow embedding within the new logic.
  Both functors and natural transformations are defined, and
  examples of each are demonstrated. This is similar to a development
  for HOL2P originally written by Norbert V\"{o}lker.

\item [\tt aopScript.sml]

  Building on the functor theory above, this shows several examples
  taken from {\it The Algebra of Programming}, by Richard Bird and Oege de Moor.
  These include homomorphisms, initial algebras, catamorphisms, and
  the banana split theorem.  This development was originally written
  by Norbert V\"{o}lker for HOL2P.

\item [\tt monadScript.sml]

  Also building on the functor theory above, this defines the concept
  of a monad in three different ways, and proves the three are equivalent.
  Multiple examples of monads are presented, and also how one can convert
  a monad from one of the styles of definition to another style.

\item [\tt type\_specScript.sml]

  This file contains examples of creating new types using the new definitial
  principle for type specification which has been added to the \HOLW{} theorem
  prover. In particular, this is used to create a new type by specifying it as
  the initial algebra of a signature. The example used is taken from a 1993
  paper by Tom Melham, ``The HOL Logic Extended with Quantification over Type
  Variables.''

\item [\tt packageScript.sml]

  This example shows more completely how packages and existential types
  may be created and used to hide the information about data types.
  Many of the examples are taken from and related to chapter 24 of the book
  "Types and Programmng Languages" by Benjamin C. Pierce, MIT Press, 2002.

\item[\tt interim]

This directory contains an extensive, worked example of a generalized
version of category theory, created by Jeremy Dawson of the Australian National
University. This generalizes the notions of functor and natural transformation
from those in \texttt{functorScript.sml}, to allow for a much richer realization
of category theory. For example, multiple categories, each with their own composition
and identity operations, may have functors defined between them.
The development of category theory is continued through the definition of adjoints,
and introduces an innovative extension of monads.
This example extensively exercises the kind structure of \HOLW, to manage
the types relating different categories and the operations among them.

\end{description}

Some of these examples will be described at length in the upcoming Tutorial.
Until then, the reader is encouraged to try out these examples on their own.

%%% Local Variables:
%%% mode: latex
%%% TeX-master: "tutorial"
%%% End:
