\documentclass[10pt]{scrartcl}

\usepackage[utf8]{inputenc}
\usepackage[english]{babel}

\usepackage[backend=biber]{biblatex}
\addbibresource{references.bib}

\usepackage{alltt}

\usepackage{booktabs}

\usepackage[scaled=0.8]{beramono}  % our monospace font
\usepackage[T1]{fontenc}  % this is necessary for beramo to work

\usepackage{listings}
\usepackage{upquote}

\lstdefinelanguage{hol4}{
  %alsoletter={@=>},
  morekeywords={
     Definition, End, Theorem, Proof, QED},
  %  fun, let, val, in, end, if, then, else, case, of},
  %sensitive=true,
  %morecomment=[l]{//},
  morecomment=[s]{(*}{*)},
  commentstyle=\color{gray},
  showstringspaces=false,
  columns=fullflexible,
  mathescape=true,
  numberstyle=\tiny,
  basicstyle=\ttfamily,
  numbersep=5pt,
  stepnumber=2,
  numbers=none,                   % where to put the line-numbers
  morestring=[b]"
}
\lstset{language=hol4, breaklines=true}
\usepackage{listings}

\usepackage{hyperref}
\usepackage{todonotes}
\newcommand{\heiko}[1]{\todo[inline,author=Heiko,bordercolor=red!20,color=orange!20]{#1}}
\newcommand{\IG}[1]{\todo[inline,author=Ivan,bordercolor=blue!20,color=orange!20]{#1}}
\newcommand{\ekey}[1]{\texttt{#1}}

\addto\extrasenglish{
  \def\sectionautorefname{Section}
  \def\subsectionautorefname{Section} % intentional
}


\title{A Programmer's Guide to Proving Theorems with HOL4}
\author{Anonymous}
\date{}

\begin{document}
\maketitle{}

\paragraph*{Summary}
Interactive theorem proving is a method to perform trustworthy, rigorous
mathematical proofs checked by computers.
From a programmer's point of view, learning an ITP system is similar to
learning a new programming language.
The main difference is that an ITP system allows to first implement a function,
and then prove theorems about it.
This guide is written for experienced programmers, that have seen pen-and-paper
proofs before, but never worked with an ITP system.
If you ever wanted to replace that nice, cozy thought of \emph{this function
should work as I intend it to} with the high-assurance of a rigorous
mathematical proof, this guide is for you\footnote{If you are asking yourself the question how this is even possible, the same reasoning applies.}.
%
\chapter{Getting and Installing \HOL{}}
\label{install}

This chapter describes how to get the \HOL{} system and how to install
it.  It is generally assumed that some sort of Unix system is being
used, but the instructions that follow should apply {\it mutatis
  mutandis\/} to other platforms.  Unix is not a pre-requisite for
using the system. \HOL{} may be run on PCs running Windows operating
systems from Windows~NT onwards (i.e., Windows~2000, XP and Vista are
also supported), as well as Macintoshes running MacOS~X.

\section{Getting \HOL{}}

The \HOL{} system can be downloaded from
\url{http://hol.sourceforge.net}.  The naming scheme for \holn{}
releases is $\langle${\it name}$\rangle$-$\langle${\it
  number}$\rangle$; the release described here is \holnversion.

\section{The {\tt hol-info} mailing list}

The \texttt{hol-info} mailing list serves as a forum for discussing
\HOL{} and disseminating news about it.  If you wish to be on this
list (which is recommended for all users of \HOL), visit
\url{http://lists.sourceforge.net/lists/listinfo/hol-info}.  This
web-page can also be used to unsubscribe from the mailing list.

\section{Installing \HOL{}}

It is assumed that the \HOL{} sources have been obtained and the
\texttt{tar} file unpacked into a directory \ml{hol}.\footnote{You may
  choose another name if you want; it is not important.} The contents
of this directory are likely to change over time, but it should
contain the following:

\begin{center}
\begin{tabular}{|l|l|l|} \hline
\multicolumn{3}{|c|}{ } \\
\multicolumn{3}{|c|}{\bf Principal Files on the HOL Distribution Directory} \\
\multicolumn{3}{|c|}{ } \\
{\it File name} & {\it Description} & {\it File type}  \\ \hline
{\tt README} & Description of directory {\tt hol} & Text\\
{\tt COPYRIGHT}& A copyright notice & Text\\
{\tt INSTALL} & Installation instructions & Text\\
{\tt tools} & Source code for building the system & Directory\\
{\tt bin} & Directory for HOL executables & Directory\\
{\tt sigobj} & Directory for \ML{} object files & Directory\\
{\tt src} & \ML{} sources of \HOL & Directory\\
{\tt help} & Help files for \HOL{} system & Directory\\
{\tt examples} & Example source files & Directory\\
\hline
\end{tabular}
\end{center}

The session in the box below shows a typical distribution directory.
The \HOL{} distribution has been placed on a PC running Linux in the
directory {\small\tt /home/mn200/hol/}.

All sessions in this documentation will be displayed in boxes with a
number in the top right hand corner.  This number indicates whether
the session is a new one (when the number will be {\small\sl 1}) or
the continuation of a session started in an earlier box.
Consecutively numbered boxes are assumed to be part of a single
continuous session.  The Unix prompt for the sessions is
\texttt{\small \dol}, so lines beginning with this prompt were typed
by the user.  After entering the \HOL{} system (see below), the user
is prompted with {\small\verb|-|} for an expression or command of the
\HOL{} meta-language \ML; lines beginning with this are thus \ML\
expressions or declarations.  Lines not beginning with \texttt{\small
  \$} or {\small\verb|-|} are system output.  Occasionally, system
output will be replaced with a line containing {\small\verb|...|} when
it is of minimal interest. The meta-language \ML{} is introduced in
Chapter~\ref{ML}.

\setcounter{sessioncount}{0}
\begin{session}
\begin{verbatim}
$ pwd
/home/mn200/hol
$ ls -F
COPYRIGHT  bin/  examples/  INSTALL  src/
README     doc/  help/      sigobj/  tools/
\end{verbatim}
\end{session}

Now you will need to rebuild \HOL{} from the sources.\footnote{It is
  possible that pre-built systems may soon be available from the
  web-page mentioned above.}

Before beginning you must have a current version of Moscow~ML or
Poly/ML\footnote{Poly/ML cannot be used with HOL on Windows.}.  In the
case of Moscow~ML, you must have version 2.01.  Moscow~ML is available
on the web from \url{http://www.itu.dk/~sestoft/mosml.html}.
Poly/ML is available from \url{http://polyml.org}.

When working with Poly/ML, the installation must ensure that dynamic library loading (typically done by setting the \texttt{LD\_LIBRARY\_PATH} environment variable) picks up \texttt{libpolyml.so} and \texttt{libpolymain.so}.
If these files are in \texttt{/usr/lib}, nothing will need to be changed, but other locations may require further system configuration.
A sample \texttt{LD\_LIBRARY\_PATH} initialisation command (in a file such as \texttt{.bashrc}) might be
\begin{verbatim}
   declare -x LD_LIBRARY_PATH=/usr/local/lib:$HOME/lib
\end{verbatim}
Further, if you are using Poly/ML version 5.5.1 or later, you will need to make sure that Poly/ML is configured with the \texttt{--enable-shared} option.
That is, your configure command (the first thing that is done when building the system), will look something like
\begin{verbatim}
   ./configure --enable-shared
\end{verbatim}

When you have your ML system installed, and are in the root directory of the distribution, the next step is to run \texttt{smart-configure}.
With Moscow~ML, this looks like:

\begin{session}
\begin{alltt}
\dol mosml < tools/smart-configure.sml
Moscow ML version 2.01 (January 2004)
Enter `quit();' to quit.
- [opening file "tools/smart-configure-mosml.sml"]

HOL smart configuration.

Determining configuration parameters: OS mosmldir holdir
OS:                 linux
mosmldir:           /home/mn200/mosml/bin
holdir:             /home/mn200/hol
dynlib_available:   true

Configuration will begin with above values.  If they are wrong
press Control-C.
\end{alltt}
\end{session}

If you are using Poly/ML, then write
\begin{verbatim}
   poly < tools/smart-configure.sml
\end{verbatim}
instead.

Assuming you don't interrupt the configuration process, this will
build the \texttt{Holmake} and \texttt{build} programs, and move them
into the \texttt{hol/bin} directory.  If something goes wrong at this
stage, consult Section~\ref{sec:editting-configure} below.

The next step is to run the \texttt{build} program.  This should
result in a great deal of output as all of the system code is compiled
and the theories built.  Eventually, a \HOL{} system\footnote{Four
  \HOL{} executables are produced: \textsf{hol}, \textsf{hol.noquote},
  \textsf{hol.bare} and \textsf{hol.bare.noquote}.  The first of these
  will be used for most examples in the \TUTORIAL{}.} is produced in
the \texttt{bin/} directory.

\begin{session}
\begin{alltt}
\dol bin/build
  ...
  ...
Uploading files to /home/mn200/hol/sigobj

Hol built successfully.
\dol
\end{alltt}
\end{session}


\subsection{Overriding \texttt{smart-configure}}
\label{sec:editting-configure}

If \texttt{smart-configure} is unable to guess correct values for the
various parameters (\texttt{holdir}, \texttt{OS} \etc) then you can
create a file called to provide correct values.  With Moscow~ML, this
should be \texttt{config-override} in the root directory of the HOL
distribution.  With Poly/ML, this should be \texttt{poly-includes.ML}
in the \texttt{tools-poly} directory. In this file, specify the
correct value for the appropriate parameter by providing an ML binding
for it.  All variables except \texttt{dynlib\_available} must be given
a string as a possible value, while \texttt{dynlib\_available} must be
either \texttt{true} or \texttt{false}.  So, one might write

\begin{session}
\begin{verbatim}
val OS = "unix";
val holdir = "/local/scratch/myholdir";
val dynlib_available = false;
\end{verbatim}
\end{session}

The \texttt{config-override} file need only provide values for those
variables that need overriding.

With this file in place, the \texttt{smart-configure} program will use
the values specified there rather than those it attempts to calculate
itself.  The value given for the \texttt{OS} variable must be one of
\texttt{"unix"}, \texttt{"linux"}, \texttt{"solaris"},
\texttt{"macosx"} or \texttt{"winNT"}.\footnote{The string
  \texttt{"winNT"} is used for Microsoft Windows operating systems
  that are at least as recent as Windows~NT.  This includes
  Windows~2000, XP and Vista.}

In extreme circumstances it is possible to edit the file
\texttt{tools/configure.sml} yourself to set configuration variables
directly.  (If you are using Poly/ML, you must edit
\texttt{tools-poly/configure.sml} instead.) At the top of this file
various incomplete SML declarations are present, but commented out.
You will need to uncomment this section (remove the \texttt{(*} and
\texttt{*)} markers), and provide sensible values.  All strings must
be enclosed in double quotes.

The \texttt{holdir} value must be the name of the top-level directory
listed in the first session above.  The \texttt{OS} value should be
one of the strings specified in the accompanying comment.

When working with Moscow~ML, the \texttt{mosmldir} value must be the
name of the directory containing the Moscow~ML binaries
(\texttt{mosmlc}, \texttt{mosml}, \texttt{mosmllex} etc).  When
working with Poly/ML, the \texttt{poly} string must be the path to the
\texttt{poly} executable that begins an interactive \ML{} session.
The \texttt{polymllibdir} must be a path to a directory that contains
the file \texttt{libpolymain.a}.

Subsequent values (\texttt{CC} and \texttt{GNUMAKE}) are needed for
``optional'' components of the system.  The first gives a string
suitable for invoking the system's C compiler, and the second
specifies a \textsf{make} program.

After editing, \texttt{tools/configure.sml} the lines above will look
something like:

\begin{session}
\begin{alltt}
\dol more configure.sml
  ...
val mosmldir = "/home/mn200/mosml";
val holdir   = "/home/mn200/hol";
val OS       = "linux"       (* Operating system; choices are:
                                "linux", "solaris", "unix", "winNT" *)

val CC       = "gcc";     (* C compiler (for building quote filter)        *)
val GNUMAKE  = "gnumake"; (* for robdd library                             *)
  ...
\dol
\end{alltt}
\end{session}

\noindent Now, at either this level (in the \texttt{tools} or
\texttt{tools-poly} directory) or at the level above, the script
\texttt{configure.sml} must be piped into the \ML{} interpreter (\ie,
\texttt{mosml} or \texttt{poly}).  For example,

\begin{session}
\begin{alltt}
\dol mosml < tools/configure.sml
Moscow ML version 2.01 (January 2004)
Enter `quit();' to quit.
- > val mosmldir = "/home/mn200/mosml" : string
  val holdir = "/home/mn200/hol" : string
  val OS = "linux" : string
- > val CC = "gcc" : string
  ...
Beginning configuration.
- Making bin/Holmake.
  ...
Making bin/build.
- Making hol98-mode.el (for Emacs)
- Setting up the standard prelude.
- Setting up src/0/Globals.sml.
- Generating bin/hol.
- Generating bin/hol.noquote.
- Attempting to compile quote filter ... successful.
- Setting up the muddy library Makefile.
- Setting up the help Makefile.
-
Finished configuration!
-
\dol
\end{alltt}
\end{session}



%%% Local Variables:
%%% mode: latex
%%% TeX-master: "tutorial"
%%% End:

%
\section{Interacting with the HOL4 REPL}

The HOL4 REPL is an extended version of the Poly/ML~\cite{polymlweb} REPL, and
behaves like the REPL's of other interpreted languages.
In general, it is recommend to first open a script file before starting
the REPL.

\textbf{Important setup note:}We strongly recommend opening a file
\texttt{tutorialScript.sml} in the folder \texttt{\$LASSIEDIR/examples}, to make
sure that the below code can be run easily.
This is done by either manually creating an empty file named \texttt{tutorialScript.sml} in \texttt{\$LASSIEDIR/examples}
and opening it with the toolbar buttons in emacs, or using \ekey{C-x C-f} (\ekey{Control} and \ekey{x}, then \ekey{Control} and \ekey{f})
to open it within emacs, by typing in the path.

We will explain in \autoref{sec:libraries} how the file should start to make its
contents reusable. For now we simply use it as a scratchpad.
The REPL is started by pressing \ekey{M-h H} (\ekey{Alt} and \ekey{h}, then \ekey{H}),
and HOL4 code is send to the REPL with the keybinding \ekey{M-h M-r}
(pressing \ekey{Alt} and \ekey{h}, then \ekey{Alt} and \ekey{r}).

For example, type
\begin{lstlisting}
  3 + 5;
\end{lstlisting}

anywhere in the currently openend script file.
Sending is then done by first highlighting the code with emacs using \ekey{C-space}.
Here, \ekey{C} stands for \ekey{Control}, so to start marking text, press
\ekey{Control} together with the \ekey{space} key.
The arrow keys are used for marking the code to be send to the REPL, and once it
has been completely selected, press \ekey{M-h M-r}.

The REPL should print:
\begin{lstlisting}[frame=single]
> 3+5;
val it = 8: int
\end{lstlisting}

All of the functionality of the Poly/ML REPL, and in general, the Standard ML
basis library (see e.g. \url{https://smlfamily.github.io/Basis/} for a reference)
are available in the HOL4 REPL.
Thus HOL4 supports creating and manipulating lists, strings, options, and
simple I/O.

It is strongly recommended to type or copy the code snippets from this tutorial
into an actual script file and sending them to the REPL with the keybinding \ekey{M-h M-r}.
This makes sure that the code can be experimented with and commands that have
been entered are not lost in the limbo of the REPL printouts.

As a quick point of reference, \autoref{tbl:keybindings} gives a short,
executive summary of the most commonly used keybindings, taken from the
documentation of the HOL4 emacs mode (\url{https://hol-theorem-prover.org/hol-mode.html}).

\begin{table}
  \centering
\begin{tabular}{@{}cll@{}}
  \toprule
  Keybinding & \multicolumn{1}{c}{Effect} & \multicolumn{1}{c}{Remark}\\
  \midrule
  \ekey{M-h H} & Start a new HOL4 session & \\
  \ekey{M-h M-r} & Send marked text to REPL & \\
  \ekey{M-h g} & Start a new proof & Must be within a \texttt{Theorem}, \texttt{Proof} block\\
  \ekey{M-h e} & Applies a tactic & Marked SML code must have type `tactic`\\
  \ekey{M-h d} & Stop current interactive proof \\
  \bottomrule
\end{tabular}
  \caption{Most common HOL4-mode keybindings}\label{tbl:keybindings}
\end{table}
%%% Local Variables:
%%% mode: latex
%%% TeX-master: "lassie-tutorial"
%%% End:

%
\section{A First (Pen and Paper Style) Proof}\label{sec:hol_ex1}
%
The HOL4 features presented so far are exactly those of an interpreted
programming language.
Next, we will define our first function in HOL4, and prove a first theorem about
it.

%A common mathematical notation is $\sum_{i=0}^n f (i)$,
%summing the numbers from $0$ to $n$, and applying function $f$.
%We define a specialized version for $f (x) = x$ in HOL4:
The function we will define is a sum of natural numbers up to $n$,
$\sum_{i=0}^n i$.
A HOL4 definition looks like this:

\begin{lstlisting}
Definition sumFun_def:
  sumFun (n:num) = if (n = 0) then 0 else n + sumFun (n-1)
End
\end{lstlisting}

The \lstinline{Definition} and \lstinline{End} keyword tell the REPL that we
define a HOL4 function and mark its end.
In the REPL, \lstinline{sumFun_def} is the name of the definition, under which it
can be accessed.
As a convention, when defining function $f$ in HOL4, its definition should be
named \lstinline{f_def}.
The type annotation \lstinline{n:num} tells the HOL4 REPL to parse variable \texttt{n}
as a natural number.
To load the definition into the HOL4 REPL, mark it completely, including the \lstinline{Definition}
keyword and the \lstinline{End} keyword, and send it to the REPL with \ekey{M-h M-r}.

Alternatively, HOL4 also supports defining a function by a system of equations,
moving the \texttt{case} expression of the \lstinline{sum} function to the
outside:
%
\begin{lstlisting}
Definition sum_def:
  sum 0 = 0 /\
  sum (n:num) = n + sum (n-1)
End
\end{lstlisting}

To avoid a name clash we have renamed the function into \lstinline{sum}.
As for the definition of \lstinline{sumFun_def}, to send the definition to the REPL,
mark it and send it with \ekey{M-h M-r}.
Choosing one definition over the other has different benefits and downsides.
As a rule of thumb, it is recommended to choose the latter version, giving a
system of equations if the function requires a top-level \texttt{case}
expression.
We will do the proof for the function \texttt{sum} here, and give a general
guideline on when to prefer which version later in \autoref{subsec:tipsAndTricks}.

As a simple, first example, we will prove a closed form for \lstinline{sum n}:
\[
  \sum_{i=0}^{n} i = \frac{n * (n + 1)}{2}
\]

In HOL4 this theorem is stated as

\begin{lstlisting}
Theorem closed_form_sum:
  ! n. sum n = n * (n + 1) DIV 2
Proof
QED
\end{lstlisting}

Again, \lstinline{Theorem}, \lstinline{Proof}, and \lstinline{QED} are the
keywords marking a theorem statement in the REPL and the indented line is
the statement that we want to prove.
Similar to a definition, the name \lstinline{closed_form_sum} is an identifier
which is used later to refer to the theorem statement proven in other proofs.
This makes theorems first class citizens of the HOL4 REPL, also allowing
functions to manipulate and inspect their statements.

When proving a theorem for the first time in HOL4, the proof is usually done
interactively.
Starting an interactive proof is as simple as marking the indented line
(\lstinline{! n. sum n = n * (n + 1) DIV 2}) and pressing \ekey{M-h g}.
The HOL4 REPL prints

\begin{lstlisting}[mathescape=true, frame=single]
> val it =
   Proof manager status: 1 proof.
   1. Incomplete goalstack:
        Initial goal:
        $\forall$ n. sum n = n * (n + 1) DIV 2
   : proofs
\end{lstlisting}

In HOL4, theorems are proven by applying so-called \emph{tactics} to the current
goal.
These tactics are a group of SML functions, implemented in the HOL4
distribution, and filled in between the \lstinline{Proof} and the \lstinline{QED}
keywords.
In this tutorial, we decouple learning how the theorem prover works from
learning the syntax of the tactics language by performing interactive proofs
with Lassie using natural language.

To load Lassie and the natural language descriptions required for the proof,
run
\begin{lstlisting}
open LassieLib arithTacticsLib logicTacticsLib arithmeticTheory;
val _ = LassieLib.loadJargon "Arithmetic";
val _ = LassieLib.loadJargon "Logic";
\end{lstlisting}
interactively.

The closed form is the standard example for proofs by induction in math classes.
Following this example, we start the proof with
\begin{lstlisting}
nltac `Induction on 'n'.`
\end{lstlisting}

Here, \lstinline{nltac} is a Lassie function that parses natural language and
translates it into a HOL4 tactic.
The parameter \lstinline{`Induction on 'n'.`} is the natural language
description of the tactic used.
To apply the tactic, the line must be marked and run with \ekey{M-h e}.
After running the code, the HOL4 REPL shows
\begin{lstlisting}
> OK..
2 subgoals:
val it =

    0.  sum n = n * (n + 1) DIV 2
   ------------------------------------
        sum (SUC n) = SUC n * (SUC n + 1) DIV 2

   sum 0 = 0 * (0 + 1) DIV 2

2 subgoals
   : proof
\end{lstlisting}

The line \lstinline{2 subgoals} tells us that we must prove two separate goals
to finish the proof.
As HOL4 keeps track of these subgoals for us, we need not manage them manually
to make sure that the proof remains error-free.
Note that in the induction step, HOL4 automatically adds the inductive
hypothesis as an assumption (labeled with \lstinline{0}) above a dashed line.

First, we prove the base case \lstinline{sum 0 = 0 * (0 + 1) DIV 2}, then we
show the induction step \lstinline{sum (n + 1) = (n + 1) * (n + 2) DIV 2}.
Function \lstinline{SUC} is the HOL4 version of Peano's successor function.
Intuitively \lstinline{SUC n} refers to the natural number after \lstinline{n},
i.e. \lstinline{n + 1}.

As for a pen-and-paper proof, the base case of the induction is trivial, and
solved with the simple statement \lstinline{nltac `use [sum_def] to simplify.`},
leaving us only with the induction step from above.
In contrast to a pen-and-paper proof, we have to explicitly state that we
simplify with the definition of our summation function (\lstinline{sum_def}).
This is part of the enforced rigour required by the theorem prover\footnote{
We will show in \autoref{subsec:tipsAndTricks} how one can get rid of this in certain cases.}.

As for a pen-and-paper proof, the first step on the induction step is to
simplify:
\begin{lstlisting}
nltac `use [sum_def, GSYM ADD_DIV_ADD_DIV] to simplify.`
\end{lstlisting}

Here, \lstinline{ADD_DIV_ADD_DIV} is a theorem from the HOL4 standard library
used to enrich the simplifier with the additional knowledge.
To find out its statement, mark the theorem name only, and send it to the
REPL with \ekey{M-h M-r}.
Sending \lstinline{GSYM} to the REPL shows that the function has type
\lstinline{:thm -> thm}, meaning that it takes a theorem as input and returns a
theorem.
Function \lstinline{GSYM} is a Poly/ML function rotating an equality theorem,
replacing equality $a = b$ with equality $b = a$.
Sending \lstinline{GSYM ADD_DIV_ADD_DIV} and \lstinline{ADD_DIV_ADD_DIV} to the REPL, one can
observe its effect easily.

Using \lstinline{GSYM} can be useful from time to time as rewriting in HOL4 is
directed from left-to-right.
If we have a theorem showing $f x = b$, HOL4 will rewrite any occurence of
$f x$ into an occurence of $b$, but it will never replace occurences of $b$
with occurences of $f x$.

After applying the tactic, the REPL will show the subgoal that remains to be proven:
\begin{lstlisting}
> OK..
1 subgoal:
val it =

    0.  sum n = (n * (n + 1)) DIV 2
   ------------------------------------
        (2 * SUC n + n * (n + 1)) DIV 2 = SUC n * (SUC n + 1) DIV 2

   : proof
\end{lstlisting}

Applying the following tactics step-by-step closes the proof:

\begin{lstlisting}
nltac `'2 * SUC n + n * (n + 1) = SUC n * (SUC n + 1)' suffices to show the goal.`
nltac `show 'SUC n * (SUC n + 1) = (SUC n + 1) + n * (SUC n + 1)' using (simplify with [MULT_CLAUSES]).`
nltac `simplify.`
nltac `show 'n * (n + 1) = SUC n * n' using (trivial using [MULT_CLAUSES, MULT_SYM]).`
nltac `'2 * SUC n = SUC n + SUC n' follows trivially.`
nltac `'n * (SUC n + 1) = SUC n * n + n' follows trivially.`
nltac `rewrite assumptions. simplify.`
\end{lstlisting}

\begin{sloppypar}
The natural language tactic \lstinline{`show 'SUC n * (SUC n + 1) = (SUC n + 1) + n * (SUC n + 1)' using (simplify with [MULT_CLAUSES]).`}
shows another feature of HOL4:
We can extend the list of assumptions with the theorem mentioned after
\lstinline{show}.
Before running the tactic, the state of the goal is
\end{sloppypar}
\begin{lstlisting}
> OK..
1 subgoal:
val it =

    0.  sum n = n * (n + 1) DIV 2
   ------------------------------------
        2 * SUC n + n * (n + 1) = SUC n * (SUC n + 1)

   : proof
\end{lstlisting}

and after running the tactic, the subgoal becomes
\begin{lstlisting}
> OK..
1 subgoal:
val it =

    0.  sum n = n * (n + 1) DIV 2
    1.  SUC n * (SUC n + 1) = SUC n + 1 + n * (SUC n + 1)
   ------------------------------------
        2 * SUC n + n * (n + 1) = SUC n * (SUC n + 1)

   : proof
\end{lstlisting}

Running all tactics, one after another, the REPL shows that the proof is finished by printiting
\begin{lstlisting}
> OK..

val it =
   Initial goal proved.
   $\vdash$ $\forall$ n. sum n = n * (n + 1) DIV 2: proof
\end{lstlisting}

To reuse the theorem later, and to make it automatically checkable by HOL4, we
have to put the natural language into a single call to \lstinline{nltac}.
The full code for the theorem is given in \autoref{fig:gaussProof}.
%
\begin{figure}[t]
\begin{lstlisting}[mathescape=true]
Theorem closed_form_sum:
  $\forall$ n. sum n = (n * (n + 1)) DIV 2
Proof
  nltac `
   Induction on 'n'.
   use [sum_def] to simplify.
   use [sum_def, GSYM ADD_DIV_ADD_DIV] to simplify.
   use [sum_def, GSYM ADD_DIV_ADD_DIV] to simplify.
   '2 * SUC n + n * (n + 1) = SUC n * (SUC n + 1)' suffices to show the goal.
   show 'SUC n * (SUC n + 1) = (SUC n + 1) + n * (SUC n + 1)' using (simplify with [MULT_CLAUSES]).
   simplify.
   show 'n * (n + 1) = SUC n * n' using (trivial using [MULT_CLAUSES, MULT_SYM]).
   '2 * SUC n = SUC n + SUC n' follows trivially.
   'n * (SUC n + 1) = SUC n * n + n' follows trivially.
   rewrite assumptions. simplify.
QED
\end{lstlisting}
\caption{Complete proof of the closed form for summing natural numbers until $n$ using Lassie}\label{fig:gaussProof}
\end{figure}

Marking the complete statement, and running it with \ekey{M-h M-r} will save the
theorem under the name \lstinline{gaussian_sum}.
%%% Local Variables:
%%% mode: latex
%%% TeX-master: "lassie-tutorial"
%%% End:

%
\section{Developing Libraries with HOL4}\label{sec:libraries}

In the previous section we have looked at a first, self-contained, example of a
HOL4 proof.
To prove the closed form of the gaussian sum, we only needed to define a function
and perform a straight-forward proof by induction.
However, in larger developments it is common to split proofs into smaller
lemmas that are used as part of a central, final theorem\footnote{This is similar to how one would never write larger programs within the \texttt{main} function. It is desirable to split up functionality into programs instead}.

In this section we describe how larger developments are performed with HOL4 by
showing the theorem that there is an infinite number of prime numbers, called
euclid's theorem.

\subsection{Preamble}
Before starting a development in HOL4, it is recommended to declare dependencies
and load theorems that come with the HOL4 theorem prover.
We do so by running

\begin{lstlisting}
open BasicProvers Defn HolKernel Parse Conv SatisfySimps Tactic monadsyntax
     boolTheory bossLib arithmeticTheory;

open LassieLib arithTacticsLib;

val _ = new_theory "euclid";
\end{lstlisting}

The first \lstinline{open} loads a bunch of theories and tactics from the HOL4
git repository, whereas the second loads Lassie and the natural language
descriptions required for the proofs that we will perform.
The third line, tells HOL4 that we start a new theory, called ``euclid''.
In HOL4 speak, this is the analogous to defining an interface in Java, or a
header file in C.
All definitions and theorems are explicitly part of the interface of a theory.
To prevent errors, the file in which we store the theory has to be called
\lstinline{euclidScript.sml}.
If this correspondance between the file name and theory name is ignored,
HOL4 will fail to build the theory later.
As in the previous section, we recommend implementing file \lstinline{euclidScript.sml}
in the directory \texttt{\$LASSIEDIR/examples}.
We load the jargon with \lstinline{val _ = LassieLib.loadJargon "Arithmetic";}.

\subsection{Basic Definitions}
Our overall goal is to prove the HOL4 equivalent of the informal statement that
``there is an infinite number of prime numbers''.
The first concept that we need to define is thus what it means for a (natural)
number to be a prime number.

A number $n$ is called a prime number if it is only divisible by $1$ and $n$
itself. Thus, we first define a predicate \lstinline{divides} where
\lstinline{a divides b} if and only if \lstinline{b} can be expressed as a
multiple of \lstinline{a}:

\begin{lstlisting}
set_fixity "divides" (Infix(NONASSOC, 450));

Definition divides_def:
  (a divides b) = (? x. b = a * x)
End
\end{lstlisting}
The first line declares \lstinline{divides} as a new infix operator, like $+,-, \ldots$
Next, the \lstinline{Definition}, \lstinline{End} block defines \lstinline{divides} as a
binary infix relation where \lstinline{a divides b} is true, if and only if
there is an \lstinline{x} such that \lstinline{b} is the result of multiplying \lstinline{a} with \lstinline{x}.

Alternatively, we could have defined \lstinline{divides} as a function instead
of an infix operation, as we did in \autoref{sec:hol_ex1}.
Using it as an infix operation however makes it more obvious which number is
divided by which.

Having defined \lstinline{divides} we use it to define a predicate
\lstinline{prime} which is true, if its argument is a prime number.
\begin{lstlisting}
Definition prime_def:
  prime p = (p<>1 /\ !x . x divides p ==> (x=1) \/ (x=p))
End
\end{lstlisting}

The left-hand side of the conjunction (\lstinline{p <> 1}) explicitly excludes
number 1 from being a prime number, and the right-hand side states the HOL4
version of \lstinline{p} being prime if it can only be divided by $1$ and itself.

\subsection{Proving Infrastructural Lemmas}

Before proving euclid's theorem itself, we start by proving some infrastructural
lemmas that will come in handy later.
\begin{lstlisting}

Theorem DIVIDES_0:
  ! x . x divides 0
Proof
  nltac `[divides_def, MULT_CLAUSES] solves the goal.`
QED

Theorem DIVIDES_ZERO:
  ! x . (0 divides x) = (x = 0)
Proof
  nltac `[divides_def, MULT_CLAUSES] solves the goal.`
QED

Theorem DIVIDES_ONE:
  ! x . (x divides 1) = (x = 1)
Proof
  nltac `[divides_def, MULT_CLAUSES, MULT_EQ_1] solves the goal.`
QED

Theorem DIVIDES_REFL:
  ! x . x divides x
Proof
  nltac `[divides_def, MULT_CLAUSES] solves the goal.`
QED

Theorem DIVIDES_TRANS:
  ! a b c . a divides b /\ b divides c ==> a divides c
Proof
  nltac `[divides_def, MULT_ASSOC] solves the goal.`
QED
\end{lstlisting}

Theorems \lstinline{DIVIDES_0, DIVIDES_ZERO}, and \lstinline{DIVIDES_ONE} show
simple base cases for predicate \lstinline{divides}.
As these follow straight-forwardly from the definition, the Lassie proof is just
\lstinline{nltac `[divides_def, MULT_CLAUSES] solves the goal.`}, resp.
\lstinline{nltac `[divides_def, MULT_CLAUSES, MULT_EQ_-1] solves the goal.`}.

Similarly, one proves theorems about the relation between $+, *$ and $\leq$ and
\lstinline{divides}:
\begin{lstlisting}

Theorem DIVIDES_ADD:
  ! d a b . d divides a /\ d divides b ==> d divides (a + b)
Proof
  nltac `[divides_def, LEFT_ADD_DISTRIB] solves the goal.`
QED

Theorem DIVIDES_SUB:
  !d a b . d divides a /\ d divides b ==> d divides (a - b)
Proof
  nltac `[divides_def, LEFT_SUB_DISTRIB] solves the goal.`
QED

Theorem DIVIDES_ADDL:
  !d a b . d divides a /\ d divides (a + b) ==> d divides b
Proof
  nltac `[ADD_SUB, ADD_SYM, DIVIDES_SUB] solves the goal.`
QED

Theorem DIVIDES_LMUL:
  !d a x . d divides a ==> d divides (x * a)
Proof
  nltac `[divides_def, MULT_ASSOC, MULT_SYM] solves the goal.`
QED

Theorem DIVIDES_RMUL:
  !d a x . d divides a ==> d divides (a * x)
Proof
  nltac `[MULT_SYM,DIVIDES_LMUL] solves the goal.`
QED

Theorem DIVIDES_LE:
  !m n . m divides n ==> m <= n \/ (n = 0)
Proof
  nltac `rewrite [divides_def]. [] solves the goal.`
QED
\end{lstlisting}

\subsection{Euclid's Theorem}
Having defined prime numbers, and after proving simple properties of
\lstinline{divides}, we next state euclid's theorem and start
exploring its proof.

\begin{lstlisting}
Theorem euclid:
  !n . ?p . n < p /\ prime p
Proof

QED
\end{lstlisting}

After starting the interactive proof with \ekey{M-h g}, we can start exploring
it with Lassie.
The textbook version of the proof is done by contradiction, so we perform the
same step in HOL4: \lstinline{nltac `suppose not.`}.
After running the tactic with \ekey{M-h e} the REPL shows the following goal state:
%
\begin{lstlisting}[frame=single, mathescape=true]
> OK..
1 subgoal:
val it =

    0.  $\exists$ n. $\forall$ p. n < p $\rightarrow$ $\neg$prime p
   ------------------------------------
        F

   : proof
\end{lstlisting}

The first assumption starts with an existential quantifier, from which we can
obtain the witness. Therefore we next call into the simplifier to automatically
take care of this with \lstinline{nltac `simplify.`} leaving us with
%
\begin{lstlisting}[frame=single, mathescape=true]
> > > > > # # OK..
1 subgoal:
val it =

    0.  $\forall$ p. n < p $\rightarrow$ $\neg$prime p
   ------------------------------------
        F

   : proof
\end{lstlisting}

The assumption now tells us that any natural number $p$ which is greater than
$n$ cannot be prime.
On a high level, the goal is to derive a contradiction from this assumption by
finding a prime number that is bigger than $n$.
An integral part of this step is that every natural number greater than $1$ has
a prime factorization\footnote{We have to exclude $1$ here because our definition of prime numbers explicitly ruled out $1$.}.
Before continuing the proof, we prove a theorem that for an arbitrary natural
number $n$, there is a prime factor of that number.
To this end we first drop the current goal with \ekey{M-h d} and start proving:
%
\begin{lstlisting}
Theorem PRIME_FACTOR:
  !n . n <> 1 ==> ?p . prime p /\ p divides n
Proof
  LassieLib.nltac `
    Complete Induction on 'n'.
    rewrite [].
    perform a case split for 'prime n'.
    Goal 1. follows from [DIVIDES_REFL]. End.
    Goal 1.
      show '? x. x divides n and x <> 1 and x <> n' using (follows from [prime_def]).
      follows from [LESS_OR_EQ, PRIME_2, DIVIDES_LE, DIVIDES_TRANS, DIVIDES_0].
    End.`
\end{lstlisting}

The theorem can be immediately loaded by running \ekey{M-h M-r} over the complete
text.
However, we recommend stepping through the \lstinline{nltac} steps one-by-one
with \ekey{M-h e} after starting an interactive proof with \ekey{M-h g}.

One particular thing to note here is that this is the first proof that
explicitly mentions subgoals in Lassie.
After performing a case split on whether $n$ is prime or not
(\lstinline{perform a case split for 'prime n'.}) HOL4 leaves us with two
subgoals to prove:

\begin{lstlisting}[frame=single]
> OK..
2 subgoals:
val it =

    0.  $\forall$ m. m < n $\rightarrow$ m $\neq$ 1 $\rightarrow$ $\exists$ p. prime p $\wedge$ p divides m
    1.  n $\neq$ 1
    2.  $\neg$ prime n
   ------------------------------------
        $\exists$ p. prime p $\wedge$ p divides n

    0.  $\forall$ m. m < n $\rightarrow$ m $\neq$ 1 $\rightarrow$ $\exists$ p. prime p $\wedge$ p divides m
    1.  n $\neq$ 1
    2.  prime n
   ------------------------------------
        $\exists$ p. prime p $\wedge$ p divides n

2 subgoals
   : proof
\end{lstlisting}

Here, the first subgoal ($\exists \texttt{p}. \texttt{prime p} \wedge \texttt{p divides n}$)
is solved first, with the natural language command \lstinline{Goal 1}.
Alternatively, we can say that we want to prove first the goal where \texttt{prime n} holds with
\lstinline{Goal 'prime n'}.

Having shown the theorem \lstinline{PRIME_FACTOR} we can go back to proving
Euclid's theorem:
\begin{lstlisting}
Theorem euclid:
  !n . ?p . n < p /\ prime p
Proof
  nltac `suppose not. simplify.`
QED
\end{lstlisting}

As a next step, we will obtain a prime factor of $n! + 1$, called $q$.
From this we know that $q$ is prime and \lstinline{q divides FACT n + 1}.
As $q$ is a prime number, we can derive that $q \leq n$.
We obtain a contradiction by deriving that $q = 1$ from the fact that any number
smaller than $n$ is a divisor of $!n$, and theorem \lstinline{DIVIDES_ADDL}.
The full proofscript becomes:
\begin{lstlisting}[mathescape=true]
nltac `
  suppose not. simplify.
  we can derive 'FACT n + 1 <> 1' from [FACT_LESS, neq_zero].
  thus PRIME_FACTOR for 'FACT n + 1'.
  we further know '?q. prime q and q divides (FACT n + 1)'.
  show 'q <= n' using [NOT_LESS_EQUAL].
  show '0 < q' using [PRIME_POS] .
  show 'q divides FACT n' using [DIVIDES_FACT].
  show 'q=1' using [DIVIDES_ADDL, DIVIDES_ONE].
  show 'prime 1' using (simplify).
  [NOT_PRIME_1] solves the goal.`
\end{lstlisting}

We recommend stepping through each step one by one and observing the changes to
the proof.

After finishing the proof the theory development is ended by putting
\lstinline{val _ = export_theory();} at the end of the file.
This directive tells HOL4 to export the defined functions and proven theorems
into a file \texttt{euclidTheory.sml} which can be used by future developments.
The file \texttt{euclidTheory.sig} gives an overview of the included theorems and
definitions in a human readable format.
%%% Local Variables:
%%% mode: latex
%%% TeX-master: "lassie-tutorial"
%%% End:

%
\section{Learning HOL4 Tactics}

In the previous examples we have used Lassie to prove theorems in HOL4 in a
rigorous but still human readable format.
HOL4 itself supports a richer set of so-called tactics that are commonly used
when proving theorems.

We will next show how Lassie can be used to learn a first set of basic tactics
and explain some of their intricacies that a beginner may struggle with.
Instead of using function \lstinline{nltac} to parse textual descriptions into HOL4
tactics, we next use Lassie's proof-REPL \lstinline{nlexplain} to interactively
develop a first proof script by looking again at the proof of the gaussian sum
in \autoref{fig:gaussProof}.

Therefore we (re-)start the interactive proof with \lstinline{g `! n. sum n = (n * (n + 1)) DIV 2`}
and sending it to the REPL with \ekey{M-h M-r}.
The HOL4 REPL prints:
\begin{lstlisting}[frame=single, mathescape=true, deletekeywords={Proof}]
> > # # # val it =
   Proof manager status: 1 proof.
   1. Incomplete goalstack:
     Initial goal:
     $\forall$ n. sum n = n * (n + 1) DIV 2
   : proofs
\end{lstlisting}

As before the interactive proof shows us the current goal that we have to proof.
Next, we start the Lassie's proof-REPL with \lstinline{nlexplain()} and
sending it with \ekey{M-h M-r}.
Note that the REPL changes from \lstinline{>} to \lstinline{|>} to denote that
Lassie is capturing the input.
Instead of sending the full proofscript with \lstinline{nltac}, one can now send
each step separately and observe its output.
Sending the first step from the Lassie proof (\lstinline{Induction on 'n'.}) with
\ekey{M-h M-r} prints
%
\begin{lstlisting}[frame=single]
Induct_on ` n `
 >- (
 sum 0 = 0 * (0 + 1) DIV 2)
 >- (
  0.  sum n = n * (n + 1) DIV 2
 ------------------------------------
      sum (SUC n) = SUC n * (SUC n + 1) DIV 2)
\end{lstlisting}

This tells us that the HOL4 equivalent to Lassies ``Induction on 'n''' is the
tactic \lstinline{Induct_on `n`}.
The tactic takes a HOL4 expression, called a term as input and tries to perform
an induction on it using the underlying types induction scheme.

Next, we see a so-called tactic combinator that can be used to chain tactics
together.
In Lassie's proofs we relied on the ``.'' to do this for you, similar to
pen-and-paper proofs.
A HOL4 tactic proof however requires manually putting the tactics together.
Here we see \lstinline{>-} which combines a tactic (\lstinline{Induct_on})
with a tactic solving a single subgoal.

If we send the next step, \lstinline{use [sum_def] to simplify.}, from the Lassie proof, the REPL shows
%
\begin{lstlisting}[frame=single]
Induct_on ` n `
 >- (
 fs [ sum_def ])
 >- (
  0.  sum n = n * (n + 1) DIV 2
 ------------------------------------
      sum (SUC n) = SUC n * (SUC n + 1) DIV 2)
\end{lstlisting}

The tactic \lstinline{fs [ sum_def ]} has been used by Lassie to solve the first subgoal
of the proof, but the second subgoal remained unchanged.
If we want to also simplify the second goal, we have to send the Lassie step again,
leaving us with
%
\begin{lstlisting}[frame=single]
Induct_on ` n `
 >- (
 fs [ sum_def ])
 >- (
 fs [ sum_def ] >>
   0.  sum n = n * (n + 1) DIV 2
  ------------------------------------
       SUC n + n * (n + 1) DIV 2 = SUC n * (SUC n + 1) DIV 2)
\end{lstlisting}

Here, we also see another tactic combinator of HOL4, \lstinline{>>}, which can
alternatively be written as \lstinline{\\}.
Applying tactic \lstinline{t1 >> t2} applies tactic \lstinline{t2} to every
subgoal  generated by \lstinline{t1} and fails if \lstinline{t2} fails on one of
the subgoals.
Note that for all tactic combinators, their second argument can themselves be
tactics composed by applications of \lstinline{>>} and \lstinline{>-}.

We recommend sending each of steps of the Lassie proof separately to construct a
full proof script and getting an intuitive meaning of the tactics.
If a step should be undone one can send the command \lstinline{back.}, and
\lstinline{help.} prints a quick help message.
Sending \lstinline{exit.} ends the \lstinline{nlexplain} session.
The proof script generated by Lassie can then be copied and played with
interactively to get a feeling for the tactics themselves.
We recommend using the goal manager used before to start an interactive proof of
the closed form of the summation.
Tactics from the proof script can then be send separately or joined together by
marking them and pressing \ekey{M-h e}.
%
\section{Small Tips and Tricks}\label{subsec:tipsAndTricks}

We end this tutorial by giving some helpful tips and tricks, and giving pointers
where to find more information about HOL4.

\subsection{Extending the Simplifier}
While proving the gaussian sum in \autoref{sec:hol_ex1} we had to explicitly
tell the simplifier that it should also use the definition of function
\lstinline{sum} while proving properties about the function.
In a pen-and-paper proof, one would never explicitly write this down and as such
it is desirable to have the same convenience in HOL4.
We can do so by slightly changing the definition of \lstinline{sum}:
%
\begin{lstlisting}
Definition sum_def[simp]:
  sum 0 = 0 /\
  sum n = n + sum (n-1)
End
\end{lstlisting}

By appending \lstinline{[simp]} to the name of the function, HOL4 automatically
adds \lstinline{sum_def} to the list of theorems used by the simplifier.
A similar mechanism exists for adding theorems to the simplifier.
However, this mechanic has to be used with caution as it is very easy to make
the simplifier diverge.

As an example, suppose we used the old definition of \lstinline{sum} which
defines the function as a recursive function not in equational style:

\begin{lstlisting}
Definition sum_def[simp]:
  sum n = if (n = 0) then 0 else n + sum (n-1)
End
\end{lstlisting}

If we restart the proof for the gaussian sum now, and run through the first two
tactics only (\lstinline{nltac `Induction on 'n'. simplify.`}) HOL4 will just
keep running.
As a rule of thumb, it is recommended to be conservative and rather mention a
definition than adding it to the simplifier.
The machinery can be useful for (non-recursive) abbreviations.
For theorems, one should refrain from adding commutativity or associativity
theorems, but adding theorems of the form $\forall x. P x \rightarrow Q x$, where
$Q$ does not depend on $P$ should be fine.

\subsection{Making Proof Scripts More Robust}
The most cumbersome work once a proof has been developed is making sure that it
remains correct even when versions of HOL4 change.
We give some simple recommendations that have proven quite useful over time.

First, we recommend commenting larger case splits and induction proofs.
While it may seem obvious now which case is being worked on by the proofscript
this might not be the case in a month, or a year of time after writing the
initial version.

Second, we recommend using tactics like
\lstinline{first_x_assum, last_assum, qpat_x_assum}.
These tactics are independent of the specific order of assumptions and thus make
the proof more robust to additional assumptions, or their removal.

\subsection{Getting More Help}
This small tutorial has only covered the basics.
More reference material can be found on \url{https://hol-theorem-prover.org/#doc}.
We especially recommend looking at the documentation of the emacs mode
(\url{https://hol-theorem-prover.org/hol-mode.html}), and the description manual.

The help index located at \lstinline{<HOLDIR>/help/HOLindex.html} provides
documentation for a lot of tactics and contains signature files for all of the
HOL4 distributions libraries and theories.

Finally, the HOL-info mailing list
(\url{https://sourceforge.net/projects/hol/lists/hol-info}) is a good place to
ask further questions, as well as the \texttt{\#hol} channel of the Slack of the
CakeML project (\url{https://join.slack.com/t/cakeml/shared_invite/MjM1NjEyODgxODkzLTE1MDQzNjgwMTUtYjI4YTdlM2VmMQ}).

%
\clearpage
\appendix
\section{Setup Script}\label{sec:script}
\lstinputlisting[language=bash,mathescape=false]{setup_hol4.sh}

%%% Local Variables:
%%% mode: latex
%%% TeX-master: "lassie-tutorial"
%%% End:

%
\printbibliography
%
\end{document}
%%% Local Variables:
%%% mode: latex
%%% TeX-master: t
%%% End:
