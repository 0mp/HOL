\documentclass[10pt]{article}
\addtolength\textwidth{1cm}
\addtolength\textheight{2cm}
\pagestyle{empty}

\begin{document}

\section*{Emacs HOL mode commands}

All of these commands are executed by first typing the HOL prefix
(Meta-h, or {\tt Esc}-h) and then the appropriate letter.


\begin{description}
\small
\item [h ``hol98''] Starts the HOL session.
\item [C-c ``hol-interrupt''] Interrupts the HOL session.  Useful for
  those tactics that don't always return.
\item [C-l ``hol-recentre''] Recentres the screen that HOL is running
  in, so that the most text is visible with the bottom of the text
  at the bottom of the screen.
\item [C-v ``hol-scroll-up''] Scrolls the HOL window.
\item [C-t ``hol-toggle-show-types''] Toggles the value of the {\tt
    show\_types} variable, affecting how terms are printed.
\item [M-b  ``backward-hol-tactic''] Moves the cursor back over the
  previous tactic in the source text in the current buffer.  With a
  \emph{prefix argument}, moves back that many tactics.
\item [M-f   ``forward-hol-tactic''] Moves the cursor forward over the
  next tactic in the source text in the current buffer.  With a \emph{prefix
  argument}, moves that many.
\item [M-r   ``send-region''] Sends the \emph{region} to the HOL
  process, where it is evaluated at the top level.  Can be used both
  to define new ML bindings, and to evaluate epxressions.
\item [M-v ``hol-scroll-down''] Scrolls the HOL window.
\item [b   ``hol-backup''] Backs up one stage in the goalstack.
\item [d  ``hol-drop-goal''] Drops the current goal, removing it from
  the goal-stack entirely.
\item [e ``expand-hol-tactic''] Sends the region to the HOL process as
  a tactic, where it is applied to the current goal.
\item [g ``hol-goal''] Sets the current goal.  Sends the term at point
  (delimited on both sides by back-quotes) to the HOL process.  With a
  \emph{prefix argument}, or if in transient mark mode with an active
  region, sends the selected region as the goal instead.
\item [l ``hol-load-file''] Loads a HOL library.  If there is a region
  marked, uses that string as the library to load.  Otherwise, if the
  cursor (point) is over a likely name, it uses that for the library
  to load.  As a last resort (or in all situations, if there is a
  \emph{prefix argument}), prompts for the name of the library.
\item [n ``hol-name-top-theorem''] Prompts for a name to give to the
  ``top theorem'' (i.e., the theorem just proved in the goalstack).
  With a \emph{prefix argument} also drops the goal.
\item [p ``hol-print''] Prints the top goal in the goalstack.
\item [r ``hol-rotate''] Rotates goals in the goalstack.  With a
  numeric \emph{prefix argument} rotates that many out of the way,
  instead of just one.
\item [R ``hol-restart''] Restarts the current proof.
\item [s ``send-string-to-hol''] Prompts for a string to be evaluated
  by SML, like ``M-r''.
\item [t ``mark-hol-tactic''] Marks the tactic at point, putting mark
  at the start of the tactic, and point at the end.  Useful as a
  prelude to applying the tactic with ``expand-hol-tactic''.
\item [u ``hol-use-file''] Prompts for a file-name to be \emph{use}-d
  at the top level.

\end{description}

\end{document}

